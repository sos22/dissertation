\chapter{Finding bugs}
\label{sect:derive}

The heart of {\technique} is a system for taking potential bugs,
described by \glspl{verificationcondition}, and turning them into
either \glspl{bugenforcer}, which check which ones are real, or fixes,
which eliminate them.  This chapter describes one approach to finding
the \glspl{verificationcondition}.  The basic algorithm proceeds as
follows:
\begin{itemize}
\item Identify all of the instructions which the \gls{crashingthread}
  might have executed in the \gls{alpha}-instruction
  \gls{analysiswindow} leading up to the crash.  These are represented
  as a \glslink{dynamic cfg}{dynamic \gls{cfg}}; the details are given
  in \autoref{sect:derive:build_crashing_cfg}, in the case where the
  crashing instruction is known in advance, and
  \autoref{sect:derive:unknown_bugs}, in the case where it is not.
\item Decompile the \gls{crashingthread}'s \gls{cfg} into an
  \emph{approximant}.  The {\StateMachine} abstraction is described in
  \autoref{sect:derive:state_machines} and the decompilation process
  in \autoref{sect:derive:compile_cfg}.
\item Simplify the {\StateMachine}.  {\STateMachines}, when initially
  derived, are faithful representations of the program's behaviour,
  and as such often contain a large amount of information which is
  irrelevant to the bug under investigation.  The simplification step
  is responsible for removing this redundant information.  It is
  described in \autoref{sect:derive:simplify_sm}.
\item Find instructions with which the \gls{crashingthread}'s
  {\StateMachine} might have raced, and hence build \glspl{cfg} and
  {\StateMachines} for all of the \glspl{interferingthread}.  This is
  described in \autoref{sect:derive:write_side}.
\item Characterise the circumstances under which racing the
  interfering {\StateMachines} against the crashing one might lead to
  a crash, expressing the result as a \gls{verificationcondition}.
  This step is described in \autoref{sect:using:check_realness}.
\end{itemize}

\section{Building the crashing thread's CFG}
\label{sect:derive:build_crashing_cfg}

{\Technique} considers concurrency bugs caused by unfortunate
interleavings of the \gls{alpha} (dynamic) instructions prior to the
crash, and hence the first step in analysing a bug is to find those
\gls{alpha} instructions.  {\Technique} does so by first finding all
of the static instructions which might execute in the window, by
examining the program's \gls{static cfg}, and then manipulating this
fragment of \gls{static cfg} in a semantics-preserving way so that
each static instruction executes at most once in the window.  At that
point, the static instructions can be reinterpreted as dynamic ones,
producing a \gls{dynamic cfg} which can be compiled into
{\AStateMachine} and passed on to the rest of the analysis.

\begin{sanefig}
\begin{tikzpicture}
  [node distance=0.9 and 0.15, inner sep = 0pt]
  \begin{scope}
    \node (A) at (0,2) [CfgInstr] {A};
    \node (B) [CfgInstr] [below=of A] {B}; 
    \node (C) [CfgInstr] [below=of B] {C}; 
    \node (D) [CfgInstr] [below=of C] {D}; 
    \node (dummy) [right = of B] {~~};
    \draw[->] (A) -- (B);
    \draw[->] (B) -- (C);
    \draw[->] (C) -- (D);
    \draw[->] (C.east) ++(-.1,0) ..controls +(.12,0) and +(0,-.12).. ++(.2,.2) -- ++(0,1.13) .. controls +(0,.12) and +(.12,0) .. ++(-.2,.2) -- ++(-.03,0);
    \begin{pgfonlayer}{bg}
      \node (box0) [fill=black!10,fit=(A) (B) (C) (D) (dummy)] {};
    \end{pgfonlayer}
  \end{scope}
  \begin{scope}[xshift=3.2cm]
    \node (A) at (0,2) [CfgInstr] {$A_0$};
    \node (B) [CfgInstr] [below=of A] {$B_0$}; 
    \node (C) [CfgInstr] [below=of B] {$C_0$}; 
    \node (D) [CfgInstr] [below=of C] {$D_0$}; 
    \node (dummy) [right = of B] {~~\!~};
    \draw[->] (A) -- (B);
    \draw[->] (B) -- (C);
    \draw[->] (C) -- (D);
    \draw[->] (C.east) ++(-.05,0) ..controls +(.12,0) and +(0,-.12).. ++(.2,.2) -- ++(0,1.13) .. controls +(0,.12) and +(.12,0) .. ++(-.2,.2) -- ++(-.03,0);
    \begin{pgfonlayer}{bg}
      \node (box1) [fill=black!10,fit=(A) (B) (C) (D) (dummy)] {};
    \end{pgfonlayer}
  \end{scope}
  \begin{scope}[xshift=5.8cm]
    \node (A) at (0,2) [CfgInstr] {$A_0$};
    \node (B) [CfgInstr] [below=of A] {$B_0$}; 
    \node (C) [CfgInstr] [below=of B] {$C_0$}; 
    \node (D) [CfgInstr] [below=of C] {$D_0$};  
    \node (C') [CfgInstr] [right=of C] {$C_1$};
    \draw[->] (A) -- (B);
    \draw[->] (B) -- (C);
    \draw[->] (C) -- (D);
    \draw[->] (B) to [bend right=10] (C');
    \draw[->] (C') to [bend right=10] (B);
    \begin{pgfonlayer}{bg}
      \node (box2) [fill=black!10,fit=(A) (B) (C) (D) (C')] {};
    \end{pgfonlayer}
  \end{scope}
  \begin{scope}[xshift=8.8cm]
    \node (A) at (0,2) [CfgInstr] {$A_0$};
    \node (B) [CfgInstr] [below=of A] {$B_0$};
    \node (B') [CfgInstr] [right=of B] {$B_1$};
    \node (C) [CfgInstr] [below=of B] {$C_0$};
    \node (D) [CfgInstr] [below=of C] {$D_0$};
    \node (C') [CfgInstr] [right=of C] {$C_1$};
    \draw[->] (A) -- (B);
    \draw[->] (B) -- (C);
    \draw[->] (C) -- (D);
    \draw[->] (C') -- (B);
    \draw[->] (A) -- (B');
    \draw[->] (B') to [bend right=10] (C');
    \draw[->] (C') to [bend right=10] (B');
    \begin{pgfonlayer}{bg}
      \node (box3) [fill=black!10,fit=(A) (B) (C) (D) (C') (B')] {};
    \end{pgfonlayer}
  \end{scope}
  \begin{scope}[xshift=11.8cm]
    \node (A) at (0,2) [CfgInstr] {$A_0$};
    \node (B) [CfgInstr] [below=of A] {$B_0$};
    \node (B') [CfgInstr] [right=of B] {$B_1$};
    \node (C) [CfgInstr] [below=of B] {$C_0$};
    \node (C') [CfgInstr] [right=of C] {$C_1$};
    \node (C'') [CfgInstr] [right=of C'] {$C_2$};
    \node (D) [CfgInstr] [below=of C] {$D_0$};
    \draw[->] (A) -- (B);
    \draw[->] (B) -- (C);
    \draw[->] (C) -- (D);
    \draw[->] (C') -- (B);
    \draw[->] (A) -- (B');
    \draw[->] (B') -- (C');
    \draw[->] (C'') to [bend right=10] (B');
    \draw[->] (B') to [bend right=10] (C'');
    \begin{pgfonlayer}{bg}
      \node (box4) [fill=black!10,fit=(A) (B) (C) (D) (C') (B') (C'')] {};
    \end{pgfonlayer}
  \end{scope}
  \draw[->,thick] (box0) -- (box1) node [midway] {\shortstack[c]{Reinterpret\\as dynamic}};
  \draw[->,thick] (box1) -- (box2) node [midway] {\shortstack[c]{unroll\\$C_0{\rightarrow}B_0$}};
  \draw[->,thick] (box2) -- (box3) node [midway] {\shortstack[c]{unroll\\$B_0{\rightarrow}C_1$}};
  \draw[->,thick] (box3) -- (box4) node [midway] {\shortstack[c]{unroll\\$C_1{\rightarrow}B_1$}};
  \draw[->,thick] (box4) -- +(2.28,0) node [above,midway] {$\cdots$};
\end{tikzpicture}
\caption{Converting a cyclic \glsentrytext{static cfg} to a dynamic
  one.  In this figure, and throughout this dissertation, instructions
  in the \glsentrytext{static cfg} are shown as roman letters and
  those in the \glsentrytext{dynamic cfg} are shown as subscripted
  italic letters.}
\label{fig:cyclic_cfg}
\end{sanefig}

It would, in principle, be possible to skip most of this step and
simply convert the program's \gls{static cfg} directly into
{\AStateMachine}.  Converting to a \gls{dynamic cfg} has two main
advantages.  First, it renders the \gls{cfg} acyclic, and hence
ensures that it completes in a finite number of steps, greatly
simplifying later analysis.  Of course, this also implies that the
resulting {\StateMachine} will only be able to capture the behaviour
of finite fragments of the program's execution but, since
{\StateMachines} are only used to investigate \glslink{simple
  atomicity violation}{SAVs} which involve at most \gls{alpha}
instructions, this is not in practice a problem.

Second, and more subtly, the use of a \gls{dynamic cfg} also makes it
meaningful to talk about whether one access happens before or after
another.  A single static instruction could execute many times in the
\gls{analysiswindow}, making it difficult to say anything meaningful
about whether it happens before or after some other instruction, and
without that there is little hope of being able to meaningfully
characterise the circumstances under which a concurrency bug will
reproduce.  Dynamic instructions happen at most once, by definition,
avoiding the issue completely.

As indicated above, the \gls{static cfg} used when analysing a
particular bug is built by taking the complete program \gls{static
  cfg} and trimming it to contain only the relevant instructions.
This presumes that {\technique} is able to generate the complete
\gls{cfg} in the first place.  {\Technique} does so using a standard
recursive traversal disassembler~\cite{Schwarz}.  Information from the
dynamic analysis is used to identify instructions in the program which
can be branched to by library code, and such instructions are used as
roots of the traversal.  Information from the dynamic analysis is also
used to locate possible targets of computed branch instructions.  As
it is collected from a dynamic analysis, this information is
potentially incomplete, and so the program \gls{cfg} might also be
incomplete, and so might, ultimately, be the eventual set of
\glspl{verificationcondition}.  This weakness is more aesthetic than
meaningful: \glspl{verificationcondition} are used primarily to build
\glspl{bugenforcer}, which are themselves inherently incomplete, so
minor lacunae in the \gls{cfg} cause few real problems.

\begin{wrapfigure}{r}{3.9cm}
\vspace{-14pt}
\begin{figgure}
\vspace{-2pt}
\begin{tikzpicture}
  [node distance=1 and 0.3]
  \node (A) at (0,2) [CfgInstr] {$A_0$};
  \node (B) [CfgInstr] [below=of A] {$B_0$};
  \node (B') [CfgInstr] [right=of B] {$B_1$};
  \node (C) [CfgInstr] [below=of B] {$C_0$};
  \node (C') [CfgInstr] [right=of C] {$C_1$};
  \node (C'') [CfgInstr] [above right=of B'] {$C_2$};
  \node (D) [CfgInstr] [below=of C] {$D_0$};
  \draw[->] (A) -- (B);
  \draw[->] (B) -- (C);
  \draw[->] (C) -- (D);
  \draw[->] (C') -- (B);
  \draw[->] (A) -- (B');
  \draw[->] (B') -- (C');
  \draw[->] (C'') -- (B');
  \begin{pgfonlayer}{bg}
    \node (box4) [fill=black!10,fit=(A) (B) (C) (D) (C') (B') (C'')] {};
  \end{pgfonlayer}
\end{tikzpicture}
\caption{Fully unrolled version of the CFG in
  \autoref{fig:cyclic_cfg}, preserving all paths of length six or
  less.}
\label{fig:unrolled_cyclic_cfg}
\vspace{-4pt}
\end{figgure}
\vspace{-13pt}
\end{wrapfigure}
Once an appropriate \gls{static cfg} has been built, {\technique}
converts it to a dynamic one by unrolling any loops.
\autoref{fig:cyclic_cfg} gives an example.  The \gls{static cfg}, on
the left, contains a loop between instructions B and C, and this loop
must be removed in a way which maintains all paths of length
\gls{alpha} or less which end at D, the \gls{crashing instruction}.
    {\Technique}'s algorithm for doing so starts by selecting a
    cycle-completing edge, in this case $C_0{\rightarrow}B_0$, and
    replacing it with one from a newly-constructed instruction $C_1$.
    $C_1$ has the same incoming edges as $C_0$ but no outgoing edges
    aside from the one to $B_0$.  The new graph is semantically
    equivalent to the old one, as $C_0$ and $C_1$ represent the same
    instruction in the original program and therefore have precisely
    the same effect when run, and the cycle has moved one instruction
    away from $D_0$.  The process repeats until all cycles are at
    least \gls{alpha} instructions from $D_0$, at which point the
    cycle closing edges can simply be discarded.  The resulting
    \gls{dynamic cfg} is shown in \autoref{fig:unrolled_cyclic_cfg}.
    The complete algorithm is shown in
    \autoref{fig:derive:read:unroll_cycle_break}.

\begin{sanefig}
\begin{algorithmic}
  \Function{staticToDynamicCrashing}{}
  \While {Graph contains a cycle}
     \State $(\mathit{start}, \mathit{end}) \gets \textsc{findCycleEdgeCrashing}(\text{crashing instruction})$
     \State {Remove edge $(\mathit{start}, \mathit{end})$}
     \If {$\mathit{start}$ is no more than $\alpha$ instructions from crashing instruction}
        \State {$newNode \gets$ duplicate $\mathit{start}$}
        \State {Create a new edge $(\mathit{newNode}, \mathit{end})$}
        \For {Edges $(\mathit{start}', \_)$ entering $\mathit{start}$}
           \State {Create a new edge $(\mathit{start}', \mathit{newNode})$}
        \EndFor
     \EndIf
  \EndWhile
  \EndFunction
\end{algorithmic}
\caption{Loop unrolling and cycle breaking algorithm.
  \textsc{findCycleEdgeCrashing} simply performs a depth-first search
  of the graph backwards from the crashing instruction and returns the
  first edge which completes a cycle.}
\label{fig:derive:read:unroll_cycle_break}
\end{sanefig}

This might appear, on the face of it, to be a rather expensive
algorithm: it must explore every path of length \gls{alpha} ending at
the crashing instruction, and the number of such paths is potentially
exponential in \gls{alpha}.  This is true, but several other
algorithms used by {\implementation} also have exponential worst-case
running time and larger constant factors, and so in practice deriving
the crashing \gls{cfg} accounts for only a small percentage of the
total analysis time.

\section[Generating \glsentryplural{cfg} for unknown bugs]{Generating \glspl{cfg} for unknown bugs}
\label{sect:derive:unknown_bugs}

The algorithm given in \autoref{sect:derive:build_crashing_cfg}
assumed that the \gls{crashing instruction} was somehow
pre-identified, but {\technique} can also be used to discover unknown
bugs, for which the \gls{crashing instruction} is initially
unspecified.  {\Technique} discovers it by brute force: it enumerates
every possible \gls{crashing instruction} in the program, for a given
class of bug, and attempts to generate \glspl{verificationcondition}
for each in turn.  This is feasible because the vast majority of
potential \glspl{crashing instruction} can be dismissed very quickly:
{\implementation} takes, on average, a few seconds to process each
instruction on fairly modest hardware, allowing even large programs
with hundreds of thousands of instructions to be analysed in a few
days.  Further, this approach is embarrassingly parallel, and so would
be expected to scale well as hardware concurrency increases, which is
precisely the scenario in which it would be most useful.

The way in which these potentially-crashing instructions are
identified depends on the type of bug which is to be investigated.
{\Implementation} considers three types of bug:
\begin{itemize}
\item Assertion failures.  These are caused by the program calling a
  function such as \verb|__assert_fail| or \verb|abort|.  Finding such
  functions is straightforward, given the usual dynamic linker
  information, and the whole-program \gls{cfg}, discussed earlier, can
  provide all of their callers.  Each such caller is treated as a
  potential bug.
\item Double free errors.  These are caused by the program calling
  \verb|free| in an incorrect manner.  Again, the dynamic linker
  information allows {\implementation} to quickly find all calls to
  \verb|free| in the program, and these calls are used as the
  potentially-crashing instructions in the program.
\item Bad pointer dereferences.  Any memory-accessing instruction
  could potentially dereference a bad pointer, and so
  {\implementation} simply enumerates all memory accessing
  instructions discovered by the initial static analysis.
\end{itemize}
Other classes of bugs would generate different sets of potentially
crashing instructions.

\subsection{Timeouts}

Many of the algorithms used by {\technique} require far more time in
their worst case than in their expected case, often by many orders of
magnitude.  This is irritating but tolerable when the analysis is
being used to investigate a specific bug, but far more of a problem
when the analysis is applied speculatively to a very large number of
potential bugs.  Suppose, for instance, that the analysis completes in
500ms in 99.9\% of cases but takes forty years in the remaining 0.1\%
of cases.  An analysis which fails 0.1\% of the time would still be
quite useful, and so this is reasonable for analysing specific bugs.
On the other hand, if the analysis is run 10,000 times then the
probability of at least one of the steps taking forty years is very
close to one and the analysis is effectively useless.

{\Technique} works around this problem by applying timeouts to the
various analysis steps, ensuring that it can produce at least some
useful results in a reasonable time even when it occasionally
encounters one of its bad cases.  {\Implementation} uses two
independent timeouts, one for per-\gls{crashingthread} work and the
other for per-\gls{interferingthread} work, both of which are set by
default to five minutes.  I discuss their effects in more detail in
the evaluation.

\section{\STateMachines}
\label{sect:derive:state_machines}

The \gls{dynamic cfg} represents all of the instructions in the
\gls{crashingthread} in the \gls{analysiswindow}, and therefore in
principle contains all of the information needed to analyse the bug.
It is, however, very close to machine code, and so is rather difficult
to work with.  The next step in the algorithm is to convert it into a
more amenable form: the {\StateMachine}.  {\STateMachines} are, in
effect, small programs which model the relevant behaviour of the
program under investigation, written in a language which can easily
model machine code but which is better suited to the needed analyses
and transformations.  Programs in this language consist of a directed
acyclic graph of {\StateMachine} states, defined in
figures~\ref{fig:state_machine_states}
and~\ref{fig:state_machine_exprs}.\fnote{The actual implementation
  used by {\implementation} include several features, such as sub-word
  memory accesses and bounded-range integers, not given in these
  figures, but these are not essential to the {\technique} technique
  and I do not give details here.}  These {\StateMachine} programs
examine the state of the program and, based on that, produce a
prediction about the program's ultimate behaviour, which is either
\textsc{Crash}, \textsc{Survive}, or \textsc{Unreached}.
\textsc{Crash} and \textsc{Survive} indicate that the program either
will or will not, respectively, reproduce the bug which is under
investigation.  \textsc{Unreached} indicates that a particular path
through the {\StateMachine} should be ignored for some reason, most
often because the matching path through the original program would
suffer some fatal error other than the bug of interest.

\begin{sanefig}
{\hfill}
\begin{tabular}{llllp{6.05cm}}
\multicolumn{2}{l}{State}       & \multicolumn{2}{l}{Fields} & Meaning \\
\hline
\multicolumn{2}{l}{\state{If}}  & \state{cond} & Boolean expression        & Conditional branch with two successor states.  Evaluates \state{cond}, branching to one successor if it is true and the other if it is false. \\
\hline
\multicolumn{2}{l}{Terminal states} &          &             & Terminal states.  {\STateMachine} execution finishes when it reaches one of these. \\
\hdashline
 & {\stSurvive}              &              &             & The bug has been avoided. \\
\hdashline
 & {\stCrash}                &              &             & The bug will definitely happen. \\
\hdashline
 & {\stUnreached}            &              &             & A contradiction has been reached. \\
\hline
\multicolumn{2}{l}{Side-effect states}\\
 & \state{Load}                 & \state{addr} & Integer expression & \multirow{3}{6.05cm}{\parbox{6.05cm}{Load from the \gls{speicher} at address \state{addr} and store the result in {\AStateMachine} variable.}} \\
 &                              & \state{var}  & {\STateMachine} variable \\
\\
\hdashline
 & \state{Store}                & \state{addr} & Integer expression & \multirow{2}{6.05cm}{\parbox{6.05cm}{Store the value of \state{data} to the \gls{speicher} at address \state{addr}.}}\\
 &                              & \state{data} & Integer expression \\
\hdashline
 & \state{Copy}                 & \state{data} & Integer expression & \multirow{3}{6.05cm}{\parbox{6.05cm}{Evaluate an expression and store the result in {\AStateMachine} variable.}} \\
 &                              & \state{var}  & {\STateMachine} variable \\
\\
\hdashline
 & \state{Assert}               & \state{expr} & Boolean expression & Note that a given condition is true at a particular point in the {\StateMachine}'s execution. \\
\hdashline
 & $\Phi$                       &              &                 & Implement an SSA $\Phi$ node. \\
\hdashline
 & {\stStartAtomic}          &              &                 & \multirow{3}{6.05cm}{Delimit atomic blocks, used to constrain the set of schedules which must be considered; see \autoref{sect:using:build_cross_product}.} \\
 & {\stEndAtomic}            \\
\\
\\
\end{tabular}
{\hfill}
\caption{Types of {\StateMachine} states.  The expression language is
  described in \autoref{fig:state_machine_exprs}.  A more formal
  description of the state behaviour is given in
  \autoref{fig:derive:sm_semantics}.}
\label{fig:state_machine_states}
\end{sanefig}

The details of the analysis language are important for analysis
performance but are not critical to the overall structure of the
{\technique} technique.  I therefore give only a brief overview of the
language's most important features:
\begin{itemize}
\item {\STateMachine} states do not directly correspond to
  instructions in the original program: one state might represent
  several instructions, and a single instruction might be represented
  by multiple states or by none at all.
\item {\STateMachines} have access to an infinite set of
  {\StateMachine} variables each of which contains a single integer
  value.  The {\StateMachine} is maintained in single static
  assignment form~\cite{cytron1991} with respect to these variables.
\item In addition to the variables, {\StateMachines} also have access
  to a mutable mapping from integers to integers called the
  \gls{speicher} which is initialised to the contents of program
  memory.  The \gls{speicher} is accessed only via the \state{Load}
  and \state{Store} side-effects.
\item {\STateMachines} execute in the context of a particular
  \gls{environment} which provides some information on the run-time
  state of the program which is being modelled.  This includes the
  contents of processor registers and program memory when the
  {\StateMachine} started, the path which each thread takes through
  the \gls{dynamic cfg}, and the order in which memory accesses happen
  when multiple threads are being modelled.  Unlike the
  \gls{speicher}, the contents of the environment does not change as
  the {\StateMachine} executes, and so expressions which reference the
  environment are effectively constants, simplifying the analysis.
\end{itemize}
This model of computation is not Turing complete (in particular,
{\StateMachines} are guaranteed to terminate in a finite number of
steps).  This simplifies analysis, but implies that they cannot
capture the entire program's complete behaviour.  {\Technique} uses
them only to model the \gls{alpha}-instruction \gls{analysiswindow},
and so this is not, in practice, a problem.

\begin{sanefig}
\begin{tabular}{lp{11.35cm}}
Expression & Meaning \\
\hline
$\smVar{i}$ & The value of {\StateMachine} variable $i$. \\
$\happensBefore{A_i}{B_j}$ & True if memory access $A_i$ happens before memory access $B_j$. \\
$\entryExpr_t{A_i}$ & True if thread $t$ starts with instruction $A_i$. \\
$\controlEdge{t}{A_i}{B_j}$ & True if thread $t$ executed dynamic instruction $B_j$ immediately after instruction $A_i$, false if it executed some other instruction after $A_i$, and undefined if it did not execute $A_i$ at all.\\
$\smBadPtr{expr}$ & True if $\mathit{expr}$ evaluates to a value which is not a valid pointer.\\
$\smLoad{expr}$ & The initial value of program memory at address $\mathit{expr}$.\\
$\smReg{r}{t}$ & The initial value of register \textsc{r} in thread $t$.\\
\end{tabular}
\caption{Expressions in the {\StateMachine} expression language.  The
  usual arithmetic operators, such as $+$, $\times$, $\wedge$, etc.,
  are also supported.}
\label{fig:state_machine_exprs}
\end{sanefig}

\begin{sanefig}
  \begin{tabular}{lc}
    Initial configuration & $\mathrm{X} \stackrel{e}{\rightarrow} \{X, \varnothing, e.\mathit{memory}\}$\\
    \\
    \multirow{2}{*}{\raisebox{-50pt}{\state{If} states}} &
    \AxiomC{$\mathit{a} \downarrow_{c,t} \true$}
    \UnaryInfC{
      \begin{math}
        \left\{
          \begin{tikzpicture}[baseline = (current bounding box.center), node distance = 0.3 and -0.5]
            \node (r) [stateIf] {\stIf{\mathit{a}}};
            \node (X) [below left = of r] {X};
            \node (Y) [below right = of r] {Y};
            \draw[->,ifTrue] (r) -- (X);
            \draw[->,ifFalse] (r) -- (Y);
          \end{tikzpicture}, t, m
        \right\} \stackrel{e}{\rightarrow} \{\mathrm{X}, t, m\}
      \end{math}
    }
    \DisplayProof
    \vspace{4pt}\\ &
    \AxiomC{$\mathit{a} \downarrow_{c,t} \false$}
    \UnaryInfC{
      \begin{math}
        \left\{
          \begin{tikzpicture}[baseline = (current bounding box.center), node distance = 0.3 and -0.5]
            \node (r) [stateIf] {\stIf{\mathit{a}}};
            \node (X) [below left = of r] {X};
            \node (Y) [below right = of r] {Y};
            \draw[->,ifTrue] (r) -- (X);
            \draw[->,ifFalse] (r) -- (Y);
          \end{tikzpicture}, t, m
        \right\}
        \stackrel{e}{\rightarrow} \{\mathrm{Y}, t, m\}
      \end{math}
    }
    \DisplayProof
    \vspace{30pt}\\

    \multirow{2}{*}{\raisebox{-14pt}{Terminal states}} &
    \AxiomC{}
    \UnaryInfC{$\left\{\tikz[baseline = (s.base)]{\node (s)[stateTerminal] {\stSurvive};}, t, m\right\} \stackrel{e}{\rightarrow} \stSurvive$}
    \DisplayProof
    \vspace{8pt}\\
    &\AxiomC{}
    \UnaryInfC{$\left\{\tikz[baseline = (s.base)]{\node (s) [stateTerminal] {\stCrash};}, t, m\right\} \stackrel{e}{\rightarrow} \stCrash$}
    ~\DisplayProof\vspace{8pt}
    \vspace{30pt}\\

    \multirow{4}{*}{\raisebox{-110pt}{Side-effect states}} &
    \AxiomC{$\mathit{addr} \downarrow_{c,t} \mathit{addr}'$}
    \AxiomC{$t' = t[\smVar{T} = m(\mathit{addr}')]$}
    \BinaryInfC{
      \begin{math}
        \left\{
          \begin{tikzpicture}[baseline = (current bounding box.center), node distance = .3]
            \node (r) [stateSideEffect] {\stLoad{T}{\mathit{addr}}};
            \node (X) [below = of r] {X};
            \draw[->] (r) -- (X);
          \end{tikzpicture}, t, m
        \right\}
        \stackrel{e}{\rightarrow} \{\mathrm{X}, t', m\}
      \end{math}
    }
    \DisplayProof
    \vspace{4pt}\\
    & \AxiomC{$\mathit{addr} \downarrow_{c,t} \mathit{addr}'$}
    \AxiomC{$\mathit{data} \downarrow_{c,t} \mathit{data}'$}
    \AxiomC{$m' = m[\mathit{addr'} = \mathit{data}']$}
    \TrinaryInfC{
      \begin{math}
        \left\{
          \begin{tikzpicture}[baseline = (current bounding box.center), node distance = .3]
            \node (r) [stateSideEffect] {\stStore{\mathit{data}}{\mathit{addr}}};
            \node (X) [below = of r] {X};
            \draw[->] (r) -- (X);
          \end{tikzpicture}, t, m
        \right\}
        \stackrel{e}{\rightarrow} \{\mathrm{X}, t, m'\}
      \end{math}
    }
    \DisplayProof
    \vspace{4pt}\\
    & \AxiomC{$\mathit{data} \downarrow_{c,t} \mathit{data}'$}
    \AxiomC{$t' = t[\smVar{T} = \mathit{data}']$}
    \BinaryInfC{
      \begin{math}
        \left\{
          \begin{tikzpicture}[baseline = (current bounding box.center), node distance = .3]
            \node (r) [stateSideEffect] {\stCopy{T}{\mathit{data}}};
            \node (X) [below = of r] {X};
            \draw[->] (r) -- (X);
          \end{tikzpicture}, t, m
        \right\}
        \stackrel{e}{\rightarrow} \{\mathrm{X}, t', m\}
      \end{math}
    }
    \DisplayProof\\
    & \AxiomC{$\mathit{cond} \downarrow_{c,t} \true$}
    \UnaryInfC{
      \begin{math}
        \left\{
          \begin{tikzpicture}[baseline = (current bounding box.center), node distance = .3]
            \node (r) [stateSideEffect] {\stAssert{\mathit{cond}}};
            \node (X) [below = of r] {X};
            \draw[->] (r) -- (X);
          \end{tikzpicture}, t, m
        \right\}
        \stackrel{e}{\rightarrow} \{\mathrm{X}, t, m\}
      \end{math}
    }
    \DisplayProof
  \end{tabular}
  \caption{Small-step semantics for single-thread {\StateMachines};
    these are generalised to multi-threaded programs in
    \autoref{sect:using:build_cross_product}.  The
    $\stackrel{e}{\rightarrow}$ relationship shows how to evaluate the
    \StateMachine{} configuration $\{\mathrm{X}, t, m\}$, where
    $\mathrm{X}$ is the current {\StateMachine} state, $t$ the
    variables, and $m$ the \glsentrytext{speicher} under the
    environment $e$, giving either a new configuration,
    \textsc{Survive}, or \textsc{Crash}.  Configurations which do not
    match any of these rules evaluate to \textsc{Unreached}.
    $\mathrm{expr} \downarrow_{c,t} \mathrm{value}$ is true when the
    expression $\mathrm{expr}$ evaluates to the value $\mathrm{value}$
    under environment $c$ and variables $t$.}
  \label{fig:derive:sm_semantics}
\end{sanefig}

\section[Converting the \glsentrytext{dynamic cfg} to \AStateMachine]{Converting the \gls{dynamic cfg} to \AStateMachine}
\label{sect:derive:compile_cfg}

The {\StateMachine} analysis language is powerful enough to make
translating individual instructions in isolation
straightforward,\kern-.6pt\fnote{{\Implementation} uses
  LibVEX~\cite[Chapter 2]{FFFSeward2012} to decode AMD64 machine code
  before performing this translation.} but connecting the instructions
together is not always trivial.  There are three cases which require
special care:
\begin{itemize}
\item
  Some control flow edges will be missing from the \gls{dynamic cfg}.
  For instance, in \autoref{fig:unrolled_cyclic_cfg}, the program's
  \gls{static cfg} contained an edge from C to D but the dynamic one
  does not include any edges from $C_1$ to any dynamic instruction
  derived from D.  These missing edges are converted to branches to
  the {\stUnreached} state, reflecting the fact that these paths are
  of no interest to the rest of the analysis.
\item
  Some additional edges will have been introduced which do not
  correspond to anything in the original program.  In the example,
  static instruction A had a single successor, B, but the dynamic
  instruction $A_0$ has two successors, $B_0$ and $B_1$.  Each of the
  $B_i$ instructions will be represented by a separate {\StateMachine}
  state, but there is no condition on the original program's state
  which can be evaluated at $A_0$ which determines which must execute
  next.  {\Technique} converts these into {\StateMachine}-level
  control flow using ${\controlEdgeName}()$ expressions which test the
  program's path through the \gls{dynamic cfg}.
\item
  The \gls{cfg} can sometimes have multiple roots, each represented by
  a separate {\StateMachine} state, but the {\StateMachine} itself
  must have a single entry state.  {\Technique} handles these using
  $\entryExpr{}$ expressions, which simply test where a thread entered
  the \gls{dynamic cfg}.
\end{itemize}

\parshape 1
  0pt 9.5cm
\noindent As a somewhat unrealistic example, suppose that the
\gls{cfg} in \autoref{fig:cyclic_cfg} were generated from the program
shown in \autoref{fig:derive:example_dissassembly1}.  The
\texttt{JMP\_NE} instruction C loads from the memory at
\texttt{rcx+8}, jumping to B if it is non-zero and proceeding to D
otherwise.  This will produce a \gls{dynamic cfg} as in
\autoref{fig:unrolled_cyclic_cfg}, as already discussed, and
\AStateMachine{} as shown in
\autoref{fig:state_machine_for_cyclic_cfg}.

\parshape 1 9.3cm 6.2cm \vspace{-4.2cm}{\parbox{6.2cm}{
    \begin{figgure}
  \begin{centering}
    \texttt{
      \begin{tabular}{l<{\hspace{-2mm}} llll}
        \!\!\!\!\!\!{\rm A}: & MOV  & rdx    &\!\!\!-> & \!\!\!rcx\\
        \!\!\!\!\!\!{\rm B}: & LOAD & *(rcx) &\!\!\!-> & \!\!\!rcx\\
        \!\!\!\!\!\!{\rm C}: & \multicolumn{4}{l}{\kern-1ptJMP\_NE *(rcx + 8), 0, {\rm B}}\\
        \!\!\!\!\!\!{\rm D\kern-1pt}: & STORE & \$0 &\!\!\!-> & \!\!\!*(rcx)\\
      \end{tabular}
    }
  \end{centering}
  \captionof{figure}{}
  \label{fig:derive:example_dissassembly1}
  \end{figgure}
}}
\\

\begin{sanefig}
\begin{tikzpicture}[minimum height=1cm, minimum width=6.5cm]
  \node[stateIf] (l1) {\stIf{\entryExpr{A_0}}};
  \node[right = of l1] (dummy1) {};
  \node[stateSideEffect,below = of l1] (l2) {$A_0$: \state{Copy} $\smReg{rdx}{} \rightarrow \smVar{0}$};
  \node[stateIf,below = of l2, inner sep = -5mm] (l3) {$A_0$: \stIf{{\controlEdgeName}(A_0 \!\rightarrow\! B_0)}};
  \node[stateSideEffect,below = of l3] (l3b) {\state{$\Phi$} $(\smVar{0}, \smVar{4}) \rightarrow \smVar{5}$};
  \node[stateSideEffect,below = of l3b] (l4) {$B_0$: \state{Load} $\ast(\smVar{5}) \rightarrow \smVar{6}$};
  \node[stateSideEffect,below = of l4] (l5) {$C_0$: \stLoad{7}{\smVar{6}+8}};
  \node[stateIf,below = of l5] (l6) {$C_0$: \stIf{\smVar{7} = 0}};
  \node[stateIf,below = of l6] (l7) {$D_0$: \stIf{\smBadPtr{\smVar{6}}}};

  \node[stateSideEffect] (l11) at (dummy1 |- l2) {$C_2$: \stLoad{1}{\smReg{rcx}{}+8}};
  \node[stateIf] at (l3 -| l11) (l12) {$C_2$: \stIf{\smVar{1} = 0}};

  \node[stateSideEffect] at (l12 |- l3b) (l8b) {\state{$\Phi$} $(\smVar{0}, \smReg{rcx}{}) \rightarrow \smVar{2}$};
  \node[stateSideEffect] at (l12 |- l4) (l8) {$B_1$: \state{Load} $\ast(\smVar{2}) \rightarrow \smVar{3}$};
  \node[stateSideEffect] at (l8 |- l5) (l9) {$C_1$: \stLoad{4}{\smVar{3}+8}};
  \node[stateIf] at (l9 |- l6) (l10) {$C_1$: \stIf{\smVar{4} = 0}};

  \node[stateTerminal,below = of l7, minimum width=3.5cm] (lBeta) {\stCrash};
  \node[stateTerminal,minimum width=3.5cm] at (lBeta -| l10)(lAlpha) {\stUnreached};
  \node[stateTerminal,minimum width=3.5cm] at (barycentric cs:lBeta=0.5,lAlpha=0.5)(lGamma) {\stSurvive};

  \draw[->,ifTrue] (l1) -- (l2);
  \draw[->,ifFalse] (l1) -- (l11);
  \draw[->] (l2) -- (l3);
  \draw[->,ifFalse] (l3) -- (l8b);
  \draw[->,ifTrue] (l3) -- (l3b);
  \draw[->] (l3b) -- (l4);
  \draw[->] (l4) -- (l5);
  \draw[->] (l5) -- (l6);
  \draw[->,ifFalse] (l6) -- (lAlpha);
  \draw[->,ifTrue] (l6) -- (l7);
  \draw[->,ifFalse] (l7) -- (lGamma);
  \draw[->,ifTrue] (l7) -- (lBeta);
  \draw[->] (l8b) -- (l8);
  \draw[->] (l8) -- (l9);
  \draw[->] (l9) -- (l10);
  \draw[->,ifTrue] (l10) -- (lAlpha);
  \draw[->,ifFalse] (node cs:name=l10,angle=210) ..controls +(0,-.3) and +(.3,0) .. ++(-.5,-0.5) % Leaving node
      -- ++(-1.9,0) % First horizontal
      .. controls +(-.3,0) and +(0,-.3) .. ++(-.5,.5) % Bend to vertical
      -- ++(0,7) % Vertical
      .. controls +(0,.3) and +(.3,0) .. ++(-.5,.5) % Bend to second horizontal
      -- ++(-.5,0) % Second horizontal
      ..controls +(-.3,0) and +(0,.3) .. (node cs:name=l3b,angle=12);
  \draw[->] (l11) -- (l12);
  \draw[->,ifTrue] (l12.east) .. controls +(.3,0) and +(0,.3) .. ++ (.5,-.5) -- ++(0,-11.08) .. controls +(0,-.3) and +(.3,0) .. ++ (-.5,-.5) -- (lAlpha);
  \draw[->,ifFalse] (l12) -- (l8b);
\end{tikzpicture}
\caption{{\STateMachine} generated from the \glsentrytext{dynamic cfg}
  shown in \autoref{fig:cyclic_cfg}.}
\label{fig:state_machine_for_cyclic_cfg}
\end{sanefig}

\subsection{Library functions}
\label{sect:derive:library_functions}

Any non-trivial program will include calls to system library
functions, and {\technique} must incorporate the effects of these
calls into its {\StateMachines}.  It would, in principle, be possible
to treat libraries identically to the main program, allowing the
\gls{dynamic cfg} to cross between the program and its libraries
whenever the program's control flow does, but this approach would have
two important disadvantages.  First, it requires the library to be
available during the analysis step, which is not always possible.
Second, and more importantly, it does not lead to a particularly
effective analysis.  Many library functions have simple interfaces but
complex implementations; incorporating these functions into
{\technique}'s {\StateMachines} would overwhelm the
\gls{analysiswindow} for little gain.  It would, for instance, be
impractical to detect double-free bugs by symbolically executing the
\texttt{free} function (or, at the very least, would require a far
more powerful symbolic execution than {\technique} otherwise needs).

{\Technique} instead relies on manually translating important library
functions into {\StateMachine} fragments which can be substituted in
place of calls to them.  These implementations are generally quite
simplistic and model only the most relevant parts of the function's
behaviour.  For instance, in a real program the \texttt{write}
function takes a buffer of a given size and writes it to a given file
descriptor, updating the file on disk and the file descriptor offset,
and returns either the number of bytes written or an error code; in
{\atechnique} {\StateMachine}, it simply returns the size of the
buffer, claiming to have completed successfully despite discarding
most of its arguments.  This is insufficient to properly investigate
bugs which depend on interactions with library functions (a race which
can only be triggered following an IO error, for instance, will not be
detected), but is sufficient when library calls incidentally happen
near to the buggy code without meaningfully influencing its behaviour.

Some library functions are more directly relevant to {\technique} and
deserve more complete implementations.  \texttt{pthread\_mutex\_lock}
and \texttt{pthread\_mutex\_unlock}, for instance, directly influence
the program's concurrency behaviour, and are hence important to many
concurrency bugs.  \autoref{fig:library_mux} shows how they are
implemented in {\technique}.  Note that the lock operation does not
include any logic to wait for the lock to be released, as such
operations are difficult to express in the acyclic {\StateMachine}
structure, but instead simply {\stAssertN}s that it is not currently
held.  That is sufficient: the key property of a lock is that it
cannot simultaneously be held by multiple threads, and this assumption
ensures that any paths on which that property is violated will
evaluate to \textsc{Unreached} and hence be discarded by the rest of
the analysis.

\begin{sanefig}
  \centerline{
    {\hfill}
  \subfigure[][pthread\_mutex\_lock]{
    \begin{tikzpicture}[minimum height=1cm, minimum width=7cm]
      \node (l1) [stateSideEffect] {\stStartAtomic};
      \node (l2) [below = of l1, stateSideEffect] {\stLoad{0}{\mathit{arg0}}};
      \node (l3) [below = of l2, stateSideEffect] {\stAssert{\smVar{0} = 0}};
      \node (l4) [below = of l3, stateSideEffect] {\stStore{\mathit{tid}}{\mathit{arg0}}};
      \node (l5) [below = of l4, stateSideEffect] {\stEndAtomic};
      \draw[->] (l1) -- (l2);
      \draw[->] (l2) -- (l3);
      \draw[->] (l3) -- (l4);
      \draw[->] (l4) -- (l5);
    \end{tikzpicture}
  }{\hfill}
  \subfigure[][pthread\_mutex\_unlock]{
    \begin{tikzpicture}[minimum height=1cm, minimum width=7cm]
      \node (l1) [stateSideEffect] {\stStartAtomic};
      \node (l2) [below = of l1, stateSideEffect] {\stLoad{0}{\mathit{arg0}}};
      \node (l3) [below = of l2, stateSideEffect] {\stAssert{\smVar{0} = \mathit{tid}}};
      \node (l4) [below = of l3, stateSideEffect] {\stStore{0}{\mathit{arg0}}};
      \node (l5) [below = of l4, stateSideEffect] {\stEndAtomic};
      \draw[->] (l1) -- (l2);
      \draw[->] (l2) -- (l3);
      \draw[->] (l3) -- (l4);
      \draw[->] (l4) -- (l5);
    \end{tikzpicture}
  }
    {\hfill}
  }
  \vspace{-12pt}
  \caption{{\STateMachine} models for the pthread\_mutex\_lock and
    pthread\_mutex\_unlock functions.  $\mathit{arg0}$ is an
    expression for the first argument register.  $\smVar{x}$ is a
    fresh {\StateMachine} variable.  $\mathit{tid}$ is the constant 1
    for the \glsentrytext{crashingthread} and 2 for the interfering one.}
  \label{fig:library_mux}
\end{sanefig}

Similarly, correctly modelling the \texttt{free} function is
particularly important when investigating double-free bugs.
{\Technique} uses two implementations of \texttt{free}: the crashing
\texttt{free}, for the call which is being investigated as a potential
crash site, and the non-crashing \texttt{free}, for every other call
(including other calls in the \gls{crashingthread}).  Both are shown
in \autoref{fig:library_free}.  Non-crashing \texttt{free}s set the
special $\mathit{last\_free}$ address to the pointer which was
released and the crashing \texttt{free} asserts that the pointer which
it releases is not the one which was most recently released.  This
scheme is clearly not capable of detecting all possible double-free
bugs, but it is sufficient for the most common kind.

\begin{sanefig}
  \centerline{
  \subfigure[][Crashing {\tt free}]{
    \begin{tikzpicture}[minimum height=1cm,minimum width=7cm]
      \node (l1) [stateSideEffect] {\stLoad{0}{\mathit{last\_free}} };
      \node (l2) [stateIf, below = of l1] {\stIf{\smVar{0} = \mathit{arg0}} };
      \node (l3) [stateTerminal,minimum width=3.5cm] at (-2cm,-4cm) {\stCrash };
      \node (l4) [stateTerminal,minimum width=3.5cm] at (2cm,-4cm) {\stSurvive };
      \draw[->] (l1) -- (l2);
      \draw[->,ifTrue] (l2) -- (l3);
      \draw[->,ifFalse] (l2) -- (l4);
    \end{tikzpicture}
    \label{fig:library_free:crashing}
  }
  \subfigure[][Non-crashing {\tt free}]{
    \raisebox{20mm}{
    \begin{tikzpicture}[minimum height=1cm,minimum width=7cm]
      \node [stateSideEffect] {\stStore{\mathit{arg0}}{\mathit{last\_free}} };
    \end{tikzpicture}
    }
    \label{fig:library_free:non_crashing}
  }
  }
  \vspace{-12pt}
  \caption{{\STateMachine} implementations of the {\tt free}
    function. $\mathit{arg0}$ is an expression for the first argument
    register.  $\mathit{last\_free}$ is any fixed memory location
    which is not used by the program.}
  \label{fig:library_free}
\end{sanefig}

\section{Simplifying the {\StateMachine}}
\label{sect:derive:simplify_sm}

Much as an optimising compiler optimises its intermediate form so as
to make the generated program run faster, {\technique} simplifies its
{\StateMachines} so as to make symbolically executing them less
expensive.  The main techniques used to do so are:
\begin{itemize}
\item \emph{Dead code elimination} is used to eliminate redundant
  updates to {\StateMachine} variables.
\item \emph{Copy propagation}, based on the algorithm from the
  dcc~\cite{Cifuentes1994} decompiler, is used to shorten long chains
  of assignments.  One minor extension over the dcc algorithm is that
  {\technique} can make use of {\stAssertN} side-effects during this
  transformation, so that, for instance, if $x$ is assumed to be less
  than $7$ then the expression $x > 22$ can be replaced by \false.
  This does not require any important changes to the algorithm, beyond
  a few simple rules describing when such rewrites are valid.
\item \emph{{\stPhi} elimination} is used to turn {\stPhi} states into
  \state{Copy} ones.  This reifies the {\stPhi} states' implicit
  dependence on {\StateMachine} control flow into the explicit
  structure of {\AStateMachine} expression, making it easier for the
  other simplification steps to manipulate and use it.
\item \emph{Alias analysis} determines how \state{Store} and
  \state{Load} operations might interact and then uses this
  information to forward values from \state{Store}s to \state{Load}s,
  potentially eliminating both.
\item Various minor peephole simplifications optimise some common
  {\StateMachine} patterns by, for instance, removing empty atomic
  regions or combining long chains of related \state{If} operations.
\end{itemize}
The effect of these simplification passes is to take {\AStateMachine}
which represents all of the program's behaviour in the
\gls{analysiswindow} and transform it to one which represents only the
behaviour which is most relevant to the bug under investigation.

\section{Building the interfering thread's \StateMachines}
\label{sect:derive:write_side}

Once {\technique} has built the crashing {\StateMachine}, the next
step is to find all of the instructions with which it might possibly
race and convert them to an appropriate set of interfering
{\StateMachines}.  In principle, this should consider every possible
interleaving of the crashing {\StateMachine} with every
\gls{alpha}-instruction long trace of any other thread in the program,
but this would clearly be completely impractical for all but the most
trivial programs.  Fortunately, most such traces can be dismissed
without ever needing to explicitly enumerate them, and this usually
makes it possible to complete the analysis in a reasonable amount of
time.

This reduction depends on information in the \gls{programmodel}, which
has not yet been described (see \autoref{sect:program_model}).  For
now, assume that it is possible to map from one memory-accessing
instruction to the set of instructions which might access the same
memory location.  This makes it possible to derive three useful sets
of instructions: $i2c$, the set of stores which might alias with any
load operation in the \gls{crashingthread}; $c2i$, the set of loads
which might alias with any store in the \gls{crashingthread}; and
$\beta = c2i \cup i2c$.  Informally, $i2c$ is the set of instructions
which, when added to the \gls{interferingthread}, might send data to
the \gls{crashingthread}, $c2i$ those which might receive data from
it, and $\beta$ those which communicate in either direction.

These three sets can then be used to restrict the set of interfering
traces which need to be considered.  Most obviously, it is safe to
discard all instructions in the \gls{interferingthread} after the last
member of $i2c$, as these cannot possibly influence the behaviour of
the \gls{crashingthread} and hence cannot influence whether the bug of
interest will reproduce.  Less obviously, it is also safe to discard
any instructions prior to the first member of $\beta$.  Suppose, for
instance, that the \gls{interferingthread} consists of the instruction
sequence $aBc$, where $a$ is some sequence of instructions not
belonging to $\beta$, $B$ is a member of $\beta$, and $c$ is another
sequence of instructions.  Any \glslink{sav}{SAVs} which could be
derived by racing the \gls{crashingthread} {\StateMachine} against the
interfering instruction sequence $aBc$ could also be derived by racing
it against the sequence $Bc$.  $a$ cannot directly influence the
\gls{crashingthread}, by definition, and so its only possible effect
is on the behaviour of the \gls{interferingthread}, by restricting the
possible values of program memory and registers when it starts.
{\Technique} will generate a \gls{verificationcondition} if \emph{any}
possible initial values could lead to an atomicity violation, and the
set of possible values after running $a$ is clearly a subset of the
set of all possible values, and so discarding $a$ can never lead to
any possible \glspl{sav} being omitted.  It is therefore always safe
to do so.

While safe, it is not always desirable to discard $a$ if the
restricted set of initial values would have been easier to analyse.
For instance, if the bug to be investigated is a bad pointer
dereference, knowing that the value stored into a shared structure had
previously been dereferenced by the interfering thread, and hence must
definitely be a valid pointer, is often useful.  The approach taken by
{\implementation} is to first generate a set of minimal \glspl{cfg}
(i.e. one in which every root is a member of $\beta$) and to then
extend the \glspl{cfg} backwards to include a small amount of
additional context, provided that doing so does not increase the
number of paths through the \gls{cfg} or cause the length of any path
to exceed \gls{alpha}.  In other words, a \gls{cfg} rooted at
instruction A will be extended to include its predecessor instruction
B provided that B is A's only predecessor and the longest path
starting at A is at most $\alpha - 1$ instructions long.  Because the
number of paths through the \gls{cfg} does not increase, this leads to
only a modest increase in the eventual cost of symbolically executing
the resulting {\StateMachine}, and this extra context can often
provide useful hints to the analysis.

\subsection[Building the \glsentrytext{interferingthread} \glsentrytext{cfg}s]{Building the \gls{interferingthread} \glspl{cfg}}

The procedure for building the \gls{interferingthread}'s
{\StateMachines} is similar to that for building the
\gls{crashingthread}'s: determine some starting point instructions,
build a \gls{static cfg} of the ``nearby'' instructions, unroll that
\gls{static cfg} to a dynamic one, and decompile the \gls{dynamic cfg}
into {\AStateMachine}.  The nature of the starting point is, however,
quite different (for a crashing {\StateMachine}, the single crashing
instruction; for an interfering one, the $i2c$ and $c2i$ sets), and
this leads to quite different algorithms and quite different results.
In particular, the \gls{interferingthread} algorithm generally works
forwards through the \gls{cfg} where the \gls{crashingthread} one
works backwards, and the \gls{interferingthread} algorithms can
produce multiple disjoint {\StateMachines} where the
\gls{crashingthread} one only ever produces a single {\StateMachine}.
I now describe these differences in more detail.

The initial \gls{static cfg} is again built by trimming the program's
complete \gls{static cfg} to contain only the needed static
instructions.  In this case, that means any instruction which can both
be reached from a member of $\beta$ and reach a member of $i2c$ within
\gls{alpha} instructions.  Note that this can sometimes generate
multiple disjoint \glspl{cfg}, rather than the single \gls{static cfg}
generated by the \gls{crashingthread} algorithm.  Each \gls{cfg} is
treated independently, generating independent \glspl{dynamic cfg} and
independent interfering {\StateMachines}, each of which will be
considered in turn by the rest of the analysis.

\begin{sanefig}
\begin{algorithmic}
  \Function{staticToDynamicInterfering}{}
    \State {Compute initial node positions}
    \While {\Gls{cfg} contains a cycle}
       \State $(\mathit{start}, \mathit{end}) \gets \textsc{findCycleEdgeInterfering}()$
       \State $\mathit{newPos} \gets \textsc{combinePositions}(\text{current positions of } \mathit{start} \text{ and } \mathit{end})$
       \State {Remove edge $(\mathit{start}, \mathit{end})$}
       \If {$\min_c(\mathit{newPos}.\mathit{min\_from}_{\mathit{end}}(c)) + \min_i(\mathit{newPos}.\mathit{min\_to}_{\mathit{end}}(i)) < \alpha$}
           \State $\mathit{newNode} \gets \text{duplicate } \mathit{end}$
           \State {Create a new edge $(\mathit{start}, \mathit{newNode})$}
           \For {Edges $(\_, \mathit{end}')$ leaving $\mathit{end}$}
              \State {Create a new edge $(\mathit{newNode}, \mathit{end}')$}
           \EndFor
           \State {Set position of $\mathit{newNode}$ to $\mathit{newPos}$}
       \EndIf
       \State {Recalculate $\mathit{min\_from}$ for $\mathit{end}$ and its successors, if necessary}
    \EndWhile
  \EndFunction
\end{algorithmic}
\caption{Loop unrolling algorithm for interfering thread CFGs.
  \textsc{findCycleEdgeInterfering} and \textsc{combinePositions} are described
  in the text.}
\label{fig:derive:store_cfg_unroll_alg}
\end{sanefig}

The static interfering \glspl{cfg} are converted to dynamic ones using
the \textsc{staticToDynamicInterfering} algorithm given in
\autoref{fig:derive:store_cfg_unroll_alg}.  This is more complicated
than the \textsc{staticToDynamicCrashing} algorithm given earlier.  In
particular, determining whether a loop has been sufficiently unrolled
is more difficult: where the \gls{crashingthread} algorithm preserves
\gls{alpha}-instruction paths which end at the (single) \gls{crashing
  instruction}, and so need only track an instruction's distance from
that \gls{crashing instruction}, the \gls{interferingthread} one must
preserve those which go between any member of $\beta$ and any member
of $i2c$, requiring a more complicated notion of node position.  In
this algorithm, the position of a CFG node $l$ is given by two maps,
$\mathit{min\_from}$ and $\mathit{min\_to}$:
\begin{itemize}
\item
  $\mathit{min\_from}_l(c)$, where $c \in \beta$, is the number of
  instructions on the shortest path from node $c$ to node $l$.
\item
  $\mathit{min\_to}_l(i)$, where $i \in i2c^\sharp$\kern-.34em, is the
  number of instructions on the shortest path from node $l$ to node $i$.
  $i2c^\sharp$ consists of all members of $i2c$, plus all of the
  \gls{cfg} nodes created by duplicating one of those instructions.
\end{itemize}
The length of the shortest path from a node $c \in \beta$ to $i \in
i2c^\sharp$ via node $l$ is then $\mathit{min\_from}_l(c) +
\mathit{min\_to}_l(i) + 1$, and so it is safe to discard any
node $l$ where
\begin{displaymath}
\min_{c \in \beta}\left(\mathit{min\_from}_l(c)\right) + \min_{i \in i2c^\sharp}\left(\mathit{min\_to}_l(i)\right) {\geq} \alpha
\end{displaymath}
The asymmetry, taking the distance from only ``true'' members of
$\beta$ but to any duplicate of a member of $i2c$, is perhaps
surprising.  The key observation is that every path which starts at a
duplicated member of $\beta$ will have a matching path which starts at
the original member, and so the ones which start at the duplicate
instruction are redundant.\kern-.5pt\fnote{The symmetrical statement is also
  true: every path which ends in a member of $i2c^\sharp$ has a
  matching path which ends at a member of $i2c$.  It would therefore
  also be correct to discard paths which end at a duplicate member of
  $i2c$.  It would not, however, be correct to combine the two
  observations and discard all paths which either start with a
  duplicate of $\beta$ or end with a duplicate of $i2c$, as there
  would then be little point in having those duplicates.}

The remaining differences between \textsc{staticToDynamicInterfering}
and \textsc{staticToDynamicCrashing} are simpler to explain:
\begin{itemize}
\item When \textsc{staticToDynamicCrashing} duplicates a node it
  duplicates all of that node's incoming edges, whereas
  \textsc{staticToDynamicInterfering} duplicates the node's outgoing
  edges.  This reflects the fact that interfering \glspl{cfg} are
  built up forwards from known instructions while crashing ones are
  built up backwards.
\item Whereas \textsc{findCycleEdgeCrashing} found a cycle-closing
  edge to break by searching backwards from the crashing instruction,
  \textsc{findCycleEdgeInterfering} finds one by performing a
  breadth-first search forwards from one of the roots of the
  \gls{cfg}.  If the graph reachable from that root is acyclic then it
  moves on to the next root.
\item \textsc{staticToDynamicInterfering} must maintain its node
  positioning information, which was not necessary for
  \textsc{staticToDynamicCrashing}.  This work is done primarily by
  the \textsc{combinePositions} function which is used to compute the
  label for the new node which would be produced by duplicating
  $\mathit{end}$.  This node will have the same outgoing edges as
  $\mathit{end}$, and so the same $min\_to$ label, and a single
  incoming edge from $\mathit{edge}.\mathit{start}$, and hence a
  $\mathit{min\_from}$ label which is just
  $\mathit{edge}.\mathit{start}$'s $\mathit{min\_from}$ with one added
  to every value.
\end{itemize}

\newcommand{\shortrightarrow}{\begin{tikzpicture}[baseline= -1ex*.75]
    \draw[->] (0,0) -- ++(.25,0);
  \end{tikzpicture}
  \hspace{-1pt}
}

\noindent
As an example, consider this cyclic \gls{cfg}:
\vfill
\centerline{
\inlinebox{
\parbox{11.59cm}{
\begin{tikzpicture}
  \node (A) at (0,2) [CommCfgInstr] {$A_0$};
  \node (B) [CfgInstr, below=of A] {$B_0$} edge [in=30,out=-30,loop] ();
  \node (C) [InterferingCfgInstr, below=of B] {$C_0$};
  \node (dummy) [left = .3 of C] {};
  \node (dummy2) [right = .5 of B] {};
  \node [left = .45 of dummy] {};
  \draw[->] (A) -- (B);
  \draw[->] (B) -- (C);
  \draw[->] (C.west) .. controls +(-.3,0) and +(0,-.3) .. ++(-.5,.5) -- ++(0,2.28) .. controls +(0,.3) and +(-.3,0) .. ++(.45,.5) -- (A.west);
  \begin{pgfonlayer}{bg}
    \node(box1) [fill=black!10,fit=(A) (B) (C) (dummy) (dummy2)] {};
  \end{pgfonlayer}
  \draw node [right=1.65 of box1] {
    \begin{tabular}{lcccc}
             & \multicolumn{1}{c}{$\mathit{min\_to}$} & \multicolumn{2}{c}{$\mathit{min\_from}$} & overall min\\
             & $C_0$ & $A_0$ & $C_0$ \\
      $A_0$    & 2   & 0   & 1 & 2\\
      $B_0$    & 1   & 1   & 2 & 2\\
      $C_0$    & 0   & 2   & 0 & 0\\
    \end{tabular}
  };
\end{tikzpicture}
}
}
}
\vfill
\noindent $C_0$, in green, is a member of $i2c$ (and therefore also
$\beta$); $A_0$, in blue, is a member of $\beta$ but not $i2c$.  The
overall min column is the minimum $\mathit{min\_to}$ value plus the
minimum $\mathit{min\_from}$ one; it gives the number of edges on the
shortest path involving a given node which starts at a member of
$\beta$ and ends at a member of $i2c^\sharp$\kern-.34em.  Suppose that we wish to
render this graph acyclic whilst preserving all paths of length 5 or
less.  \textsc{findCycleEdgeCrashing} will find $A_0 \shortrightarrow B_0
\shortrightarrow B_0$, of length three, as the first cyclic path, and so
the algorithm will break the $B_0 \shortrightarrow B_0$ edge by
duplicating $B_0$, producing this new graph:
\vfill
\noindent \centerline{ \inlinebox{
    \begin{tikzpicture}
      \node (A) at (0,2) [CommCfgInstr] {$A_0$};
      \node (B) [CfgInstr, below=of A] {$B_0$};
      \node (B1) [NewCfgInstr, right=of B] {$B_1$};
      \node (C) [InterferingCfgInstr, below=of B] {$C_0$};
      \node (dummy) [left = .6 of B] {};
      \draw[->] (A) -- (B);
      \draw[->] (B) -- (C);
      \draw[->,newCfgEdge] (B) to [bend left=10] (B1);
      \draw[->,newCfgEdge] (B1) to [bend left=10] (B);
      \draw[->,newCfgEdge] (B1) -- (C);
      \draw[->] (C.west) -- ++(-.4,0) .. controls +(-.3,0) and +(0,-.3) .. ++(-.5,.5) -- ++(0,2.28) .. controls +(0,.3) and +(-.3,0) .. ++(.45,.5) -- (A.west);
      \draw[->,killEdge] (node cs:name=B,angle=120) .. controls +(-.3,.6) and +(0,.6).. ++(-.9,-.3) ..controls +(0,-.6) and +(-.3,-.6) .. ++(.95,-.25);
      \begin{pgfonlayer}{bg}
        \node(box1) [fill=black!10,fit=(A) (B) (B1) (C) (dummy)] {};
      \end{pgfonlayer}
      \draw node [right=of box1] {
        \begin{tabular}{lcccc}
          & \multicolumn{1}{l}{$\mathit{min\_to}$} & \multicolumn{2}{l}{$\mathit{min\_from}$} & overall min\\
                & $C_0$ & $A_0$ & $C_0$ \\
          $A_0$ & 2     & 0     & 1    & 2\\
          $B_0$ & 1     & 1     & 2    & 2\\
          $B_1$ & 1     & 2     & 3    & 3\\
          $C_0$ & 0     & 2     & 0    & 0\\
        \end{tabular}
      };
    \end{tikzpicture}
  }
}\\
\noindent
New nodes and edges are shown in red and edges which have been removed
are shown crossed through.  There are now no more length three cyclic
paths in the graph, and so \textsc{findCycleEdgeCrashing} will report
the two length four paths $A_0 \shortrightarrow B_0
\shortrightarrow \hspace{1pt} C_0 \shortrightarrow A_0$ and $A_0
\shortrightarrow B_0 \shortrightarrow B_1 \hspace{-.5pt}
\shortrightarrow B_0$ and .  This will cause the $C_0 \shortrightarrow
A_0$ and $B_1 \hspace{-.5pt} \shortrightarrow B_0$ edges to be broken,
as shown:\vfill
\noindent
\centerline{ \inlinebox{
\begin{tikzpicture}
  \node (A) at (0,2) [CommCfgInstr] {$A_0$};
  \node (B) [CfgInstr, below=of A] {$B_0$};
  \node (B1) [CfgInstr, right=of B] {$B_1$};
  \node (C) [InterferingCfgInstr, below=of B] {$C_0$};
  \node (A1) [NewCfgInstr,right=of C] {$A_1$};
  \node (dummy) [left = .66 of B] {};
  \draw[->] (A) -- (B);
  \draw[->,newCfgEdge] (A1) -- (B);
  \draw[->] (B) -- (C);
  \draw[->] (B) to [bend left=10] (B1);
  \draw[->] (B1) -- (C);
  \draw[->] (B1) to [bend left=10] (B);
  \draw[->,newCfgEdge] (C) -- (A1);
  \draw[->,killEdge] (C.west) -- ++(-.4,0) .. controls +(-.3,0) and +(0,-.3) .. ++(-.5,.5) -- ++(0,2.28) .. controls +(0,.3) and +(-.3,0) .. ++(.45,.5) -- (A.west);
  \begin{pgfonlayer}{bg}
    \node(box1) [fill=black!10,fit=(A) (B) (B1) (C) (dummy)] {};
  \end{pgfonlayer}
  \draw node [right=of box1] {
    \begin{tabular}{lcccc}
      labels & \multicolumn{1}{l}{$\mathit{min\_to}$} & \multicolumn{2}{l}{$\mathit{min\_from}$} & overall min\\
            & $C_0$ & $A_0$ & $C_0$\\
      $A_0$   & 2   & 0   & $\infty$ & 2\\
      $A_1$ & 2   & 3   & 1        & 3\\
      $B_0$   & 1   & 1   & 2        & 2\\
      $B_1$ & 1   & 2   & 3        & 3\\
      $C_0$   & 0   & 2   & 0        & 0\\
    \end{tabular}
    ~
  };
\end{tikzpicture}
}
}\vfill
\centerline{
\inlinebox{
\begin{tikzpicture}
  \node (A) at (0,2) [CommCfgInstr] {$A_0$};
  \node (B) [CfgInstr, below=of A] {$B_0$};
  \node (B1) [CfgInstr, right=of B] {$B_1$};
  \node (B2) [NewCfgInstr, right=of B1] {$B_2$};
  \node (C) [InterferingCfgInstr, below=of B] {$C_0$};
  \node (A1) [DupeCommCfgInstr,right=of C] {$A_1$};
  \draw[->] (A) -- (B);
  \draw[->] (A1) -- (B);
  \draw[->] (B) -- (C);
  \draw[->] (B) to [bend left=10] (B1);
  \draw[->,killEdge] (B1) to [bend left=10] (B);
  \draw[->,newCfgEdge] (B1) to [bend left=10] (B2);
  \draw[->] (B1) -- (C);
  \draw[->,newCfgEdge] (B2) to [bend left=10] (B1);
  \draw[->,newCfgEdge] (B2) -- (C);
  \draw[->] (C) -- (A1);
  \begin{pgfonlayer}{bg}
    \node(box1) [fill=black!10,fit=(A) (A1) (B) (B1) (B2) (C)] {};
  \end{pgfonlayer}
  \draw node [right=of box1] {
    \begin{tabular}{lcccc}
            & \multicolumn{1}{l}{$\mathit{min\_to}$} & \multicolumn{2}{l}{$\mathit{min\_from}$} & overall min\\
            & $C_0$ & $A_0$ & $C_0$\\
      $A_0$   & 2   & 0   & $\infty$ & 2\\
      $A_1$ & 2   & 3 & 1 & 3\\
      $B_0$   & 1   & 1 & 2 & 2\\
      $B_1$ & 1   & 2 & 3 & 3\\
      $B_2$ & 1   & 3 & 4 & 4\\
      $C_0$   & 0   & 2 & 0 & 0\\
    \end{tabular}
  };
\end{tikzpicture}
}
}\vfill
\noindent
\textsc{findCycleEdgeCrashing} will now discover two length five cyclic
paths, $A_0 \shortrightarrow B_0 \shortrightarrow \hspace{1pt} C_0 \shortrightarrow A_1
\shortrightarrow B_0$ and $A_0 \shortrightarrow B_0 \shortrightarrow B_1
\hspace{-.5pt} \shortrightarrow B_2 \shortrightarrow B_1$.  The first
is eliminated by once more duplicating $B_0$: \vfill
\noindent
\centerline{
\inlinebox{
\begin{tikzpicture}
  \node (A) at (0,2) [CommCfgInstr] {$A_0$};
  \node (B) [CfgInstr, below=of A] {$B_0$};
  \node (B1) [CfgInstr, right=of B] {$B_1$};
  \node (B2) [CfgInstr, right=of B1] {$B_2$};
  \node (A1) [DupeCommCfgInstr,right=of C] {$A_1$};
  \node (C) [InterferingCfgInstr, below=of B] {$C_0$};
  \node (B3) [NewCfgInstr, below=of A1] {$B_3$};
  \draw[->] (A) -- (B);
  \draw[->,killEdge] (A1) -- (B);
  \draw[->,newCfgEdge] (A1) -- (B3);
  \draw[->] (B) -- (C);
  \draw[->] (B) -- (B1);
  \draw[->] (B1) to [bend left=10] (B2);
  \draw[->] (B1) -- (C);
  \draw[->] (B2) to [bend left=10] (B1);
  \draw[->] (B2) -- (C);
  \draw[->,newCfgEdge] (B3) -- (C);
  \draw[->,newCfgEdge] (B3) to [bend right=45] (B1);
  \draw[->] (C) -- (A1);
  \begin{pgfonlayer}{bg}
    \node(box1) [fill=black!10,fit=(A) (A1) (B) (B1) (B2) (B3) (C)] {};
  \end{pgfonlayer}
  \draw node [right=of box1] {
    \begin{tabular}{lcccc}
            & \multicolumn{1}{l}{$\mathit{min\_to}$} & \multicolumn{2}{l}{$\mathit{min\_from}$} & overall min\\
            & $C_0$ & $A_0$ & $C_0$\\
      $A_0$   & 2 & 0 & $\infty$ & 2\\
      $A_1$ & 2 & 3 & 1        & 3\\
      $B_0$   & 1 & 1 & $\infty$ & 2\\
      $B_1$ & 1 & 2 & 3        & 3\\
      $B_2$ & 1 & 3 & 4        & 4\\
      $B_3$ & 1 & 4 & 2        & 3\\
      $C_0$   & 0 & 2 & 0        & 0\\
    \end{tabular}
  };
\end{tikzpicture}
}
}\vfill
\noindent
The second, by contrast, can be eliminated without needing to
duplicate any further instructions.  Were $B_1$ to be duplicated, the
shortest path from $A_0$ to $C_0$ which used the new instruction would be
of length 6, exceeding the desired maximum path length, and so the
cycle-closing edge can simply be deleted:
\\
\noindent
\centerline{
\inlinebox{
\begin{tikzpicture}
  \node (A) at (0,2) [CommCfgInstr] {$A_0$};
  \node (B) [CfgInstr, below=of A] {$B_0$};
  \node (B1) [CfgInstr, right=of B] {$B_1$};
  \node (B2) [CfgInstr, right=of B1] {$B_2$};
  \node (A1) [DupeCommCfgInstr,right=of C] {$A_1$};
  \node (C) [InterferingCfgInstr, below=of B] {$C_0$};
  \node (B3) [CfgInstr, below=of A1] {$B_3$};
  \draw[->] (A) -- (B);
  \draw[->] (A1) -- (B3);
  \draw[->] (B) -- (C);
  \draw[->] (B) -- (B1);
  \draw[->] (B1) to [bend left=10] (B2);
  \draw[->] (B1) -- (C);
  \draw[->] (B2) to [bend left=10] (B1);
  \draw[->,killEdge] (B2) to [bend left=10] (B1);
  \draw[->] (B2) -- (C);
  \draw[->] (B3) -- (C);
  \draw[->] (B3) to [bend right=45] (B1);
  \draw[->] (C) -- (A1);
  \begin{pgfonlayer}{bg}
    \node(box1) [fill=black!10,fit=(A) (A1) (B) (B1) (B2) (B3) (C)] {};
  \end{pgfonlayer}
  \draw node [right=of box1] {
    \begin{tabular}{p{1.8cm}>{\centering}p{1.25cm}ccc}
         & $\mathit{min\_to}$ & \multicolumn{2}{l}{$\mathit{min\_from}$} & overall min\\
         & $C_0$ & $A_0$ & $C_0$\\
      $A_0$   & 2 & 0 & $\infty$ & 2\\
      $A_1$ & 2 & 3 & 1        & 3\\
      $B_0$   & 1 & 1 & $\infty$ & 2\\
      $B_1$ & 1 & 2 & 3        & 3\\
      $B_2$ & 1 & 3 & 4        & 4\\
      $B_3$ & 1 & 4 & 2 & 3\\
      $C_0$  & 0 & 2 & 0        & 0\\
      New label & 1 & 4 & 5 & 5\\
    \end{tabular}
  };
\end{tikzpicture}
}
}\\
\noindent
This process iterates, removing one cycle-completing edge at a time,
until the graph becomes completely acyclic:
\\
\noindent
\centerline{
\inlinebox{
\begin{tikzpicture}
  \node (A) at (0,2) [CommCfgInstr] {$A_0$};
  \node (B) [CfgInstr, below=of A] {$B_0$};
  \node (B1) [CfgInstr, right=of B] {$B_1$};
  \node (B2) [CfgInstr, right=of B1] {$B_2$};
  \node (A1) [DupeCommCfgInstr,right=of C] {$A_1$};
  \node (C) [InterferingCfgInstr, below=of B] {$C_0$};
  \node (B3) [CfgInstr, below=of A1] {$B_3$};
  \node (C1) [DupeInterferingCfgInstr, below=of B3] {$C_1$};
  \node (B4) [CfgInstr, right=of B3] {$B_4$};
  \draw[->] (A) -- (B);
  \draw[->] (A1) -- (B3);
  \draw[->] (B) -- (C);
  \draw[->] (B) -- (B1);
  \draw[->] (B1) -- (B2);
  \draw[->] (B1) -- (C);
  \draw[->] (B2) -- (C);
  \draw[->] (B3) -- (B4);
  \draw[->] (C) -- (A1);
  \draw[->] (B3) -- (C1);
  \draw[->] (B4) -- (C1);
  \begin{pgfonlayer}{bg}
    \node(box1) [fill=black!10,fit=(A) (A1) (B) (B1) (B2) (B3) (C) (C1)] {};
  \end{pgfonlayer}
  \draw node [right=of box1] {
    \begin{tabular}{p{1.8cm}ccccc}
            & \multicolumn{2}{c}{$\mathit{min\_to}$} & \multicolumn{2}{l}{$\mathit{min\_from}$} & overall min\\
            & $C_0$ & $C_1$ & $A_0$ & $C_0$ \\
      $A_0$   & 2 & 5 & 0 & $\infty$ & 2\\
      $A_1$ & $\infty$ & 2 & 3 & 1 & 3\\
      $B_0$   & 1 & 4 & 1 & $\infty$ & 2\\
      $B_1$ & 1 & 4 & 2 & $\infty$ & 3\\
      $B_2$ & 1 & 4 & 3 & $\infty$ & 4\\
      $B_3$ & $\infty$ & 1 & 4 & 2 & 3\\
      $B_4$ & $\infty$ & 1 & 5 & 3 & 4\\
      $C_0$   & 0 & 3 & 2 & 0 & 0\\
      $C_1$ & $\infty$ & 0 & 5 & 3 & 3\\
    \end{tabular}
  };
\end{tikzpicture}
}
}
\\
\noindent
As desired, the graph has been rendered acyclic while preserving all
paths of length up to five instructions between $\beta$ and $i2c$.
The resulting \gls{dynamic cfg} can now be converted to
{\AStateMachine}.

\section{Generating a verification condition}
\label{sect:using:check_realness}

Previous sections have described how to generate pairs of
{\StateMachines} representing fragments of the program which might
interact in interesting ways when run concurrently.  The next step is
to determine, for each pair, whether running the two {\StateMachines}
in parallel might cause an \gls{sav}, and, if so, under what
circumstances.

The core of the approach is to use symbolic execution~\cite{King1976}
to convert the {\StateMachines} into two predicates over the
{\StateMachine} execution environment: the \gls{verificationcondition}
itself, which is true when interleaving the two {\StateMachines} might
lead to a crash, and the \gls{inferredassumption}, which is true when
executing them atomically in series will not.  The program potentially
has an \gls{sav} bug if these are ever simultaneously true, and so
{\technique} generates a run-time \gls{bugenforcer} whenever it cannot
show that their conjunction is unsatisfiable.  Note that discovering a
crashing interleaving of the two threads will \emph{not} cause an
enforcer to be generated if running them atomically also crashes; as
discussed in \autoref{sect:types_of_bugs}, {\technique} is only
concerned with atomicity violation bugs, and this provides a very
useful reduction in the number of false positives which must be
checked at run-time.

\subsection{Symbolically executing {\StateMachines}}
\label{sect:derive:symbolic_execute}

The symbolic execution engine used by {\implementation} is, for the
most part, quite conventional.  I give only a brief overview of the
most important features here:
\begin{itemize}
\item The symbolic execution engine considers only a single
  {\StateMachine} at a time, even when investigating the parallel
  behaviour of two threads.  Rather than investigating thread
  interleavings in the symbolic execution engine, as is done in, for
  instance, ESD~\cite{Zamfir2010}, {\technique} instead encodes the
  behaviour of multiple {\StateMachines} into a single cross-product
  {\StateMachines}, described in
  \autoref{sect:using:build_cross_product}.

\item The {\StateMachine} speicher is represented by a sequence of
  update operations in the style of McCarthy's theory of
  arrays~\cite{McCarthy1962}, rather than attempting to maintain
  separate models for particular objects or memory locations.
  \state{Store} operations simply add themselves to the update list;
  \state{Load} operations scan back through it to find a matching
  \state{Store}.

  This is not a particularly efficient approach.  {\Implementation}
  relies on two facts to mitigate its performance problems.  First,
  where the relationship between \state{Store}s and \state{Load}s is
  simple, the {\StateMachine} simplifiers will forward data between
  them before the symbolic execution starts, eliminating both from the
  {\StateMachine}.  Second, {\implementation} maintains a cache of
  previous aliasing queries, and so if, for instance, two paths
  through the {\StateMachine} both need to determine whether
  \state{Store} A might alias with \state{Load} B, the symbolic
  execution engine usually need only do so once.

\item Aliasing problems are resolved lazily.  This means that if the
  engine must execute a \state{Load} operation and cannot determine
  which \state{Store} operation to use it does not cause an immediate
  fork of its state, but instead causes the \state{Load} to return
  {\AStateMachine} expression which describes the aliasing query and
  selects an appropriate result.  {\Implementation} only determines
  which precise \state{Store} operation should provide the loaded data
  when that data is needed.  This is often faster then resolving these
  queries eagerly.\kern.7pt\fnote{Note that this is not quite the same
    as the lazy abstraction algorithm used in, for instance,
    BLAST~\cite{Henzinger2002}, which adjusts which aspects of program
    state are represented in the abstract state.  {\Implementation}
    uses a fixed abstraction; the laziness here lies in the \emph{way}
    the abstract state is represented, not in \emph{what} is
    represented.}

\item The engine does not use any kind of incremental abstraction
  technique such as CEGAR~\cite{Clarke2003}.  This is primarily
  because of the use of a flat memory representation: the speicher is
  a single object, so it makes little sense to talk about modelling
  one part of it accurately and another part inaccurately, and without
  that CEGAR provides little benefit.  The use of lazy aliasing
  resolution provides some of the benefit of CEGAR, as it allows some
  aliasing queries which do not affect program behaviour to be
  skipped.

\item Unlike most symbolic execution engines, the one used by
  {\implementation} does not attempt to detect when it revisits a
  previously-visited configuration.  This is safe because
  {\StateMachines} are acyclic: any path through {\AStateMachine} can
  visit a given state at most once, so there is no possibility of
  revisiting a configuration and entering an infinite loop.  It is
  also, surprisingly, reasonably performant, because it is extremely
  rare for multiple paths to visit the same configuration and so there
  is little scope for re-using configurations to reduce duplicated
  work.  This is largely because {\technique} simplifies the
  {\StateMachine} before attempting to symbolically execute it and
  these simplifications tend to remove most easily-exploited forms of
  redundancy.
\end{itemize}
The resulting engine is generally adequate but rarely impressive, and
it is likely that using a more powerful one could considerably improve
{\implementation}'s performance.  Nevertheless, even this relatively
simple system is sufficient to generate some results in a useful
selection of cases.

\subsection{Deriving the inferred assumption}
\label{sect:derive:inferred_assumption}

As previously mentioned, the \gls{inferredassumption} is the condition
under which the program would avoid the bug if the crashing and
interfering {\StateMachines} were run atomically.  It is formed by the
conjunction of two sub-conditions: \gls{ci-atomic}, the condition
under which running the crashing {\StateMachine} and then the
interfering one avoids the bug, and \gls{ic-atomic}, the condition
under which running them in the opposite order does.  These
sub-conditions can be easily generated by concatenating the two
{\StateMachines} in the appropriate order, symbolically executing
them, and taking the disjunction of the path conditions for all paths
which evaluate to \textsc{Survive}.  Note that {\StateMachine}
environments which cause one of the concatenations to evaluate to
\textsc{Unreached} are excluded by the \gls{inferredassumption},
reflecting the intended semantics of \textsc{Unreached} as a path
which is of no interest to the rest of the analysis.

\subsection{Building cross-product {\StateMachines}}
\label{sect:using:build_cross_product}

{\Technique}'s symbolic execution engine is only capable of exploring
the behaviour of one {\StateMachine} at a time, but the
\gls{verificationcondition} is defined in terms of the parallel
composition of two {\StateMachines}.  {\Technique} therefore builds a
new {\StateMachine}, the \emph{cross-product {\StateMachine}}, which
emulates this parallel composition.  The chief complication here is
that it is not possible to express an arbitrary cross-product
operation in the {\StateMachine} language,\kern.3pt\fnote{The number
  of steps which such a cross-product emulating {\StateMachine} could
  take would be statically bounded, and so the number of states of the
  input {\StateMachines} which it could consider would also be
  statically bounded, and so there is no single {\StateMachine} which
  can compute the cross-product of two other arbitrary
  {\StateMachines}.} and so it must be split between a first, host,
stage, which runs in the same (Turing-complete) model of computation
as {\technique} itself, and a second, embedded, stage, which runs in
the (less powerful but symbolically executable) model defined by the
{\StateMachine} language.

\begin{sanefig}
  \newlength{\extrapadA}
  \setlength{\extrapadA}{3mm}
  \newlength{\extrapadB}
  \setlength{\extrapadB}{3mm}
  \newcommand{\midcolumn}{~\hspace{\extrapadA}$\Rightarrow$\hspace{\extrapadB}~}
  \newcommand{\lastcolumn}[1]{\production{#1}}
  \newcommand{\minheight}{2cm}
  \newcommand{\minwidth}{2.8cm}
  {\hfill}
  \begin{tabular}{m{3.7cm}m{10.9cm}}
    Terminals: & {\STateMachine} states \\
    Non-terminals: & Pairs \graphNT{$A, B$}, where $A$ is a fragment of the crashing {\StateMachine} and $B$ a fragment of the interfering one. \\
    {\raggedright Initial non-terminal:} & \graphNT{$A_0, B_0$}, where $A_0$ is the entire crashing {\StateMachine} and $B_0$ the entire interfering one, after renaming apart any common {\StateMachine} variables. \\
    \raisebox{12pt}{Productions:} &
    \begin{tabular}{lcc @{~\hspace{2.02cm}~} r}
      \tikz[baseline=(current bounding box.center)]{
        \node [style=graphNT, minimum height = \minheight, minimum width=\minwidth] {
          $\begin{tikzpicture}[baseline=(current bounding box.center), minimum height = 0, minimum width = 0]
            \node at (0,0) (r) [stateIf] {\stIf{m}};
            \node at (-5mm,-10mm) (A) {$A_0$};
            \node at (5mm,-10mm) (B) {$A_1$};
            \draw[->,ifTrue] (r) -- (A);
            \draw[->,ifFalse] (r) -- (B);
          \end{tikzpicture}, B$
        };
      } & \midcolumn & \begin{tikzpicture}[baseline=(current bounding box.center)]
        \node at (0,0) (r) [stateIf] {\stIf{m}};
        \node at (-10mm, -10mm) (A) [style=graphNT] { $A_0, B$ };
        \node at (10mm, -10mm) (B) [style=graphNT] { $A_1, B$ };
        \draw[->,ifTrue] (r) -- (A);
        \draw[->,ifFalse] (r) -- (B);
      \end{tikzpicture} & \lastcolumn{1_a} \\

      \tikz[baseline=(current bounding box.center)]{
        \node [style=graphNT, minimum height = \minheight, minimum width=\minwidth] {
          $\hspace{0.2mm}A, \begin{tikzpicture}[baseline=(current bounding box.center), minimum height = 0, minimum width = 0]
            \node at (0,0) (r) [stateIf] {\stIf{m}};
            \node at (-5mm,-10mm) (A) {$B_0$};
            \node at (5mm,-10mm) (B) {$B_1$};
            \draw[->,ifTrue] (r) -- (A);
              \draw[->,ifFalse] (r) -- (B);
          \end{tikzpicture}$
        };
      } & \midcolumn & \begin{tikzpicture}[baseline=(current bounding box.center)]
        \node at (0,0) (r) [stateIf] {\stIf{m}};
        \node at (-10mm, -10mm) (A) [style=graphNT] { $A, B_0$ };
        \node at (10mm, -10mm) (B) [style=graphNT] { $A, B_1$ };
        \draw[->,ifTrue] (r) -- (A);
        \draw[->,ifFalse] (r) -- (B);
      \end{tikzpicture} & \lastcolumn{1_b} \\

      \tikz[baseline=(current bounding box.center)]{
        \node [style=graphNT, minimum height = \minheight, minimum width=\minwidth] {
          \hspace{2.68mm}
          \begin{tikzpicture}[baseline=(current bounding box.center), minimum height=0, minimum width = 0]
            \node at (0,0) (r) [stateSideEffect, minimum height=15pt, minimum width=1em] {$a$};
            \node at (0,-10mm) (A) {$A$};
            \draw[->] (r) -- (A);
          \end{tikzpicture}\hspace{2.78mm},\hspace{2.68mm}\begin{tikzpicture}[baseline=(current bounding box.center), minimum height=0, minimum width = 0]
            \node at (0,0) (r) [stateSideEffect, minimum height=15pt, minimum width=1em] {$b$};
            \node at (0,-10mm) (A) {$B$};
            \draw[->] (r) -- (A);
          \end{tikzpicture}
          \hspace{2.58mm}
        };
      } & \midcolumn & \begin{tikzpicture}[baseline=(current bounding box.center)]
        \node at (0,0) (r) [stateIf] {\stIf{\happensBefore{a}{b}}};;
        \node at (-10mm,-12mm) [stateSideEffect,minimum height=15pt, minimum width=1em] (A) { $a$ };
        \node at (10mm,-12mm) [stateSideEffect,minimum width=1em, minimum height=15pt] (B) { $b$ };
        \node at (-10mm,-28mm) (A2) [style=graphNT, minimum height=1cm, minimum width=0] {
          $A, \begin{tikzpicture}[minimum height=0]
            \node at (0,0) (rr) [stateSideEffect,minimum width=1em, minimum height=15pt] {$b$};
            \node at (0,-10mm) (rb) {$B$};
            \draw[->] (rr) -- (rb);
          \end{tikzpicture}$
        };
        \node at (10mm,-28mm) (B2) [style=graphNT, minimum height=1cm, minimum width = 0] {
          $\begin{tikzpicture}[minimum height=0]
            \node at (0,0) (rr) [stateSideEffect,minimum width=1em, minimum height=15pt] {$a$};
            \node at (0,-10mm) (rb) {$A$};
            \draw[->] (rr) -- (rb);
          \end{tikzpicture}, B$
        };
        \draw[->,ifTrue] (r) -- (A);
        \draw[->] (A) -- (A2);
        \draw[->,ifFalse] (r) -- (B);
        \draw[->] (B) -- (B2);
      \end{tikzpicture} & \lastcolumn{2} \\

      \tikz[baseline=(current bounding box.center)]{
        \node [style=graphNT, minimum height = \minheight, minimum width=\minwidth] {
          \hspace{3mm}%
          \tikz[baseline = (charrr.base), minimum height = 0, minimum width = 0]{\node [stateTerminal] (charrr) {\state{t}};}%
          \hspace{1.8mm},\hspace{3.1mm}$B$\hspace{2.58mm}
        };
      } & \midcolumn & \begin{tikzpicture}[baseline=(current bounding box.center)]
        \node at (0,-4mm) (r) [stateTerminal] {\state{t}};
      \end{tikzpicture} & \lastcolumn{3} \\

      \tikz[baseline=(current bounding box.center)]{
        \node [style=graphNT, minimum height=\minheight, minimum width=\minwidth] {
          \hspace{.8mm}
          \begin{tikzpicture}[baseline=(current bounding box.center), minimum height = 0, minimum width = 0]
            \node at (0,0) (r) [stateSideEffect,minimum height=15pt, minimum width=1em] {$a$};
            \node at (0,-10mm) (A) {$A$};
            \draw[->] (r) -- (A);
          \end{tikzpicture}\hspace{2.2mm},\hspace{2.1mm}\tikz[minimum height = 0, minimum width = 0, baseline = (charrr.base)] {\node [stateTerminal] (charrr) {\state{t}};}
        };
      } & \midcolumn & \begin{tikzpicture}[baseline=(current bounding box.center)]
        \node at (0,2mm) (r) [stateSideEffect,minimum height=15pt, minimum width = 1em]{$a$};
        \node at (0,-11mm) (A) [style=graphNT] {\raisebox{-6pt}{
            $A$,\raisebox{.35em}{\tikz{\node [stateTerminal] {\state{t}};}}
          }
        };
        \draw[->] (r) -- (A);
      \end{tikzpicture} & \lastcolumn{4} \\
    \end{tabular}
  \end{tabular}
  {\hfill}
  \caption{A basic {\StateMachine} cross-product algorithm, expressed
    as a node replacement graph generating grammar.  $m$ matches
    boolean expressions; $A_0$, $A_1$, and $A$ match fragments of the
    crashing {\StateMachine}; $B_0$, $B_1$, and $B$ match fragments of
    the interfering {\StateMachine}; $a$ and $b$ match individual
    states within the crashing and interfering {\StateMachines},
    respectively; \state{t} matches any terminal state.}
  \label{fig:derive:basic_cross_product}
\end{sanefig}

The algorithm used by {\technique} is rather complicated, and so
before discussing it in detail I first present a simplified version,
described in \autoref{fig:derive:basic_cross_product} as a node
replacement graph grammar (see \autoref{sect:intro:graph_grammar}).
The non-terminal type \graphNT{A, B} represents the cross-product of
the {\StateMachine} starting at state A with that starting at state B.
The productions show how to convert these non-terminals into a single
{\StateMachine} with equivalent behaviour.  For instance, if
{\StateMachine} A begins with an \state{If} state, the output
{\StateMachine} will be formed by an \state{If} state which tests the
same discriminant and advances A to the appropriate successor state
(production \production{1_a}).  Similarly, if both {\StateMachines}
are at simple side-effects, the output {\StateMachine} uses an
\state{If} and happens-before test $\happensBeforeEdge$ to select
which side-effect to do first, does it, and then advances the selected
input {\StateMachine} (production \production{2}).  These rules are
reasonably direct encodings of the usual parallel interleaving
semantics into the {\StateMachine} language.

Note the asymmetry between handling of terminal states in the crashing
{\StateMachine} and those in the interfering one.  The output
{\StateMachine} stops immediately if the crashing {\StateMachine}
terminates, producing the same result (production \production{3}), but
continues if the interfering {\StateMachine} one does (production
\production{4}).  This reflects the different roles of the two
           {\StateMachines}: the crashing one determines whether the
           bug would have reproduced had the program followed the
           modelled execution, whereas the interfering one simply
           provides something for the \gls{crashingthread} to race
           with.  The final result of the crashing {\StateMachine} is
           therefore of critical import while that of the interfering
           one is largely irrelevant.

\begin{sanefig}
  \begin{displaymath}
    \textsc{Configuration} = \graphNT{\begin{tabular}{rrll}
      \multirow{2}{*}{\bigg\{} & \textit{crashingState}: & \textsc{{\STateMachine} state}, & \multirow{2}{*}{\bigg\},}\\
                               & \textit{crashingHasIssued}: & \textsc{Bool}\\
      \multirow{2}{*}{\bigg\{} & \textit{interferingState}: & \textsc{{\STateMachine} state}, & \multirow{2}{*}{\bigg\},} \\
                               & \textit{interferingHasIssued}: & \textsc{Bool}\\
      \multicolumn{2}{r}{\textit{atomic}:} & \multicolumn{2}{l}{\{$\varnothing$, \textit{crashing}, \textit{interfering}\}}
    \end{tabular}}
  \end{displaymath}
  \caption{\textsc{Configuration} type for the cross-product algorithm.}
  \label{fig:cross_product:configuration}
\end{sanefig}
The actual algorithm used by {\technique} includes several refinements
over this basic one:
\begin{itemize}
\item The input {\StateMachines} may contain atomic blocks, delimited
  by {\stStartAtomic} and {\stEndAtomic} side-effects, and the simple
  grammar does not respect them.  This can be fixed by extending the
  grammar's non-terminal structure with a final field which indicates
  which, if any, of the {\StateMachines} are currently within atomic
  blocks.  This $\mathit{atomic}$ field is maintained as the
  {\StateMachines} perform {\stStartAtomic} and {\stEndAtomic} states
  and is used to restrict the interleavings generated by the grammar.
\item The basic grammar will consider every possible interleaving of
  the two {\StateMachines}, including the trivial ones in which one or
  other completes atomically.  This is correct but inefficient, as the
  atomic orderings have already been considered in the
  \gls{inferredassumption}.  This problem can be fixed by extending
  the grammar's non-terminal structure so that it tracks whether the
  {\StateMachines} have issued any \gls{speicher}-accessing
  side-effects and arranging that if either {\StateMachine} terminates
  without having done so the cross-product {\StateMachine} ends in the
  {\stUnreached} state.

\begin{sanefig}
  \begin{displaymath}
    \mathit{initial} = \graphNT{\begin{tabular}{rrll}
      \multirow{2}{*}{\bigg\{} & \textit{crashingState} = & $\mathit{crashing}_0$, & \multirow{2}{*}{\bigg\},} \\
                               & \textit{crashingHasIssued} = & \false\\
      \multirow{2}{*}{\bigg\{} & \textit{interferingState} = & $\mathit{interfering}_0$, & \multirow{2}{*}{\bigg\},} \\
                               & \textit{interferingHasIssued} = & \false\\
      \multicolumn{2}{r}{\textit{atomic} = } & $\varnothing$ \\
    \end{tabular}}
  \end{displaymath}
  \caption{Initial \textsc{Configuration} for the cross-product
    algorithm.  $\mathit{crashing}_0$ is the first state of the
    crashing {\StateMachine} and $\mathit{interfering}_0$ that of the
    interfering one.}
  \label{fig:cross_product:initial}
\end{sanefig}

  Note that this refinement excludes executions in which the
  {\StateMachines} run in series, rather than those in which they run
  linearizably~\cite{Herlihy1990}.  The latter would perhaps be more
  useful, as it could potentially eliminate more paths, but is far
  more difficult to calculate.  This is purely a performance
  enhancement, and so the simpler implementation is more appropriate.
\item The basic grammar fails to take account of partial order
  redundancies caused by {\StateMachine}-local side-effects such as
  \state{Copy} or \state{Assume}.  The complete grammar ameliorates
  this weakness by only generating happens-before tests for non-local
  side-effects, defined to be those which could possibly influence or
  be influenced by the other {\StateMachine}.  This definition is
  context-dependent: a \state{Load} in the crashing {\StateMachine},
  for instance, will be non-local if it could alias with any of the
  interfering {\StateMachine}'s possible future \state{Store}s, and so
  might switch from being non-local to being local when the
  interfering {\StateMachine} performs a \state{Store}.  A
  {\stStartAtomic} side-effect is considered non-local if any of the
  side-effects in the atomic block which it starts are non-local.
\item This grammar can sometimes duplicate side-effect states, which
  might break static single assignment form.  {\Technique} uses a
  separate post-pass to restore the SSA invariant after the graph
  grammar has finished.
\end{itemize}
The extended non-terminal type, \textsc{Configuration}, is shown in
\autoref{fig:cross_product:configuration}, its initial value in
\autoref{fig:cross_product:initial}, and the extended productions in
\autoref{fig:cross_product:algorithm}.  Many of the productions have
symmetrical versions which simply swap the roles of the crashing and
interfering {\StateMachines}; for the sake of brevity, I show only the
crashing {\StateMachine} production and mark it with a *.

\begin{sidewaysfigure}
  \begin{figgure}
    \begin{tabular}{ c @{~$\Rightarrow$~} c >{\raggedright}p{5.3cm} c c}
      \tikz [baseline = (current bounding box.center)] {
        \node[style = graphNT] {
          $\left\{\raisebox{.5ex}{\begin{tikzpicture}[baseline = (current bounding box.center),minimum height = 0, minimum width = 0,node distance=0.5cm,font=\small]
          \node at (0,0) (r) [stateIf] {\stIf{m}};
          \node at (-5mm, -9mm) (A) {$A_0$};
          \node at (5mm, -9mm) (B) {$A_1$};
          \draw[->,ifTrue] (r) -- (A);
          \draw[->,ifFalse] (r) -- (B);
          \end{tikzpicture}}, i_a\right\},\left\{B, i_b\right\},z$
        };
      } & \begin{tikzpicture}[node distance=0.5cm,font=\small, baseline = (current bounding box.center)]
          \node at (0,0) (r) [stateIf] {\stIf{m} };
          \node at (-20mm, -8mm) (A) [graphNT] { $\{A_0, i_a\}, \{B, i_b\}, z$ };
          \node at (20mm, -8mm) (B) [graphNT] { $\{A_1, i_a\}, \{B, i_b\}, z$ };
          \draw[->,ifTrue] (r) -- (A);
          \draw[->,ifFalse] (r) -- (B);
      \end{tikzpicture} & & \production{1'} & *\\

      \tikz [baseline = (current bounding box.center)] {
        \node[style = graphNT] {
          $\left\{\raisebox{.5ex}{\begin{tikzpicture}[font=\small, baseline = (current bounding box.center) ]
          \node at (0,0) (r) [stateSideEffect, minimum height=15pt, minimum width=1em] {$a$};
          \node at (0,-7.5mm) (A) {$A$};
          \draw[->] (r) -- (A);
          \end{tikzpicture}}, i_a\right\},\left\{\raisebox{.5ex}{\begin{tikzpicture}[font=\small, baseline = (current bounding box.center) ]
          \node at (0,0) (r) [stateSideEffect, minimum height=15pt, minimum width=1em] {$b$};
          \node at (0,-7.5mm) (A) {$B$};
          \draw[->] (r) -- (A);
          \end{tikzpicture}}, i_b\right\}, \varnothing$
        };
      }
      & \begin{tikzpicture}[font=\small, baseline = (current bounding box.center)]
          \node at (0,0) (r) [stateIf] {\stIf{\happensBefore{a}{b}}};
          \node at (-25mm, -8mm) [stateSideEffect, minimum height=15pt, minimum width=1em] (A) {$a$};
          \node at (25mm, -8mm) [stateSideEffect, minimum height=15pt, minimum width=1em] (C) {$b$};
          \node[style=graphNT] at (-25mm, -21mm) (B) {
            $\{A, \true\}, \left\{\raisebox{.5ex}{\begin{tikzpicture}[font=\small, baseline = (current bounding box.center)]
            \node at (0,0) (Br) [stateSideEffect, minimum height=15pt, minimum width=1em] {$b$};
            \node at (0,-7.5mm) (BA) {$B$};
            \draw[->] (Br) -- (BA);
            \end{tikzpicture}}, i_b\right\}, \varnothing$
          };
          \node[style=graphNT] at (25mm, -21mm) (D) {
            $\left\{\raisebox{.5ex}{\begin{tikzpicture}[font=\small, baseline = (current bounding box.center)]
            \node at (0,0) (Dr) [stateSideEffect, minimum height=15pt, minimum width=1em] {$a$};
            \node at (0,-7.5mm) (DA) {$A$};
            \draw[->] (Dr) -- (DA);
            \end{tikzpicture}}, i_a\right\}, \{B, \true\}, \varnothing$
          };
          \draw[->,ifTrue] (r) -- (A);
          \draw[->,ifFalse] (r) -- (C);
          \draw[->] (A) -- (B);
          \draw[->] (C) -- (D);
        \end{tikzpicture} & When $a$ and $b$ are non-local. & \production{2'} & \\

      \vspace{1mm}\tikz [baseline = (current bounding box.center)] {
        \node[style = graphNT] {
          $\left\{\begin{tikzpicture}[font=\small, baseline = (char.base)]
          \node [stateTerminal] (char) {\state{t}};
          \end{tikzpicture}, \true\right\}, \{A, \true\}, z$
        };
      }
      & \begin{tikzpicture}[font=\small, baseline = (current bounding box.center)]
          \node [stateTerminal] {\state{t}};
        \end{tikzpicture} &  & \production{3'} \\
        
      \vspace{1mm}\tikz [baseline = (current bounding box.center)] {
        \node[style = graphNT] {
          $\left\{\raisebox{.5ex}{\begin{tikzpicture}[font=\small][baseline = (current bounding box.center)]
          \node at (0,0) (r) [stateSideEffect, minimum height=15pt, minimum width=1em] {$a$};
          \node at (0,-7.5mm) (A) {$A$};
          \draw[->] (r) -- (A);
          \end{tikzpicture}}, i_a\right\},\left\{B, i_b\right\},z$
        };
      }
      & \begin{tikzpicture}[font=\small, baseline = (current bounding box.center)]
          \node at (0,0) (r) [stateSideEffect, minimum height=15pt, minimum width=1em] {$a$};
          \node at (0,-9mm) [graphNT] (A) {$\{A, i_a\}, \{B, i_b\}, z$};
          \draw[->] (r) -- (A);
          \end{tikzpicture} & If $a$ is a local side-effect. & \production{4'} & *\\

      \vspace{1mm}\raisebox{3pt}{\tikz [baseline = (current bounding box.center)] {
        \node[style=graphNT] {
          $\left\{\begin{tikzpicture}[font=\small, baseline = (char.base), minimum height=0]
          \node [stateTerminal] (char) {\state{t}};
          \end{tikzpicture}, i_a\right\}, \{A, i_b\}, z$
        };
      }} & \raisebox{-1pt}{\begin{tikzpicture}[font=\small, baseline = (A.base)]
          \node (A) [stateTerminal] {{\stUnreached}};
        \end{tikzpicture}} & If $i_a \wedge i_b = \false$. & \production{5'} & *\\
      
      \vspace{1mm}\tikz [baseline = (current bounding box.center)] {
        \node[style=graphNT] {
          $\left\{\raisebox{.5ex}{\begin{tikzpicture}[font=\small, baseline = (current bounding box.center)]
          \node at (0,0) (r) [stateSideEffect] {{\stStartAtomic}};
          \node at (0,-7.5mm) (A) {$A$};
          \draw[->] (r) -- (A);
          \end{tikzpicture}}, i_a\right\},\left\{B, i_b\right\}, \varnothing$
        };
      }
      & \graphNT{$\{A, i_a\}, \{B, i_b\}, \mathit{crashing}$} & If the atomic block is local. & \production{6'} & *\\

      \vspace{1mm}\tikz [baseline = (current bounding box.center)] {
        \node[style=graphNT] {
          $\left\{\raisebox{.5ex}{\begin{tikzpicture}[font=\small, baseline = (current bounding box.center)]
          \node at (0,0) (r) [stateSideEffect] {{\stEndAtomic}};
          \node at (0,-7.5mm) (A) {$A$};
          \draw[->] (r) -- (A);
          \end{tikzpicture}}, i_a\right\},\left\{B, i_b\right\}, \mathit{crashing}$
        };
      }
      & \graphNT{$\{A, i_a\}, \{B, i_b\}, \varnothing$} & & \production{7'} & *\\

      \vspace{1mm}\tikz [baseline = (current bounding box.center)] {
        \node[style=graphNT] {
          $\left\{\raisebox{.5ex}{\begin{tikzpicture}[font=\small, baseline = (current bounding box.center)]
          \node at (0,0) (r) [stateSideEffect, minimum height=15pt, minimum width=1em] {$a$};
          \node at (0,-7.5mm) (A) {$A$};
          \draw[->] (r) -- (A);
          \end{tikzpicture}}, i_a\right\},\left\{B, i_b\right\},\mathit{crashing}$
        };
      }
      & \begin{tikzpicture}[font=\small, baseline = (current bounding box.center)]
          \node at (0,0) (r) [stateSideEffect, minimum height=15pt, minimum width=1em] {$a$};
          \node at (0,-9mm) [graphNT] (A) {$(\{A, \true\}, \{B, i_b\}, \mathit{crashing})$};
          \draw[->] (r) -- (A);
        \end{tikzpicture} & If $a$ is a non-local side-effect. & \production{8'} & *
    \end{tabular}
    \caption{The cross-product algorithm as a node replacement graph
      grammar.  $A$, $A_0$, and $A_1$ match fragments of the crashing
      {\StateMachine} and $a$ matches a single state from the crashing
      {\StateMachine}.  $B$ and $b$ match fragments of and a single
      state in, respectively, the interfering {\StateMachine}.  $i_a$
      and $i_b$ match either {\true} or {\false}.  $z$ matches any of
      $\varnothing$, $\mathit{crashing}$, or $\mathit{interfering}$.
      $m$ matches a boolean expression.  \state{T} matches any terminal
      state. *: production also applies with the crashing and
      interfering {\StateMachines} swapped.}
    \label{fig:cross_product:algorithm}
  \end{figgure}
\end{sidewaysfigure}
\begin{sanefig}
  \tikzstyle{stateSideEffect}+=[minimum width=4cm, minimum height=1.0cm]
  \tikzstyle{stateIf}+=[minimum width=4cm, minimum height=1.0cm]
  \tikzstyle{stateTerminal}+=[minimum width=4cm, minimum height=1.0cm]
  \hspace{-3mm}
  \begin{subfloat}
    \begin{tikzpicture}
      \node[stateSideEffect] (lA) {$A$: \stLoad{0}{x} };
      \node[stateIf,below = of lA] (lB) {$B$: \stIf{\smVar{0} = 0} };
      \node[stateSideEffect,below = of lB] (lC) {$C$: \stLoad{1}{x} };
      \node[stateIf,below = of lC] (lD) {\!\!\!\!$D$: \stIf{\smBadPtr{\smVar{1}}}\!\!\!\!};
      \node[stateTerminal,below = of lD] (lH) {$H$: \stCrash };
      \node[stateTerminal,right = 0.3 of lC] (lG) {$G$: \stSurvive };
      \draw[->] (lA) -- (lB);
      \draw[->,ifTrue] (lB) -- (lG);
      \draw[->,ifFalse] (lB) -- (lC);
      \draw[->] (lC) -- (lD);
      \draw[->,ifTrue] (lD) -- (lH);
      \draw[->,ifFalse] (lD) -- (lG);
    \end{tikzpicture}
    \caption{Crashing thread {\StateMachine} }
  \end{subfloat}
  \begin{subfloat}
    \begin{tikzpicture}
      \node[stateIf] (lE) {$E$: \stIf{y \not= 0}};
      \path (node cs:name=lE) ++(2.3,-2.0) node [stateSideEffect] (lF) {$F$: \stStore{0}{x}};
      \node[stateTerminal,below = 3 of lE] (lI) {$I$: \stSurvive };
      \draw[->,ifTrue] (lE) -- (lI);
      \draw[->,ifFalse] (lE) -- (lF);
      \draw[->] (lF) -- (lI);
      \node at (0,-6.4) {~};
    \end{tikzpicture}
    \caption{Interfering thread {\StateMachine} }
  \end{subfloat}
  \caption{A pair of {\StateMachines}.  $x$ is a global memory
    location.  Figure~\ref{fig:cross_product_output} shows their
    cross-product.}
  \label{fig:cross_product_input}
\end{sanefig}

The productions of the extended grammar encode these refinements quite
directly:
\begin{itemize}
\item Production \production{1'} corresponds to productions
  \production{1_a} and \production{1_b} in the simple grammar,
  production \production{2'} to \production{2}, and \production{3'} to
  \production{3}.  These need no further explanation.
\item Production \production{4'} allows either of the {\StateMachines}
  to advance past a local side-effect without needing to perform a
  happens-before test, implementing a limited form of partial-order
  reduction.
\item Production \production{5'} causes paths in which one
  {\StateMachine} finishes before the other starts to end in the
  {\stUnreached} state, effectively eliminating those paths from
  consideration.
\item Productions \production{6'}, \production{7'}, and
  \production{8'} implement atomic blocks by allowing {\StateMachines}
  to, respectively, enter a local atomic block, leave a block, and
  progress within a block.  Non-local atomic blocks are entered by a
  variant of production \production{2'}, not shown in the figure,
  which races the {\stStartAtomic} side-effect against an appropriate
  non-local side-effect in the other {\StateMachine} using a
  $\happensBeforeEdge$ test and, depending on the result of that test,
  either starts the atomic block, in the same way as production
  \production{6'}, or runs the other side-effect.
\end{itemize}
\begin{sanefig}
  \newcommand{\boxLabel}[2]{\small \raisebox{-1.5pt}{\production{#1'}} #2}
  \newcommand{\labelBox}[6]{
    \fill [color=blue!20] (#1, #2) rectangle (#1 + #3, #2 + #4);
    \node at (#1, #2 + #4 + .1) [below right] {\boxLabel{#5}{#6}};
  }
  \newcommand{\nodeWidth}{4.3cm}
  \newcommand{\nodeHeight}{1.0cm}
  \begin{tikzpicture}[align=center, node distance = 1 and 0.9]
    \labelBox{-2.4}{-.6}{4.8}{1.65}{1}{\{$A$, {\false}\}, \{$E$, {\false}\}}
    \labelBox{2.8}{-.6}{4.8}{1.65}{5}{(\{$A$, {\false}\}, \{$I$, {\false}\})}

    \fill [color=blue!20] (-2.4, -0.9) -- (7.6, -0.9) -- (7.6, -2.65) -- (2.4, -2.65) -- (2.4, -4.65) -- (-2.4, -4.65);
    \node at (-2.4, -0.8) [below right] {\boxLabel{2}{(\{$A$, {\false}\}, \{$F$, {\false}\})} };

    \labelBox{2.8}{-4.65}{4.8}{1.65}{5}{(\{$A$, {\false}\}, \{$I$, {\true}\})}
    \labelBox{-2.4}{-6.65}{4.8}{1.65}{1}{(\{$B$, {\true}\}, \{$F$, {\false}\})}
    \labelBox{2.8}{-6.65}{4.8}{1.65}{5}{(\{$G$, {\true}\}, \{$F$, {\false}\})}

    \fill [color=blue!20] (-2.4, -6.9) -- (7.6, -6.9) -- (7.6, -8.7) -- (2.4, -8.7) -- (2.4, -10.7) -- (-2.4, -10.7);
    \node at (-2.4, -6.8) [below right] {\boxLabel{2}{(\{$C$, {\true}\}, \{$F$, {\false}\})}};

    \labelBox{8.0}{-8.7}{4.8}{1.8}{4}{(\{$C$, {\true}\}, \{$I$, {\true}\})}
    \labelBox{8.0}{-10.7}{4.8}{1.8}{1}{(\{$D$, {\true}\}, \{$I$, {\true}\})}
    \labelBox{8.0}{-12.7}{4.8}{1.8}{3}{(\{$H$, {\true}\}, \{$I$, {\true}\})}
    \labelBox{2.8}{-12.7}{4.8}{1.8}{3}{(\{$G$, {\true}\}, \{$I$, {\true}\})}
    \labelBox{-2.4}{-12.7}{4.8}{1.8}{1}{(\{$D$, {\true}\}, \{$F$, {\false}\})}

    \labelBox{-2.4}{-14.7}{4.8}{1.8}{5}{(\{$H$, {\true}\}, \{$F$, {\false}\})}
    \labelBox{2.8}{-14.7}{4.8}{1.8}{5}{(\{$G$, {\true}\}, \{$F$, {\false}\})}

    \node at (0,0) [stateIf, minimum width=\nodeWidth, minimum height=\nodeHeight] (A) {\stIf{y \not= 0} };
    \node[stateTerminal, right = of A, minimum width=\nodeWidth, minimum height=\nodeHeight] (B) {{\stUnreached} };

    \node[stateIf, below = of A, minimum width=\nodeWidth, minimum height=\nodeHeight] (C) {\stIf{\happensBefore{A}{F}} };
    \node[stateSideEffect, below = of C, minimum width=\nodeWidth, minimum height=\nodeHeight] (D) {\stLoad{0}{x} };
    \node[stateSideEffect, right = of C, minimum width=\nodeWidth, minimum height=\nodeHeight] (E) {\stStore{0}{x} };
    \node[stateIf, below = of D, minimum width=\nodeWidth, minimum height=\nodeHeight] (F) {\stIf{\smVar{0} = 0} };
    \node[stateTerminal, below = of E, minimum width=\nodeWidth, minimum height=\nodeHeight] (G) {\stUnreached };
    \node[stateTerminal, right = of F, minimum width=\nodeWidth, minimum height=\nodeHeight] (H) {\stUnreached };
    \node[stateIf, below = of F, minimum width=\nodeWidth, minimum height=\nodeHeight] (I) {\stIf{\happensBefore{C}{F}} };
    \node[stateSideEffect, below = of I, minimum width=\nodeWidth, minimum height=\nodeHeight] (J) {\stLoad{1}{x}};
    \node[stateSideEffect, right = of I, minimum width=\nodeWidth, minimum height=\nodeHeight] (K) {\stStore{0}{x}};
    \node[stateIf, below = of J, minimum width=\nodeWidth, minimum height=\nodeHeight] (L) {\!\!\!\stIf{\smBadPtr{\smVar{1}}}\!\!\!};
    \node[stateSideEffect, right = of K, minimum width=\nodeWidth, minimum height=\nodeHeight] (M) {\stLoad{2}{x}};
    \node[stateTerminal, below = of L, minimum width=\nodeWidth, minimum height=\nodeHeight] (N) {\stUnreached };
    \node[stateTerminal, right = of N, minimum width=\nodeWidth, minimum height=\nodeHeight] (O) {\stUnreached };
    \node[stateIf, below = of M, minimum width=\nodeWidth, minimum height=\nodeHeight] (P) {\!\!\!\stIf{\smBadPtr{\smVar{2}}}\!\!\!};
    \node[stateTerminal, minimum width=\nodeWidth, minimum height=\nodeHeight, below = of P] (R) {\stCrash };
    \node[stateTerminal, minimum width=\nodeWidth, minimum height=\nodeHeight] at (R-|K) (Q) {\stSurvive };
    \draw[->,ifTrue] (A) -- (B);
    \draw[->,ifFalse] (A) -- (C);
    \draw[->,ifTrue] (C) -- (D);
    \draw[->,ifFalse] (C) -- (E);
    \draw[->] (D) -- (F);
    \draw[->] (E) -- (G);
    \draw[->,ifTrue] (F) -- (H);
    \draw[->,ifFalse] (F) -- (I);
    \draw[->,ifTrue] (I) -- (J);
    \draw[->,ifFalse] (I) -- (K);
    \draw[->] (J) -- (L);
    \draw[->] (K) -- (M);
    \draw[->,ifTrue] (L) -- (N);
    \draw[->,ifFalse] (L) -- (O);
    \draw[->] (M) -- (P);
    \draw[->,ifFalse] (P) -- (Q);
    \draw[->,ifTrue] (P) -- (R);
  \end{tikzpicture}
  \caption{Cross product of the {\StateMachines} shown in
    \autoref{fig:cross_product_input}.  Blue boxes show the
    non-terminals and productions used to generate each part of the
    graph.  The $\mathit{atomic}$ field of the non-terminal is always
    $\varnothing$ for these input {\StateMachines} and is not shown.}
  \label{fig:cross_product_output}
\end{sanefig}
The cross-product of the {\StateMachines} in
\autoref{fig:cross_product_input} is given in
\autoref{fig:cross_product_output}, showing the
\textsc{Configuration}s and productions used to produce all of the
output states.  This captures every possible interleaving of the two
input {\StateMachines} into a single cross-product {\StateMachine},
allowing {\technique} to build the \gls{verificationcondition} using a
simple single-threaded symbolic execution engine.  Note that while
this example produced a tree-structured {\StateMachine}, in the
general case the result is a directed acyclic graph, with sub-graphs
shared if the same non-terminal is generated multiple times.  This
reduces the worst-case number of states in the cross-product
{\StateMachine} from $O(\binom{c+i}{c})$ to $O(ci)$, where $c$ is the
number of states in the crashing {\StateMachine} and $i$ the number in
the interfering one.\fnote{There is a loose but perhaps helpful
  analogy between {\technique}'s cross product {\StateMachines} and
  the state reachability graphs maintained by some symbolic execution
  engines.  Both act to structure the symbolic execution engine's
  search of the program's state space, and both represent regions
  where results can potentially be reused between paths as joins in
  the graph.  In this view, the most important feature of
  cross-product {\StateMachines} is that they are themselves programs
  expressed in the same language as the one which is to be
  symbolically executed, whereas state reachability graphs do not
  usually have a convenient interpretation as a runnable program.}

Importantly, the cross-product {\StateMachine} can itself be
simplified using the usual {\StateMachine} simplification passes.
Simplifying the example cross-product {\StateMachine} will produce the
{\StateMachine} shown in Figure~\ref{fig:cross_product_output_opt}.
In this case, the actual symbolic execution step will be trivial, and
will report that the program will reach a {\stCrash} state precisely
when $y = 0 \wedge \happensBefore{A}{F} \wedge \smLoad{x} \not= 0 \wedge
\happensBefore{F}{C}$; in other words, if $y$ is nonzero, the initial
value of $x$ is non-zero, and statement F intercedes between
statements A and C.  Comparison to the input {\StateMachines} will
show that this is precisely the desired result.

The simplifications needed in this example are quite simple and it
would have been possible to include equivalent optimisations in the
symbolic execution engine itself.  This would be more difficult for
more complex simplifications, for two reasons:
\begin{itemize}
\item Simplification passes can easily look ahead in the
  {\StateMachine}, whereas symbolic execution primarily considers a
  single state at a time.  Dead code elimination, for example, is much
  easier to implement as a simplification to the {\StateMachine} than
  as a change to the symbolic execution engine.
\item The results of a simplification pass are inherently shared
  across all paths which reach a particular state, whereas the
  symbolic execution engine needs to perform additional work in order
  to share results.
\end{itemize}
There is also an engineering consideration which argues in favour of
building and simplifying the cross-product {\StateMachine} rather than
integrating equivalent optimisations into the symbolic execution
engine: {\implementation} already needs all of the simplifiers in
order to build the input {\StateMachines} and so re-using them here
halves the implementation effort.

\begin{sanefig}
  \tikzstyle{stateSideEffect}+=[minimum width=4.3cm, minimum height=1.0cm]
  \tikzstyle{stateIf}+=[minimum width=4.3cm, minimum height=1.0cm]
  \tikzstyle{stateTerminal}+=[minimum width=4.3cm, minimum height=1.0cm]
  \centerline{
  \begin{tikzpicture}
    \node[stateSideEffect] (A) {\stAssert{y = 0 \wedge \happensBefore{A}{F} \wedge \smLoad{x} \not= 0 \wedge \happensBefore{F}{C}} };
    \node[stateTerminal, below = of A] (B) {\stCrash };
    \draw[->] (A) -- (B);
  \end{tikzpicture}
  }
  \caption{Result of simplifying {\StateMachine} shown in
    \autoref{fig:cross_product_output}.}
  \label{fig:cross_product_output_opt}
\end{sanefig}

\subsection{Path explosion}

Path explosion is a common problem in symbolic execution systems.  The
number of paths through a program rises exponentially in the size of
the program, and this can prevent na\"ive symbolic execution systems
from being applied to realistically large programs.  In the case of
\technique, there are two main causes of path explosion:
\begin{itemize}
\item
  \textit{Aliasing}.  If the various simplification passes and the
  \gls{programmodel} cannot determine how memory accessing
  instructions alias then the symbolic execution engine must
  exhaustively consider every possible way in which every \state{Load}
  can be satisfied.  If there are $n$ \state{Load}s and $m$
  \state{Store}s then the number of such patterns to be considered is
  $O((m+1)^n)$, which grows very quickly in the number of accesses.
  The use of lazy alias resolution helps mitigate this to some extent
  but does not eliminate it completely.  This represents one of the
  major limitations to \technique's scalability.
\item
  \textit{Thread interleaving}.  The cross-product {\StateMachine}
  will have $O(ci)$ states, where $c$ is number of states in the
  crashing {\StateMachine} and $i$ the number in the interfering one.
  The number of paths through the combined {\StateMachine} is then
  $O(2^{ci})$, which again grows rather quickly.
\end{itemize}
The result is that the symbolic execution engine might have to
consider up to $O(2^{ci}.m^n)$ distinct paths when evaluating the
cross-product {\StateMachine}.  This is impractical for even
moderately complex inputs.  For good performance, {\technique} relies
on the various simplification and analysis techniques to eliminate
most of these paths without needing to fully symbolically execute
them.  Fortunately, as discussed in the evaluation, they are able to
do so in a useful set of cases.

\section{The \glsentrytext{w-isolation} property}
\label{sect:derive:w_isolation}

In practice, {\technique}'s most important limitation is the need to
solve a large number of aliasing problems.  This limitation can be
somewhat ameliorated by assuming that the crashing {\StateMachine}
never stores to any memory locations which are subsequently loaded by
the interfering one; in other words, by restricting the class of bugs
considered from those where two threads are simultaneously working
with a structure to those where one thread is reading from the
structure whilst another updates it.  Such bugs are said to be
I-isolated, or to have the \gls{w-isolation} property, as the
\gls{interferingthread} is effectively isolated from the crashing one.

Restricting the class of bugs considered in this way enables three
main optimisations:
\begin{itemize}
\item
  It directly restricts the aliasing problem, as the analysis no
  longer needs to consider aliasing between stores in the crashing
  {\StateMachine} and loads in the interfering one.
\item
  It reduces the set of interfering \glspl{cfg} which is generated for
  each \gls{crashingthread}, because if the \gls{interferingthread}
  cannot load any location stored to by the \gls{crashingthread} then
  $c2i$ is empty and $\beta = i2c$ (see
  \autoref{sect:derive:write_side}).
\item
  It simplifies the calculation of the \gls{inferredassumption}.
  Normally, the \gls{inferredassumption} is the conjunction of
  \gls{ic-atomic} and \gls{ci-atomic}, but when the \gls{w-isolation}
  property holds the latter can be replaced by \gls{c-atomic}, the
  condition for the crashing {\StateMachine} to survive when run in
  isolation.  This is because the \gls{w-isolation} property implies
  that \gls{ci-atomic} = \gls{c-atomic} $\wedge$ I-atomic, where
  I-atomic is the condition for the interfering {\StateMachine} to
  survive in isolation, and I-atomic is implied by \gls{ic-atomic}.
  \gls{c-atomic} is independent of the interfering {\StateMachine},
  and so only needs to be calculated once for every crashing
  {\StateMachine} rather than once for every pair of crashing and
  interfering {\StateMachines}, which can sometimes provide a modest
  performance improvement.
\end{itemize}
The analysis is almost always faster with the \gls{w-isolation}
assumption, but cannot handle as broad a class of program behaviour.
I evaluate these effects experimentally in
\autoref{sect:eval:w_isolation}.

\section{The \glsentrytext{programmodel}}
\label{sect:program_model}

In addition to the {\StateMachine}-level aliasing analysis,
{\technique} also makes use of an \gls{programmodel}, built before the
main analysis starts, which describes how the program accesses memory
during normal operation.  This is used both for alias analysis during
{\StateMachine} simplification (\autoref{sect:derive:simplify_sm}) and
symbolic execution (\autoref{sect:using:check_realness}), and also to
find the $\beta$ and $i2c$ sets when building the interfering
\gls{cfg} (\autoref{sect:derive:write_side}).  The \gls{programmodel}
is composed of separate models of stack and non-stack memory; I
describe each in turn.

\subsection{The stack model}
The stack model is built using a fairly conventional static
function-local pointer escape analysis~\cite[pages
  140--141]{Appel2004} based on the observation that stack frames are
``created'' when a function starts.  This implies that there should
not be any pointers to function-local variables unless the function
being analysed creates one, and so the static analysis attempts to
track which registers and memory locations might contain pointers to
the local stack frame.  This is usually sufficient for the
{\StateMachine} simplifiers to be able to determine which, if any,
stack frames a given memory access might refer to; simple arithmetic
considerations can then usually determine the specific local variable.
As part of this analysis, {\technique} discovers any registers which
have constant value at a particular instruction, or those which are
equal to another register plus a constant, and this information is
also used during {\StateMachine} simplification.

\subsection{The non-stack model}
It is much more difficult to characterise the structure of non-stack
memory, such as the heap, using static analysis.  Not only is the heap
structure itself more complicated, in terms of the number of objects
which point at other objects, but the information is harder to locate,
as it is not localised to any particular function or program module.
These problems make it difficult to accurately model the heap
statically even when the analysis tool has full access to the
program's source code; attempting to do so given only a binary is
completely infeasible.

{\Technique} instead relies on a dynamic analysis to model accesses to
non-stack locations.  In the case of {\implementation}, this is
implemented as a Valgrind skin~\cite{Nethercote2007}.  This analysis
works by dividing memory into fixed-size chunks\footnote{For
  {\implementation}, these chunks are eight bytes, matching the
  platform word size. This represents a reasonable trade-off between
  precision, which argues for smaller chunks, and analysis
  performance, which argues for larger ones.} and then tracking which
instructions access each chunk.  If two instructions are ever observed
to access the same chunk without an intervening call to a function
such as \texttt{free} then they are marked as potentially aliasing;
otherwise, they are marked as non-aliasing.  In practice, this
analysis will, if provided with a sufficient cross-section of the
program's potential behaviour, approximate a source-level static one
which considers two accesses to potentially alias if and only if they
access the same field of the same compound structure type.  Most
programs are structured so that most structure fields are accessed
from a relatively small number of program instructions, and so this is
generally reasonably precise despite being completely flow- and
context-insensitive.

The chief weakness of this approach (aside from the fact that, as a
dynamic analysis, it is inherently incomplete) is that it must be able
to tell when the type of structure stored at a particular memory
location changes.  For most programs, that means that it must be able
to identify dynamic memory management functions, and in particular
functions such as \texttt{free} which release memory.  Failing to
correctly identify these functions will effectively merge all of the
structures managed by a particular allocator, potentially leading to a
very imprecise analysis.  {\Technique} relies on the user to manually
identify these functions.  This is not an unreasonable burden: most
programs rely on allocation functions in libraries, which only have to
be identified once for all users of the library, and the remainder
generally use only a very small number of custom allocators.  MySQL,
for instance, had only two functions which needed to be annotated, and
Thunderbird and pbzip2 had none.

\subsubsection{Escape analysis}
{\Implementation} includes one minor refinement to the basic dynamic
analysis described above.  It is fairly common for programs to
allocate new heap structures using a function such as \texttt{malloc}
and to then initialise this structure using a series of stores.  These
stores will never race, so it would be helpful to avoid spending
excessive time considering what would happen if they did.
{\Technique} avoids doing so by marking blocks of memory returned from
\texttt{malloc} as thread-local, and they remain so until a pointer to
them is stored in non-stack memory.  Entries in the aliasing table
include a flag indicating whether the access is thread-private or
potentially racing and this is used by later phases of the analysis to
constrain the aliasing problem.

This policy might seem to be overly conservative, in that a block of
memory is marked as shared whenever a pointer to it is stored into any
non-stack memory, even when that non-stack memory is itself marked as
thread-private.  This is necessary because the analysis does not
attempt to track the heap reachability graph, and in particular cannot
map from a block to the set of blocks reachable from it.  There is
therefore no safe way to ``upgrade'' a block from thread-private to
thread-shared if there is any possibility of that block containing a
thread-private pointer; upgrading blocks early and pessimistically
means that it is never necessary to do so.

\section{Implementation limitations}
{\Implementation} makes several important simplifying assumptions,
some of which can impact the correctness of the analysis.  I now
briefly discuss these assumptions and their effects, and possible ways
of removing them.

\subsection{Incomplete library models}
As has already been discussed, {\implementation} relies on
hand-written {\StateMachine} implementations of system library
functions.  If no such implementation is available then
{\implementation} uses a stub implementation which simply sets the
return address register to a new free variable.  While this approach
does have some advantages, most notably in being able to incorporate
useful higher-level information on library interfaces without needing
to analyse complex library implementations and in being able to
perform analysis without access to a library implementation, it
carries obvious risks of incorrectly characterising bugs.  This is
particularly regrettable since the {\technique} technique itself is
just as applicable to library code as it is to code in the main
program.  This weakness could be eliminated by arranging to load the
library implementation, when available, into the {\implementation}
analysis tool along with the program itself, and then extending the
\gls{static cfg} building algorithms to trace across the
program/library boundary in the same way that they already trace
across function boundaries within the program.

\subsection{Simplified memory model}
\label{sect:derive:simpl_mem_model}

In addition to the library limitation, {\implementation} simplifies
the processor's memory model in three ways.  First, the aliasing
resolution algorithm does not support unaligned memory accesses, and
will produce incorrect results if the program makes use of them in the
\gls{analysiswindow}.  Such accesses are unusual in most programs, as
they have very poor performance on most modern
processors,\kern-.1pt\fnote{Indeed, many processors do not support
  unaligned accesses at all and require them to be emulated in the
  operating system, and performance in that case is generally dreadful
  rather than merely poor.} and removing this limitation would
complicate alias analysis even for programs which never make unaligned
accesses.  Alias analysis is already one of the main factors limiting
{\implementation}'s scalability and so this would be a poor trade-off.

Second, {\implementation} assumes that every address in the program's
address space is either completely inaccessible or completely
accessible, and that this status does not change during the lifetime
of the program.  {\Implementation} will therefore not be able to model
bugs which depend on read-only memory or those where the
\gls{analysiswindow} includes calls to functions such as
\texttt{mremap}.  Lifting the first restriction would be relatively
straightforward by extending the $\smBadPtr{}$ expression to include
an indication of the type of access contemplated, and, while this
would lead to a modest increase in analysis complexity, I would not
expect the performance cost to be extravagant.  Lifting the second
restriction would be more challenging.  At present, $\smBadPtr{}$ is a
function only of the address tested and the {\StateMachine}
environment.  Crucially, a $\smBadPtr{}$ in one thread cannot be
affected by any behaviour in the other, and so the {\StateMachine}
simplifiers can make useful inferences about them before deriving the
interfering {\StateMachine} (and these inferences can often be used to
reduce the number of interfering {\StateMachines} which must be
derived).  Having to consider potential races with \texttt{mremap} and
\texttt{munmap} at every $\smBadPtr{}$ would make this far more
difficult, and would probably lead to a noticeable reduction in
analysis performance, even when analysing bugs which do not depend on
such races.

Third, {\implementation} assumes that the program does not make use of
any run-time generated code or run-time code modifications.  This
would be a difficult limitation to remove.  The \glspl{dynamic cfg}
and {\StateMachines} are generated from the program binary's
\gls{static cfg}, and this will not accurately reflect the program's
actual instructions if those instructions can change while the program
is running.  Solving this problem is impossible in the general
case,\kern-.1pt\fnote{Consider a program P in a model of computation C
  which does not permit run-time code generation.  Construct a new P'
  in a model of computation C' which is identical to C except for
  allowing run-time code generation such that P' runs P and then uses
  some run-time generated code.  P' then uses run-time generated code
  if and only if P terminates.  An algorithm to check for run-time
  code generation in C' would therefore solve the halting problem in
  C, and, since there are Turing-powerful models of computation which
  do not support run-time code generation, that is impossible.} but it
might be possible to handle some simple cases by having the dynamic
analysis identify and log such code.  The later analyses could then
generate the \glspl{cfg} and {\StateMachines} from these logs.  Such
an approach would have few costs in programs which do not generate
code at run-time and so could sensibly be incorporated into
{\implementation}, although the rarity of run-time code generation
suggests that this would be a dubious use of engineering effort.  Even
with this extension {\implementation} would not be able to support
true self-modifying code, as the {\StateMachines} cannot be modified
while they are being symbolically executed; removing \emph{that}
limitation is probably impractical while retaining the {\StateMachine}
abstraction.  In any case, true self-modifying code is sufficiently
rare in real programs that not supporting it is highly unlikely to be
a problem in practice.

\subsection{Simplified concurrency model}
The final simplification made by {\implementation} is in its
concurrency model.  The only form of concurrency supported by
{\implementation} is simple multi-thread concurrency within a single
program.  It does not, for instance, consider races with signal
handlers, or inter-process races mediated via shared memory regions or
\texttt{ptrace}-like APIs, or races with hardware DMA.  While these
mechanisms are much less common than thread-style concurrency, they do
still occur in some real programs, and so it would potentially be
useful to have some support for them.  Signal handlers, and other
up-call style concurrency, could be added relatively easily by
treating each signal handling function as a special kind of
\gls{interferingthread}, although the dynamic analysis might need some
modest extensions to determine when to do so.  Supporting
inter-process concurrency would require more fundamental changes to
{\implementation}, as it would need to be able to model multiple
address spaces at the same time, but would require few fundamental
changes to the {\technique} technique itself.

\section{Discussion}

This chapter has shown how to derive {\StateMachines} representing
potential \gls{sav} bugs and how to convert these into
\glspl{verificationcondition} specifying the circumstances under which
those bugs might reproduce.  I also discussed some of the limitations
of the techniques presented here and some possible ways of alleviating
these weaknesses.  The most important limitation of this approach,
though, was not discussed: the majority of bugs detected are false
positives which can never occur in real runs of the program.  These
come about because the {\StateMachine} analysis presented here
considers only instructions within the, quite small,
\gls{analysiswindow}, and makes highly conservative assumptions about
the program's behaviour outside of that window.  The next chapter will
show how to convert these \glspl{verificationcondition} into
\glspl{bugenforcer} which make the potential bugs reproduce far more
easily, allowing the true bugs to be confirmed and the false ones
discarded with minimal manual intervention.
