The algorithm used to perform this factorisation is reasonably simple.
The core idea is to use BDD re-ordering to find the individual
factors.  For instance, the side-condition in the example might be
represented by either of the BDDs shown in
Figure~\ref{fig:place_conditions_example:bdd1}.  The representations
are equivalent, and will always evaluate to the same result, but the
order can affect how much of the BDD can be evaluated when only some
variables are available.  In particular, the BDD on the left cannot be
evaluated at all at A, because the first thing it tests involves
RBX:2, which is unavailable, whereas in the BDD on the right the first
test in the sequence can be evaluated, as it only tests RAX:1, which
is available.  The BDD can then be factorised into the two components
shown in Figure~\ref{fig:place_conditions_example:bdd2}.  The first
components shows the part of the side-condition which can be evaluated
at A and the second shows the part which cannot be.  The first
component is attached to instruction A, and will be checked there at
run time, and the algorithm then tries to enforce the second component
at A's successor B.  This will repeat until either the side-condition
has been completely placed.

\begin{figure}
  \hfill%
  \begin{subfloat}
    \begin{tikzpicture}
      \node (x) [BddNode] {$\mathrm{RBX:2} = 72$};
      \node (y) [BddNode, below = of x] {$\mathrm{RAX:1} = 93$};
      \node (z) [BddNode, below = of y] {$\mathrm{RCX:1} = \mathrm{RCX:2}$};
      \node (true) [BddLeaf, below = of z] {true};
      \node (false) [BddLeaf, right = 2cm of true] {false};
      
      \draw [BddTrue] (x) -- (y);
      \draw [BddFalse] (x.east) to [bend left=45] (false);
      \draw [BddTrue] (y) -- (z);
      \draw [BddFalse] (y.east) to [bend left=45] (false);
      \draw [BddTrue] (z) -- (true);
      \draw [BddFalse] (z.east) to [bend left=45] (false);
    \end{tikzpicture}
  \end{subfloat}%
  \hfill%
  \begin{subfloat}
    \begin{tikzpicture}
      \node (x) [BddNode] {$\mathrm{RAX:1} = 72$};
      \node (y) [BddNode, below = of x] {$\mathrm{RBX:2} = 93$};
      \node (z) [BddNode, below = of y] {$\mathrm{RCX:1} = \mathrm{RCX:2}$};
      \node (true) [BddLeaf, below = of z] {true};
      \node (false) [BddLeaf, right = 2cm of true] {false};
      
      \draw [BddTrue] (x) -- (y);
      \draw [BddFalse] (x.east) to [bend left=45] (false);
      \draw [BddTrue] (y) -- (z);
      \draw [BddFalse] (y.east) to [bend left=45] (false);
      \draw [BddTrue] (z) -- (true);
      \draw [BddFalse] (z.east) to [bend left=45] (false);
    \end{tikzpicture}
  \end{subfloat}
  \hfill%
  \caption{BDD representations of $\mathrm{RAX:1} = 93 \wedge
    \mathrm{RBX:2} = 72 \wedge \mathrm{RCX:1} = \mathrm{RCX:2}$.}
  \label{fig:place_conditions_example:bdd1}
\end{figure}

\begin{figure}
  \hfill%
  \begin{tikzpicture}
    \node (x) [BddNode] {$\mathrm{RAX:1} = 72$};
    \node (true) [BddLeaf, below = of x] {true};
    \node (false) [BddLeaf, below right = of x] {false};
    \draw [BddTrue] (x) -- (true);
    \draw [BddFalse] (x) -- (false);
  \end{tikzpicture}{\hfill}\raisebox{17mm}{$\bigwedge$}{\hfill}%
  \begin{tikzpicture}
    \node (y) [BddNode] {$\mathrm{RBX:2} = 93$};
    \node (z) [BddNode, below = of y] {$\mathrm{RCX:1} = \mathrm{RCX:2}$};
    \node (true) [BddLeaf, below = of z] {true};
    \node (false) [BddLeaf, right = of true] {false};
      
    \draw [BddTrue] (y) -- (z);
    \draw [BddFalse] (y.east) to [bend left=45] (false);
    \draw [BddTrue] (z) -- (true);
    \draw [BddFalse] (z.east) to [bend left=45] (false);
  \end{tikzpicture}
  {\hfill}
  \caption{Factorisation of the BDD on the right of
    Figure~\ref{fig:place_conditions_example:bdd1}.}
  \label{fig:place_conditions_example:bdd2}
\end{figure}


\begin{figure}
  {\hfill}
  \begin{subfloat}
    \begin{tikzpicture}
      \node (a) [BddNode, blue] {a};
      \node (d0) [BddNode, below left = of a] {d};
      \node (d1) [BddNode, below right = of a] {d};
      \node (b0) [BddNode, below = of d0, blue] {b};
      \node (b1) [BddNode, below = of d1, blue] {b};
      \node (c0) [BddNode, below = of b0, blue] {c};
      \node (c1) [BddNode, below = of b1, blue] {c};
      \node (true) [BddLeaf, below = of c0] {true};
      \node (false) [BddLeaf, below = of c1] {false};
      \draw [BddTrue] (a) -- (d0);
      \draw [BddTrue] (d0) -- (b0);
      \draw [BddTrue] (d1) to [bend left=45] (false);
      \draw [BddTrue] (b0) to [bend right=45] (true);
      \draw [BddTrue] (b1) to [bend left=45] (false);
      \draw [BddTrue] (c0) -- (true);
      \draw [BddTrue] (c1) -- (false);
      \draw [BddFalse] (a) -- (d1);
      \draw [BddFalse] (d0) -- (b1);
      \draw [BddFalse] (d1) -- (c0);
      \draw [BddFalse] (b0) -- (c1);
      \draw [BddFalse] (b1) -- (c0);
      \draw [BddFalse] (c0) -- (false);
      \draw [BddFalse] (c1) -- (true);
    \end{tikzpicture}
    \caption{Before reordering}
  \end{subfloat}{\hfill}
  \begin{subfloat}
    \begin{tikzpicture}
      \node (a) [BddNode, blue] {a};
      \node (b) [BddNode, below left = of a, blue] {b};
      \node (dummy) [below right = of a] {};
      \node (c0) [BddNode, below = of b, blue] {c};
      \node (c1) [BddNode, below = of dummy, blue] {c};
      \node (d0) [BddNode, below = of c0] {d};
      \node (d1) [BddNode, below = of c1] {d};
      \node (true) [BddLeaf, below = of d0] {true};
      \node (false) [BddLeaf, below = of d1] {false};

      \draw [BddTrue] (a) -- (b);
      \draw [BddTrue] (b) to [bend right=45] (d0);
      \draw [BddTrue] (c0) -- (d1);
      \draw [BddTrue] (c1) -- (d1);
      \draw [BddTrue] (d0) -- (true);
      \draw [BddTrue] (d1) -- (false);

      \draw [BddFalse] (a) -- (c1);
      \draw [BddFalse] (b) -- (c0);
      \draw [BddFalse] (c0) -- (d0);
      \draw [BddFalse] (c1) to [bend left=45] (false);
      \draw [BddFalse] (d0) -- (false);
      \draw [BddFalse] (d1) -- (true);
    \end{tikzpicture}
    \caption{After reordering}
  \end{subfloat}
  {\hfill}
  \caption{A more complex example of BDD reordering.}
  \label{fig:place_conditions_example2:bdd1}
\end{figure}

This example is perhaps too simple to make the factorisation algorithm
clear.  Consider, then, the example in
Figure~\ref{fig:place_conditions_example2:bdd1}.  a, b, c, and d are
the input expressions.  Assume that at the current node in the
control-flow graph, a, b, and c are available but that d is not; this
is indicated by colouring available variables blue in the figure.  It
is now possible to construct the evaluatable sub-condition by
replacing all of the unevaluatable variables with a true leaf and the
applying the usual BDD reduction rules; the results are shown in
Figure~\ref{fig:place_conditions_example2:evalable}.  This condition
can then be evaluated at the current node.  Similarly, the
unevaluatable component can be generated by removing all of the edges
from an evaluatable variable to false and reducing; this is
illustrated in Figure~\ref{fig:place_conditions_example2:unevalable}.
This condition will need to be evaluated later in the control-flow
graph.

\todo{Do I need a proof of correctness for this?  Proving that it's
  correct in the sense of being a proper conjunction is easy, but
  showing that it's optimal, in any sense, is going to be a pain.}

\begin{figure}
  {\hfill}
  \begin{subfloat}
    \begin{tikzpicture}
      \node (a) [BddNode, blue] {a};
      \node (b) [BddNode, below left = of a, blue] {b};
      \node (dummy) [below right = of a] {};
      \node (c0) [BddNode, below = of b, blue] {c};
      \node (c1) [BddNode, below = of dummy, blue] {c};
      \node (true) [BddLeaf, below = of d0] {true};
      \node (false) [BddLeaf, below = of d1] {false};

      \draw [BddTrue] (a) -- (b);
      \draw [BddTrue] (b) to [bend right=45] (d0);
      \draw [BddTrue] (c0) -- (true);
      \draw [BddTrue] (c1) -- (true);
      \draw [BddTrue] (d0) -- (true);
      \draw [BddTrue] (d1) -- (false);

      \draw [BddFalse] (a) -- (c1);
      \draw [BddFalse] (b) -- (c0);
      \draw [BddFalse] (c0) -- (true);
      \draw [BddFalse] (c1) to [bend left=45] (false);
    \end{tikzpicture}
    \caption{Before reduction}
  \end{subfloat}
  {\hfill}
  \begin{subfloat}
    \begin{tikzpicture}
      \node (a) [BddNode, blue] {a};
      \node (dummy) [below = of a] {};
      \node (c) [BddNode, blue, below right = of a] {c};
      \node (true) [BddNode, below = of dummy] {true};
      \node (false) [BddNode, below = of c] {false};
      \draw [BddTrue] (a) -- (true);
      \draw [BddTrue] (c) -- (true);
      \draw [BddFalse] (c) -- (false);
      \draw [BddFalse] (a) -- (c);
    \end{tikzpicture}
    \caption{After reduction}
  \end{subfloat}
  {\hfill}
  \caption{The evaluatable portion of the BDD in
    Figure~\ref{fig:place_conditions_example2:bdd1}.}
  \label{fig:place_conditions_example2:evalable}
\end{figure}

\begin{figure}
  {\hfill}
  \begin{subfloat}
    \begin{tikzpicture}
      \node (a) [BddNode, blue] {a};
      \node (b) [BddNode, below left = of a, blue] {b};
      \node (dummy) [below right = of a] {};
      \node (c0) [BddNode, below = of b, blue] {c};
      \node (c1) [BddNode, below = of dummy, blue] {c};
      \node (d0) [BddNode, below = of c0] {d};
      \node (d1) [BddNode, below = of c1] {d};
      \node (true) [BddLeaf, below = of d0] {true};
      \node (false) [BddLeaf, below = of d1] {false};

      \draw [BddTrue] (a) -- (b);
      \draw [BddTrue] (b) to [bend right=45] (d0);
      \draw [BddTrue] (c0) -- (d1);
      \draw [BddTrue] (c1) -- (d1);
      \draw [BddTrue] (d0) -- (true);
      \draw [BddTrue] (d1) -- (false);

      \draw [BddFalse] (a) -- (c1);
      \draw [BddFalse] (b) -- (c0);
      \draw [BddFalse] (c0) -- (d0);
      \draw [BddFalse] (c1) to [bend left=45] (d1);
      \draw [BddFalse] (d0) -- (false);
      \draw [BddFalse] (d1) -- (true);
    \end{tikzpicture}
    \caption{Before reduction}
  \end{subfloat}
  {\hfill}
  \begin{subfloat}
    \begin{tikzpicture}
      \node (a) [BddNode, blue] {a};
      \node (b) [BddNode, below left = of a, blue] {b};
      \node (dummy) [below right = of a] {};
      \node (c0) [BddNode, below = of b, blue] {c};
      \node (d0) [BddNode, below = of c0] {d};
      \node (d1) [BddNode, below = of c1] {d};
      \node (true) [BddLeaf, below = of d0] {true};
      \node (false) [BddLeaf, below = of d1] {false};

      \draw [BddTrue] (a) -- (b);
      \draw [BddTrue] (b) to [bend right=45] (d0);
      \draw [BddTrue] (c0) -- (d1);
      \draw [BddTrue] (d0) -- (true);
      \draw [BddTrue] (d1) -- (false);

      \draw [BddFalse] (a) -- (d1);
      \draw [BddFalse] (b) -- (c0);
      \draw [BddFalse] (c0) -- (d0);
      \draw [BddFalse] (d0) -- (false);
      \draw [BddFalse] (d1) -- (true);
    \end{tikzpicture}
    \caption{After reduction}
  \end{subfloat}
  {\hfill}
  \caption{Unevaluatable component of the BDD in
    Figure~\ref{fig:place_conditions_example2:bdd1}.}
  \label{fig:place_conditions_example2:unevalable}
\end{figure}
