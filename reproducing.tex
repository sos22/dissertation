\chapter{Reproducing bugs}
\label{sect:reproducing_bugs}

Previous chapters have shown how to derive
\glspl{verificationcondition} showing when the local structure of a
program is compatible with a particular type of bug.  I now show how
to convert these conditions into \glspl{bugenforcer}: run-time
monitors for the program which attempt to shepherd it towards certain
schedules and hence make the bug more likely to reproduce.  In this
way {\technique} is able to take account of more global properties of
the program and the environment in which it is running, and hence to
eliminate the many false positives generated by the basic analysis.

\begin{figure}
  \begin{tabular}{ll}
    \multicolumn{2}{l}{\texttt{int *global\_ptr[];}}\\
    \hspace{-5mm}\subfigure[][Crashing thread source]{
      \texttt{
        \begin{tabular}{lll}
          \multicolumn{3}{l}{void crashing(int idx1) \{}\\
          &\multicolumn{2}{l}{if (global\_ptr[idx1])}\\
          &&*global\_ptr[idx1] = 7;\\
          \}\\
        \end{tabular}
      }
    } & \hspace{-5mm}%
    \subfigure[][Interfering thread source]{
      \texttt{
        \begin{tabular}{ll}
          \multicolumn{2}{l}{void interfering(int idx2) \{}\\
          &global\_ptr[idx2] = NULL;\\
          \}\\
          \\
        \end{tabular}
      }
    }\\
    \hspace{-5mm}\subfigure[][Crashing thread machine code]{
      \texttt{
        \begin{tabular}{rlll}
          & \multicolumn{3}{l}{crashing:} \\
          l1: & ADD & global\_ptr+idx1 & \!\!\!$\rightarrow$ reg1 \\
          l2: & LOAD & *reg1 & \!\!\!$\rightarrow$ reg2 \\
          l3: & CMP & 0, reg2 \\
          l4: & JE & l7 \\
          l5: & LOAD & *reg1 & \!\!\!$\rightarrow$ reg3 \\
          l6: & STORE & 7 & \!\!\!$\rightarrow$ *reg3 \\
          l7:
        \end{tabular}
      }
    } & \hspace{-5mm}%
    \subfigure[][Interfering thread machine code]{
      \texttt{
        \begin{tabular}{rlll}
          & \multicolumn{3}{l}{interfering:} \\
          l8: & ADD & global\_ptr+idx2 & \!\!\!$\rightarrow$ reg4 \\
          l9: & STORE & 0 & \!\!\!$\rightarrow$ *reg4 \\
          \\
          \\
          \\
          \\
          \\
        \end{tabular}
      }
    }\\
  \end{tabular}
  \caption{Example threads.}
  \label{fig:enforce:example_threads}
\end{figure}
  
\begin{figure}
  {\hfill}
  \begin{tikzpicture}
    \node[CfgInstr] (l1) {l1: ADD global\_ptr + idx1 $\rightarrow$ reg1};
    \node[CfgInstr, below=of l1] (l2) {l2: LOAD {$\ast$}reg1 $\rightarrow$ reg2};
    \node[CfgInstr, below=of l2] (l3) {l3: CMP 0, reg2};
    \node[CfgInstr, below=of l3] (l4) {l4: JMP\_IF\_EQ };
    \node[CfgInstr, below=of l4] (l5) {l5: LOAD {$\ast$}reg1 $\rightarrow$ reg3 };
    \node[CfgInstr, below=of l5] (l6) {l6: STORE 7 $\rightarrow$ {$\ast$}reg3 };
    \node[CfgInstr, below=of l6] (l7) {};
    \node[CfgInstr, right=of l1] (l8) {l8: ADD global\_ptr + idx2 $\rightarrow$ reg4};
    \node[CfgInstr, below=30mm of l8] (l9) {l9: STORE 0 $\rightarrow$ {$\ast$}reg4};
    \draw[->] (l1) -- (l2);
    \draw[->] (l2) -- (l3);
    \draw[->] (l3) -- (l4);
    \draw[->,ifFalse] (l4) -- (l5);
    \draw[->,ifTrue] (l4.west) to [bend right=70] (l7);
    \draw[->] (l5) -- (l6);
    \draw[->] (l6) -- (l7);
    \draw[->] (l8) -- (l9);
    \draw[->,happensBeforeEdge] (l2) -- (l9);
    \draw[->,happensBeforeEdge] (l9) -- (l5);
  \end{tikzpicture}
  {\hfill}
  \caption{CFG with happens-before edges for the example.}
  \label{fig:using:example_hb_graph}
\end{figure}

\begin{figure}
  \begin{tabular}{p{0.6\textwidth}p{0.35\textwidth}}
    \centerline{
      \begin{tikzpicture}
        \node[draw] (A) {A};
        \node[draw, above right = of A] (X) {X};
        \node[draw, below right = of X] (B) {B};
        \node[draw, above right = of B] (Y) {Y};
        \node[draw, below right = of Y] (C) {C};
        \node [below = 1.2 of B] {};
        \draw[->] (A) -- (B);
        \draw[->] (B) -- (C);
        \draw[->] (X) -- (Y);
        \draw[->,happensBeforeEdge] (A) -- (X);
        \draw[->,happensBeforeEdge] (X) -- (B);
        \draw[->,happensBeforeEdge] (B) -- (Y);
        \draw[->,happensBeforeEdge] (Y) -- (C);
      \end{tikzpicture}
    }
    &
    \begin{tikzpicture}
      \node[place,tokens=1,label=above:{before receive}] (A) {};
      \node[transition,below right=of A, label=right:{\shortstack[l]{start\\message}}] (C) {};
      \node[place,tokens=1,above right = of C, label=above:{before transmit}] (B) {};
      \node[place,below = of C] (D) {};
      \node[transition,below = of D, label=right:{\shortstack[l]{check\\side-condition}}] (E) {};
      \node[place,below = of E] (F) {};
      \node[transition,below = of F, label=right:{\shortstack[l]{finish\\message}}] (G) {};
      \node[place,tokens=0,below left = of G, label=below:{after receive}] (H) {};
      \node[place,tokens=0,below right = of G, label=below:{after transmit}] (I) {};
      \draw[->] (A) -- (C);
      \draw[->] (B) -- (C);
      \draw[->] (C) -- (D);
      \draw[->] (D) -- (E);
      \draw[->] (E) -- (F);
      \draw[->] (F) -- (G);
      \draw[->] (G) -- (H);
      \draw[->] (G) -- (I);
    \end{tikzpicture} \\
    \refstepcounter{figure}\label{fig:enforce_crash:complex_hb}
    \small Figure \arabic{chapter}.\arabic{figure}: \textit{A happens-before graph which cannot be reproduced by
      inserting a simple delay.  The bug of interest reproduces only for
      the access ordering A, X, B, Y, C.} &
    \refstepcounter{figure}\label{fig:message_petri_net}
    \small Figure \arabic{chapter}.\arabic{figure}: \textit{Petri net for the} \textit{message operation.}
  \end{tabular}
\end{figure}

Consider, for example, the threads shown in
\autoref{fig:enforce:example_threads}\footnote{This example is
  discussed further in the evaluation.}.  There is a risk here that
the crashing thread might crash if \verb|l9| is interleaved between
\verb|l2| and \verb|l5| and \verb|idx1 == idx2|.  The previous
analysis phase will detect this bug and produce the verification
condition $idx1 = idx2 \wedge \happensBefore{l2}{l9} \wedge
\happensBefore{l9}{l5}$.  This can then be used to augment the control
flow graph of the program with happens-before edges, as shown in
\autoref{fig:using:example_hb_graph}.

One could make this happens-before graph more likely by inserting
small delays into the program's execution.  In this case, all that
needs to happen is for some thread to execute \texttt{l9} while
another thread is between \texttt{l2} and \texttt{l5}, and so
inserting a delay anywhere between \texttt{l2} and \texttt{l5} would
be sufficient.  In general, the longer the delay, the more likely it
becomes that some other thread will execute \texttt{l9} during it, and
hence the more easily the bug will reproduce.  Alternatively,
\texttt{l9} could be delayed to run as soon as possible after
\texttt{l2}.  Which strategy is most appropriate would depend on the
frequency at which the two fragments of code run.

\autoref{fig:enforce_crash:complex_hb} shows a more complex
example.  The graph would be difficult to enforce by simply inserting
fixed delays between instructions within a given thread.  Consider,
for instance, the delay between X and Y.  This must be sized so as to
increase the probability of Y happening after B, but without overly
increasing the probability of Y happening
after C.  Similarly, the delay between A and B must be sized such that
B occurs after X but before Y.  Solving that kind of constraint
requires detailed information about the program's timing, and this
information is both difficult to collect and highly
configuration-specific \todo{Cite, maybe?}.

\todo{Try harder to bind side-condition}

Even once the happens-before graph has been enforced, that is not
always sufficient to reproduce the bug.  In the previous example, for
instance, the two threads will only interact at all when
$\mathit{idx1} = \mathit{idx2}$, and so there is no point in trying to
enforce the happens-before graph when the two indices do not match up.
This is not just a performance requirement: inserting delays often
means that the buggy code will run less often, and so inserting delays
when the bug has no chance of reproducing will often cause the bug to
reproduce \emph{less} frequently; precisely the opposite of the
desired effect.  Note that the side-condition will often involve
thread-local state from multiple threads (in this case,
$\mathit{idx1}$ is local to the crashing thread and $\mathit{idx2}$ to
the interfering one), so that no single thread can evaluate the
condition by itself.

\Technique{} solves all of these problems by modelling the program as
a message-passing system.  A happens-before ordering X before Y is
modelled by a message sent by X, after it completes, and received by
Y, before it starts.  In the first example, there will be two
messages, one from \texttt{l2} to \texttt{l9} and one from \texttt{l9}
to \texttt{l5}.

These messages are synchronous, in the sense that a message operation
cannot complete unless both the sender and the receiver are
simultaneously at the relevant instructions, as illustrated by the
Petri net in \autoref{fig:message_petri_net}\editorial{SMH doesn't
  like Petri nets.}.  One way of reading this diagram is that the
process starts with two threads, in the two before locations, which
are then merged into a single thread in the message operation
transition, and which then split apart into separate threads in the
two after places.  This suggests a possible strategy for evaluating
the side-condition: if the message operation transition represents the
merging of the two threads, it should be able to access both threads'
local state, and so be able to evaluate the $\mathit{idx1} =
\mathit{idx2}$ side-condition.  This is the strategy adopted by
       {\technique}.

\section{Outline of algorithm}

At a high level, the algorithm used has the following phases:

\begin{itemize}
\item
  Examine the verification condition and extract the happens-before
  graphs which are to be reproduced.  This is discussed in
  \autoref{sect:enforce:slice_hb_graph}.
\item
  Plan how to evaluate the side-condition part of the verification
  condition.  This is discussed in \autoref{sect:enforce:place_vcs}.
\item
  Decide on a strategy for gaining control of the program at
  appropriate points.  Discussion of how to do so is deferred to
  \autoref{sect:enforce:gain_control}; for now, assume that there is
  some mechanism to gain control of the program at arbitrary
  instructions.
\item
  Combine the results of the previous phases with the \emph{plan
    interpreter}, discussed in \autoref{sect:enforce:interpreting},
  and compile the result into the final \gls{bugenforcer}.
\end{itemize}

In the case of {\implementation}, the \gls{bugenforcer} takes the form
of an ELF shared object which can be loaded into the target program to
enforce the plan.  It is also possible to combine multiple
\glspl{bugenforcer} into a single shared object which will try to
enforce all of the plans at the same time; this is useful when the
initial analysis phase generates a large number of
\glspl{verificationcondition} for bugs in different parts of the
program.

\section{Extracting the happens-before graph}
\label{sect:enforce:slice_hb_graph}

\begin{wrapfigure}{r}{4cm}
  \vspace{-5mm}
  \begin{tikzpicture}
    \node (thread1) {\shortstack{Crashing\\thread}};
    \node (thread2) [right = 0.5 of thread1] {\shortstack{Interfering\\thread}};
    \node (A) [below = 0 of thread2] {$A_1$};
    \node (B) [below = of A] {$B_1$};
    \node (C) [below = of B] {$C_1$};
    \node (D0) [below = 0.7 of thread1] {$x = 7$ $D_0$};
    \node (D1) [below = of D0] {$y = 9$ $D_0$};
    \draw[->] (A) -- (B);
    \draw[->] (B) -- (C);
    \draw[->,happensBeforeEdge] (A) -- (D0);
    \draw[->,happensBeforeEdge] (D0) -- (B);
    \draw[->,happensBeforeEdge] (B) -- (D1);
    \draw[->,happensBeforeEdge] (D1) -- (C);
  \end{tikzpicture}
  \vspace{-10mm}
  \caption{}
  \vspace{-5mm}
  \label{fig:reproducing:placement_eg_bug}
\end{wrapfigure}
The first step in building a \gls{bugenforcer} is to extract the
happens-before graph from the \gls{verificationcondition}, so that the
\gls{bugenforcer} knows what instruction ordering it needs to enforce,
and the side-condition which must be checked for each ordering.  This
is most easily understood as a transformation on the BDD which
represents the \gls{verificationcondition}.  Consider, for instance,
the graph shown in \autoref{fig:reproducing:placement_eg_bug}.  This
is intended to indicate that the \gls{crashingthread} has a single
instruction, $D_0$, that the \gls{interferingthread} has three
instructions, $A_1$, $B_1$, and $C_1$, and that the bug reproduces if
either $x = 7$ and $D_0$ happens between $A_1$ and $B_1$ or if $y = 9$
and $D_0$ happens between $B_1$ and $C_1$.

\begin{figure}
  \todo{SMH had forgotten what the dotted lines meant by the time he
    got here (despite also commenting on the explanation of what the
    dotted lines meant that it was too long).}

  \subfigure[][Before re-ordering]{
    \begin{tikzpicture}
      \node (n10) [BddNode] {$x = 7$};
      \node (n01) [BddNode, below left = of n10] {$y = 9$};
      \node (n21) [BddNode, below right = of n10] {$y = 9$};
      \node (n02) [BddNode, below = of n01] {$\happensBefore{D_0}{B_1}$};
      \node (n12) at (n02 -| n10) [BddNode] {$\happensBefore{D_0}{B_1}$};
      \node (n22) at (n02 -| n21) [BddNode] {$\happensBefore{D_0}{B_1}$};
      \node (n03) [BddNode, below = of n02] {$\happensBefore{A_1}{D_0}$};
      \node (n13) at (n03 -| n12) [BddNode] {$\happensBefore{D_0}{C_1}$};
      \node (true) [BddLeaf, below = of n03] {\true};
      \node (false) at (true -| n22) [BddLeaf] {\false};
      \draw[BddTrue] (n01) -- (n02);
      \draw[BddFalse] (n01) -- (n12);
      \draw[BddTrue] (n02) -- (n03);
      \draw[BddFalse] (n02) -- (n13);
      \draw[BddTrue] (n03) -- (true);
      \draw[BddFalse] (n03) to [bend right=5] (false);
      \draw[BddTrue] (n10) -- (n01);
      \draw[BddFalse] (n10) -- (n21);
      \draw[BddTrue] (n12) -- (n03);
      \draw[BddFalse] (n12) to [bend left=10] (false);
      \draw[BddTrue] (n13) -- (true);
      \draw[BddFalse] (n13) -- (false);
      \draw[BddTrue] (n21) -- (n22);
      \draw[BddFalse] (n21) to [bend left=35] (false);
      \draw[BddTrue] (n22) -- (n13);
      \draw[BddFalse] (n22) -- (false);
    \end{tikzpicture}
    \label{fig:reproducing:slice_eg:before}
  }
  \subfigure[][After re-ordering]{
    \begin{tikzpicture}
      \node (n10) [BddNode] {$\happensBefore{D_0}{B_1}$};
      \node (n01) [BddNode, below left = of n10] {$\happensBefore{A_1}{D_0}$};
      \node (n21) [BddNode, below right = of n10] {$\happensBefore{D_0}{C_1}$};
      \node (n02) [BddNode, below = of n01] {$x = 7$};
      \node [left = 0 of n02] {\circled{1}};
      \node (n22) [BddNode, below = of n21] {$y = 9$};
      \node [right = 0 of n22] {\circled{2}};
      \node (true) [BddLeaf, below = of n02] {\true};
      \node (false) [BddLeaf, below = of n22] {\false};
      \draw[BddTrue] (n10) -- (n01);
      \draw[BddFalse] (n10) -- (n21);
      \draw[BddTrue] (n01) -- (n02);
      \draw[BddFalse] (n01) -- (false);
      \draw[BddTrue] (n02) -- (true);
      \draw[BddFalse] (n02) -- (false);
      \draw[BddTrue] (n21) -- (n22);
      \draw[BddFalse] (n21) to [bend left=30] (false);
      \draw[BddTrue] (n22) -- (true);
      \draw[BddFalse] (n22) -- (false);
    \end{tikzpicture}
    \label{fig:reproducing:slice_eg:after}
  }
  \caption{Example BDD}
\end{figure}

This bug might produce a \gls{verificationcondition} something like
the one in \autoref{fig:reproducing:slice_eg:before}.  The
\gls{verificationcondition} correctly represents the intended
behaviour, but mixes the happens-before tests $\happensBeforeEdge$
with tests of the program's ordinary state, like the tests of $x = 7$
and $y = 9$, making it more difficult to enforce.
\autoref{fig:reproducing:slice_eg:after} shows a
semantically-equivalent BDD which is slightly easier to work with.
Now all of the happens-before tests occur before any of the tests of
ordinary program state, and so can be separated out relatively easily.

This BDD has, in effect, three parts: two conditions on ordinary
program state, \circled{1} and \circled{2}, and a network of tests of
$\happensBeforeEdge$ conditions which selects between them.  Each path
through the initial network corresponds to a different ordering of the
program's instructions and the conditions on ordinary program state
provide the side-condition which must be enforced when the program
follows that path.  In this case, there are only two paths:
$\happensBefore{D_0}{B_1}$, $\happensBefore{A_1}{D_0}$, \circled{1}
and ${\neg}\happensBefore{D_0}{B_1}$, $\happensBefore{D_0}{C_1}$,
\circled{2}.  {\Technique} will therefore try to enforce two
happens-before graphs, corresponding to the two ways in which the bug
can reproduce.  The two side conditions will be checked independently.

This example also illustrates the importance of side-condition
checking.  One of the ways of reproducing this bug requires that $D_0$
happens before $B_1$, while the other requires that $D_0$ happens
after $B_1$.  These clearly cannot both happen at once.  Without
side-condition checking, {\technique} would always try to enforce the
first ordering, which would be unfortunate if the second represented a
real bug and the first did not.  With side-condition checking,
{\technique} will only try to enforce the first ordering if $x = 7$,
and, if the side-conditions are correct, enforcing the ordering in
that case will definitely lead to a crash, and so the fact that the
two graphs are incompatible does not cause any actual problems.

The computation complexity of BDD re-ordering operations is, in the
worst case, $O(2^n)$ if the BDD moves from a particularly good
ordering to a particularly bad one.  {\Technique} very rarely
encounters this worst case, for several reasons.  Most simply, the BDD
variable ordering used when deriving \glspl{verificationcondition}
usually puts most of the happens-before tests near the root of the
BDD, and so there is usually little need for extensive re-ordering
during this phase.  Beyond that, most bugs do not have very complex
data-dependent happens-before patterns, and so the initial
$\happensBeforeEdge$ network part of the re-ordered BDD usually only
has to select between a small number of conditions over the program's
state, which further restricts the scope for excessive BDD growth.
Finally, most bugs do not need to enforce a large number of
happens-before edges; Musuvathi et al\cite{Musuvathi2007}, achieved
good results enforce just two edges at a time.

This does not, therefore, lead to an excessive increase in the cost of
generating the enforcer.  Treating each graph's side condition
independently does, however, lead to some inefficiency in the enforcer
itself, as it is not possible to share common expressions between the
different bugs.  A more elegant approach would be to allow the
different graphs to share a common prefix and only have them diverge
once they encounter the data-dependent happens-before components.  I
have not implemented such an optimisation.

\subsection{Unenforcable graphs}
\begin{figure}
  \centerline{
    {\hfill}
  \subfigure[][An unenforcable control-flow and happens-before graph.]{
    \begin{tikzpicture}
      \node (thread0) {Thread 0};
      \node[CfgInstr, below = 1.8 of thread0] (A) {$A_0$};
      \node[CfgInstr, below = of A] (B) {$B_0$};
      \node[right = 2 of thread0] (thread1) {Thread 1};
      \node[CfgInstr, below = 1.8 of thread1] (C) {$C_1$};
      \node[CfgInstr, below = of C] (D) {$D_1$};
      \node[below = 0.5 of D] {};
      \draw[->] (A) -- (B);
      \draw[->] (C) -- (D);
      \draw[->,happensBeforeEdge] (A) to node [above,pos=0.25] {$X$} (D);
      \draw[->,happensBeforeEdge] (C) to node [above,pos=0.25] {$Y$} (B);
    \end{tikzpicture}
    \label{fig:reproduce:cyclic_synchronous}
  }
    {\hfill}
  \subfigure[][The graph which is actually enforced; the double arrow indicates
    that $C_1$ must happen \emph{immediately} before $B_0$, rather than any time
    before it.]{
    \begin{tikzpicture}
      \node (thread0) {Thread 0};
      \node[CfgInstr, below = of thread0] (A) {$A_0$};
      \node[CfgInstr, below = of A] (B) {$B_0$};
      \node[right = 2 of thread0] (thread1) {Thread 1};
      \node[CfgInstr] at (thread1 |- B) (C) {$C_1$};
      \node[CfgInstr, below = of C] (D) {$D_1$};
      \draw[->] (A) -- (B);
      \draw[->] (C) -- (D);
      \draw[->,double] (C) -- (B);
    \end{tikzpicture}
    \label{fig:reproduce:cyclic_decyclic}
  }
    {\hfill}
  }
  \vspace{-12pt}
  \caption{}
  \label{fig:reproduce:cyclic1}
\end{figure}

The use of synchronous messages restricts the class of bugs which can
be reproduced.  Consider, for instance, the combined control-flow and
happens-before graph in \autoref{fig:reproduce:cyclic_synchronous},
which shows that both threads must complete their first instruction
before either thread starts its second.  It is clearly possible for
the program to execute its instructions in this order, but this
ordering cannot be enforced using synchronous messages: in order to
enforce edge $X$, $A_0$ and $D_1$ must execute at the same time, and
so $A_0$ must wait for $D_1$, but in order to enforce edge $Y$ $C_1$
must wait for $B_0$.  This creates a cycle in the graph, and so this
program would deadlock if the enforcer attempted to enforce the entire
graph.

{\Technique} detects these cycles and resolves them by discarding one
of the happens-before edges, chosen arbitrarily.  In this particular
case, that is sufficient to avoid the problem.  Suppose that
{\technique} selects edge $X$ to discard (the other case is
symmetrical).  The enforcer will now ensure that $C_1$ waits for $B_0$
and then executes immediately before it, which is also sufficient to
ensure that $D_1$ executes after $A_0$, as shown in
\autoref{fig:reproduce:cyclic_decyclic}.  There is, however, no
guarantee that this technique will always work for more complex
graphs.

\section{Placing the evaluation of verification conditions}
\label{sect:enforce:place_vcs}

The verification condition has now been factored into a happens-before
graph and a side-condition, such that if the happens-before graph is
enforced while the side-condition holds then the bug is highly likely
to reproduce.  The next step is to decide how to evaluate the
side-condition.  The challenge here is that the various input
expressions to the side-condition, such as the values of registers or
$\controlEdge{}{\!}{\!}$ expressions, become available at different
points in the program's \gls{cfg}, and we would like to ensure that
the various components of the side-condition are evaluated as soon as
possible, so that the enforcer can avoid imposing unnecessary delays.

\begin{figure}
  \begin{tikzpicture}
    \node (t1) {Thread 1};
    \node[right = 4.5cm of t1] (t2) {Thread 2};
    \node[below = of t1] (A) {$A$};
    \node[below = of A] (B) {$B$};
    \node[below = of B] (C) {$C$};
    \node[below = of C] (D) {$D_0$};
    \node[left = 4cm of D] (D1) {$D_1$};
    \node at (t2 |- A) (X) {$X$};
    \node at (t2 |- B) (Y) {$Y$};
    \node at (t2 |- D) (Z) {$Z$};
    \draw[->] (A) -- (B);
    \draw[->] (B) -- (C);
    \draw[->] (C) -- (D);
    \draw[->] (C) -- (D1);
    \draw[->] (X) -- (Y);
    \draw[->] (Y) -- (Z);
    \draw[->] (D1) to [bend right=55] (D);
    \draw[->,happensBeforeEdge] (B) to node [above] {M} (Y);
    \draw[->,happensBeforeEdge] (Z) to node [above] {N} (D);
  \end{tikzpicture}
  \vspace{-8pt}
  \center{Side condition: $\textsc{rax}_1 = 93 \wedge (\textsc{rbx}_2 =
    72 \vee \controlEdge{1}{C}{D_0}) \wedge \textsc{rcx}_1 = \textsc{rcx}_2$}
  \vspace{-4pt}
  \caption{Example control-flow (solid lines) and happens-before
    (dashed lines) graphs.  $D_0$ and $D_1$ represent duplicates of
    the same program instruction created by the loop unrolling
    algorithm (see \autoref{sect:derive:handling_loops}).  They are
    indistinguishable in the original program, but only $D_0$ can
    receive the happens-before edge N.}
  \label{fig:place_conditions_example}
  \vspace{4pt}
\end{figure}

Consider the example in \autoref{fig:place_conditions_example}.  The
side-condition to this bug depends on various values which have to be
collected from the running program, referred to as the condition's
inputs: $\textsc{rax}_1$, $\textsc{rbx}_1$, $\textsc{rcx}_1$, which
are registers in the \gls{crashingthread}, $\textsc{rcx}_2$, a
register in the interfering one, and $\controlEdge{1}{C}{D_0}$, an
expression on the \gls{crashingthread}'s control flow.  It is
important not to evaluate any part of the side condition before that
part's inputs become available, but also important not to defer
evaluating the condition until after a message operation if there is
enough information to show that the condition will fail and hence that
the message operation is redundant.

\begin{figure}[t]
  \begin{tikzpicture}
    \node (t1) {Thread 1};
    \node[right = 4.5cm of t1] (t2) {Thread 2};
    \node[below = of t1] (A) {$A$};
    \node[left = 0 of A] {$\textsc{rax}_1$, $\textsc{rcx}_1$};
    \node[below = of A] (B) {$B$};
    \node[left = 0 of B] {$\textsc{rax}_1$, $\textsc{rcx}_1$};
    \node[below = of B] (C) {$C$};
    \path (node cs:name=C,anchor=north east) node [below right] {\shortstack[r]{$\textsc{rax}_1$, $\textsc{rcx}_1$\\$\textsc{rbx}_2$, $\textsc{rcx}_2$}};
    \node[below = of C] (D) {$D_0$};
    \path (node cs:name=D,anchor=north west) node [below left] {\shortstack[c]{$\controlEdge{1}{C}{D_0}$,\\$\textsc{rax}_1$, $\textsc{rcx}_1$\\$\textsc{rbx}_2$, $\textsc{rcx}_2$}\!\!\!};
    \node[left = 4cm of D] (D1) {$D_1$};
    \node[above = 0 of D1] {\shortstack[c]{$\controlEdge{1}{C}{D_0}$,\\$\textsc{rax}_1$, $\textsc{rcx}_1$\\$\textsc{rbx}_2$, $\textsc{rcx}_2$}\!\!\!};
    \node at (t2 |- A) (X) {X};
    \node[below right = 0 of X.north east] {$\textsc{rbx}_2$, $\textsc{rcx}_2$};
    \node at (t2 |- B) (Y) {Y};
    \node[below right = 0 of Y.north east] {\shortstack[l]{$\textsc{rax}_1$, $\textsc{rcx}_1$\\$\textsc{rbx}_2$, $\textsc{rcx}_2$}};
    \node at (t2 |- D) (Z) {Z};
    \node[below right = 0 of Z.north east] {\shortstack[l]{$\textsc{rax}_1$, $\textsc{rcx}_1$\\$\textsc{rbx}_2$, $\textsc{rcx}_2$}};
    \draw[->] (A) -- (B);
    \draw[->] (B) -- (C);
    \draw[->] (C) -- (D);
    \draw[->] (C) -- (D1);
    \draw[->] (X) -- (Y);
    \draw[->] (Y) -- (Z);
    \draw[->] (D1) to [bend right=55] (D);
    \draw[->,happensBeforeEdge] (B) to node [above] {M} node [below] {\shortstack[c]{$\textsc{rax}_1$, $\textsc{rcx}_1$\\$\textsc{rbx}_2$, $\textsc{rcx}_2$}} (Y);
    \draw[->,happensBeforeEdge] (Z) to node [above] {N} node [below] {\shortstack[c]{$\textsc{rax}_1$, $\textsc{rcx}_1$\\$\textsc{rbx}_2$, $\textsc{rcx}_2$,\\$\controlEdge{1}{C}{D_0}$}} (D);
  \end{tikzpicture}
  \vspace{-12pt}
  \caption{\autoref{fig:place_conditions_example} extended to show
    input availability.}
  \label{fig:place_conditions_example:availability}
\end{figure}

The availability of inputs for this example is shown in
\autoref{fig:place_conditions_example:availability}\editorial{bad break}:
\begin{itemize}
\item $\textsc{rax}_1$ and $\textsc{rcx}_1$ represent the initial
  values of registers \textsc{rax} and \textsc{rcx} in thread 1, and
  are therefore trivially available at the start of thread 1.  The
  enforcer can copy them to its local state, when necessary, and so
  they are also available at every subsequent instruction in thread 1.
  Similarly, $\textsc{rbx}_2$ and $\textsc{rcx}_2$ are available
  throughout thread 2.
\item All of the program registers are available when evaluating the
  message edge M.  In general, the expressions available on a message
  edge will be the union of the expressions available at the sender and
  receiver of the message.
\item All of the program registers are available at instruction Y.
  This instruction receives the message M before it starts, and so
  anything which is available to M will also be available to Y.  On
  the other hand, in thread 1 the expressions available to M do not
  become available until instruction C, the instruction after the one
  which sent the message.  This asymmetry is because messages are
  received at the start of an instruction and sent at the end.
\item Likewise, the control flow expression $\controlEdge{1}{C}{D_0}$
  is available at $D_0$ and N, as these happen after instruction C
  completes and control flow expressions become available near the end
  of the instruction cycle.
\end{itemize}
The precise details of the instruction cycle, and hence precise rules
for calculating when input expressions become available, are given in
\autoref{sect:enforce:llis}\editorial{Well, they're kind of implied
  rather than given, but they're pretty damn obvious.  Wouldn't hurt
  to be a bit more explicit, I suppose.}.

\begin{figure}
  \centerline{
    \begin{tikzpicture}
      \node (root) [BddNode] {$\controlEdge{1}{C}{D_0}$};
      \path (node cs:name=root) ++(-2,-1.3) node (a0) [BddNode] {$\textsc{rax}_1 = 93$};
      \node (c0) [BddNode, below = of a0] {$\textsc{rcx}_1 = \textsc{rcx}_2$};
      \path (node cs:name=c0) ++(0, -3) node (true) [BddLeaf] {\true};
      \path (node cs:name=root) ++(2,-1.3) node (a1) [BddNode] {$\textsc{rax}_1 = 93$};
      \node (c1) [BddNode, below = of a1] {$\textsc{rcx}_1 = \textsc{rcx}_2$};
      \path (node cs:name=c1) ++(0, -3) node (false) [BddLeaf] {\false};
      \path (node cs:name=root) ++(0,-4.2) node (b) {$\textsc{rbx}_2 = 72$};
      \draw[BddTrue] (root) -- (a0);
      \draw[BddFalse] (root) -- (a1);
      \draw[BddTrue] (a0) to (c0);
      \draw[BddFalse] (a0) to [bend left=20] (false.north);
      \draw[BddTrue] (a1) -- (c1);
      \draw[BddFalse] (a1.east) to [bend left=90] (false.east);
      \draw[BddTrue] (c0) -- (true);
      \draw[BddFalse] (c0.east) to [bend left=30] (false.north);
      \draw[BddTrue] (c1) -- (b);
      \draw[BddFalse] (c1) -- (false);
      \draw[BddTrue] (b) -- (true);
      \draw[BddFalse] (b) -- (false);
    \end{tikzpicture}
  }
  \caption{The side-condition in
    \autoref{fig:place_conditions_example} expressed as a BDD,
    assuming the variable ordering $\controlEdge{1}{C}{D_0}$,
    $\textsc{rax}_1 = 93$, $\textsc{rcx}_1 = \textsc{rcx}_2$,
    $\textsc{rbx}_2 = 72$.}
  \label{fig:place_conditions_example:bdd1}
\end{figure}

Once the availability of input expressions has been determined, the
next step is to decide how to evaluate the side condition, and in
particular at which \gls{cfg} nodes it should be evaluated.  As
already discussed, {\technique} represents its side-conditions using
BDDs, and the ordering of variables within the BDDs is largely
arbitrary.  There is therefore no guarantee that the BDD will test the
components of the side-condition in an order which is convenient for
run-time checking.  For instance, the side-condition in the example
might be represented by \autoref{fig:place_conditions_example:bdd1}.
The first variable in the BDD is the control-flow expression, but this
is one of the last input expressions to become available, making it
difficult to evaluate any part of the BDD early.

\begin{figure}
  \centerline{
    \begin{tabular}{ccccc}
      \begin{tikzpicture}
        \node (root) [BddNode,color=blue] {$\textsc{rax}_1 = 93$};
        \node (alpha) [BddNode, below = of root] {$\controlEdge{1}{C}{D_0}$};
        \node (c0) [BddNode, below = of alpha] {$\textsc{rcx}_1 = \textsc{rcx}_2$};
        \path (node cs:name=c0,anchor=east) node (c1) [BddNode,right] {$\textsc{rcx}_1 = \textsc{rcx}_2$};
        \path (barycentric cs:c0=0.5,c1=0.5) ++(0.2,-0.8) node (b) [BddNode, below] {$\textsc{rbx}_2 = 72$};
        \path (node cs:name=c1,anchor=south) ++(0,-2) node (false) [BddLeaf] {\false};
        \node (true) at (c0 |- false) [BddLeaf] {\true};
        \draw[BddTrue] (root) -- (alpha);
        \draw[BddFalse] (root.east) to [bend left=45] (false);
        \draw[BddTrue] (alpha) -- (c0);
        \draw[BddFalse] (alpha) -- (c1);
        \draw[BddTrue] (c0) -- (true);
        \draw[BddFalse] (c0) to [bend left=30] (false);
        \draw[BddTrue] (c1) -- (b);
        \draw[BddFalse] (c1) -- (false);
        \draw[BddTrue] (b) -- (true);
        \draw[BddFalse] (b) -- (false);
      \end{tikzpicture} & \raisebox{30mm}{$=$} & \raisebox{22mm}{\begin{tikzpicture}
        \node (root) [BddNode,color=blue] {$\textsc{rax}_1 = 93$};
        \node (true) at (-0.6,-1) [BddLeaf] {\true};
        \node (false) at (0.6,-1) [BddLeaf] {\false};
        \draw[BddTrue] (root) -- (true);
        \draw[BddFalse] (root) -- (false);
      \end{tikzpicture}} & \raisebox{30mm}{$\bigwedge$} & \raisebox{16mm}{\begin{tikzpicture}
        \node (root) [BddNode] {$\controlEdge{1}{C}{D_0}$};
        \path (node cs:name=root) ++ (-1.5,-1) node (c0) [BddNode] {$\textsc{rcx}_1 = \textsc{rcx}_2$};
        \path (node cs:name=root) ++ (1.5,-1) node (c1) [BddNode] {$\textsc{rcx}_1 = \textsc{rcx}_2$};
        \node (b) at (root |- 0,-1.8) [BddNode] {$\textsc{rbx}_2 = 72$};
        \node (true) [BddLeaf, below = of c0] {\true};
        \node (false) [BddLeaf, below = of c1] {\false};
        \draw[BddTrue] (root) -- (c0);
        \draw[BddFalse] (root) -- (c1);
        \draw[BddTrue] (c0) -- (true);
        \draw[BddFalse] (c0) to [bend right=20] (false);
        \draw[BddTrue] (c1) -- (b);
        \draw[BddFalse] (c1) -- (false);
        \draw[BddTrue] (b) -- (true);
        \draw[BddFalse] (b) -- (false);
      \end{tikzpicture}} \\
      \small \it \raisebox{6pt}{(a) Reordered BDD} & \multicolumn{3}{c}{\parbox{4cm}{\small \it (b) Condition evaluable at $A$}} & \small \it \raisebox{8pt}{(c) Residual condition}
    \end{tabular}
  }
  \caption{Factorisation of the BDD in
    \autoref{fig:place_conditions_example:bdd1} at CFG node $A$.
    $\textsc{rax}_1$ and $\textsc{rcx}_1$ are the only available input
    expressions.  Evaluable BDD variables are shown in blue.}
  \label{fig:place_conditions_example:instr_a}
\end{figure}

The problem becomes far easier if the side-condition BDD is re-ordered
to reflect the order in which the input expressions become available.
The leftmost part of \autoref{fig:place_conditions_example:instr_a}
shows how this is done for \gls{cfg} node $A$.  The available inputs
are $\textsc{rax}_1$ and $\textsc{rcx}_1$, and so the only evaluable
expression is $\textsc{rax}_1 = 93$ and this expression is moved to
the root of the BDD.  The BDD can then be factorised into a component
which is evaluable at \gls{cfg} node $A$, (b), and a residual which
must be evaluated after $A$, (c), such that (b)~$\wedge$~(c) is equal
to the original side-condition.

The factorisation algorithm itself is reasonably simple.  To build the
evaluable component, {\technique} replaces all of the nodes which
test unevaluable variables with edges to \true; to build the
residual component, it removes all edges from an evaluable variable
to \false.  The usual BDD reduction rules are then used to convert the
result back into a BDD.

The most important property of this algorithm is that it produces a
correct factorisation of the side condition.  In other words, if the
algorithm decomposes $p$ into $p_0$ and $p_1$ then $p = p_0 \wedge
p_1$.  This is easiest to see by considering the paths through the
BDD: paths which test only evaluable variables and end at {\false}
will be preserved in the evaluable component of the result, and all
of the other paths will be preserved in the unevaluable one.  If all
of the paths through the BDD are preserved then the BDD behaviour is
as well.

\begin{figure}
  \centerline{
    \begin{tikzpicture}
      \node (b) [BddNode, color=blue] {$\textsc{rbx}_2 = 72$};
      \node (a) [BddNode, below = 2 of b] {$\textsc{rax}_1 = 93$};
      \node (c) [BddNode, below = of a] {$\textsc{rcx}_1 = \textsc{rcx}_2$};
      \node (true) [BddLeaf, below = of c] {\true};
      \node (false) [BddLeaf, right = of true] {\false};
      \node (alpha) at (false |- 1,-1.3) [BddNode] {$\controlEdge{1}{C}{D_0}$};
      \draw[BddTrue] (b) -- (a);
      \draw[BddFalse] (b) -- (alpha);
      \draw[BddTrue] (alpha) -- (a);
      \draw[BddFalse] (alpha) -- (false);
      \draw[BddTrue] (a) -- (c);
      \draw[BddFalse] (a) to [bend left=20] (false);
      \draw[BddTrue] (c) -- (true);
      \draw[BddFalse] (c) -- (false);
    \end{tikzpicture}
  }
  \caption{Reordering of the BDD in
    \autoref{fig:place_conditions_example:bdd1} at CFG node $X$.
    $\textsc{rbx}_2$ and $\textsc{rcx}_2$ are the only available input
    expressions.  Evaluable BDD variables are shown in blue. In this
    case, no non-trivial factorisation is possible.  \todo{Unref?}}
\end{figure}

The residual BDD which is unevaluable at $A$ must now be propagated
through the graph until we find some place at which it can be
evaluated.  The first place we encounter, $B$, has the same available
input expression set as $A$, and so no further side-conditions can be
evaluated here.  The M edge, however, does have more available input
expressions, and so can make some progress.  The factorisation is
shown in \autoref{fig:place_conditions_example:message_m}.  Note
that the residual condition here includes an evaluable variable,
$\textsc{rbx}_2 = 72$.  In principle, this could be evaluated as part
of this message operation, but doing so would never allow the enforcer
to exit early, and so would not be helpful.

\begin{figure}
  \centerline{
    \begin{tabular}{ccccc}
      \begin{tikzpicture}
        \node (c) [BddNode,color=blue] {$\textsc{rcx}_1 = \textsc{rcx}_2$};
        \node (b) [BddNode,below = of c,color=blue] {$\textsc{rbx}_2 = 72$};
        \node (true) [BddLeaf, below = 2 of b] {\true};
        \node (false) [BddLeaf, right = of true] {\false};
        \path (node cs:name=false) ++(0,1.3) node (alpha) [BddNode] {$\controlEdge{1}{C}{D_0}$};
        \draw[BddTrue] (c) -- (b);
        \draw[BddFalse] (c) to [bend left=30] (false);
        \draw[BddTrue] (b) -- (true);
        \draw[BddFalse] (b) -- (alpha);
        \draw[BddTrue] (alpha) -- (true);
        \draw[BddFalse] (alpha) -- (false);
      \end{tikzpicture}
      & \raisebox{23mm}{$=$} &
      \raisebox{15mm}{
        \begin{tikzpicture}
          \node (c) [BddNode, color=blue] {$\textsc{rcx}_1 = \textsc{rcx}_2$};
          \path (node cs:name=root) ++(-0.6,-1) node (true) [BddLeaf] {\true};
          \path (node cs:name=root) ++(0.6,-1) node (false) [BddLeaf] {\false};
          \draw[BddTrue] (c) -- (true);
          \draw[BddFalse] (c) -- (false);
        \end{tikzpicture}
      }
      & \raisebox{23mm}{$\bigwedge$} &
      \raisebox{8mm}{
        \begin{tikzpicture}
          \node (b) [BddNode,color=blue] {$\textsc{rbx}_2 = 72$};
          \node (true) [BddLeaf, below = 2 of b] {\true};
          \node (false) [BddLeaf, right = of true] {\false};
          \path (node cs:name=false) ++(0,1.3) node (alpha) [BddNode] {$\controlEdge{1}{C}{D_0}$};
          \draw[BddTrue] (b) -- (true);
          \draw[BddFalse] (b) -- (alpha);
          \draw[BddTrue] (alpha) -- (true);
          \draw[BddFalse] (alpha) -- (false);
        \end{tikzpicture}
      }\\
      Reordered BDD & \multicolumn{3}{c}{Condition evaluable at M} & Residual condition
    \end{tabular}
  }
  \caption{Factorisation of the BDD in
    \autoref{fig:place_conditions_example:bdd1} at message edge M.}
  \label{fig:place_conditions_example:message_m}
\end{figure}

Once the algorithm has decided on the side condition for message M it
can move on to calculating the side-conditions for M's successor \gls{cfg}
nodes, $Y$ and $C$.  In this case, the set of available input
expressions does not change at either of these nodes, and so this is
trivial, and likewise for $Y$'s successor $Z$.  The message N,
however, does allow further side-conditions to be evaluated, as shown
in \autoref{fig:place_conditions_example:message_n}.  All of the
input expressions are available when this message operation is
performed, and so the entire side-condition is evaluable and the
residual condition is just \true.  The entire side-condition has now
been placed on the control flow graph.  The results are shown in
\autoref{fig:place_conditions_example:result}.

\begin{figure}
  \centerline{
    \begin{tabular}{ccccc}
      \begin{tikzpicture}
        \node (b) [BddNode,color=blue] {$\textsc{rbx}_2 = 72$};
        \node (true) [BddLeaf, below = 2 of b] {\true};
        \node (false) [BddLeaf, right = of true] {\false};
        \path (node cs:name=false) ++(0,1.3) node (alpha) [BddNode, color=blue] {$\controlEdge{1}{C}{D_0}$};
        \draw[BddTrue] (b) -- (true);
        \draw[BddFalse] (b) -- (alpha);
        \draw[BddTrue] (alpha) -- (true);
        \draw[BddFalse] (alpha) -- (false);
      \end{tikzpicture}
      & \raisebox{15mm}{$=$} &
      \begin{tikzpicture}
        \node (b) [BddNode,color=blue] {$\textsc{rbx}_2 = 72$};
        \node (true) [BddLeaf, below = 2 of b] {\true};
        \node (false) [BddLeaf, right = of true] {\false};
        \path (node cs:name=false) ++(0,1.3) node (alpha) [BddNode, color=blue] {$\controlEdge{1}{C}{D_0}$};
        \draw[BddTrue] (b) -- (true);
        \draw[BddFalse] (b) -- (alpha);
        \draw[BddTrue] (alpha) -- (true);
        \draw[BddFalse] (alpha) -- (false);
      \end{tikzpicture}
      & \raisebox{15mm}{$\bigwedge$} &
      \raisebox{13mm}{
        \begin{tikzpicture}
          \node (true) [BddLeaf] {\true};
        \end{tikzpicture}
      }\\
      Reordered BDD & \multicolumn{3}{c}{Condition evaluable at $A$} & Residual condition\\
    \end{tabular}
  }
  \caption{Factorisation of the BDD in
    \autoref{fig:place_conditions_example:bdd1} at message edge N.
    Every variable is now evaluable, and so the residual condition
    is just \true.}
  \label{fig:place_conditions_example:message_n}
\end{figure}

\begin{figure}
  \begin{tikzpicture}
    \node (t1) {Thread 1};
    \node[right = 4.5cm of t1] (t2) {Thread 2};
    \node[below = of t1] (A) {$A$};
    \node[left = 0 of A] {$\textsc{rax}_1 = 93$};
    \node[below = of A] (B) {$B$};
    \node[below = of B] (C) {$C$};
    \node[below = of C] (D) {$D_0$};
    \node[left = 4cm of D] (D1) {$D_1$};
    \node[above = 0 of D1] {\shortstack{$\controlEdge{1}{C}{D_0}$\\$\textsc{rbx}_2 = 72 \vee$}};
    \node at (t2 |- A) (X) {$X$};
    \node at (t2 |- B) (Y) {$Y$};
    \node at (t2 |- D) (Z) {$Z$};
    \draw[->] (A) -- (B);
    \draw[->] (B) -- (C);
    \draw[->] (C) -- (D);
    \draw[->] (C) -- (D1);
    \draw[->] (X) -- (Y);
    \draw[->] (Y) -- (Z);
    \draw[->] (D1) to [bend right=55] (D);
    \draw[->,happensBeforeEdge] (B) to node [above] {M} node [below] {$\textsc{rcx}_1 = \textsc{rcx}_2$} (Y);
    \draw[->,happensBeforeEdge] (Z) to node [above] {N} node [below] {\shortstack{$\controlEdge{1}{C}{D_0}$\\$\textsc{rbx}_2 = 72 \vee$}} (D);
  \end{tikzpicture}
  \caption{Final placement of side-condition checks for the example
    control flow graph in \autoref{fig:place_conditions_example}}
  \label{fig:place_conditions_example:result}
\end{figure}

\section{Enforcing the plan}
\label{sect:enforce:interpreting}

Previous sections described how to find the desired happens-before
graph, how to turn that into a message-passing system, and how to
place the components of the side-condition on the control-flow graph.
Collectively, these form the crash enforcement plan.  It remains to
show how to actually use this plan to trigger the bug.

One way to think about this problem is to treat the plan as a program
which does the same thing as the original program, but with additional
operations which make it more likely that the bug will reproduce.
Consider, for example the bug shown in
\autoref{fig:enforcement:example_bug}, and its crash enforcement
plan shown in \autoref{fig:enforcement:example_bug:plan}.  The plan
can be regarded as a replacement for instructions {\tt a} and {\tt c}
in the original program.  The replacement for {\tt a} will behave like
this:

\begin{itemize}
\item Check that $\mathtt{loc1} \not= 7$.  If this condition fails,
  the replacement returns to the original version of instruction {\tt
    a} and the program can then resume normal operation.
\item Emulate the program's instruction {\tt a}.  In this case, that
  means loading from {\tt loc1} and storing the result in local
  variable {\tt x}.
\item Send a message to the other thread, so as to implement the first
  happens-before edge.  Message operations are synchronous in
  {\implementation}, and so this means waiting for some other thread
  to arrive at the other side of the message passing operation.  If no
  other thread arrives within a timeout then the operation fails, so
  the whole plan fails and the thread resumes normal operation.
\item If the thread manages to send its first message, it must
  immediately wait to receive its second one.  This time, the message
  operation is the last step of the plan for this thread, and so
  regardless of whether it succeeds or fails the thread will resume
  normal operation at instruction {\tt b}.
\end{itemize}

The replacement for {\tt c} is similar:

\begin{itemize}
\item The replacement starts by receiving the first message.  This
  will, again, involve waiting for another thread to arrive, and might
  fail.
\item Assuming the message operation succeeds, the replacement will
  then move on to emulating the original program instruction {\tt c}
  by setting {\tt loc1} to 7.
\item The thread can then complete its part of the enforcement plan by
  sending the second message.  Once again, this will involve waiting
  for the other thread to reach its side of the message operation.
\item The thread can then resume normal operation at whatever
  instruction comes after {\tt c}.
\end{itemize}
\begin{figure}
  {\hfill}
  \begin{subfloat}
    \parbox{4cm}{
      \begin{tabular}{ll}
        {\tt a:} & {\tt x = loc1;}\\
        {\tt b:} & {\tt y = loc1;}\\
        & {\tt assert(x == y);}
      \end{tabular}
    }
    \caption{Crashing thread}
  \end{subfloat}
  {\hfill}
  \begin{subfloat}
    \parbox{3.2cm}{\tt c: loc1 = 7;}
    \vspace{24pt}
    \caption{Interfering thread}
  \end{subfloat}
  {\hfill}
  \begin{subfloat}
    $\happensBefore{a}{c} \wedge \happensBefore{c}{b} \wedge \mathtt{loc1} \not= 7$
    \caption{Generated verification condition}
  \end{subfloat}
  {\hfill}
  \caption{An example bug\todo{SMH says: fix vertical alignment}}
  \label{fig:enforcement:example_bug}
\end{figure}
\begin{figure}
  \begin{tikzpicture}
    \node (n0m) {Crashing thread};
    \node (n00) [CfgInstr, below = of n0m] {\shortstack{Gain control at\\instruction {\tt a}}};
    \node (n01) [CfgInstr, below = of n00] {Require $loc1 \not= 7$};
    \node (nm1) [CfgInstr, left = of n01] {\shortstack{Return to\\instruction {\tt a}}};
    \node (n02) [CfgInstr, below = of n01] {\tt a: x = loc1};
    \node (n03) [CfgInstr, below = of n02] {Send message 1};
    \node (n04) [below = of n03] {};
    \node (n05) [below = of n04] {};
    \node (n06) [below = of n05] {};
    \node (n07) [below = of n06, CfgInstr] {Receive message 2};
    \node (n08) [below = of n07, CfgInstr] {\shortstack{Return to\\instruction {\tt b}}};
    \node (n1m) [right = of n0m] {Interfering thread};
    \node (n10) [CfgInstr, right = of n00] {\shortstack{Gain control at\\instruction {\tt b}}};
    \node (n11) [below = of n10] {};
    \node (n12) [below = of n11] {};
    \node (n13) [below = of n12] {};
    \node (n14) [below = 1.35 of n13, CfgInstr] {Receive message 1};
    \node (n24) [right = of n14, CfgInstr] {\shortstack{Return to\\instruction {\tt c}}};
    \node (n15) [below = of n14, CfgInstr] {\tt c: loc1 = 7};
    \node (n16) [below = of n15, CfgInstr] {Send message 2};
    \node (n17) [below = of n16] {};
    \node (n18) [below = of n17, CfgInstr] {\shortstack{Return to instruction\\after {\tt c}}};

    \draw[->] (n00) -- (n01);
    \draw[->] (n01) -- (n02);
    \draw[->] (n02) -- (n03);
    \draw[->] (n03) -- (n07);
    \draw[->] (n07) -- (n08);
    \draw[->] (n10) -- (n14);
    \draw[->] (n14) -- (n15);
    \draw[->] (n15) -- (n16);
    \draw[->] (n16) -- (n18);

    \draw[->,ifFalse] (n01) -- (nm1);
    \draw[->,ifFalse] (n03.west) to [bend right = 45] (n08.west);
    \draw[->,ifFalse] (n07.west) to [bend right = 45] (n08.west);
    \draw[->,ifFalse] (n14) -- (n24);
    \draw[->,ifFalse] (n16.east) to [bend left = 45] (n18);

    \draw[happensBeforeEdge,->] (n03) -- (n14);
    \draw[happensBeforeEdge,->] (n16) -- (n07);
  \end{tikzpicture}
  \caption{Crash enforcement plan for the bug in
    \autoref{fig:enforcement:example_bug}.  Dashed lines indicate
    message passing operations and dotted ones indicate error paths.}
  \label{fig:enforcement:example_bug:plan}
\end{figure}
At the highest level of abstraction, this is precisely what
{\implementation}'s enforcers will do if asked to enforce that crash
enforcement plan.  There are, however, two important complications.
Most obviously, the enforcer must somehow identify the crashing and
interfering threads.  When building the plan, {\technique} assumes
that there are precisely two threads in the program, and that those
threads are easily identified.  The enforcers operate on a real
program, which might contain an arbitrary number of threads, and must
somehow select which of these concrete threads correspond to which of
the two abstract threads in the plan.  Less obviously, the
happens-before edges in the plan are expressed between nodes in the
dynamic \gls{cfg} (see \autoref{sect:derive:handling_loops}), and do
not necessarily correspond closely to instructions in the actual
program.  The plan might require different actions at different
\gls{cfg} nodes which are hard to distinguish in a running program.

The solution to these problems is the same: to decide as much as
possible lazily.  When {\implementation} encounters a branch in the
dynamic \gls{cfg} which does not correspond to one in the static
\gls{cfg}, it does not try to figure out which way it should go
immediately, but instead defers the decision for as long as possible
by forking its state, with one fork taking one branch of the dynamic
\gls{cfg} and the other taking the other, and maintains both forks
until it becomes clear which fork was correct.  Similarly, if it has a
choice of synchronising between several possible threads, it creates
sufficient forks of its internal state to represent synchronising with
each of them, and then decides later on which fork is
correct\footnote{Note that these forks are entirely internal to the
  enforcer itself, rather than, for instance, being implemented with
  the \texttt{fork} system call.}.

The result is conceptually similar to the power set construction used
to simulate non-deterministic finite state automata using
deterministic ones\needCite{}.  A non-deterministic FSA can contain
branches which are ambiguous at the point where they occur but can be
resolved by looking ahead through the automaton to discover which ones
will eventually succeed.  These semantics can be implemented in a
deterministic FSA, where every branch must be immediately unambiguous,
using the power set construction.  The idea here is for each state in
the deterministic FSA to represent a set of states in the
non-deterministic one, so that any ambiguity in the original automaton
can be resolved by moving to every possible successor state at the
same time.  If any of those successors eventually succeed then the
whole automaton succeeds.  There are, in effect, two models of
computation involved here: a low level one, which allows this kind of
non-deterministic choice, and a high level one, which does not.  The
states of the high level model are formed from sets of states of the
low level one.

{\Implementation}'s plan interpreter is structured in a similar way.
The low level interpreter implements the obvious plan semantics
discussed in the example above, using an operator similar to
non-deterministic choice when it does not have enough information to
immediately implement an operation, and the high level interpreter
then implements the choice operator by maintaining sets of low level
interpreters.  Following the analogy, I start by describing what the
low level interpreter does, using non-deterministic choice where
convenient, and then explain how the high level interpreter implements
those choices.

\subsection{The low-level interpreter}
\label{sect:enforce:llis}

The low-level interpreter, or LLI, runs some simple stages in a tight
loop:

\begin{itemize}
\item \textbf{Sample} Look at the thread's registers, determine which
  ones might be needed to evaluate later side-conditions, and copy the
  appropriate registers to the LLI's local state.

\item \textbf{RX} Receive any messages required by the plan.  Message
  operations are discussed below.

\item \textbf{Emul} Emulate the instruction in the original program.
  Implementing this is moderately subtle as it is has effects which
  are visible to the original program (beyond adjusting the timing).
  This means that the enforcer must be careful to emulate instructions
  the right number of times, regardless of how many LLIs happen to be
  active at the time.  {\Technique}'s design ensures that every LLI in
  a program thread will reach the \textbf{Emul} state at the same
  time, and will be executing the same program
  instruction\footnote{They might, however, be executing at different
    points in the dynamic \gls{cfg}, as a single program instruction
    can correspond to multiple \gls{cfg} nodes.}, and so this is
  straightforward to arrange.

\item \textbf{TX} Send any messages required by the plan.

\item \textbf{Succ} Determine which node in the dynamic \gls{cfg} the
  thread is going to execute next.  In the low-level interpreter, this
  is very easy: compare the instruction pointer of the next
  instruction (generated by the \textbf{Emul} phase) to the
  instruction pointers of all of the potential successor nodes in the
  dynamic \gls{cfg} and select one using a non-deterministic choice.
  If there are no possible successors then the plan has failed and the
  low-level interpreter exits.
\end{itemize}

A new low-level interpreter is started whenever any thread in the
program reaches the starting point of any of the \glspl{cfg} involved
in one of the happens-before graphs which are to be enforced.

The LLI message operation is illustrated in
\autoref{fig:enforce:lli_message}.  The operation starts by selecting
a remote LLI with which to communicate; the mechanism for doing so is
described in \autoref{sect:enforce:hli_messages}.  The LLI then waits
for its remote LLI to arrive.  The two LLIs are then merged together
to perform the message operation itself, by evaluating the side
condition and copying state between the two LLIs, after which they are
unmerged and continue at the appropriate place in the LLI loop.

\begin{figure}
  \centerline{
  \begin{tikzpicture}
    \node (n1m) {\textbf{RX}};
    \node (n10) [below = of n1m] {$t \leftarrow$ select remote LLI};
    \node (n11) [below = of n10] {Wait for t};
    \node (n01) [left = of n11] {fail};
    \node (n22) [below right = of n11] {Merge LLIs};
    \node (n31) [above right = of n22] {Wait for $t'$};
    \node (n41) [right = of n31] {fail};
    \node (n30) [above = of n31] {$t' \leftarrow$ select remote LLI};
    \node (n3m) [above = of n30] {\textbf{TX}};
    \node (n23) [below = of n22] {Evaluate side condition};
    \node (n43) at (n23 -| n41) {fail};
    \node (n24) [below = of n23] {Copy state between LLIs};
    \node (n24b) [below = of n24] {Unmerge LLIs};
    \node (dummy) [below = of n24b] {};
    \node (n15) at (dummy -| n11) {Advance to \textbf{Emul}};
    \node (n35) at (dummy -| n31) {Advance to \textbf{Succ}};
    \draw[->] (n1m) -- (n10);
    \draw[->] (n10) -- (n11);
    \draw[->] (n11) -- (n22);
    \draw[->] (n3m) -- (n30);
    \draw[->] (n30) -- (n31);
    \draw[->] (n31) -- (n22);
    \draw[->] (n22) -- (n23);
    \draw[->] (n23) -- (n24);
    \draw[->] (n24) -- (n24b);
    \draw[->] (n24b) -- (n15);
    \draw[->] (n24b) -- (n35);
    \draw[->,ifFalse] (n11) -- (n01);
    \draw[->,ifFalse] (n31) -- (n41);
    \draw[->,ifFalse] (n23) -- (n43);
    \draw[->,style={dashed}] (n11) -- (n15);
    \draw[->,style={dashed}] (n31) -- (n35);
  \end{tikzpicture}
  }
  \caption{Message operation in the low-level interpreter.}
  \label{fig:enforce:lli_message}
\end{figure}

There are two ways for this algorithm to fail.  The simplest is that
the side-condition might evaluate to false, in which case both LLIs
fail and return the program to normal operation.  The timeouts are
more interesting.  There are several reasons why one of the wait steps
might time out:

\begin{itemize}
\item
  If this is the first message operation in the LLI, a timeout usually
  indicates that, for whatever reason, the program is not going to run
  the two desired fragments of code in parallel.  {\Technique}
  responds by having the LLI which timed out exit, allowing the
  program thread to return to normal operation.

\item
  If this is not the first message operation, a timeout usually
  indicates that the program has some existing synchronisation which
  will prevent the bug from reproducing.  The \textbf{Emul} step in
  the LLI cycle includes running any synchronisation operations which
  were present in the original program, which might cause there to be
  additional edges in the happens-before graph of which {\technique}
  is unaware.  If these additional edges complete a cycle then
  attempting to enforce the bug-causing ordering will lead to a
  deadlock, which is then detected as a timeout in a message
  operation.  When a timeout does happen, the LLI which suffered the
  timeout must, of course, exit.  Less obviously, the LLI with which
  it had previously communicated should also exit: any future message
  operations which that LLI attempts are doomed to fail, and so there
  is no point continuing to run it in the enforcer.
\end{itemize}


\subsection{Implementing the \textbf{Succ} phase in the high-level interpreter}
\label{sect:enforce:succ}

The low-level interpreter described in \autoref{sect:enforce:llis}
is both correct and conceptually simple, but impossible to implement
directly due to the use of non-deterministic choice.  This section
describes how to emulate it in a real system.  As discussed
previously, the basic approach is to switch from emulating the LLI
itself to emulating sets of LLIs, so that the choice operators can be
implemented by adding more LLIs to these sets and an LLI exit can be
implemented by removing one.  The resulting meta-interpreter is
referred to as the high-level interpreter, or HLI.

The simplest use of non-determinism is in the \textbf{Succ} phase,
which determines which dynamic \gls{cfg} node a given program thread
is going to execute next.  The equivalent operation in the HLI is to
convert the current set of LLIs into a new set of LLIs, where each of
the new LLIs is formed by taking one of the old LLIs and updating its
location in the dynamic \gls{cfg} in a way which is consistent with
the results of the \textbf{Emul} phase.  The new set of LLIs is given
by:
\begin{displaymath}
\{l' | l \in \mathit{llis}, (l.n, n') \in \mathit{CFG}, n'.\mathrm{RIP} = \mathbf{Emul}.\mathrm{RIP}, l' = l[n = n'] \}
\end{displaymath}
Where:
\begin{itemize}
\item $\mathit{llis}$ is the current set of LLIs, and $l$ is a member of it.
  $l'$ is a member of the new set of LLIs.
\item $n'$ is a node in the dynamic \gls{cfg}.
\item $l.n$ is LLI $l$'s current node in the dynamic \gls{cfg}.  $l[n =
  n']$ is a new LLI constructed by taking the LLI $l$ and setting its
  current \gls{cfg} node to $n'$.
\item $n'.\mathrm{RIP}$ is the raw instruction pointer associated with
  the \gls{cfg} node $n'$.  $\mathbf{Emul}.\mathrm{RIP}$ is the raw
  instruction pointer which was generated by the $\mathbf{Emul}$
  phase.
\item $(l.n, n') \in \mathit{CFG}$ is true precisely when $n'$ is one
  of the successors of $l.n$ in the dynamic \gls{cfg}.
\end{itemize}
In other words, the resulting set will contain one new LLI for every
combination of existing LLI and \gls{cfg} node, provided that the new
\gls{cfg} node is a successor of the current \gls{cfg} node and that
the new \gls{cfg} node's raw instruction pointer matches that produced
by the \textbf{Emul} phase.  Note that all of the successors will have
the same raw instruction pointer; this is necessary to correctly
implement the \textbf{Emul} phase of the next instruction cycle.

It is perhaps informative to consider what happens to the individual
LLIs in the input set.  There are three interesting cases:

\begin{enumerate}
\item The common case is that the \gls{cfg} node has precisely one
  successor node which matches the raw address of the next
  instruction, in which case the LLI has a single successor.  In
  effect, all that has happened is that the LLI has moved from its
  current node in the \gls{cfg} to one of that node's successors.

\item The LLI's \gls{cfg} node might not have any successors which
  match the address of the next instruction to be executed.  This
  could be because it has completed its part of the plan, in which
  case the program will crash soon, assuming that some other thread is
  able to complete the other part, or it could be because the program
  thread is no longer following the plan, in which case this attempt
  to reproduce the bug has failed.  In either case, the LLI will not
  generate any successors and has, in effect, exited; if it was the
  last LLI in this HLI then the HLI will also exit and the program
  thread will resume normal operation.

\item The \gls{cfg} node might have multiple matching successors.  The LLI
  will have multiple successors, one for each possible successor,
  allowing the enforcer to determine lazily which successor was
  correct.
\end{enumerate}

\subsection{Implementing message operations in the high-level interpreter}
\label{sect:enforce:hli_messages}

As discussed previously, the first step of a low-level interpreters
message operation is to select the remote LLI with which it will
communicate.  If it has performed any previous communication then this
is simple: it will always communicate with the same LLI; the two LLIs
have been \emph{bound} together.  The first message operation in an
LLI, which I refer to as its \emph{unbound} message operation, is,
however, more complex.  The low-level operation described above leaves
unspecified precisely how this selection is to be made, but one
reasonable approach would be for each message operation to define a
fixed interval in the program's execution and to allow communication
whenever these windows overlap.  This is illustrated in
\autoref{fig:enforce:message_windows}.  Suppose, for instance, that
the interval were set at 20ms.  A \textbf{TX} operation at time 512ms
would then be able to communicate with an \textbf{RX} operation at
520ms.  This would involve delaying the transmitting thread by 8ms so
that the two operations occur at the same time.  Similarly, the
\textbf{TX} operation could communicate with an \textbf{RX} at 498ms
by delaying the receiving thread by 14ms.  It would not, however, be
possible for the \textbf{TX} operation to communicate with \textbf{RX}
operations at 490ms or 540ms, because those fall outside of the
interval.

\begin{figure}
  {\hfill}
  \subfigure[][Intervals overlap; message is sent]{
    \begin{tikzpicture}
      \draw[->,decorate,decoration={snake,segment length=2mm, amplitude=1mm}]
        (0mm,0mm) -- (0,-5mm)
        (0mm,-8mm) node (TX) {\textbf{TX}} (0,-10mm) --
        (0mm,-40mm);
      \draw (-4mm,-8mm) -- (-6mm,-8mm) -- (-6mm,-23mm) -- (-4mm,-23mm);

      \draw[->,decorate,decoration={snake,segment length=2mm, amplitude=1mm}]
        (13mm,0mm) -- (13mm,-16mm)
        (13mm,-19mm) node (RX) {\textbf{RX}} (13mm,-23mm) --
        (13mm,-40mm);
      \draw (19mm,-19mm) -- (21mm,-19mm) -- (21mm,-34mm) -- (19mm,-34mm);
      \draw[->] (TX) -- (RX);
      \node [right = 0.7 of RX] {};
      \node [left = 0.7 of TX] {};
    \end{tikzpicture}
    \label{fig:enforce:message_windows:tx_first}
  }
  {\hfill}
  \subfigure[][Intervals overlap; message is sent]{
    \begin{tikzpicture}
      \draw[->,decorate,decoration={snake,segment length=2mm, amplitude=1mm}]
        (0mm,0mm) -- (0,-16mm)
        (0mm,-19mm) node (TX) {\textbf{TX}} (0,-23mm) --
        (0mm,-40mm);
      \draw (-4mm,-19mm) -- (-6mm,-19mm) -- (-6mm,-34mm) -- (-4mm,-34mm);

      \draw[->,decorate,decoration={snake,segment length=2mm, amplitude=1mm}]
        (13mm,0mm) -- (13mm,-5mm)
        (13mm,-8mm) node (RX) {\textbf{RX}} (13mm,-10mm) --
        (13mm,-40mm);
      \draw (19mm,-8mm) -- (21mm,-8mm) -- (21mm,-23mm) -- (19mm,-23mm);
      \draw[->] (TX) -- (RX);
      \node [right = 0.7 of RX] {};
      \node [left = 0.7 of TX] {};
    \end{tikzpicture}
    \label{fig:enforce:message_windows:rx_first}
  }
  {\hfill}
  \subfigure[][Intervals do not overlap; no message can be sent.]{
    \begin{tikzpicture}
      \draw[->,decorate,decoration={snake,segment length=2mm, amplitude=1mm}]
        (0mm,0mm) -- (0,-5mm)
        (0mm,-8mm) node (TX) {\textbf{TX}} (0,-10mm) --
        (0mm,-40mm);
      \draw (-4mm,-8mm) -- (-6mm,-8mm) -- (-6mm,-15mm) -- (-4mm,-15mm);

      \draw[->,decorate,decoration={snake,segment length=2mm, amplitude=1mm}]
        (13mm,0mm) -- (13mm,-25mm)
        (13mm,-28mm) node (RX) {\textbf{RX}} (13mm,-30mm) --
        (13mm,-40mm);
      \draw (19mm,-28mm) -- (21mm,-28mm) -- (21mm,-34mm) -- (19mm,-34mm);

      \draw[->,dashed] (TX) -- (RX);
      \node [right = 0.7 of RX] {};
      \node [left = 0.7 of TX] {};
    \end{tikzpicture}
    \label{fig:enforce:message_windows:failed}
  }
  {\hfill}
  \vspace{-12pt}
  \caption{Overlapping TX and RX intervals allow the message to be
    sent, regardless of which operation is ordered first.  If the
    intervals do not overlap then no message can be sent.  Square
    brackets show the intervals.}
  \label{fig:enforce:message_windows}
\end{figure}

There are two obstacles to implementing this behaviour: it is
difficult for a \textbf{TX} operation to look ahead through the
execution to determine how long it should wait for the \textbf{RX}
operation to arrive, or vice versa, and it might be that several
possible partner operations happen during the interval, and the LLI
must decide which to bind to.  In practice, it is not possible to
completely solve these problems and get precisely the desired
behaviour, but it is possible to approximate it reasonably accurately.
{\Implementation} uses the following algorithm to implement the peer
selection part of the \textbf{TX} algorithm; the \textbf{RX} algorithm
is symmetric:
\begin{itemize}
\item[1] Iterate over the global \texttt{rx\_thread} table, which
  contains LLIs which are currently performing an unbound \textbf{RX}.
  Compare the message which this LLI wants to transmit to the message
  which the other LLI wants to receive.  If they match, and if any
  side-condition on the message passes, the local LLI completes both
  message operations.  This means that it duplicates both LLIs, copies
  any necessary state between the two new LLIs, advances them past the
  message operation, and adds them to the appropriate HLIs.
\item[2] Register the current LLI in the global \texttt{tx\_thread}
  table.
\item[3] Sleep for the interval.
\item[4] De-register the current LLI from the \texttt{tx\_thread}
  table.
\item[5] The current LLI now corresponds to the situation where no
  other threads arrived during the interval, and so it has failed and
  should exit.
\end{itemize}
The intent is that when a new \textbf{TX} operation starts, it first
looks for any existing \textbf{RX} operations which are waiting for
it, corresponding to \autoref{fig:enforce:message_windows:rx_first}.
If it finds any, it completes both operations, creating new LLIs on
both the receive and transmit side to store its results.  It then goes
to sleep waiting for any further \textbf{RX} operations, corresponding
to \autoref{fig:enforce:message_windows:tx_first}.  If any \textbf{RX}
operations arrive during this window then they will complete both the
\textbf{TX} and \textbf{RX} operations, and so when the sleep
completes the current LLI has no more to do and should exit,
corresponding to
\autoref{fig:enforce:message_windows:failed}\footnote{{\Implementation}
  actually includes an optimisation to avoid creating and destroying
  LLIs unnecessarily in the common case where precisely one peer
  operation is available, but this does not fundamentally alter the
  algorithm and is not shown here.  Likewise, this discussion omits
  the enforcer-internal synchronisation necessary to implement this
  algorithm in a race-free way.}.

This algorithm solves the problem of not knowing how long to wait by
always waiting for the maximum possible time and the problem of not
knowing which of many possible peers to synchronise with by
synchronising with all of them.  That is similar but not quite
identical to the desired behaviour.  In particular, in the desired
semantics, when a message operation is destined to fail it should not
wait at all, whereas here failing operations wait for the maximum
possible time; in other words, the non-deterministic part of the
desired behaviour is correctly implemented, but the non-causal part is
not.  This can sometimes lead to threads delaying for longer than
necessary.  If it were necessary, the ``correct'' behaviour could be
achieved using a deterministic replay system\cite{Choi1998} and
techniques similar to a time-travelling debugger\cite{Xu2003}, but at
the probable expense of further impairing performance.  I have not
investigated this at all.

This algorithm assumes that it is possible to delay an arbitrary LLI
for as long as desired, whenever desired, with no adverse effects on
any other part of the program.  This is not entirely true.  It is
common for a single HLI, and hence a single concrete program thread,
to run multiple LLIs, and it is not possible to delay one LLI in a
thread without also delaying all of the others (as otherwise the
different LLIs would reach the \textbf{Emul} phase at different
times).  What effect will these spurious delays have on the program
and the enforcer?  Delaying an unbound LLI is generally safe, as it
just causes the enforcement plan to start a little later, which might
lead to poor performance but will not completely prevent the plan from
succeeding.  Delaying a bound LLI is not.  The problem is that
delaying a bound LLI will have knock-on effects on the LLI to which it
has bound, which might cause \emph{that} LLI to fail its part of the
plan, and in the extreme case might completely prevent certain bugs
from reproducing.  Suppose that the plan requires that LLI $l_1$ be
delayed but does not impose any delay on LLI $l_2$, which is in the
same HLI, and that $l_2$ is bound to the LLI $l_3$ in another HLI.
$l_3$ will eventually advance to a point which requires further
communication with $l_2$.  If $l_2$ is excessively delayed then this
message operation will time out, causing both $l_2$ and $l_3$ to fail.
If this happens consistently then it might make it impossible to
reproduce the bug.

Fortunately, this problem is easily averted by defining the bound
message timeout more carefully.  Rather than timing out after $l_3$
has been waiting for more than time $\tau$, the plan should fail if
$l_2$ takes more than $\tau$ to advance from one message operation to
the next, excluding the time it wastes due to $l_1$'s message
operations.  The HLI can track this information for each LLI, and
$l_3$ can adjust its timeout as appropriate.  This avoids the
issue\editorial{Or at least, it does the one case I've actually seen
  this, and I think it does in the general case as well.}.

The interactions with the \textbf{Succ} phase present a further
complication.  The \textbf{Succ} phase sometimes needs to duplicate an
LLI in order to handle ambiguities in the \gls{cfg}, and it must
ensure that the relationships between bound LLIs are correctly
maintained as it does so.  For instance, if $l_1$ and $l_2$ are bound
together and $l_1$ is duplicated to $l_1'$, the HLI should also
duplicate $l_2$ to $l_2'$ so that it can bind $l_1'$ and $l_2'$.  The
alternatives, of leaving $l_1'$ unbound or of binding both $l_1$ and
$l_1'$ to $l_2$, would not allow future message operations to be
correctly implemented in any of the LLIs.

\subsection{Selecting the size of the message interval}
\label{sect:using:timeout_balancing}

\todo{not sure this adds all that much.}

The discussion so far has assumed that all unbound message operations
use the same message interval, and hence the same timeout.  This is
not always the best possible policy.  Most obviously, it is quite
inefficient: the rate at which the bug reproduces is, under some
plausible assumptions about the structure of the program, proportional
to the sum of the delays at the sending and receiving ends of the
message operation, and so if one happens more frequently than the
other then the overhead of the enforcer can be reduced without
affecting its efficacy by moving some of the delay from the more
frequent operation to the less frequent one.  {\Implementation} takes
advantage of this by only delaying a thread at a \textbf{TX} operation
if that operation has run fewer times, in this run of the program,
than the matching \textbf{RX} operation, and vice versa.

\section{Implementing the $\smLoad{}$ operation}

The side-condition placement algorithm\editorial{Or whatever.} removes
all of the happens-before $\happensBeforeEdge$ queries from the
side-condition, and most of the remaining parts of the {\StateMachine}
expression language are simple to implement.  The only exception is
the $\smLoad{}$ expression, which is defined to return the initial
contents of memory at a given address.  This is not entirely
well-defined in a running program with multiple threads.  Even worse,
the address itself might not become evaluable until one or both of
the threads involved in the crash have been running in the interpreter
for some time.  {\Implementation} only implements an approximation of
the desired behaviour: it captures the contents of memory at the point
the address becomes evaluable, rather than trying to capture any
kind of initial value.  In practice, this is usually sufficient,
because the address usually becomes evaluable to the interpreter at
the same time as it becomes evaluable to the program itself.  The
program thread will therefore not be able to access the memory before
the interpreter is able to capture its value, and so even when the
captured value differs from the initial value it will not usually
matter.

A second, less severe, problem is that the address, when it becomes
evaluable, might refer to invalid memory, even when the original
program is guaranteed to only ever access valid memory.  The problem
here is that the side-condition placement strategy can sometimes lift
memory accesses past control-flow branches, and if that control-flow
branch was a test of the validity of the pointer then the
side-condition might involve invalid pointers.  {\Implementation}
solves this problem by simply catching page faults generated while
evaluating side conditions.  If any such faults occur then the
relevant LLI is considered to have failed.


\section{Discussion}

\todo{Put some kind of closing and linking text in here.}
