\chapter{Reproducing bugs}
\label{sect:reproducing_bugs}

The previous chapter showed how to derive
\glspl{verificationcondition} which characterise the precise
conditions under which a program might suffer an \glslink{simple
  atomicity violation}{SAV}.  I now show how to use the
\glspl{verificationcondition} to build \glspl{bugenforcer} which
shepherd programs towards schedules in which these bugs reproduce
easily.  These run-time enforcers complement the local analysis of the
previous chapter, allowing {\technique} to incorporate global dynamic
properties of the program, completely eliminating the many false
positives generated by the {\StateMachine} analysis.

\begin{sanefig}
  \begin{tabular}{ll}
    \multicolumn{2}{c}{\texttt{int *global\_ptr[];~~~~~~~~~}}\\
    \hspace{-5mm}\subfigure[][Crashing thread source]{
      \texttt{
        \begin{tabular}{lll}
          \multicolumn{3}{l}{void crashing(int idx1) \{}\\
          &\multicolumn{2}{l}{if (global\_ptr[idx1])}\\
          &&*global\_ptr[idx1] = 7;\\
          \}\\
        \end{tabular}
      }
    } & \hspace{-5mm}%
    \subfigure[][Interfering thread source]{
      \texttt{
        \begin{tabular}{ll}
          \multicolumn{2}{l}{void interfering(int idx2) \{}\\
          &global\_ptr[idx2] = NULL;\\
          \}\\
          \\
        \end{tabular}
      }
    }\\
    \hspace{-5mm}\subfigure[][Crashing thread machine code]{
      \texttt{
        \begin{tabular}{rlll}
          & \multicolumn{3}{l}{crashing:} \\
          \rm A: & ADD & global\_ptr+idx1 & \!\!\!$\rightarrow$ reg1 \\
          \rm B: & LOAD & *reg1 & \!\!\!$\rightarrow$ reg2 \\
          \rm C: & CMP & 0, reg2 \\
          \rm D: & JE & \rm G \\
          \rm E: & LOAD & *reg1 & \!\!\!$\rightarrow$ reg3 \\
          \rm F: & STORE & 7 & \!\!\!$\rightarrow$ *reg3 \\
          \rm G:
        \end{tabular}
      }
    } & \hspace{-5mm}%
    \subfigure[][Interfering thread machine code]{
      \texttt{
        \begin{tabular}{rlll}
          & \multicolumn{3}{l}{interfering:} \\
          \\
          \\
          \rm H: & ADD & global\_ptr+idx2 & \!\!\!$\rightarrow$\!\! reg4 \\
          \rm J: & STORE & 0 & \!\!\!$\rightarrow$\!\! *reg4 \\
          \\
          \\
          \\
        \end{tabular}
      }
    }\\
  \end{tabular}
  \caption{Threads which, when run in parallel, might exhibit an
    atomicity violation.}
  \label{fig:enforce:example_threads}
\end{sanefig}

\begin{sanefig}
  {\hfill}
  \begin{tikzpicture}
    \node[CfgInstr] (l1) {$A_0$: \tt ADD global\_ptr + idx1 $\rightarrow$ reg1};
    \node[CfgInstr, below=of l1] (l2) {$B_0$: \tt LOAD *reg1 $\rightarrow$ reg2};
    \node[CfgInstr, below=of l2] (l3) {$C_0$: \tt CMP 0, reg2};
    \node[CfgInstr, below=of l3] (l4) {$D_0$: \tt JMP\_IF\_EQ };
    \node[CfgInstr, below=of l4] (l5) {$E_0$: \tt LOAD *reg1 $\rightarrow$ reg3 };
    \node[CfgInstr, below=of l5] (l6) {$F_0$: \tt STORE 7 $\rightarrow$ *reg3 };
    \node[CfgInstr, below=of l6] (l7) {};
    \node[CfgInstr] at (7.5,-.8) (l8) {$H_0$: \tt ADD global\_ptr + idx2 $\rightarrow$ reg4};
    \node[CfgInstr, below=26.5mm of l8] (l9) {$J_0$: \tt STORE 0 $\rightarrow$ *reg4};
    \draw[->] (l1) -- (l2);
    \draw[->] (l2) -- (l3);
    \draw[->] (l3) -- (l4);
    \draw[->,ifFalse] (l4) -- (l5);
    \draw[->,ifTrue] (l4.west)
      -- ++(-0.7,0)
      .. controls ++(-.3,0) and ++(0,.3) .. ++(-.5,-.5)
      -- ++(0,-3.97)
      .. controls ++(0,-.3) and ++(-.3,0) .. ++(.5,-.5)
      -- (l7.west);
    \draw[->] (l5) -- (l6);
    \draw[->] (l6) -- (l7);
    \draw[->] (l8) -- (l9);
    \draw[->,happensBeforeEdge] (l2) to node [pos=.65, above=3.5mm] {$\texttt{idx1} = \texttt{idx2}$} (l9);
    \draw[->,happensBeforeEdge] (l9) -- (l5);
  \end{tikzpicture}
  {\hfill}
  \caption{The \glsentrytext{verificationcondition} generated from
    the program in \autoref{fig:enforce:example_threads}, showing the
    \glsentrytext{dynamic cfg} (as solid and dotted lines), the
    happens-before graph (as dashed lines) and the \glsentrytext{side
      condition} ($\texttt{idx1} = \texttt{idx2}$).}
  \label{fig:using:example_hb_graph}
\end{sanefig}

Consider, for example, the threads shown in
\autoref{fig:enforce:example_threads}.  There is a risk here that the
crashing thread might crash if instruction J is interleaved between
instructions B and E and \texttt{idx1} and \texttt{idx2} coincide.
The previous analysis phase will detect this bug and produce the
verification condition $\texttt{idx1} = \texttt{idx2} \wedge
\happensBefore{B_0}{J_0} \wedge \happensBefore{J_0}{E_0}$, where
$B_0$, $E_0$, and $J_0$ are dynamic instructions corresponding to the
static instructions B, E, and J, respectively, illustrated by the
augmented happens-before graph in
\autoref{fig:using:example_hb_graph}.  In order for the bug to
reproduce, the program must follow the happens-before graph shown as
dashed lines whilst the \gls{side condition} $\texttt{idx1} =
\texttt{idx2}$ holds; imposing the correct happens-before graph when
the \gls{side condition} is false will not reproduce the bug, and
neither will arranging for the \gls{side condition} to be true if the
schedule does not match the happens-before graph.

As discussed in \autoref{sect:intro:theory_of_fixing}, {\technique}
only modifies the behaviour of running programs in ways which are
``safe'', defined to be equivalent to inserting delays into its
execution.  As such, it can easily impose the desired happens-before
graph but cannot directly influence the likelihood of the \gls{side
  condition} holding.  It might, therefore, appear that knowing the
\gls{side condition} is unhelpful.  This is not the case.  Imposing a
happens-before graph implies slowing down one or more of the program's
threads, which means that the program will run more slowly and that
the buggy code will run less frequently; if the happens-before graph
reproduces easily but the \gls{side condition} passes rarely then this
could actually \emph{increase} the time needed to reproduce the bug.
Careful use of the \gls{side condition} can reduce the number of
delays which the \glspl{bugenforcer} must insert into the program,
largely avoiding this problem.

\section{Outline of algorithm}

At a high level, the algorithm used has the following phases:
\begin{itemize}
\item
  Extract the happens-before graphs and \glspl{side condition} from
  the \gls{verificationcondition}.  This is discussed in
  \autoref{sect:enforce:slice_hb_graph}.
\item
  Plan how to evaluate the \gls{side condition} part of the verification
  condition.  This is discussed in \autoref{sect:enforce:place_vcs}.
\item
  Decide on a strategy for gaining control of the program at
  appropriate points.  The mechanism which {\technique} uses to do so
  is described in \autoref{sect:enforce:gain_control}.
\item
  Combine the results of the previous phases with the \gls{plan
    interpreter}, discussed in \autoref{sect:enforce:interpreting},
  and compile the result into the final \gls{bugenforcer}.
\end{itemize}
The resulting \gls{bugenforcer} can be loaded into the running program
and the two exercised together, hopefully reproducing the bug under
investigation with relatively little manual effort required.

\section{Extracting the happens-before graph}
\label{sect:enforce:slice_hb_graph}

\begin{wrapfigure}{o}{5.2cm}
  \vspace{-12pt}
  \begin{figgure}
  \begin{tikzpicture}
    \node (thread1) {\shortstack{Crashing\\thread}};
    \node (thread2) [right = 0.5 of thread1] {\shortstack{Interfering\\thread}};
    \node (A) [below = 0 of thread2] {$A_1$};
    \node (B) [below = of A] {$B_1$};
    \node (C) [below = of B] {$C_1$};
    \node (D0) [below = 0.7 of thread1] {$x = 7$ $D_0$};
    \node (D1) [below = of D0] {$y = 9$ $D_0$};
    \draw[->] (A) -- (B);
    \draw[->] (B) -- (C);
    \draw[->,happensBeforeEdge] (A) -- (D0);
    \draw[->,happensBeforeEdge] (D0) -- (B);
    \draw[->,happensBeforeEdge] (B) -- (D1);
    \draw[->,happensBeforeEdge] (D1) -- (C);
  \end{tikzpicture}
  \caption{}
  \label{fig:reproducing:placement_eg_bug}
  \end{figgure}
  \vspace{-12pt}
\end{wrapfigure}
\noindent The first step in building a \gls{bugenforcer} is to extract
the happens-before graphs and \glspl{side condition} from the
\gls{verificationcondition}, so that the \gls{bugenforcer} knows what
instruction ordering it needs to enforce and when it should do so.
Consider, for instance, the graph shown in
\autoref{fig:reproducing:placement_eg_bug}.  This is intended to
indicate that the \gls{crashingthread} has a single instruction,
$D_0$, that the \gls{interferingthread} has three instructions, $A_1$,
$B_1$, and $C_1$, and that the bug reproduces if either $x = 7$ and
$D_0$ happens between $A_1$ and $B_1$ or if $y = 9$ and $D_0$ happens
between $B_1$ and $C_1$.

This bug will produce a \gls{verificationcondition}

\noindent
{\hfill}
\begin{tabular}{lll}
$z$ = & $\happensBefore{A_1}{D_0} \wedge \happensBefore{D_0}{C_1} \wedge (\!\!\!\!$ & $(x = 7 \wedge \happensBefore{D_0}{B_1})\,\vee$ \\
      &                                                                            & $(y = 9 \wedge \happensBefore{B_1}{D_0}))$
\end{tabular}
{\hfill}

\noindent
which must be factorised into a set of happens-before graphs $h_i$,
expressed as a conjunction of $\happensBeforeEdge$ tests, each paired
with a $\happensBeforeEdge$-free \gls{side condition} $c_i$ such that
$\bigvee_i(h_i \wedge c_i) = z$.  {\Technique} does so by
constructing a \gls{bdd} representation of the
\gls{verificationcondition} in which all $\happensBeforeEdge$ tests
happen before any non-$\happensBeforeEdge$ ones and then examining the
structure of the graph.  Suppose, for example, that the
\gls{verificationcondition} were represented by the \gls{bdd} shown in
\autoref{fig:reproducing:slice_eg:bdd}.  Every path through the
initial $\happensBeforeEdge$-fragment of this diagram corresponds to a
happens-before graph and the remainder of the diagram provides the
\glspl{side condition}.  In this case, there are four such paths, and so four
happens-before graphs and four \glspl{side condition}:

\noindent
\centerline{
  \newcommand{\trueRightArrow}{\!\!\raisebox{.6ex}{\tikz{\draw[->,ifTrue] (0,0) -- (0.4,0);}}}
  \newcommand{\falseRightArrow}{\!\!\raisebox{.6ex}{\tikz{\draw[->,ifFalse] (0,0) -- (0.4,0);}}}
  \begin{tabular}{l @{~~~~~~~~~~~~} l @{~~~~~~~~~~~~~~} l @{~~~~~~~~~~~~~} r}
    Path                                                          & $h_i$                                                     & $c_i$   \\
    $\happensBefore{D_0}{B_1}$ \trueRightArrow $\happensBefore{A_1}{D_0}$ \trueRightArrow & $\happensBefore{D_0}{B_1} \wedge \happensBefore{A_1}{D_0}$ & $x = 7$  & \circled{1}\\
    $\happensBefore{D_0}{B_1}$ \trueRightArrow $\happensBefore{A_1}{D_0}$ \falseRightArrow & $\happensBefore{D_0}{B_1} \wedge \happensBefore{D_0}{A_1}$ & $\false$ & \circled{2}\\
    $\happensBefore{D_0}{B_1}$ \falseRightArrow $\happensBefore{D_0}{C_1}$ \trueRightArrow & $\happensBefore{B_1}{D_0} \wedge \happensBefore{D_0}{C_1}$ & $y = 9$  & \circled{3}\\
    $\happensBefore{D_0}{B_1}$ \falseRightArrow $\happensBefore{D_0}{C_1}$ \falseRightArrow & $\happensBefore{B_1}{D_0} \wedge \happensBefore{C_1}{D_0}$ & $\false$ & \circled{4}\\
  \end{tabular}
}

\noindent
In this case, two of the paths, \circled{2} and \circled{4}, produce
\glspl{side condition} of {\false} and are immediately discarded.  The
\glspl{bugenforcer} will attempt to enforce both of the other two.

\begin{sanefig}
  \centerline{
    \begin{tikzpicture}[node distance = 1.0 and 0]
      \node (n10) [BddNode] {$\happensBefore{D_0}{B_1}$};
      \node (n01) [BddNode, below left = of n10] {$\happensBefore{A_1}{D_0}$};
      \node (n21) [BddNode, below right = of n10] {$\happensBefore{D_0}{C_1}$};
      \node (n02) [BddNode, below = of n01] {$x = 7$};
      \node (n22) [BddNode, below = of n21] {$y = 9$};
      \node (true) [BddLeaf, below = of n02] {\true};
      \node (false) [BddLeaf, below = of n22] {\false};
      \draw[BddTrue] (n10) -- (n01);
      \draw[BddFalse] (n10) -- (n21);
      \draw[BddTrue] (n01) -- (n02);
      \draw[BddFalse] (n01) -- (false);
      \draw[BddTrue] (n02) -- (true);
      \draw[BddFalse] (n02) -- (false);
      \draw[BddTrue] (n21) -- (n22);
      \draw[BddFalse] (n21.east) .. controls +(.3,0) and +(0,.3) .. ++(.5,-.5) -- ++(0,-2.3) ..controls +(0,-.3) and +(.3,0) .. ++(-.5,-.5) -- (false.east);
      \draw[BddTrue] (n22) -- (true);
      \draw[BddFalse] (n22) -- (false);
    \end{tikzpicture}
  }
  \caption{\protect\Gls{bdd} representation of the
    \protect\gls{verificationcondition}.  Solid lines show the path
    taken through the \protect\gls{bdd} when the node conditions are
    true and dotted lines the path when the conditions are false.}
  \label{fig:reproducing:slice_eg:bdd}
\end{sanefig}

This can sometimes be a very expensive approach.  In principle, a
single \gls{verificationcondition} with $n$ $\happensBeforeEdge$ tests
could produce up to $2^n$ possible happens-before graphs, each with
its own \gls{side condition}, and this can become completely
unmanageable for even quite small values of $n$.  In practice, though,
the number produced is usually far more modest.  Most bugs do not
require a particularly complicated mapping from happens-before graphs
to \glspl{side condition}, in the sense that they will usually only
have a small number of distinct \glspl{side condition} and will only
need to test a small number of happens-before edges to determine which
condition to use in a particular execution.  Provided
{\technique} can identify an appropriate order in which to test
those edges, and hence a suitable ordering for the \gls{bdd}
variables, there is no need to enumerate all $2^n$ possible paths, and
the worst case is avoided.  As I show in the evaluation, it can do so
for slightly over 99\% of \glspl{verificationcondition} generated from
real programs, and so this algorithm is quite usable in spite of its
unfortunate theoretical worst case.\kern-.9pt\fnote{There is another,
  less optimistic, reason why this usually works: {\technique}
  makes extensive use of \protect\glspl{bdd} during symbolic
  execution, and so if the \protect\gls{verificationcondition} were
  one which is hard to represent as a \protect\gls{bdd} then one of
  the earlier symbolic execution steps would have failed and it would
  not have been possible to even start building the
  \protect\gls{bugenforcer}.}

\subsection{Unenforceable graphs}
\label{sect:reproduce:unenforcable}

\begin{sanefig}
  \centerline{
    {\hfill}
  \subfigure[][An unenforceable control-flow and happens-before graph.]{
    \begin{tikzpicture}
      \node (thread0) {\shortstack{Crashing\\thread}};
      \node[CfgInstr, below = 1.3 of thread0] (A) {$A_0$};
      \node[CfgInstr, below = of A] (B) {$B_0$};
      \node[right = 2 of thread0] (thread1) {\shortstack{Interfering\\thread}};
      \node[CfgInstr, below = 1.3 of thread1] (C) {$C_1$};
      \node[CfgInstr, below = of C] (D) {$D_1$};
      \node[below = 0.5 of D] {};
      \draw[->] (A) -- (B);
      \draw[->] (C) -- (D);
      \draw[->,happensBeforeEdge] (A) to node [above,pos=0.25] {$\alpha$} (D);
      \draw[->,happensBeforeEdge] (C) to node [above,pos=0.25] {$\beta$} (B);
    \end{tikzpicture}
    \label{fig:reproduce:cyclic_synchronous}
  }
    {\hfill}
  \subfigure[][The graph which is actually enforced; the double arrow indicates
    that $C_1$ must happen \emph{immediately} before $B_0$, rather than any time
    before it.]{
    \begin{tikzpicture}
      \node (thread0) {\shortstack{Crashing\\thread}};
      \node[CfgInstr, below = .5 of thread0] (A) {$A_0$};
      \node[CfgInstr, below = of A] (B) {$B_0$};
      \node[right = 2 of thread0] (thread1) {\shortstack{Interfering\\thread}};
      \node[CfgInstr] at (thread1 |- B) (C) {$C_1$};
      \node[CfgInstr, below = of C] (D) {$D_1$};
      \draw[->] (A) -- (B);
      \draw[->] (C) -- (D);
      \draw[->,double] (C) -- (B);
    \end{tikzpicture}
    \label{fig:reproduce:cyclic_decyclic}
  }
    {\hfill}
  }
  \vspace{-12pt}
  \caption{}
  \label{fig:reproduce:cyclic1}
\end{sanefig}

\noindent
{\Technique} enforces happens-before edges synchronously, in the sense
that to enforce $\happensBefore{A_i}{B_j}$ it enforces that $A_i$
happens \emph{immediately} before $B_j$.  This can sometimes restrict
the class of bugs which can be reproduced.  Consider, for instance,
the combined control-flow and happens-before graph in
\autoref{fig:reproduce:cyclic_synchronous}, which shows that both
threads must complete their first instruction before either thread
starts its second.  It is clearly possible for the program to execute
its instructions in this order, but this ordering cannot be enforced
synchronously: in order to enforce edge $\alpha$, $A_0$ must happen
immediately before $D_1$, and so $A_0$ must wait for $D_1$, but in
order to enforce edge $\beta$ $C_1$ must wait for $B_0$.  This creates
a cycle in the graph, and so this program would deadlock if the
enforcer attempted to enforce the entire graph.

{\Technique} resolves these cycles by discarding one of the
happens-before edges, chosen arbitrarily.  In this particular case,
that is sufficient to avoid the problem.  Suppose that {\technique}
selects edge $\alpha$ to discard (the other case is symmetrical).  The
enforcer will now ensure that $C_1$ waits for $B_0$ and then executes
immediately before it, which is also sufficient to ensure that $D_1$
executes after $A_0$, as shown in
\autoref{fig:reproduce:cyclic_decyclic}.  There is, however, no
guarantee that this approach will always work for more complex graphs.

\section{Placing \glsentrytext{side condition}s}
\label{sect:enforce:place_vcs}

Once the \glspl{side condition} have been extracted from the
\gls{verificationcondition}, {\technique} must decide how and when to
evaluate them.  Ideally, \glspl{side condition} would be evaluated as
soon as the enforcer gains control of the program's execution, so as
to avoid unnecessarily perturbing executions which cannot lead to
reproductions of the bug, but this is not always possible.  Most
\glspl{side condition} will involve some cross-thread state, and so
cannot be fully evaluated until the enforcer has determined how
program threads map onto the {\StateMachine}'s crashing and
interfering threads.  Even in the absence of cross-thread constraints,
expressions such as ${\controlEdgeName}()$ which examine a thread's
path through its \gls{dynamic cfg} often cannot be evaluated until the
thread reaches the relevant \gls{cfg} node, which might be some time
after the enforcer gains control of it.  The \gls{bugenforcer} must
therefore often defer evaluating part of the \gls{side condition}
until all of the necessary inputs are available.

\newcommand{\crashing}{\mathrm{C}}
\newcommand{\interfering}{\mathrm{I}}
\begin{sanefig}
  \hspace{6mm}
  \begin{tikzpicture}
    \node (t1) {\shortstack{Crashing\\thread}};
    \node[right = 4.5cm of t1] (t2) {\shortstack{Interfering\\thread}};
    \node[below = .5 of t1] (A) {$A_0$};
    \node[below = of A] (B) {$B_0$};
    \node[below = of B] (C) {$C_0$};
    \node[below = of C] (D) {$D_0$};
    \node[left = 4cm of D] (D1) {$D_1$};
    \node at (t2 |- A) (X) {$X_0$};
    \node at (t2 |- B) (Y) {$Y_0$};
    \node at (t2 |- D) (Z) {$Z_0$};
    \draw[->] (A) -- (B);
    \draw[->] (B) -- (C);
    \draw[->] (C) -- (D);
    \draw[->] (C) -- (D1);
    \draw[->] (X) -- (Y);
    \draw[->] (Y) -- (Z);
    \draw[->] (D1.south)
      -- ++(0,-.4)
      ..controls +(0,-.3) and +(-.3,0).. ++(.5,-.5)
      -- ++(3.8,0)
      ..controls +(.3,0) and +(0,-.3).. ++(.5,.5)
      -- (D);
    \draw[->,happensBeforeEdge] (B) to node [above] {$\alpha$} (Y);
    \draw[->,happensBeforeEdge] (Z) to node [above] {$\beta$} (D);
  \end{tikzpicture}
  \center{\Gls{side condition}: $\textsc{rax}_{\crashing} = 93 \wedge (\textsc{rbx}_{\interfering} =
    72 \vee \controlEdge{\crashing}{C_0}{D_0}) \wedge \textsc{rcx}_{\crashing} = \textsc{rcx}_{\interfering}$}
  \vspace{-4pt}
  \caption{Example control-flow (solid lines) and happens-before
    (dashed lines) graphs.  $D_0$ and $D_1$ are dynamic instructions
    representing the same static instruction (see
    \autoref{sect:derive:build_crashing_cfg}).  They are
    indistinguishable in the original program, but only $D_0$ can
    receive the happens-before edge $\beta$.}
  \label{fig:place_conditions_example}
  \vspace{4pt}
\end{sanefig}

\begin{sanefig}
  \begin{tikzpicture}
    \node (t1) {\shortstack{Crashing\\thread}};
    \node[right = 4.5cm of t1] (t2) {\shortstack{Interfering\\thread}};
    \node[below = .5 of t1] (A) {$A_0$};
    \node[left = 0 of A] {$\textsc{rax}_\crashing$, $\textsc{rcx}_\crashing$};
    \node[below = of A] (B) {$B_0$};
    \node[left = 0 of B] {$\textsc{rax}_\crashing$, $\textsc{rcx}_\crashing$};
    \node[below = of B] (C) {$C_0$};
    \path (node cs:name=C,anchor=north east) node [below right] {\shortstack[l]{$\textsc{rax}_\crashing$, $\textsc{rcx}_\crashing$,\\$\textsc{rbx}_\interfering$, $\textsc{rcx}_\interfering$}};
    \node[below = of C] (D) {$D_0$};
    \path (node cs:name=D,anchor=north west) node [below left] {\shortstack[r]{$\controlEdge{\crashing}{C_0}{D_0}$,\\$\textsc{rax}_\crashing$, $\textsc{rcx}_\crashing$,\\$\textsc{rbx}_\interfering$, $\textsc{rcx}_\interfering$\phantom{,}}\!\!\!};
    \node[left = 4cm of D] (D1) {$D_1$};
    \path (D1.north) ++(.7,0) node [above] {\shortstack[l]{$\controlEdge{\crashing}{C_0}{D_0}$,\\$\textsc{rax}_\crashing$, $\textsc{rcx}_\crashing$,\\$\textsc{rbx}_\interfering$, $\textsc{rcx}_\interfering$}\!\!\!};
    \node at (t2 |- A) (X) {$X_0$};
    \node[below right = 0 of X.north east] {$\textsc{rbx}_\interfering$, $\textsc{rcx}_\interfering$};
    \node at (t2 |- B) (Y) {$Y_0$};
    \node[below right = 0 of Y.north east] {\shortstack[l]{$\textsc{rax}_\crashing$, $\textsc{rcx}_\crashing$,\\$\textsc{rbx}_\interfering$, $\textsc{rcx}_\interfering$}};
    \node at (t2 |- D) (Z) {$Z_0$};
    \node[below right = 0 of Z.north east] {\shortstack[l]{$\textsc{rax}_\crashing$, $\textsc{rcx}_\crashing$,\\$\textsc{rbx}_\interfering$, $\textsc{rcx}_\interfering$}};
    \draw[->] (A) -- (B);
    \draw[->] (B) -- (C);
    \draw[->] (C) -- (D);
    \draw[->] (C) -- (D1);
    \draw[->] (X) -- (Y);
    \draw[->] (Y) -- (Z);
    \draw[->] (D1.south)
      -- ++(0,-.4)
      ..controls +(0,-.3) and +(-.3,0).. ++(.5,-.5)
      -- ++(3.8,0)
      ..controls +(.3,0) and +(0,-.3).. ++(.5,.5)
      -- (D);
    \draw[->,happensBeforeEdge] (B) to node [above] {$\alpha$} node [below] {\shortstack[l]{$\textsc{rax}_\crashing$, $\textsc{rcx}_\crashing$,\\$\textsc{rbx}_\interfering$, $\textsc{rcx}_\interfering$}} (Y);
    \draw[->,happensBeforeEdge] (Z) to node [above] {$\beta$} node [below] {\shortstack[c]{$\controlEdge{\crashing}{C_0}{D_0}$,\\$\textsc{rax}_\crashing$, $\textsc{rcx}_\crashing$,\\$\textsc{rbx}_\interfering$, $\textsc{rcx}_\interfering$\phantom{,}}} (D);
  \end{tikzpicture}
  \vspace{-12pt}
  \caption{\autoref{fig:place_conditions_example} extended to show
    input availability.}
  \label{fig:place_conditions_example:availability}
\end{sanefig}

Consider the example in \autoref{fig:place_conditions_example}.  The
\gls{side condition} to this bug depends on various values which have
to be collected from the running program: $\textsc{rax}_\crashing$,
and $\textsc{rcx}_\crashing$, registers in the \gls{crashingthread};
$\textsc{rbx}_\interfering$ and $\textsc{rcx}_\interfering$, registers
in the interfering one; and $\controlEdge{\crashing}{C_0}{D_0}$, an
expression on the \gls{crashingthread}'s control flow.
\autoref{fig:place_conditions_example:availability} shows how these
inputs become available as the program progresses through its
\gls{cfg}:
\begin{itemize}
\item $\textsc{rax}_\crashing$ and $\textsc{rcx}_\crashing$ represent
  the initial values of registers \textsc{rax} and \textsc{rcx} in the
  \gls{crashingthread}, and are therefore available in the
  \gls{crashingthread} as soon as the enforcer gains control over it.
  Similarly, $\textsc{rbx}_\interfering$ and
  $\textsc{rcx}_\interfering$ are available throughout the
  \gls{interferingthread}.
\item All program registers are available when evaluating the
  happens-before edge $\alpha$.  As discussed previously, {\technique}
  \glspl{bugenforcer} enforce happens-before edges synchronously, in
  the sense that both threads must be present simultaneously to
  evaluate the edge, and so anything available in either thread will
  be available during the edge operation.
\item Similarly, all registers are available at instructions $C_0$ and
  $Y_0$, as they can be copied from one thread to the other during the
  happens-before edge operation.  Note that the additional registers
  become available at $Y_0$, the edge's ending instruction, in the
  \gls{interferingthread}, but not until $C_0$, the instruction after
  the edge's beginning instruction, in the crashing one.  This is
  because the edge $\alpha$ links the end of instruction $B_0$ to the
  start of instruction $Y_0$, and so inputs made available by the edge
  are available at the end of $B_0$ and the start of $Y_0$.
\item The control flow expression $\controlEdge{\crashing}{C_0}{D_0}$
  becomes available once instruction $C_0$ has selected which
  instruction will be executed next, and hence is available in $C_0$'s
  successors $D_0$ and $D_1$.  It is also available on happens-before
  edge $\beta$ because $D_0$ is $\beta$'s ending instruction.
\end{itemize}
\begin{sanefig}
  \sloppy
  \subfigure[][Reordered \protect\gls{bdd}]{
    \begin{tikzpicture}
      \node (root) [BddNode,color=blue] {$\textsc{rax}_\crashing\!= 93$};
      \node (alpha) [BddNode, below = of root] {$\controlEdge{\crashing}{C_0}{D_0}$\hspace{-1.45cm}~};
      \node (c0) [BddNode, below = of alpha] {$\textsc{rcx}_\crashing\!\!=\!\textsc{rcx}_\interfering$\hspace{-4mm}~};
      \path (node cs:name=c0,anchor=east) node (c1) [BddNode,right] {~~$\textsc{rcx}_\crashing\!\!=\!\textsc{rcx}_\interfering$};
      \path (barycentric cs:c0=0.5,c1=0.5) ++(0,-0.8) node (b) [BddNode, below] {$\textsc{rbx}_\interfering\!= 72$};
      \path (node cs:name=c1,anchor=south) ++(0,-2) node (false) [BddLeaf] {\false};
      \node (true) at (c0 |- false) [BddLeaf] {\true};
      \draw[BddTrue] (root) -- (alpha);
      \draw[BddFalse] (root.east) ..controls +(2.5,-.5) and +(.4,3).. (false);
      \draw[BddTrue] (alpha) -- (c0);
      \draw[BddFalse] (alpha) -- (c1);
      \draw[BddTrue] (c0) -- (true);
      \draw[BddFalse] (c0) to [bend left=30] (false);
      \draw[BddTrue] (c1) -- (b);
      \draw[BddFalse] (c1) -- (false);
      \draw[BddTrue] (b) -- (true);
      \draw[BddFalse] (b) -- (false);
    \end{tikzpicture}
    \label{fig:place_conditions:example:instr_a:reordered}
  } \subfigure[][Condition evaluable at $A_0$]{
    \hspace{-4.5mm}
    \raisebox{2.55cm}{\Large =}
    \hspace{-7.5mm}
    \begin{tikzpicture}
      \node (root) [BddNode,color=blue] {$\textsc{rax}_\crashing\!= 93$};
      \node (true) [BddLeaf, below = 5 of root] {\true};
      \node (false) [BddLeaf, right = of true] {\false};
      \draw[BddTrue] (root) -- (true);
      \draw[BddFalse] (root.east) ..controls +(2.5,-.5) and +(.4,3).. (false);
    \end{tikzpicture}
    \hspace{-5mm}\raisebox{2.55cm}{$\bigwedge$}
    \label{fig:place_conditions:example:instr_a:evaluable}
  } \hspace{-4mm}\subfigure[][Residual condition]{
    \begin{tikzpicture}
      \node (alpha) [BddNode] {$\controlEdge{\crashing}{C_0}{D_0}$\hspace{-1.45cm}~};
      \node (c0) [BddNode, below = of alpha] {$\textsc{rcx}_\crashing\!\!=\!\textsc{rcx}_\interfering$\hspace{-4mm}~};
      \path (node cs:name=c0,anchor=east) node (c1) [BddNode,right] {~~$\textsc{rcx}_\crashing\!\!=\!\textsc{rcx}_\interfering$};
      \path (barycentric cs:c0=0.5,c1=0.5) ++(0,-0.8) node (b) [BddNode, below] {$\textsc{rbx}_\interfering\!= 72$};
      \path (node cs:name=c1,anchor=south) ++(0,-2) node (false) [BddLeaf] {\false};
      \node (true) at (c0 |- false) [BddLeaf] {\true};
      \draw[BddTrue] (alpha) -- (c0);
      \draw[BddFalse] (alpha) -- (c1);
      \draw[BddTrue] (c0) -- (true);
      \draw[BddFalse] (c0) to [bend left=30] (false);
      \draw[BddTrue] (c1) -- (b);
      \draw[BddFalse] (c1) -- (false);
      \draw[BddTrue] (b) -- (true);
      \draw[BddFalse] (b) -- (false);
    \end{tikzpicture}
    \label{fig:place_conditions:example:instr_a:residual}
  }
  \caption{Factorisation of the \glsentrytext{side condition} in
    \autoref{fig:place_conditions_example} at CFG node $A_0$.
    $\textsc{rax}_\crashing$ and $\textsc{rcx}_\crashing$ are the only available input
    expressions.  Evaluable BDD variables are shown in blue.}
  \label{fig:place_conditions_example:instr_a}
\end{sanefig}

\noindent
Once the availability of input expressions has been determined,
{\technique} can decide how to evaluate the \gls{side condition}.  It
starts by placing a copy of the \gls{side condition} at every
\gls{cfg} entry instruction, factorising it into evaluable and
unevaluable components based on the available inputs, and evaluating
the evaluable component at that instruction and pushing the remainder
of the \gls{side condition} to the instruction's successors.  The
remainder is then split using the inputs available at that successor,
and the process repeats until either the entire \gls{side condition}
has been evaluated or {\technique} reaches the end of the \gls{cfg}.

Consider now what happens when {\technique} attempts to evaluate the
\gls{side condition} in \autoref{fig:place_conditions_example} at $A_0$.
The only available input is $\textsc{rax}_\crashing$, so $\textsc{rax}_\crashing = 93$
is evaluable but nothing else is.  {\Technique} will re-express the
\gls{side condition} as a \gls{bdd}, shown in
\autoref{fig:place_conditions:example:instr_a:reordered}, with the
variables ordered so the evaluable expression is at the root.
Extracting the evaluable component,
\autoref{fig:place_conditions:example:instr_a:evaluable}, is then a
simple matter of replacing every edge from an evaluable expression to
an unevaluable one with an edge to \true; likewise, the unevaluable
component, \autoref{fig:place_conditions:example:instr_a:residual},
can be derived by removing every edge from an evaluable expression to
{\false} and then renormalising.  The \gls{bugenforcer} will evaluate
the evaluable component, $\textsc{rax}_\crashing = 93$, at $A_0$, and will
defer the unevaluable component to later in the \gls{cfg}\kern-.8pt.\kern-.5pt\fnote{This
  is in essence a predicate abstraction problem, and {\technique}'s
  algorithm for doing so can be regarded as a modest generalisation of
  Cavada's \protect\Gls{bdd} Modulo Theory
  algorithm~\cite{cavada2007}, applied in a slightly unusual context.}

To understand why this is a correct factorisation of the condition,
consider the paths through the \gls{bdd}.  Any which test only
evaluable expressions and end at {\false} will be preserved in the
evaluable component, and every other path is preserved in the
unevaluable one.  A \gls{bdd}'s behaviour can be completely
characterised by the set of paths through it, and so if every path is
preserved then the behaviour is as well.

Applying the same algorithm shows that no sub-condition can be
usefully checked at $B_0$, as its set of available inputs is the same
as $A_0$'s, or at $X_0$, as the available inputs there do not allow an
evaluable component other than \true.  {\Technique} will therefore
attempt to evaluate the unevaluable component from $A_0$ as part of
the happens-before edge $\alpha$, as shown in
\autoref{fig:place_conditions:example:hb_m}.  The evaluable component
is now $\textsc{rcx}_\crashing = \textsc{rcx}_\interfering$, and so the \gls{bugenforcer}
will only attempt to enforce this happens-before edge between threads
whose RCX registers coincide.  Note that in this case the evaluable
expression $\textsc{rbx}_\interfering = 72$ remains present in the unevaluable
component.  This is because, while the expression could be evaluated
as part of the $\alpha$ edge, doing so would never allow the enforcer
to abort the enforcement process early, and so would have no actual
benefits over delaying the evaluation further, but would incur some
additional bookkeeping complexity.

The residual condition is again propagated through the graph until it
reaches a point with a larger set of available inputs.  In this case,
that will be either happens-before edge $\beta$ or instruction $D_1$
(these can be processed in either order due to their positions in the
graph).  These have the same set of available inputs and hence produce
the same factorisation, shown in
\autoref{fig:place_conditions:example:hb_n}.  Every input is now
available, so every expression evaluable, and the evaluable component
is the entire \gls{bdd} and the residual is just {\true}.  The
resulting placement of \glspl{side condition} is shown in
\autoref{fig:place_conditions_example:result}.

\begin{sanefig}
  \subfigure[][Reordered \protect\gls{bdd}]{
    \begin{tikzpicture}
      \node (c) [BddNode,color=blue] {$\textsc{rcx}_\crashing = \textsc{rcx}_\interfering$\hspace{-4mm}\!};
      \node (b) [BddNode,below = of c,color=blue] {$\textsc{rbx}_\interfering = 72$};
      \node (true) [BddLeaf, below = 2 of b] {\true};
      \node (false) [BddLeaf, right = of true] {\false};
      \path (node cs:name=false) ++(0,1.3) node (alpha) [BddNode] {$\controlEdge{\crashing}{C}{D_0}$};
      \draw[BddTrue] (c) -- (b);
      \draw[BddFalse] (c) ..controls +(2,-1) and +(.5,3).. (false);
      \draw[BddTrue] (b) -- (true);
      \draw[BddFalse] (b) -- (alpha);
      \draw[BddTrue] (alpha) -- (true);
      \draw[BddFalse] (alpha) -- (false);
    \end{tikzpicture}
  } \subfigure[][Condition evaluable at $\alpha$]{
    \hspace{-3mm}
    \raisebox{2.1cm}{\Large =}
    \hspace{-2mm}
    \begin{tikzpicture}
      \node (c) [BddNode, color=blue] {$\textsc{rcx}_\crashing = \textsc{rcx}_\interfering$\hspace{-4mm}\!};
      \node (true) [BddLeaf, below = 3.6 of root] {\true};
      \node (false) [BddLeaf, right = of true] {\false};
      \draw[BddTrue] (c) -- (true);
      \draw[BddFalse] (c) ..controls +(2,-1) and +(.5,3).. (false);
    \end{tikzpicture}
    \raisebox{2.1cm}{$\bigwedge$} 
  } \subfigure[][Residual condition]{
    \begin{tikzpicture}
      \node (b) [BddNode,color=blue] {$\textsc{rbx}_\interfering = 72$\hspace{-1cm}~};
      \node (true) [BddLeaf, below = 2 of b] {\true};
      \node (false) [BddLeaf, right = of true] {\false};
      \path (node cs:name=false) ++(0,1.3) node (alpha) [BddNode] {$\controlEdge{\crashing}{C}{D_0}$};
      \draw[BddTrue] (b) -- (true);
      \draw[BddFalse] (b) -- (alpha);
      \draw[BddTrue] (alpha) -- (true);
      \draw[BddFalse] (alpha) -- (false);
    \end{tikzpicture}
    \label{fig:place_conditions:example:hb_m:residual}    
  }
  \caption{Factorisation of the \protect\gls{bdd} in
    \autoref{fig:place_conditions:example:instr_a:residual} at
    happens-before edge $\alpha$.}
  \label{fig:place_conditions:example:hb_m}
\end{sanefig}

\begin{sanefig}
  \subfigure[][Reordered \protect\gls{bdd}]{
    \begin{tikzpicture}
      \node (b) [BddNode,color=blue] {$\textsc{rbx}_\interfering = 72$};
      \node (true) [BddLeaf, below = 2 of b] {\true};
      \node (false) [BddLeaf, right = of true] {\false};
      \path (node cs:name=false) ++(0,1.3) node (alpha) [BddNode, color=blue] {$\controlEdge{\crashing}{C}{D_0}$};
      \draw[BddTrue] (b) -- (true);
      \draw[BddFalse] (b) -- (alpha);
      \draw[BddTrue] (alpha) -- (true);
      \draw[BddFalse] (alpha) -- (false);
    \end{tikzpicture}
  } \subfigure[][Condition evaluable at $\beta$ and $D_1$]{
    \hspace{-2.5mm}
    \raisebox{15mm}{\Large =}
    \hspace{-1.5mm}
    \begin{tikzpicture}
      \node (b) [BddNode,color=blue] {$\textsc{rbx}_\interfering = 72$};
      \node (true) [BddLeaf, below = 2 of b] {\true};
      \node (false) [BddLeaf, right = of true] {\false};
      \path (node cs:name=false) ++(0,1.3) node (alpha) [BddNode, color=blue] {$\controlEdge{\crashing}{C}{D_0}$};
      \draw[BddTrue] (b) -- (true);
      \draw[BddFalse] (b) -- (alpha);
      \draw[BddTrue] (alpha) -- (true);
      \draw[BddFalse] (alpha) -- (false);
    \end{tikzpicture}
  } \subfigure[][Residual condition]{
    \raisebox{15mm}{$\bigwedge$~~\true~~~~~~~~~~~~~}
  }
  \caption{Factorisation of the BDD in
    \autoref{fig:place_conditions:example:hb_m:residual} at
    happens-before edge $\beta$ and instruction $D_1$.  Every variable is
    now evaluable, and so the residual condition is just \true.}
  \label{fig:place_conditions:example:hb_n}
\end{sanefig}

\begin{sanefig}
  \begin{tikzpicture}
    \node (t1) {\shortstack{Crashing\\thread}};
    \node[right = 4.5cm of t1] (t2) {\shortstack{Interfering\\thread}};
    \node[below = .5 of t1] (A) {$A_0$};
    \node[left = 0 of A] {$\textsc{rax}_\crashing = 93$};
    \node[below = of A] (B) {$B_0$};
    \node[below = of B] (C) {$C_0$};
    \node[below = of C] (D) {$D_0$};
    \node[left = 4cm of D] (D1) {$D_1$};
    \path (D1.north) ++(1,0) node [above] {\shortstack[l]{$\controlEdge{\crashing}{C_0}{D_0} \vee$\\$\textsc{rbx}_\interfering = 72$}};
    \node at (t2 |- A) (X) {$X_0$};
    \node at (t2 |- B) (Y) {$Y_0$};
    \node at (t2 |- D) (Z) {$Z_0$};
    \draw[->] (A) -- (B);
    \draw[->] (B) -- (C);
    \draw[->] (C) -- (D);
    \draw[->] (C) -- (D1);
    \draw[->] (X) -- (Y);
    \draw[->] (Y) -- (Z);
    \draw[->] (D1.south)
      -- ++(0,-.4)
      ..controls +(0,-.3) and +(-.3,0).. ++(.5,-.5)
      -- ++(3.8,0)
      ..controls +(.3,0) and +(0,-.3).. ++(.5,.5)
      -- (D);
    \draw[->,happensBeforeEdge] (B) to node [above] {$\alpha$} node [below] {$\textsc{rcx}_\crashing = \textsc{rcx}_\interfering$} (Y);
    \draw[->,happensBeforeEdge] (Z) to node [above] {$\beta$} node [below] {\shortstack{$\controlEdge{\crashing}{C_0}{D_0} \vee$\\$\textsc{rbx}_\interfering = 72$}} (D);
  \end{tikzpicture}
  \caption{Final placement of \glsentrytext{side condition} checks for
    the example control flow graph in
    \autoref{fig:place_conditions_example}.}
  \label{fig:place_conditions_example:result}
\end{sanefig}

\section{Gaining control of a program}
\label{sect:enforce:gain_control}

In order to manipulate a program's execution a \gls{bugenforcer} must
first gain control of it.  {\Technique} does so by patching the
program's binary and replacing certain critical instructions with
branches to the \gls{bugenforcer}'s main loop.  This is highly
efficient, when compared to other techniques such as dynamic binary
rewriting~\cite{Luk2005,Nethercote2007} or processor breakpoint
registers~\cite[Chapter 16.2: Debug Registers]{Intel2009}, and avoids
unnecessarily disturbing the program's state, but requires care to be
implemented safely.  In particular, the branch operation might be
larger than the instruction it is replacing\footnote{The normal AMD64
  jump instruction is five bytes long, and some preliminary
  experiments showed that 56\% of instructions in a small selection of
  programs were four bytes or smaller.  The precise ratio will vary
  depending on the program examined; the important point is that these
  small instructions are far too common to ignore.} and so the
replacement risks corrupting the instructions immediately after the
one which is to be replaced.  The patching mechanism must ensure that
the program never executes one of these corrupted instructions.
{\Technique} does so by introducing further patches which ensure
that it has control whenever the program attempts to branch to one of
these corrupted instructions.

\begin{sanefig}
  \begin{displaymath}
    \textsc{Strategy} = \left\{\begin{array}{rl}
    \mathit{Patch}: & \{\textsc{Instruction}\} \\
    \mathit{Cont}: & \{\textsc{Instruction}\} \\
    \mathit{Pending}: & \{\textsc{Instruction}\}
    \end{array}\right\}
  \end{displaymath}
  \vspace{-12pt}
  \caption{The \textsc{Strategy} type}
  \label{fig:patch_strategy_type}
\end{sanefig}

{\Technique} represents solutions and partial solutions to the
patch problem using the \textsc{Strategy} type, illustrated in
\autoref{fig:patch_strategy_type}:
\begin{itemize}
\item $\mathit{Patch}$ is the set of instructions in the original
  program which will be replaced with branch instructions.
\item $\mathit{Cont}$ is the set of instructions at which
  {\technique} will have control of the program.  The enforcer
  will not gain control of the program simply because it runs one of
  these instructions, unless the instruction happens to also be a
  member of $\mathit{Patch}$, but it will also not relinquish control
  at one of these instructions, and will instead continue to emulate
  the program until it leaves the $\mathit{Cont}$ set.
\item $\mathit{Pending}$ contains instructions at which
  {\technique} must have control of the program's execution but
  which are not controlled by the current strategy.
\end{itemize}
A \textsc{Strategy} is said to be \emph{valid} if gaining control at
all $\mathit{Pending}$ and $\mathit{Patch}$ instructions, and refusing
to relinquish it at $\mathit{Cont}$ ones, would ensure that the
program never executes a corrupted instruction; it is \emph{complete}
if the $\mathit{Pending}$ set is empty; and it is \emph{sufficient} if
it is complete and $\mathit{Cont}$ includes every instruction at which
the enforcer must gain control of the program.  {\Technique}
must, at a minimum, find a strategy which is both valid and
sufficient, and should, ideally, find one which is efficient, in the
sense of minimising the number of instructions which must be emulated.
The algorithm used to do so is shown in
\autoref{fig:patch_search_algorithm}.

The \textsc{patch} and \textsc{prefix} helper functions perhaps
require additional explanation.  These transform a \textsc{Strategy}
into a set of new strategies by considering each $\mathit{Pending}$
instruction $i$ in turn and either replacing $i$ with a branch
(\textsc{patch}) or arranging to gain control before the program can
run $i$ (\textsc{prefix}).  Both functions arrange to gain control at
$i$, but they have different costs and constraints:
\begin{itemize}
\item The \textsc{patch} function gains control directly at
  instruction $i$, and so avoids forcing any additional programs to be
  run in the instruction emulator, but potentially corrupts other
  instructions in the program.  The function must then ensure that it
  gains control before the program executes any of these corrupted
  instructions, and hence must add them to the $\mathit{Pending}$ set.
  It is, mechanically, only possible to patch $i$ if doing so will
  neither corrupt not be corrupted by any of the existing entries in
  the $\mathit{Patch}$ set.
\item The \textsc{prefix} function, by contrast, does not attempt to
  gain control at $i$ itself but instead arranges to gain control at
  the predecessors of $i$ and then emulate the program until it
  reaches $i$.  This does not involve modifying the program, and so
  cannot cause any instruction corruption, but requires the function
  to gain control at any instruction which can precede $i$ and hence
  to add $i.\mathit{pred}$ to $\mathit{Pending}$.
\end{itemize}
The algorithm starts with the (valid, complete, insufficient) empty
\textsc{Strategy} (line 2) and then \textsc{extend}s it with one
needed instruction at a time (line 4).  \textsc{extend}ing a
\textsc{Strategy} renders it incomplete, and so the main loop of the
algorithm restores completeness using the \textsc{patch} and
\textsc{prefix} functions (lines 5 to 13).  Once every instruction has
been added the resulting strategy is both valid and sufficient, and is
hence a correct solution to the patching problem.\fnote{Equivalently,
  \textsc{buildStrategy} implements a search over the space of
  \textsc{Strategy}s using the \textsc{prefix} and \textsc{patch}
  operators with a cut point between every instruction at which the
  enforcer must gain control.}
\begin{sanefig}
  \begin{algorithmic}[1]
    \Function{buildPatchStrategy}{$\mathit{gainControl}$}
    \State {$\mathit{soln} \gets \textsc{Strategy}(\mathit{Patch} = \{\}, \mathit{Cont} = \{\}, \mathit{Pending} = \{\})$}
    \For {$i \in \mathit{gainControl}$}
      \State {$q \gets \queue{\textsc{extend}(i, \mathit{soln})}$}
      \While {\true}
        \State {$\mathit{s} \gets \mathit{pop}(q)$}
        \If {$s.\mathit{Pending} = \{\}$}
          \State {$\mathit{soln} \gets s$}
          \State \textbf{break}
        \Else
          \State {$q \gets q \cup \textsc{prefix}(s) \cup \textsc{patch}(s)$}
        \EndIf
      \EndWhile
    \EndFor
    \State \Return $\mathit{soln}$
    \EndFunction
  \end{algorithmic}
  \vspace{8pt}
  \shortstack[l]{
    \begin{math}
      \textsc{extend}(i,s) = \textsc{Strategy}\!\!\left(\!\!\!\!\begin{array}{rcl}
      \mathit{Patch} & = & s.\mathit{Patch},\\
      \mathit{Cont}  & = & s.\mathit{Cont},\\
      \mathit{Pending} & = & s.\mathit{Pending} + i
      \end{array}\!\!\right)
    \end{math}\\
    \begin{math}
      \textsc{prefix}(s) \hspace{5mm} = \left\{\textsc{Strategy}\!\!\left(\!\!\!\!\begin{array}{rcl}
        \mathit{Patch}   & = & s.\mathit{Patch},\\
        \mathit{Cont}    & = & s.\mathit{Cont} + i,\\
        \mathit{Pending} & = & s.\mathit{Pending} -i \\
                         &   & \cup i.\mathit{pred}
      \end{array}\!\!\right)\middle| i \in s.\mathit{Pending}\!\!\right\}
    \end{math}\\
    $\textsc{patch}(s) \hspace{6mm} =$\\
    \hspace{.9cm}\begin{math}
      \left\{\textsc{Strategy}\!\!\left(\!\!\!\!\begin{array}{rcl}
      \mathit{Patch}   & = & s.\mathit{Patch} + i,\\
      \mathit{Cont}    & = & s.\mathit{Cont} + i,\\
      \mathit{Pending} & = & s.\mathit{Pending} - i \\
      &   & \cup \textsc{corrupt}(i)
      \end{array}\!\!\right)\middle| \begin{array}{lc}
        i \in s.\mathit{Pending} & \wedge \\
        \textsc{corrupt}(i) \cap s.\mathit{Patch} = \varnothing & \wedge \\
        i \not\in \bigcup\limits_{j \in s.\mathit{Patch}}\textsc{corrupt}(j)
      \end{array}\!\!\right\}
    \end{math}
  }
  \caption{The patch search algorithm \textsc{buildPatchStrategy}.
    $\mathit{gainControl}$ is the set of instructions at which the
    enforcer must gain control of the program.  Not shown:
    {\technique} records all of the strategies which it has visited so
    far so as to avoid re-visiting them.}
  \label{fig:patch_search_algorithm}
\end{sanefig}

This algorithm, as stated, leaves the order in which
\textsc{Strategy}s are removed from the queue ambiguous.  This can
sometimes have a large effect on the number of instructions which must
be executed in the emulator, and hence on the performance overheads of
the patch.  It is hard to predict precisely how large an impact a
given strategy will have on performance without a detailed model of
the program's behaviour, but it can reasonably be approximated to be
proportional to the number of static instructions affected, which can
itself be approximated by $|\mathit{Cont} \cup \mathit{Pending}|$.
{\Technique} will therefore always choose to expand the strategy
where that quantity is smallest, which usually causes it to find the
strategy which minimises the cost of the patch.

\section{Enforcing the plan}
\label{sect:enforce:interpreting}

Previous sections described how to find the desired happens-before
graph and \gls{side condition} and how to place the components of the
\gls{side condition} on the control-flow graph.  Collectively, these
form the crash enforcement plan.  {\Technique} must now somehow
arrange for the program to follow this plan.

As an example, consider the bug and crash enforcement plan shown in
\autoref{fig:enforcement:example_bug}.  If asked to impose this plan
on the program, {\technique} will start by arranging to gain control
of any program threads which execute instructions {\tt a} or {\tt c}.
When it does gain control, it will check whether \texttt{loc1} is
equal to \texttt{7} and, if it is, immediately return control to the
original program.  What happens next depends on which instruction
{\technique} gained control at: if {\tt a}, it will emulate the
original instruction {\tt a} and then wait for another thread to run
instruction {\tt c}; if {\tt c}, it will simply wait for something to
run {\tt a}.  \texttt{a} is said to send a message to \texttt{c}.
Note three things:
\begin{enumerate}
\item When the enforcer gains control at {\tt a}, it emulates {\tt a}
  before waiting for {\tt c}, whereas when it gains control at {\tt c}
  it waits for {\tt a} without first emulating {\tt c}.  This is
  because the happens-before edge orders {\tt a} before {\tt c} and so
  links the end of {\tt a} to the beginning of {\tt c}.
\item {\tt a} must wait for {\tt c}, even though the only requirement
  imposed by the happens-before edge is for {\tt c} to happen some
  time after {\tt a}; as discussed in
  \autoref{sect:reproduce:unenforcable}, {\technique} enforces
  happens-before edges synchronously.
\item Both wait operations have a timeout.  If no other thread runs
  the appropriate matching instruction within this timeout then the
  enforcer abandons the plan and returns the program to normal
  operation.
\end{enumerate}
This wait operation both enforces the happens-before edge itself and,
just as importantly, determines how the dynamic threads in a running
program map onto the analysis threads in the plan.  Whichever program
thread executed {\tt a} will become the crashing thread and whichever
executed {\tt c} will become the interfering one.  The two threads are
said to have been bound together and will execute the rest of the plan
together.

Once the wait has completed and the threads have been bound, the
\gls{bugenforcer} continues through the plan, emulating \texttt{c} in
the interfering thread and moving on to the final happens-before edge.
This is handled similarly to the first one: each thread waits for the
other to arrive at its matching happens-before operation, the
operation is discharged, and the two threads continue.  In this case,
the happens-before edge is the final operation in the plan, and so
both threads will exit the enforcer and return to normal operation
once it has finished (regardless of whether the edge completes
successfully or times out).  The crashing thread will resume at
instruction {\tt b} by loading \texttt{y} from \texttt{loc1}.
\texttt{loc1} was last modified when the \gls{interferingthread}
emulated \texttt{c}, and so \texttt{y} is now set to \texttt{7}.  The
\gls{side condition} imposed at the start of the plan will have
ensured that \texttt{x} is not \texttt{7}, and so the
\gls{crashingthread}'s assertion will fail and the bug will be
correctly reproduced. The complete procedure is shown in
\autoref{fig:enforcement:example_bug:replacements}.  Note that the
\protect\gls{bugenforcer} finishes before the bug reproduces and that
the fatal \texttt{assert} is run without direct interference from
{\technique}.  This simplifies the integration of {\technique} with
existing debugging tools which might be confused by the interposition
of an unknown interpreter.

\begin{sanefig}
  \subfigure[][Crashing thread]{
    \raisebox{7mm}{
    \begin{tabular}{ll}
      {\tt a:} & {\tt x = loc1;}\\
      {\tt b:} & {\tt y = loc1;}\\
      & {\tt assert(x == y);}\\
    \end{tabular}
    }
  }
  {\hfill}
  \subfigure[][Interfering thread]{
    \raisebox{7mm}{
     \begin{tabular}{ll}
      \\
      {\tt c:} & {\tt loc1 = 7;}\\
      \\
    \end{tabular}
    }
  }
  {\hfill}
  \raisebox{0mm}{
    \subfigure[][Crash enforcement plan]{
      \raisebox{-3mm}{
      \begin{tikzpicture}
        \node (a) [CfgInstr] {\tt a};
        \node (b) [CfgInstr, below = of a] {\tt b};
        \path (a) ++(1,-.8) node (c) [CfgInstr] {\tt c};
        \node [left = 0 of a] {$\texttt{loc1} \not= 7$};
        \node [right = 0 of c] {$\texttt{loc1} \not= 7$};
        \draw[->] (a) -- (b);
        \draw[->,happensBeforeEdge] (a) -- (c);
        \draw[->,happensBeforeEdge] (c) -- (b);
      \end{tikzpicture}
      }
    }
  }
  {\hfill}
  \caption{An example bug.}
  \label{fig:enforcement:example_bug}
\end{sanefig}
\begin{sanefig}
  {\hfill}
  \begin{tikzpicture}
    \node (n0m) {Crashing thread};
    \node (n1m) [right = of n0m] {Interfering thread};

    \node (n00) [CfgInstr, below = of n0m] {\shortstack{Gain control at\\instruction {\tt a}}};
    \node (n10) at (n1m |- n00) [CfgInstr] {\shortstack{Gain control at\\instruction {\tt c}}};

    \node (n01) [CfgInstr, below = of n00] {Require $\texttt{loc1} \not= 7$};
    \node (nm1) [CfgInstr, left = of n01] {\shortstack[r]{Return to\\instruction {\tt a}}};
    \node (n11) at (n1m |- n01) [CfgInstr] {Require $\texttt{loc1} \not= 7$};
    \node (n21) [CfgInstr, right = of n11] {};

    \node (n02) [CfgInstr, below = of n01] {\tt a: x = loc1};

    \node (n03) [CfgInstr, below = of n02] {Send message 1};
    \node (n13) at (n1m |- n03) {};

    \node (n14) [below = of n13, CfgInstr] {Receive message 1};
    \node (n24) at (n21 |- n14) [CfgInstr] {};
    \node (n2a) at (barycentric cs:n24=0.5,n21=0.5) [CfgInstr,right = -1] {\shortstack[l]{Return to\\instruction {\tt c}}};

    \node (n15) [below = of n14, CfgInstr] {\tt c: loc1 = 7};

    \node (n16) [below = of n15, CfgInstr] {Send message 2};
    \node (n06) at (n0m |- n16) {};

    \node (n07) [below = of n06, CfgInstr] {Receive message 2};

    \node (n08) [below = of n07, CfgInstr] {\shortstack{Return to\\instruction {\tt b}}};
    \node (n18) at (n1m |- n08) [CfgInstr] {\shortstack{Return to instruction\\after {\tt c}}};

    \draw[->] (n00) -- (n01);
    \draw[->] (n01) -- (n02);
    \draw[->] (n02) -- (n03);
    \draw[->] (n03) -- (n07);
    \draw[->] (n07) -- (n08);

    \draw[->] (n10) -- (n11);
    \draw[->] (n11) -- (n14);
    \draw[->] (n14) -- (n15);
    \draw[->] (n15) -- (n16);
    \draw[->] (n16) -- (n18);

    \draw[->,ifFalse] (n01) -- (nm1);
    \draw[->,ifFalse] (n03.west)
      -- ++(-.7,0)
      ..controls +(-.3,0) and +(0,.3).. ++(-.5,-.5)
      -- ++(0,-7.05)
      ..controls +(0,-.3) and +(-0.3,0).. ++(.5,-.5)
      -- (n08.west);
    \draw[->,ifFalse] (n07.west)
      ..controls +(-.3,0) and +(0,0.3).. ++(-.5,-.5)
      -- ++(0,-.86)
      ..controls +(0,-.3) and +(-0.3,0).. ++(.45,-.45)
      -- (n08.west);

    \draw[->,ifFalse] (n11.east)
      -- ++(1.11,0)
      ..controls +(0.3,0) and +(0,.3).. ++(.5,-.5)
      -- (n2a);
    \draw[->,ifFalse] (n14.east)
      -- ++(0.96,0)
      ..controls +(0.3,0) and +(0,-.3).. ++(.5,.5)
      -- (n2a);
    \draw[->,ifFalse] (n16.east)
      -- ++(.7,0)
      ..controls +(.3,0) and +(0,.3).. ++(.5,-.5)
      -- ++(0,-2.28)
      ..controls +(0,-.3) and +(.3,0).. ++(-.5,-.5)
      -- (n18.east);

    \draw[happensBeforeEdge,->] (n03) -- (n14);
    \draw[happensBeforeEdge,->] (n16) -- (n07);
  \end{tikzpicture}
  {\hfill}
  \caption{Instruction replacements for the example bug in
    \autoref{fig:enforcement:example_bug}.  Dashed lines indicate
    message passing operations and dotted ones indicate error paths.}
  \label{fig:enforcement:example_bug:replacements}
\end{sanefig}

This behaviour is correct when enforcing a single plan at a time, but
this is unlikely to be sufficient in a realistic system.  A single
program could easily produce hundreds of
\glspl{verificationcondition}, each of which can produce multiple
plans, and it would be quite tedious to test each one individually.
{\Technique} \glspl{bugenforcer} therefore track the program's
progression through a set of plans, imposing delays and checking
\glspl{side condition} in a way which allows all of the plans to make
progress.

This set construction also provides a convenient way of handling
ambiguities caused by the \gls{cfg} unrolling algorithm (see
\autoref{sect:derive:build_crashing_cfg}).  The plan is defined in
terms of the \gls{dynamic cfg}, but only the \gls{static cfg} can be
directly observed at run time, and this can sometimes make it
difficult to determine a thread's position within a plan.
{\Technique} deals with these ambiguities by forking the plan state
and having one fork follow each possible outcome of the ambiguous
choice.  Instruction $C_0$ in
\autoref{fig:place_conditions_example:result} provides an example: in
the \gls{dynamic cfg}, $C_0$ can be followed by either $D_0$ or $D_1$,
and the plan must perform different actions depending on which is
chosen, but the enforcer simply observes that static instruction C is
followed by static instruction D and cannot tell which to use.  It
avoids the issue by pretending that there were two copies of the plan
in its initial plan set and having one go to $D_0$ and the other to
$D_1$\kern-.2pt.

There is a potentially useful analogy with the power set construction
used to convert a non-deterministic finite state automaton into a
deterministic one.  In that model, a deterministic FSA emulates a
non-deterministic one by tracking the set of states which the
non-deterministic automaton might possibly occupy, allowing it to
defer non-deterministic choices until it has enough information to
resolve them unambiguously.  In the same way, {\technique}'s set of
plans allows it to defer an ambiguous choice until one or other of the
possible options leads to a failed plan, at which point it can arrange
to have chosen the other one.

I now give more details of how the plan enforcer works, starting with
a more complete description of the single-plan version
(\autoref{sect:enforce:llis}) and then showing how to convert it to
operate with sets of plans (Sections~\ref{sect:enforce:succ}
and~\ref{sect:enforce:hli_messages}), before discussing some ways in
which incompatible plans can interfere with each other
(\autoref{sect:enforce:plan_interference}).

\subsection{The single-plan enforcer}
\label{sect:enforce:llis}

The single-plan enforcer runs some simple stages in a tight loop:
\begin{itemize}
\item \textbf{Sample}: Copy any thread registers needed to evaluate later
  \glspl{side condition} into plan-local state.

\item \textbf{RX}: Receive any messages required by the plan.  Message
  operations are discussed below.

\item \textbf{Emul}: Emulate the instruction in the original program.
  My implementation of this emulator is based on the one used in the
  Xen hypervisor~\cite{XenInstructionEmul}.

\item \textbf{TX}: Send any messages required by the plan.

\item \textbf{Succ}: Determine which node in the \gls{dynamic cfg} the
  plan is going to execute next.
\end{itemize}
{\Technique} uses a synchronous message operation, illustrated in
\autoref{fig:enforce:lli_message}, both to impose the desired
happens-before edges and to synchronise needed state between the
different program threads involved in the bug.  The operation starts
by selecting a plan in a remote thread with which to communicate; the
mechanism for doing so depends on details of the multi-plan
enforcer and so is deferred to \autoref{sect:enforce:hli_messages}.
Each thread then waits for the other to arrive, at which point they
collectively evaluate the \gls{side condition}, synchronise any state
necessary to evaluate future \glspl{side condition}, and bind the
plans together, if they are not already bound.  They then advance to
the next stage in the enforcer cycle (either \textbf{Emul} for a
receive operation or \textbf{Succ} for a transmit one).

\begin{sanefig}
  {\hfill}
  \begin{tikzpicture}
    \node (n1m) {\textbf{RX}};
    \node (n10) [below = of n1m] {$t \leftarrow$ Select remote plan};
    \node (n11) [below = of n10] {Wait for t};
    \node (n01) [left = of n11] {fail};

    \node (n22) [below right = of n11] {Evaluate \gls{side condition}};
    \node (n31) [above right = of n22] {Wait for $t'$};
    \node (n41) [right = of n31] {fail};
    \node (n30) [above = of n31] {$t' \leftarrow$ Select remote plan};
    \node (n3m) [above = of n30] {\textbf{TX}};

    \node (n42) at (n22 -| n41) {fail};
    \node (n02) at (n22 -| n01) {fail};

    \node (n23) [below = of n22] {Copy state between threads};
    \node (n24) [draw, dashed, below = of n23] {Bind plans together};
    \node (dummy) [below = of n24] {};
    \node (n15) at (dummy -| n11) {Advance to \textbf{Emul}};
    \node (n35) at (dummy -| n31) {Advance to \textbf{Succ}};
    \draw[->] (n1m) -- (n10);
    \draw[->] (n10) -- (n11);
    \draw[->] (n11) -- (n22);

    \draw[->] (n3m) -- (n30);
    \draw[->] (n30) -- (n31);
    \draw[->] (n31) -- (n22);

    \draw[->] (n22) -- (n23);
    \draw[->] (n23) -- (n24);
    \draw[->] (n24) -- (n15);
    \draw[->] (n24) -- (n35);
    \draw[->,ifFalse] (n11) -- (n01);
    \draw[->,ifFalse] (n31) -- (n41);
    \draw[->,ifFalse] (n22) -- (n42);
    \draw[->,ifFalse] (n22) -- (n02);
  \end{tikzpicture}
  {\hfill}
  \caption{The message operation.  The plans are only bound together
    if they have not already been bound.}
  \label{fig:enforce:lli_message}
\end{sanefig}

A message operation can fail in three ways:
\begin{enumerate}
\item
  The \gls{side condition} might evaluate to false.  This simply indicates
  that the data-dependent part of the \gls{verificationcondition}
  cannot be satisfied by this execution.

\item
  A plan's first, unbound, message operation might time out.  In the
  example of \autoref{fig:enforcement:example_bug}, it might be that
  some program thread executes instruction \texttt{a}, and the
  \gls{side condition} is satisfied, but no thread executes
  instruction \texttt{c} before the timeout expires.  This usually
  indicates that the program is not going to run the two desired
  fragments of code in parallel.

\item
  Later message operations can also time out.  In the example, it
  might be that \texttt{a} and \texttt{c} happen sufficiently close
  together for the first message operation to succeed, but \texttt{c}
  itself takes so long that the second message times out.  This is
  much less common.  It usually indicates that the program fragments
  being coerced contain some program-level synchronisation of their
  own and that this has deadlocked against the plan-level kind.
\end{enumerate}
Regardless of the cause of the failure, the enforcer will recover in
the same way: the plan fails and exits, along with its bound plan, if
any, and the program is allowed to return to normal operation.  This
attempt to reproduce the bug has failed and the enforcer must wait
until the program next runs the potentially-buggy fragment of code.

\subsection{The multi-plan enforcer}
\label{sect:enforce:succ}

The single-plan enforcer described in \autoref{sect:enforce:llis}
is sufficient when {\technique} is tasked with reproducing a single
potential bug with a single happens-before graph and an unambiguous
mapping between the dynamic and static \glspl{cfg}, but this is not
the common case: a single program can generate many
\glspl{bugenforcer}, each of which can generate multiple
happens-before graphs, each with potentially many mappings on to the
\gls{static cfg}.  The enforcer must track its progress through all of
these potential plans.

It is useful at this point to draw a distinction between static plans,
which are built before the program's execution starts and reflect
static properties of the program and the bug to be enforced, and
dynamic plans, which are created and destroyed as the program runs and
reflect dynamic properties of a particular execution's interactions
with one of the static plans.  Each enforcer will have a single,
fixed, set of static plans, but will maintain one set of dynamic plans
for each program thread, and these dynamic plan sets will vary over
time.  Freshly-created dynamic plans are added to the dynamic plan set
whenever a thread reaches the start instruction of one of the static
plans.  Each of these dynamic plans then cycles through the stages of
the single-plan enforcer, starting from \textbf{Sample}, until it
exits, at which point it is removed from the dynamic plan set.  If the
dynamic plan set ever becomes empty then the enforcer exits and the
program is returned to normal operation.

Most of the work of maintaining this set of dynamic plans is performed
as part of the \textbf{Succ} phase.  The \textbf{Emul} stage will
determine which \gls{static cfg} node the program thread will move to
next, but, as already noted, this can map to several dynamic ones, and
the multi-plan enforcer must somehow resolve this ambiguity.  It
does so by transforming the set of dynamic plans:
\begin{displaymath}
\mathit{newPlans} = \{p[n = n'] | p \in \mathit{plans}, n' \in p.n.\mathit{succ}, n'\!.\mathit{static} = \mathbf{Emul}.\mathit{next} \}
\end{displaymath}
Where:
\begin{itemize}
\item $\mathit{newPlans}$ is the new set of dynamic plans;
\item $\mathit{plans}$ is the current set of dynamic plans;
\item $n' \in p.n.\mathit{succ}$ is true precisely when $n'$ is a
  successor node of $p$'s current \gls{dynamic cfg} node;
\item $n'\!.\mathit{static}$ is the \gls{static cfg} node
  corresponding to the dynamic node $n'$ and
  $\mathbf{Emul}.\mathit{next}$ the next static node to execute, as
  determined by the \textbf{Emul} stage; and
\item $p[n = n']$ is a new dynamic plan constructed from $p$ by
  setting its current node in the \gls{dynamic cfg} to $n'$.
\end{itemize}
The resulting set will contain one new dynamic plan for every
combination of existing dynamic plan and \gls{dynamic cfg} node,
provided that the new \gls{cfg} node is a successor of the current
\gls{cfg} node and that the new \gls{cfg} node's raw instruction
pointer matches that produced by the \textbf{Emul} phase.  Note that
all of the successors will have the same \gls{static cfg} node; this
is necessary to correctly implement the \textbf{Emul} phase of the
next instruction cycle.

It is perhaps informative to consider what happens to the individual
dynamic plans in the input set.  There are three interesting cases:
\begin{enumerate}
\item In the common case, the mapping from static to \gls{dynamic cfg}
  nodes is unambiguous and so precisely one of the current dynamic
  node's successors will match the next instruction address produced
  by \textbf{Emul}.  In effect, all that happens is that the dynamic
  plan moves from its current node in the \gls{dynamic cfg} to that
  one successor node.

\item Alternatively, the \gls{dynamic cfg} node might not have any
  successors which match the desired static node.  This might be
  because the dynamic plan has reached the end of the static plan, in
  which case this thread's part of the plan has succeeded, or it might
  be because the program has diverged from the static plan, in which
  case the plan has failed.  In either case, the dynamic plan produces
  no successors.  If this causes the set of dynamic plans to become
  empty then the enforcer will also exit and the program thread return
  to normal operation.

\item Finally, the \gls{dynamic cfg} node might have multiple matching
  successors.  The dynamic plan will produce one new plan for each
  such successor, in effect forking its state so as to lazily resolve
  the ambiguity.
\end{enumerate}

\subsection{Message operations in the multi-plan enforcer}
\label{sect:enforce:hli_messages}

As mentioned previously, the first step in an unbound message
operation is to select a dynamic plan with which to communicate.
{\Technique}'s approach to doing so is simple: wait for some
short interval, gathering up all of the dynamic plans which reach an
appropriate unbound message operation during that time, and then fork
the local and remote dynamic plans as many times as necessary to
represent every possible pattern of communication.  The decision as to
which precise plan communicates with which is thus made lazily, in the
same way that the successor \gls{dynamic cfg} node is decided lazily
in the \textbf{Succ} phase.

\subsection{Plan interference}
\label{sect:enforce:plan_interference}

Attempting to enforce multiple crash enforcement plans at the same
time is not always without disadvantage.  In particular, plans can
sometimes interfere with each other in ways which make it more
difficult to reproduce one or both of the bugs.  Consider, for
instance, the program fragments in
\autoref{fig:enforce:plan_interference:eg}.  The first happens-before
graph requires F and G to occur between A and B while the second
requires them to occur between C and D.  {\Technique} will
encounter a deadlock while attempting to reproduce these bugs.
Suppose that the crashing thread arrives first and that the enforcer
gains control at instruction F.  It will then create two dynamic plans
$p_0$ and $p_1$, at \gls{dynamic cfg} nodes $F_0$ and $F_1$
respectively, to represent its progress through the two enforcement
plans.  Both plans start with a message receive operation, with $p_0$
receiving $\alpha_0$ and $p_1$ $\alpha_1$, and so the enforcer will
put the thread to sleep waiting for one of those messages to arrive.
Now suppose that, after some suitably short delay, the interfering
thread arrives at instruction A.  The enforcer will now gain control
of that thread and create a third dynamic plan, $p_2$, at $A_0$.
$p_2$ immediately sends message $\alpha_0$, unblocking $p_0$.  $p_1$
now has a problem: it cannot advance until the \gls{interferingthread}
reaches C, creates a $C_1$ plan, and sends message $\alpha_1$, but
that cannot happen until the \gls{interferingthread} has finished with
instruction A, which is only possible once message operation
$\alpha_0$ is complete.  The \gls{crashingthread}, meanwhile, has
effectively tied messages $\alpha_0$ and $\alpha_1$ together, and will
not complete its receive of $\alpha_0$ until something sends
$\alpha_1$.  The enforcer has deadlocked.  It will not proceed until
$p_1$'s message operation times out, causing the enforcer to abandon
that plan and continue with $p_0$ alone.

\begin{sanefig}
  {\hfill}%
  \subfigure[][\Gls{static cfg}]{
    \begin{tikzpicture}
      \node at (0,0) {\shortstack{Crashing\\thread}};
      \node at (0,-2.5) (F) [CfgInstr] {F};
      \node at (0,-3.5) (G) [CfgInstr] {G};

      \node at (2,0) {\shortstack{Interfering\\thread}};
      \node at (2,-1) (A) [CfgInstr] {A};
      \node at (2,-2) (B) [CfgInstr] {B};
      \node at (2,-4) (C) [CfgInstr] {C};
      \node at (2,-5) (D) [CfgInstr] {D};

      \draw[->] (A) -- (B);
      \draw[->] (B) -- (C);
      \draw[->] (C) -- (D);
      \draw[->] (F) -- (G);
    \end{tikzpicture}
  }
  {\hfill}%
  \subfigure[][First happens-before\\graph to enforce]{
    \begin{tikzpicture}
      \node at (0,0) {\shortstack{Crashing\\thread}};
      \node at (2,0) {\shortstack{Interfering\\thread}};

      \node at (2,-1.5) (A) [CfgInstr] {$A_0$};
      \node at (0,-2.5) (F) [CfgInstr] {$F_0$};
      \node at (0,-3.5) (G) [CfgInstr] {$G_0$};
      \node at (2,-4.5) (B) [CfgInstr] {$B_0$};

      \node at (2,-5) {\phantom{B}};

      \draw[->] (A) -- (B);
      \draw[->] (F) -- (G);
      \draw[->,happensBeforeEdge] (A) to node [above] {$\alpha_0$} (F);
      \draw[->,happensBeforeEdge] (G) to node [below] {$\beta_0$} (B);
    \end{tikzpicture}
  }
  {\hfill}%
  \subfigure[][Second happens-before\\graph to enforce]{
    \begin{tikzpicture}
      \node at (0,0) {\shortstack{Crashing\\thread}};
      \node at (2,0) {\shortstack{Interfering\\thread}};
      \node at (2,-1.5) (A) [CfgInstr] {$C_1$};
      \node at (0,-2.5) (F) [CfgInstr] {$F_1$};
      \node at (0,-3.5) (G) [CfgInstr] {$G_1$};
      \node at (2,-4.5) (B) [CfgInstr] {$D_1$};

      \node at (2,-5) {\phantom{B}};

      \draw[->] (A) -- (B);
      \draw[->] (F) -- (G);
      \draw[->,happensBeforeEdge] (A) to node [above] {$\alpha_1$} (F);
      \draw[->,happensBeforeEdge] (G) to node [below] {$\beta_1$} (B);
    \end{tikzpicture}
  }%
  {\hfill}~%
  \caption{An example of plan interference.  The rest of the program
    contains branches to instructions {\rm A} and {\rm F} but not to
    any of the other instructions shown here.  The program's structure
    means that enforcing the first happens-before graph makes it
    difficult, but not impossible, to enforce the other.}
  \label{fig:enforce:plan_interference:eg}
\end{sanefig}

This is a less serious problem than might it might at first appear.
If the first plan succeeds then the program will crash anyway and the
behaviour of the second plan becomes moot, so assume that the first
plan fails for some reason.  The \gls{interferingthread} will then
continue to advance through the program's \gls{cfg} until it reaches
instruction C, at which point the enforcer will again gain control and
create another dynamic plan $p_3$ starting at $C_1$.  This plan will
then wait to try to send message $\alpha_1$.  If another thread
reaches instruction F then these two threads will bind together and
attempt the second plan, and, this time, there is no risk of deadlock.
The enforcer will thus, if the buggy code runs sufficiently
frequently, alternate between the two crash enforcement plans.
Lacking any other information, that is a perfectly reasonable strategy
for reproducing these bugs.  Similar deadlocks in more complicated
bugs can lead to enforcers behaving quite inefficiently, in the sense
that they either impose many pointless delays or make poor use of rare
events, but will not usually render any of the graphs completely
unenforceable.

\section{Discussion}

The enforcer mechanism described here allows {\technique} to shepherd
programs towards schedules which the prior analysis identified as
potentially dangerous, and hence to determine which suspected bugs can
actually reproduce.  This eliminates all of the (many) false positives
produced by the basic analysis described in \autoref{sect:derive}.
Its main weakness is that it also eliminate some true positives: as a
run-time technique, \glspl{bugenforcer} only consider parts of the
program which are exercised during the shepherded execution, and so
cannot distinguish between behaviour which \emph{cannot} happen and
behaviour which merely \emph{did not} happen.

A more fundamental problem with \glspl{bugenforcer} is what happens
when they \emph{do} work: the program crashes.  While reliable
reproduction of bugs is often useful in a development environment, it
is highly undesirable in a production one.  The next chapter
demonstrates a different, related, technique which instead makes bug
reproduction much less likely, in many cases eliminating the bug
completely.
