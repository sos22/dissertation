\chapter{Future work and conclusions}

\section{Future work}

This dissertation leaves a lot of potential refinements to
{\technique} uninvestigated.  Some of these are straightforward
refinements to {\implementation}: implementing more {\StateMachine}
library models, or considering alternative architectures, or improving
performance, or generalising to more types of bugs.  Others involve
more fundamental changes to {\technique} itself.  I describe some
possible ideas below.

\subsection{Finding bugs}

The initial analysis which builds \glspl{verificationcondition} is
probably the weakest part of {\technique}, as it is only capable of
considering instruction interleavings over fairly small windows of
instructions.  The obvious way of improving that would be to re-cast
the algorithm to take as input either source code or an intermediate
format such as LLVM bitcode\needCite{}.  These provide the analysis
with far more information than raw machine code and so might
reasonably be expected to expand the fragments of program which it can
consider.  Alternatively, the \gls{programmodel} could be extended
with more powerful decompilation-style analyses, which would achieve
much the same effect.

\subsection{Reproducing bugs}

As described in \autoref{sect:reproduce:unenforcable}, {\technique}
cannot enforce every possible happens-before graph.  The fundamental
problem is that rather than enforcing a requirement that $A$ happens
before $B$, {\technique} enforces the stronger requirement that $A$
happen \emph{immediately} before $B$.  This introduces additional edges
into the happens-before graph, and if one of those edges closes a
cycle then the graph becomes unenforcible.  In other words, the
problem is that {\technique} uses synchronous messages to enforce what
should be an asynchronous ordering constraint.

Switching to asynchronous messages would eliminate this problem.
Rather than having threads synchronise at message operations, the
sending thread would instead generate and publish a message structure
which could be collected by the receiving thread.  Any side-condition
would be evaluated by the receiving thread, and any needed
cross-thread state would be included as a payload in the message
structure.  There would then be no need for the sending thread to wait
for the receiving one, and hence no risk of introducing cycles in the
happens-before graph.  The main disadvantage of this scheme would be
the greater implementation complexity in the enforcer itself.

\subsection{Fixing bugs}

I have described how {\technique} can fix bugs by introducing an
additional global lock.  This is effective but inefficient.  In
particular, it can eliminate large numbers of safe instruction
interleavings, leading to an unnecessary loss of concurrency.  This is
an irritating limitation, as the \glspl{verificationcondition} specify
precisely which interleavings are safe.  The fix generation process
described in \autoref{sect:fix_global_lock} completely discards this
information.  It would be interesting to investigate alternative
schemes which make better use of it.
  
\begin{wrapfigure}{O}{4cm}
  \vspace{-12pt}
  \begin{mdframed}
  \centerline{
  \begin{tikzpicture}
    \node (LD1) {Load 1};
    \node [right = 1.5 of LD1] (dummy) {};
    \node [below = of dummy] (ST1) {Store 1};
    \node [below = of ST1] (ST2) {Store 2};
    \node [below = 0.9 of ST2] (dummy2) {};
    \node at (dummy2 -| LD1) (LD2) {Load 2};
    \draw[->] (LD1) -- (LD2);
    \draw[->] (ST1) -- (ST2);
    \draw[->,happensBeforeEdge] (LD1) to node [above,sloped] {$i_1 = i_2$} (ST1);
    \draw[->,happensBeforeEdge] (ST2) -- (LD2);
  \end{tikzpicture}
  }
  \caption{}
  \label{fig:concl:hb_graph}
  \end{mdframed}
  \vspace{-12pt}
\end{wrapfigure}
One approach to doing so makes use of a kind of extended hazard
pointer\cite{Michael2004}.  Consider, for example, the threads in
Figure~\ref{fig:concl:hb_graph}.  This is intended to illustrate a
construction in which there is one thread which issues two loads and
another which issues two stores, and the program will crash if both
stores occur in between the two loads and the variable $\mathit{idx}$
has the same value in both threads.  One way of preventing this bug
would be to have the first thread publish a ``hazard'', containing the
value of its $\mathit{idx}$ variable, before running the first load
and clear it after the second one.  The second thread could then check
for such hazards before running the first store and, if it finds one
with a matching $\mathit{idx}$, make sure to wait for the hazard to
clear before running the second one.  This would fix the bug and allow
the two threads to run in parallel for most of the critical section,
or for all of it when the $\mathit{idx}$s differ.  The loss of
concurrency associated with the fix would therefore usually be much
lower.  On the other hand, managing the hazards would potentially be
quite complex, which might itself lead to reduced performance in some
cases.

\section{Conclusions}

I have described {\technique}, a set of techniques for finding,
reproducing, and fixing synchronisation bugs starting from a runnable
copy of the program, without access to the source and with minimal
manual user intervention.  I have also described my prototype
implementation, {\implementation}, and discussed in detail its
performance and effects on a variety of real and artificial test bugs
and programs.  The result is that {\technique} is highly effective at
all of its goals when used on the artificial test cases, and performs
well enough to be useful for some more realistic tests taken from very
large existing pieces of software.  I have also demonstrated some of
the limitations of {\technique} in its current form; primarily, that
it struggles to solve the aliasing problems necessary to handle more
complicated bugs.  Finally, I have compared {\technique} to a variety
of existing systems, highlighting similarities and differences.

The most important contribution of this work is the crash enforcement
mechanism, which can take a description of a concurrency bug in a
particular format and convert it into a modified version of the
program which is far more likely to exhibit the bug.  This process can
take advantage of any side-conditions necessary for the bug to
reproduce, over and above the actual concurrency pattern required,
allowing it to reproduce the bug far more quickly and reliably than
simpler schedule enforcement schemes.  The additional computational
time necessary to generate these enforcers, given the bug description,
is modest, and easily outweighed by the reduction in reproduction
time.

A variant of the same technique can also be used to fix bugs.  These
fixes have very low overhead, comparable to that which could be
obtained by a hand-crafted fix for the same bug, and can be deployed
to protect existing programs very easily.  On the other hand, the set
of bugs which can be completely protected by these fixes is somewhat
smaller than the set which can be easily reproduced.  Nevertheless, I
have demonstrated that this technique can be used to generate
effective and low-overhead fixes for at least some concurrency errors
in real programs.

I also outlined a new scheme for finding concurrency-related bugs in
binary programs.  The results of this part of the dissertation are
perhaps less convincing; it found very few bugs in real programs, and
is computationally very expensive.  I briefly discussed how combining
it with existing work might help to improve on these weaknesses.  Even
without those possible future refinements, the algorithm is
embarrassingly parallel, and so its performance is likely to improve
as hardware execution facilities become more parallel; precisely the
situation in which it would be most useful.
