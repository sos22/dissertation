\section{Things to evaluate}

What I have:

\begin{itemize}
\item About a dozen artificial bugs for each mode
\item The glibc bug kernel --- fixed in DRS mode
\item The Thunderbird bug -- fixed in DRS mode
\item The first mysql bug -- fixed in core dump mode
\item The second mysql bug -- found and fixed in whole-pass mode
\end{itemize}

And that's about it.
Where I've said a bug is fixed or found in one mode that indicates that I've tested and confirmed that it works in that mode; it doesn't say anything about whether it would work in one of the other modes.

So I desperately need more real bugs.
Possible ways of getting them:

\begin{itemize}
\item
  enforce\_crash.
  I need to go and apply this to as many programs as possible, which will hopefully give me something more to work with.
\item
  Take an existing program and rip out the synchronisation, then see if we can find the resulting bugs.
  STAMP would be a reasonable place to look here.
\item
  Another bug tracker crawl.
  This would be bloody tedious but might turn up something.
\end{itemize}

Other things I can look at and evaluate:

\begin{itemize}
\item
  Turn the various simplification passes on and off and see what effect that has.
\item
  CDF of how long analysing a given potentially-crashing instruction takes.
\item
  Look at different strategies for combining machines.
\item
  Look at effect of different analysis window sizes: how many bugs do we find, how many false positives, how much time does the analysis take.
\item
  Compare the CFG-based approach to building machines to the old backtracking one.
\item
  Direct eval of the effectiveness of the induction rule.
\item
  Look at how many bugs are killed off by the different clauses of the crash definition.
\item
  Cross-eval of the accuracy of the static analysis passes by doing a dynamic analysis which checks their predictions.
\item
  Investigate performance hit of the enforcement mode, and how that depends on the mechanism used to generate the enforcers.
\item
  Compare compiled enforcers to interpreted ones.
\item
  How much does the W isolation property actually buy us, in terms of analysis cost?
  How much does it cost us, in terms of lost bugs?
\item
  How large are the CFGs and \StateMachines which we actually generate?
  Also, how large are the intermediate machines generated when we're generating verification conditions?
\item
  Do we actually win anything from using the slightly odd form of SSA?
\item
  We handle multi-rooted CFGs completely differently between read and write modes.
  Does that win/lose anything?
\item
  There's almost certainly something to say about the dynamic analysis phase, probably to do with how quickly it converges.
\item
  Might also be able to say something about how complex the resulting aliasing tables are.
\item
  Look at how long the various phases take.
\item
  There are a bunch of places where we have to decide between using the simplifier and the sat checker.
  Eval which works better for each of them.
\item
  When you're doing the cross-product machines you have a choice between explicit and implicit happens-before edges.
  Explicit is basically always better; figure out how much by, and why.
\item
  Crash enforcement: timeout strategies.
\item
  Crash enforcement: how useful is side-condition checking?
\item
  Crash enforcement: how often do we redundantly evaluate side conditions?
\item
  Crash enforcement: Is await-bound-exit actually useful?
  We know it is in silly little artificial test cases, but perhaps not in real programs?
\end{itemize}
