\chapter{Evaluation}
\label{chapter:eval}

\section{Experiments I still need to run}

\begin{itemize}
\item Re-do unoptimised MySQL.  Time static and main analysis phases.
  Check number of candidates generated.  Run candidates.
  \todo{Running on hogun now.}
\item Re-do optimised MySQL.  Time static and main analysis phases.
  Check number of candidates generated.  Run candidates.
\end{itemize}

\section{Eval}

Previous chapters have described the basic {\technique} technique.  I
will now evaluate its effectiveness, and the performance of my
implementation {\implementation}.  This evaluation will consist of the
following parts:

\begin{itemize}
\item Section~\ref{sect:eval:artificial} explores the behaviour of the
  tool and the technique on a number of artificial bugs in simple test
  programs.  This includes a comparison to a
  DataCollider\needCite{}-like tool which I implemented for the
  purpose of the evaluation.
\item Section~\ref{sect:eval:semiartificial} investigates some
  slightly more realistic bugs.  These include a simplified version of
  a real bug in glibc\needCite{} and some bugs which I deliberately
  introduced into the STAMP benchmark suite\needCite{}.
\item Section~\ref{sect:eval:real} applies the tool to some larger
  programs: pbzip2\needCite{}, MySQL\needCite{}, and
  Thunderbird\needCite{}.  I show that the analysis completes in a
  tolerable amount of time even for some very large programs, and
  demonstrate that it can both reproduce and fix a (small) number of
  real bugs.
\item Section~\ref{sect:eval:time_details} then investigates the
  tool's performance on large programs in slightly more detail,
  showing how the time taken breaks down across the various phases of
  the analysis, and the effects of some of the input parameters.
\item Section~\ref{sect:eval:w_isolation} explores the effects of the
  W isolation assumption, showing how making the assumption allows the
  analysis to complete more quickly and how it affects the number of
  \glspl{verificationcondition} discovered.
\item Section~\ref{sect:eval:dynamic_analysis} investigates the
  performance and effectiveness of the dynamic aliasing analysis.
  This includes how much it benefits the main analysis, how long it
  needs to run for to achieve reasonable coverage, and the performance
  overhead incurred by the program under investigation while it is
  running.
\end{itemize}

\section{Artificial bugs}
\label{sect:eval:artificial}

I now present the results of running {\implementation} on a number of
artificial bugs, showing that it assists in both reproducing the bugs
and fixing them, and that the analysis phases complete very quickly.

\subsection{Simple time-of-check, time-of-use (TOCTOU) bug (simple\_toctou)}\footnote{This bug was previously discussed in
  Section~\ref{sect:derive:simple_toctou_example}.}
\label{sect:eval:simple_toctou}


\begin{figure}
  \subfigure[][Crashing thread]{
    \texttt{
      \begin{tabular}{lll}
        \multicolumn{3}{l}{while (1) \{}\\
        &\multicolumn{2}{l}{STOP\_ANALYSIS();}\\
        &\multicolumn{2}{l}{if (global\_ptr != NULL) \{}\\
        &&*global\_ptr = 5;\\
        &\multicolumn{2}{l}{\}}\\
        &\multicolumn{2}{l}{STOP\_ANALYSIS();}\\
        \multicolumn{3}{l}{\}}\\
      \end{tabular}
    }
  }\hfill %
  \subfigure[][Interfering thread]{
    \texttt{
      \begin{tabular}{ll}
        \multicolumn{2}{l}{while (1) \{}\\
        &global\_ptr = \&t;\\
        &sleep(1 second);\\
        &STOP\_ANALYSIS();\\
        &global\_ptr = NULL;\\
        &STOP\_ANALYSIS();\\
        \multicolumn{2}{l}{\}}\\
      \end{tabular}
    }
  }\hfill
  \caption{The two sides of the simple\_toctou bug.}
  \label{fig:eval:simple_toctou}
\end{figure}

\begin{figure}
  \subfigure[][Crashing thread]{
    \texttt{
    \begin{tabular}{rlll}
              & \multicolumn{3}{l}{crashing\_thread:} \\
      400694: & movq  & global\_ptr, &\!\!\!\%rax\\
      40069b: & testq & \%rax,       &\!\!\!\%rax \\
      40069e: & je    & \multicolumn{2}{l}{4006ad}\\
      4006a0: & movq  & global\_ptr, &\!\!\!\%rax\\
      4006a7: & movl  & \$0x5,       &\!\!\!(\%rax)\\
    \end{tabular}
    }
  }%
  \hspace{-5mm}\subfigure[][Interfering thread]{
    \texttt{
      \begin{tabular}{rlll}
        & \multicolumn{3}{l}{interfering\_thread:} \\
        400816: & lea  & c(\%rsp), &\!\!\!\%rbp \\
        ...\\
        400884: & movq & \%rbp, &\!\!\!global\_ptr\\
        ...\\
        4008fb: & movq & \$0x0, &\!\!\!global\_ptr\\
      \end{tabular}
      }
    }
  \caption{Disassembly of the program fragments in Figure~\ref{fig:eval:simple_toctou}.}
  \label{fig:eval:simple_toctou:compiled}
\end{figure}

This is the simplest possible kind of concurrency bug: a
single-variable time-of-check, time-of-use race.  The two threads
involved are shown in Figure~\ref{fig:eval:simple_toctou}.  The intent
here is to model a very simple structure which is accessed frequently
but updated rarely.  The bug is, of course, that the interfering
thread might set \texttt{global\_ptr} to \texttt{NULL} in between the
two reads of it in the crashing thread, causing the crashing thread to
crash when it dereferences the pointer it loaded.
\texttt{STOP\_ANALYSIS()} is a marker which prevents {\technique}'s
analysis from exploring past that point when building the CFGs from
which {\StateMachines} are constructed.  It is used here to help keep
the crash summaries generated simple and easily explained;
{\implementation} is able to reproduce and fix the bug even without
those markers.

\subsubsection{Generating candidate bugs}

The first step in the analysis is to build the CFG for the
\gls{crashingthread} (see Section~\ref{sect:derive:build_static_cfg}),
which in this case is straightforward; the result is shown in
Figure~\ref{fig:eval:simple_toctou:cfg}.  This can then be compiled to
produce the {\StateMachine} shown in
Figure~\ref{fig:eval:simple_toctou:sm}.

\begin{figure}
  \centerline{
  \subfigure[][Control flow graph]{
    \centering
    \begin{tikzpicture}
      \node (cfg1) [CfgInstr] {\texttt{400694}: cfg1};
      \node (cfg2) [CfgInstr,below=of cfg1] {\texttt{40069b}: cfg2};
      \node (cfg3) [CfgInstr,below=of cfg2] {\texttt{40069e}: cfg3};
      \node (cfg3b) [right = of cfg3] {$\varnothing$};
      \node (cfg4) [CfgInstr,below=of cfg3] {\texttt{4006a0}: cfg4};
      \node (cfg4b) [below = of cfg4] {$\varnothing$};
      \draw[->] (cfg1) -- (cfg2);
      \draw[->] (cfg2) -- (cfg3);
      \draw[->,ifTrue] (cfg3) -- (cfg3b);
      \draw[->,ifFalse] (cfg3) -- (cfg4);
      \draw[->] (cfg4) -- (cfg4b);
    \end{tikzpicture}
    \label{fig:eval:simple_toctou:cfg}
  }
  {\hfill}
  \subfigure[][{\STateMachine}]{
    \begin{tikzpicture}
      \node (l1) at (0,2) [stateSideEffect] {\stLoad{1}{\mathrm{global\_ptr}} @ cfg1 };
      \node (l2) [stateIf, below=of l1] {\stIf{\smTmp{1} = 0}};
      \node (l4) [stateSideEffect, below=of l2] {\stLoad{2}{\mathrm{global\_ptr}} @ cfg4 };
      \node (l3) [stateTerminal, right=of l4] {\stSurvive };
      \node (l5) [stateIf, below=of l4] {\stIf{\smBadPtr{\smTmp{2}}}};
      \node (l6) [stateTerminal, below=of l5] {\stCrash};
      \draw[->] (l1) -- (l2);
      \draw[->,ifTrue] (l2) -- (l3);
      \draw[->,ifFalse] (l2) -- (l4);
      \draw[->] (l4) -- (l5);
      \draw[->,ifFalse] (l5) -- (l3);
      \draw[->,ifTrue] (l5) -- (l6);
    \end{tikzpicture}
    \label{fig:eval:simple_toctou:sm}
  }
  }
  \caption{\Gls{cfg} and {\STateMachine} for the crashing thread in
    Figure~\ref{fig:eval:simple_toctou:compiled}.  $\varnothing$
    indicates that the thread has left the \gls{cfg}. Dotted lines
    indicate the false successor of a conditional.}
\end{figure}

The next step is to build the \glspl{interferingstore} set (see
Section~\ref{sect:derive:write_side}).  The crashing {\StateMachine}
contains two loads, at \texttt{400694} and \texttt{4006a0}, and so the
analysis will query the \gls{programmodel} to find what stores might
interact with them, and the \gls{programmodel} will use the results of
the dynamic alias analysis (see
Section~\ref{sect:program_model:dynamic_alias}) to return the set
\{\texttt{400884}, \texttt{4008fb}\}.  The \texttt{STOP\_ANALYSIS()}
markers prevent these from being clustered together, and there are no
other \glspl{communicatinginstruction}, and so there will be two
\glspl{interferingthread} \glspl{cfg}, shown in
Figures~\ref{fig:eval:simple_toctou:interfering_cfg1}
and~\ref{fig:eval:simple_toctou:interfering_cfg2}.

\begin{figure}
  \begin{tabular}{b{0.5\textwidth}b{0.5\textwidth}}
    \subfigure[][CFG for interfering store \texttt{400884}]{
      \centerline{
      \begin{tikzpicture}
        \node (a) [CfgInstr] {\texttt{400884}: cfg5};
        \node (b) [below = of a] {$\varnothing$};
        \draw[->] (a) -- (b);
      \end{tikzpicture}
      }
      \label{fig:eval:simple_toctou:interfering_cfg1}
    } &
    \subfigure[][{\STateMachine} for interfering store \texttt{400884}, without \gls{programmodel}]{
      \centerline{
      \begin{tikzpicture}
        \node [stateSideEffect] {\stStore{\smReg{rbp}{2}}{\mathrm{global\_ptr}} @ cfg5};
      \end{tikzpicture}
      }
      \label{fig:eval:simple_toctou:interfering_sm1}
    } \\
    \subfigure[][CFG for interfering store \texttt{4008fb}]{
      \centerline{
      \begin{tikzpicture}
        \node (a) [CfgInstr] {\texttt{4008fb}: cfg6};
        \node (b) [below = of a] {$\varnothing$};
        \draw[->] (a) -- (b);
      \end{tikzpicture}
      }
      \label{fig:eval:simple_toctou:interfering_cfg2}
    } &
    \subfigure[][{\STateMachine} for interfering store \texttt{4008fb}]{
      \centerline{
      \begin{tikzpicture}
        \node [stateSideEffect] {\stStore{0}{\mathrm{global\_ptr}} @ cfg6};
      \end{tikzpicture}
      }
      \label{fig:eval:simple_toctou:interfering_sm2}
    }
  \end{tabular}
  \caption{Interfering CFGs and {\StateMachines}.}
\end{figure}

Consider the interfering store at \texttt{400884} first.  Without the
\gls{programmodel}, this would produce the {\StateMachine} shown in
Figure~\ref{fig:eval:simple_toctou:interfering_sm1}.  There is no way
for the program to crash due to interleaving this {\StateMachine} with
the crashing thread, as $\smReg{rbp}{2}$ is never a bad pointer, but
the {\StateMachines} do not contain enough information to show that.
The analysis will produce the verification condition shown in
Figure~\ref{fig:eval:simple_toctou:inferred_assumption1}. The
\gls{programmodel}'s static analyses, on the other hand, are
sufficient to show that $\smReg{rbp}{2}$ is a valid pointer,
eliminating the spurious \gls{verificationcondition}.  The
\gls{interferingstore} at \texttt{4008fb}, on the other hand, does
represent a valid bug, as interleaving it with the
\gls{crashingthread} could lead to a crash, and it produces the
\gls{verificationcondition} shown in
Figure~\ref{fig:eval:simple_toctou:inferred_assumption2}.

\begin{figure}
  \begin{tabular}{lll}
    \gls{ci-atomic}: & $\smLoad{\mathrm{global\_ptr}} = 0$ &\!\!\!$\vee\,\, \neg\smBadPtr{\smLoad{\mathrm{global\_ptr}}}$ \\
    \gls{ic-atomic}: & $\smReg{rbp}{2} = 0$                &\!\!\!$\vee\,\, \neg\smBadPtr{\smReg{rbp}{2}}$\\
    \Gls{verificationcondition}: & \multicolumn{2}{l}{$\happensBefore{\mai{cfg1}{1}}{\mai{cfg5}{2}} \wedge \happensBefore{\mai{cfg5}{2}}{\mai{cfg4}{2}} \wedge \smBadPtr{\smReg{rbp}{2}}  \wedge$}\\
                                      & $\smLoad{\mathrm{global\_ptr}} \not= 0$\\
  \end{tabular}
  \caption{\Gls{inferredassumption} and \gls{verificationcondition}
    produced using the crashing {\StateMachine} in
    Figure~\ref{fig:eval:simple_toctou:sm} and the interfering
    {\StateMachine} in
    Figure~\ref{fig:eval:simple_toctou:interfering_sm1}.}
  \label{fig:eval:simple_toctou:inferred_assumption1}
\end{figure}

\begin{figure}
  \begin{tabular}{lll}
    \gls{ci-atomic}: & $\smLoad{\mathrm{global\_ptr}} = 0$ &\!\!\!$\vee\,\, \neg\smBadPtr{\smLoad{\mathrm{global\_ptr}}}$ \\
    \gls{ic-atomic}: & \true\\
    \Gls{verificationcondition}: & \multicolumn{2}{l}{$\happensBefore{\mai{cfg1}{1}}{\mai{cfg6}{2}} \wedge \happensBefore{\mai{cfg6}{2}}{\mai{cfg4}{2}} \wedge \smLoad{\mathrm{global\_ptr}} \not= 0$}\\
  \end{tabular}
  \caption{\Gls{inferredassumption} and \gls{verificationcondition}
    produced using the crashing {\StateMachine} in
    Figure~\ref{fig:eval:simple_toctou:sm} and the interfering
    {\StateMachine} in
    Figure~\ref{fig:eval:simple_toctou:interfering_sm2}.}
  \label{fig:eval:simple_toctou:inferred_assumption2}
\end{figure}

The time taken to perform this analysis is quite modest: $0.52 \pm
0.04$ seconds for the static analysis phase and $0.18 \pm 0.01$
seconds for the {\StateMachine} analysis (mean and standard deviation
of mean for ten runs in both cases).  The dynamic analysis phase
achieved complete coverage essentially as soon as the program started.
This is a reasonable lower bound on the time which {\technique} will
take to process a very simple bug; any realistic program will take far
longer than this to process.

\subsubsection{Reproducing the bug}
This \gls{verificationcondition} can now be turned into a
\gls{bugenforcer}.  The only happens-before graph will be the one
shown in Figure~\ref{fig:eval:simple_toctou:hb_graph}, and it will
have the side condition that $\smLoad{\mathrm{global\_ptr}} \not= 0$.
This side condition can be evaluated completely at either
$\mai{cfg1}{1}$ or $\mai{cfg6}{2}$, and so the crash enforcement plan
will be as shown in Figure~\ref{fig:eval:simple_toctou:enforce_plan}.
In other words, whenever a program thread reaches \texttt{400694} and
global\_ptr is non-zero, the enforcer will wait for an interfering
thread to reach \texttt{4008fb}, after loading global\_ptr.  If one
does arrive, the enforcer will make the crashing thread wait for the
interfering thread to complete its store before proceeding to the load
at \texttt{4006a0}.  This will be sufficient to reproduce the bug.

\begin{wrapfigure}{r}{7cm}
  \begin{tikzpicture}
    \node[draw] (l1) {\texttt{400694}: $\mai{cfg1}{1}$};
    \node[draw, below right = of l1] (l2) {\texttt{4008fb}: $\mai{cfg6}{2}$ };
    \node[draw, below left = of l2] (l3) {\texttt{4006a0}: $\mai{cfg4}{1}$ };
    \draw[->] (l1) -- (l3);
    \draw[->, happensBeforeEdge] (l1) -- (l2);
    \draw[->, happensBeforeEdge] (l2) -- (l3);
  \end{tikzpicture}
  \caption{Happens-before graph to be enforced for simple\_toctou}
  \label{fig:eval:simple_toctou:hb_graph}
\end{wrapfigure}
This enforcer is effective at reproducing the bug.  Without the
enforcer, the mean time taken to reproduce the bug is $4.3 \pm 0.4$
seconds, mean and standard deviation of mean for 100 runs; with it,
the mean time was $1.158 \pm 0.003$.  Most of this reduction is caused
by eliminating outliers from the distribution: the median time to
reproduce the bug decreased from 3.0 seconds to 1.2, a 2.5-fold
reduction, while the $95^{th}$ percentile was reduced from 12 seconds
to 1.2, a ten-fold reduction.
Figure~\ref{fig:eval:crash_cdf:simple_toctou} shows the complete CDFs
for both configurations.

Even if the enforcer had failed to reduce the time to reproduction,
this increase in predictability would itself be a useful property.
Consider a programmer attempting to fix this bug.  Their basic
approach is likely to be some variant of this procedure:

\begin{itemize}
\item Form a hypothesis as to the cause of the bug.
\item Produce a fix based on that hypothesis.
\item Test whether the bug has been fixed.
\item If so, stop.  Otherwise, start again from the beginning.
\end{itemize}

\begin{figure}
  \begin{tikzpicture}
    \node[draw] (l1) {\texttt{400694}: $\mai{cfg1}{1}$};
    \node[left = 0 of l1] {$\smLoad{\mathrm{global\_ptr}} \not= 0$};
    \node[draw, below right = of l1] (l2) {\texttt{4008fb}: $\mai{cfg6}{2}$ };
    \node[right = 0 of l2] {$\smLoad{\mathrm{global\_ptr}} \not= 0$};
    \node[draw, below left = of l2] (l3) {\texttt{4006a0}: $\mai{cfg4}{1}$ };
    \draw[->] (l1) -- (l3);
    \draw[->, happensBeforeEdge] (l1) -- (l2);
    \draw[->, happensBeforeEdge] (l2) -- (l3);
  \end{tikzpicture}
  \caption{Crash enforcement plan for simple\_toctou}
  \label{fig:eval:simple_toctou:enforce_plan}
\end{figure}

Bugs with a long tail of reproduction times make the third step
difficult.  The obvious way of testing whether a bug has been fixed is
to simply run the program and see if it reproduces, but if the time
taken for the bug to reproduce is highly unpredictable then it is
difficult to decide how long to run the program for before concluding
that the bug has been fixed.  {\Technique} cannot hope to completely
such unpredictability, but, qualitatively, a bug where the $95^{th}$
percentile differs from the $5^{th}$ percentile by a tenth of a
percent is likely to be far easier to work with than one where they
differ by a factor of twelve\footnote{The obvious alternative strategy
  of measuring what fraction of program runs reach some cut-off time
  is also ineffective.  For this test program, the optimal cut-off
  time for that strategy is 3.003 seconds, and it would have to be run
  four times to achieve a p-value of 0.95, for a total time of
  slightly over 12 seconds; almost precisely the same time as would be
  necessary for the na\"{i}ve strategy of just running the program
  once to reach the same p value.}.

The time taken by the analysis is quite reasonable here.  The
\gls{verificationcondition} generation step takes $0.77 \pm 0.02$
seconds (mean and standard deviation of ten runs), including the
initial static analysis phase, and converting it to a
\gls{bugenforcer} takes $0.13 \pm 0.01$ seconds.  Real programs would,
of course, take far longer.

\subsubsection{Fixing the bug}
{\Implementation} can also generate a fix for this bug.  In this case,
the CFG fragments to be protected will be the complete CFGs shown in
Figures~\ref{fig:eval:simple_toctou:cfg}
and~\ref{fig:eval:simple_toctou:interfering_cfg2}, as the first and
last CFG nodes in both threads are involved in happens-before edges.
This corresponds to modifying the program as shown in
Figure~\ref{fig:eval:simple_toctou:fix}.  This correctly fixes the
bug.

\begin{figure}
  \centerline{
    {\hfill}
  \subfigure[][Crashing thread]{
    \texttt{
      \begin{tabular}{lll}
        \multicolumn{3}{l}{while (1) \{}\\
        &\multicolumn{2}{l}{STOP\_ANALYSIS();}\\
        &\multicolumn{2}{l}{acquire\_lock();}\\
        &\multicolumn{2}{l}{if (global\_ptr != NULL) \{}\\
        &&t = global\_ptr;\\
        &&release\_lock();\\
        &&*t = 5;\\
        &\multicolumn{2}{l}{\}}\\
        &\multicolumn{2}{l}{STOP\_ANALYSIS();}\\
        \multicolumn{3}{l}{\}}\\
      \end{tabular}
    }
  }
  {\hfill}
  \subfigure[][Interfering thread]{
    \texttt{
      \begin{tabular}{ll}
        \multicolumn{2}{l}{while (1) \{}\\
        &global\_ptr = \&t;\\
        &sleep(1 second);\\
        &STOP\_ANALYSIS();\\
        &acquire\_lock();\\
        &global\_ptr = NULL;\\
        &release\_lock();\\
        &STOP\_ANALYSIS();\\
        \multicolumn{2}{l}{\}}\\
      \end{tabular}
    }
  }\hfill
  }
  \caption{The fix generated by {\implementation} for the simple\_toctou bug.}
  \label{fig:eval:simple_toctou:fix}
\end{figure}

It does, however, have a rather high performance overhead: without a
fix, the crashing thread completes $352.5 \pm 0.2 {\times} 10^6$
iterations of the loop per second; with one, it completes $95.9 \pm
0.2 {\times} 10^6$ (mean and standard deviation of ten runs each of
ten seconds, discarding any runs in which the unfixed program
crashed).  Neither distribution suffered from a noticeable long tail
in either direction (checked using the Anderson-Darling and
Jarque-Bera tests at the 90\% level and by manual inspection).  That
gives an overhead of roughly a factor of 3.7.  This is obviously
rather large, but is probably close to {\technique}'s worst case: the
read-side critical section is very small and runs with very high
frequency, and so the patch must acquire and release the lock with
similarly high frequency and these lock operations dominate the time
taken.  Any realistic test would usually have much lower overhead,
assuming lock contention does not become a factor, simply because the
critical sections would run less frequently and the overhead could be
more effectively amortised.  Even in this case, a factor of four
overhead is not completely unreasonable when the alternative is a
program which crashes frequently.

\todo{Worst case ignoring the loss of concurrency, of course.}

For comparison, I also produced a version of the patch which does
everything except for acquiring and releasing the lock.  This version
completed $349.2 \pm 0.3 {\times} 10^6$ loops per second, and so in
this case the slow down caused by the patch was a little less than
1\%.  This strongly suggests that the overhead in this case is mostly
caused by the lock operations themselves, rather than the {\technique}
infrastructure.

\begin{table}
  \begin{tabular}{lll}
                     & Gain control with breakpoints & Gain control with branches \\
    Locking enabled  & $6.08 \pm 0.01$               & $95.9 \pm 0.2$\\
    Locking disabled & $6.38 \pm 0.01$               & $349.2 \pm 0.3$\\
  \end{tabular}
  \caption{Performance of some variants of the fix, in millions of
    loop iterations completed per second.  The original program
    completed $352.5 \pm 0.2$ million iterations per second, when it
    did not crash.  All measurements mean and population standard
    deviation of ten runs.}
  \label{table:eval:simple_toctou:other_fixes}
\end{table}

As a further point of comparison, I also produced a version of this
fix which used debug breakpoints to gain control of the program rather
than branches.  This completed $6379000 \pm 7000$ loop iterations per
second, even with the actual locking disabled, giving it an overhead
of roughly a factor of 55.  This would be problematic in a production
environment.

\todo{Give details of the machine the test is running on.}\smh{Yes}

\subsection{Indexed TOCTOU bug (indexed\_toctou)}\footnote{This bug was used as an example in Section~\ref{sect:reproducing_bugs}.}

\label{sect:eval:indexed_toctou}

In this variant of a TOCTOU bug, there are multiple instances of the
structure which is being raced on and the bug will only manifest if
the reading and writing threads happen to coincide.  This bug
exercises the side condition-checking part of {\technique}'s crash
enforcers.  The code involved in the race is shown in
Figure~\ref{fig:eval:indexed_toctou}.  Except where otherwise noted,
\verb|NR_PTRS| is set to 100.

\begin{figure}
  \subfigure[][Crashing thread]{
    \texttt{
      \begin{tabular}{lll}
        \multicolumn{3}{l}{while (1) \{}\\
        &\multicolumn{2}{l}{idx = random() \% NR\_PTRS;}\\
        &\multicolumn{2}{l}{STOP\_ANALYSIS();}\\
        &\multicolumn{2}{l}{if (global\_ptrs[idx] != NULL) \{}\\
        &&*(global\_ptrs[idx]) = 5;\\
        &\multicolumn{2}{l}{\}}\\
        &\multicolumn{2}{l}{STOP\_ANALYSIS();}\\
        \}\\
      \end{tabular}
    }
  }%
  \subfigure[][Interfering thread]{
    \texttt{
      \begin{tabular}{ll}
        \multicolumn{2}{l}{while (1) \{}\\
        & idx = random() \% NR\_PTRS;\\
        & STOP\_ANALYSIS();\\
        & global\_ptrs[idx] = NULL;\\
        & STOP\_ANALYSIS();\\
        & global\_ptrs[idx] = \&t;\\
        \multicolumn{2}{l}{\}}\\
      \end{tabular}
    }
  }
  \caption{The two sides of the indexed\_toctou bug.}
  \label{fig:eval:indexed_toctou}
\end{figure}

As with the simple\_toctou test, this test produces a single candidate
bug, with a similar enforcer and fix.  The only important difference
is that the enforcer includes a side condition $\mathtt{idx}_1 =
\mathtt{idx}_2$, where $\mathtt{idx}_1$ is an expression for
\texttt{idx} in the crashing thread and $\mathtt{idx}_2$ that in the
interfering thread, which is checked on the first happens-before edge.

The enforcer was effective at making this bug reproduce more easily.
With no enforcer loaded, the mean time to reproduce the bug was $1.2
\pm 0.2$ seconds, mean and standard deviation of mean for 100 runs;
with an enforcer, the mean time to reproduce was $0.24 \pm 0.01$
seconds.  As with the simple\_toctou test, most of this reduction was
due to removing the long tail: the $95^{th}$ percentile reproduction
time was reduced from 5.8 seconds to 0.48 seconds, whereas the median
actually increased slightly, from 0.16 seconds to 0.21 seconds.
\todo{Investigate how that changes when you change the delay
  parameter.}

I also investigated the behaviour of this test with an enforcer loaded
but no side condition checking performed.  In that case, the mean time
taken to reproduce the bug was $18 \pm 2$ seconds, mean and standard
deviation of mean for 100 runs.  This reduced enforcer not only fails
to make the bug reproduce more quickly; it actually makes it
\emph{less} likely to be triggered, per unit time!  This is because an
enforcer without side-condition checking will often slow the program
down in order to impose the happens-before graph even in situations
where doing so is unlikely to trigger the bug, and this causes the
buggy code to run far less frequently than it otherwise would.  The
full CDFs are shown in Figure~\ref{fig:eval:indexed_toctou:no_scs}.

\begin{figure}
  \input{eval/artificial_bugs/special/indexed_toctou_no_scs.tex}
  \caption{Effect of side-condition checking on the time taken to
    reproduce the indexed\_toctou bug.}
  \label{fig:eval:indexed_toctou:no_scs}
\end{figure}

\begin{figure}
  \subfigure[][Without enforcer]{ \input{eval/artificial_bugs/special/indexed_toctou_vary_nr_ptrs_no_enforcer.tex} }
  \subfigure[][Without enforcer]{ \input{eval/artificial_bugs/special/indexed_toctou_vary_nr_ptrs_enforcer.tex} }
  \caption{Reproduction times with and without an enforcer loaded, for
    varying values of \texttt{NR\_PTRS}.  Note that the two graphs use
    different scales, and that both use a log scale.}
  \label{fig:eval:indexed_toctou:nr_ptrs}
\end{figure}

As a final test, I investigated the effect changing the
\texttt{NR\_PTRS} parameter.  The results are shown in
Figure~\ref{fig:eval:indexed_toctou:nr_ptrs}.  I ran each
configuration 100 times at each of the sampled abscissae and timed how
long it took to reproduce the bug; these charts show the $25^{th}$,
$50^{th}$, and $75^{th}$ percentiles and the mean.

The most obvious property of these charts is that the behaviour of the
case without the enforcer is quite ``noisy''.  Some of this noise is
the usual experimental error; the distributions being measured have a
very long positive tail, and 100 samples is barely sufficient for the
$75^{th}$ percentile to become meaningful.  Much of it, though,
reflects the actual behaviour of the program.  The dips in
reproduction time around $\texttt{NR\_PTRS} = 200$ and
$\texttt{NR\_PTRS} = 400$, for instance, are both reproducible, and
persisted across multiple repeats of the experiment on the same
machine\footnote{This means that plotting a regression line on the
  chart is something of an abuse, but I believe that this is the
  clearest way of presenting this data.}.  I am unable to explain this
precise behaviour, beyond speculating that it might be due to some
aliasing effect between those values of \texttt{NR\_PTRS} and some
periodic structure as processor cache lines, or possibly even some
peculiarity the random number generator.  The complex behaviour of an
apparently simple test is reflective of the general complexity of
concurrency-related bugs, and is one of the reasons why they are often
difficult for programmers to correctly diagnose\needCite{}.  The chart
with the enforcer, by contrast, shows much simpler behaviour.  This
would make the bug far easier for a programmer to fix, even without
the significant reduction in the time take to reproduce it.

The automatic fix generator works well with this bug, and produces
roughly the same fix as it did in the simple\_toctou bug: one critical
section which covers the two critical loads in the read thread and one
which covers the critical store in the write thread.  To characterise
the performance overheads of the fix I again counted the number of
times the read and write loops execute per second with and without the
fix applied, running the test for ten seconds and discarding any runs
in which the test program crashed.  Without a fix applied, the test
completed the crashing thread loop $9.6 \pm 0.5 {\times} 10^5$ times
per second and the interfering thread loop $9.0 \pm 0.2 {\times} 10^5$
times per second; with a fix, it completed $8.3 \pm 0.3 \times 10^5$
crashing loops and $7.8 \pm 0.2 \times 10^5$ interfering loops (mean
and standard deviation of ten runs).  The overhead was therefore
roughly 15\% on both the read and write sides of the test.  This is
far smaller than the factor of four reported in the simple\_toctou
case, largely because the test loop in this case includes a call to
\verb|random|, which is rather expensive relative to simple lock
operations and helps to amortise the cost of the additional
synchronisation.

\subsection{Biassed indexed TOCTOU bugs (crash\_indexed\_toctou, interfering\_indexed\_toctou)}

These bugs are similar to the indexed\_toctou with $\texttt{NR\_PTRS}
= 100$, except with an additional one second delay in either the
interfering or crashing thread's loops, such that either the
interfering thread (for interfering\_indexed\_toctou) or the crashing
thread (for crash\_indexed\_toctou) runs far more often than the
other.  These tests are intended to illustrate the importance of
placing delays at appropriate operations in the crash enforcement
message-passing system.  The results are shown in
Table~\ref{fig:biassed_indexed_toctou:times}.  These results show
that, while the delay placement mechanism is not always guaranteed to
find the best possible placement, it does avoid some very poor ones.

\begin{figure}
  \input{eval/artificial_bugs/special/delay_positioning.tex}
  \caption{CDF of time taken to reproduce the crash\_indexed\_toctou
    and interfering\_indexed\_toctou bugs in different configurations.
    All configurations were repeated one hundred times.}
  \label{fig:biassed_indexed_toctou:times}
\end{figure}

\subsection{Multi-variable consistency constraint (multi\_variable)}

\begin{figure}
  \subfigure[][Crashing thread]{
    \texttt{
      \begin{tabular}{ll}
        \multicolumn{2}{l}{while (1) \{} \\
        & STOP\_ANALYSIS();\\
        & v1 = global1;\\
        & v2 = global2;\\
        & assert(v1 == v2);\\
        & STOP\_ANALYSIS();\\
        & sleep(10 milliseconds);\\
        \multicolumn{2}{l}{\}}\\
        \\
        \\
        \\
        \\
      \end{tabular}
    }
  }
  \hfill
  \subfigure[][Interfering thread]{
    \texttt{
      \begin{tabular}{ll}
        \multicolumn{2}{l}{while (1) \{}\\
        & STOP\_ANALYSIS();\\
        & global1 = 5;\\
        & STOP\_ANALYSIS();\\
        & global2 = 5;\\
        & STOP\_ANALYSIS();\\
        & sleep(100 milliseconds);\\
        & STOP\_ANALYSIS();\\
        & global1 = 7;\\
        & global2 = 7;\\
        & STOP\_ANALYSIS();\\
        \multicolumn{2}{l}{\}}\\
      \end{tabular}
    }
  }
  \caption{The two sides of the multi\_variable bug. The delays were
    chosen so that the program crashed in a reasonable amount of time
    when run unmodified.}
  \label{fig:eval:multi_variable}
\end{figure}

This bug is intended to explore {\technique}'s effects on
multi-variable atomicity violations.  The two sides of the bug are
shown in Figure~\ref{fig:eval:multi_variable}.  Note that in this case
the race leads to an assertion failure, whereas previous bugs lead to
a bad pointer dereference.  {\Technique} reports a single candidate
bug in this program, corresponding to interleaving the crashing thread
with the two stores which set \texttt{global1} and \texttt{global2} to
7 in the interfering thread.  This enforcer causes the bug to
reproduce quickly (in an average of $322 \pm 4$ms, mean and standard
deviation of mean for 100 runs), and in all cases within the three
minute timeout, whereas without an enforcer the bug failed to
reproduce before the timeout in 73\% of cases.

It is perhaps worth explaining why {\technique} only reported a single
bug here.  {\Technique} discovered that the crashing thread can crash
when interleaved with the stores which set the globals to 7, but
missed the fact that it can also crash when interleaved with the ones
which set them to 5.  This is because of an interaction between the
second \texttt{STOP\_ANALYSIS()} in the interfering thread and the
\gls{inferredassumption}, discussed in
Section~\ref{sect:derive:inferred_assumption}.  The
\texttt{STOP\_ANALYSIS()} causes the two stores to be converted into
independent {\StateMachines}.  Consider the first store; the other is
symmetrical.  Figure~\ref{fig:eval:multi_variable:other_bug} shows how
the \gls{inferredassumption} is derived for this bug\footnote{The
  actual analysis performs this derivation using {\StateMachines},
  rather than by concatenating the program's code, but that makes no
  difference in this case.}.  The \gls{ci-atomic} constraint shows
that the initial values of \texttt{global1} and \texttt{global2} must
be equal, and the \gls{ic-atomic} constraint shows that the initial
value of \texttt{global2} must be 5.  Combining these two shows that
the initial value of \texttt{global} must also be 5, and so the store
operation becomes a no-op and no candidate bug is reported.

\begin{figure}
  \centerline{
    {\hfill}
  \subfigure[][CI atomic]{
    \begin{tabular}{l}
      \hspace{-5mm}\texttt{
        \begin{tabular}{l}
          v1 = global1;\\
          v2 = global2;\\
          assert(v1 == v2);\\
          v1 = 5;\\
        \end{tabular}
      }\\
      \\
      $\smLoad{\texttt{global1}} = \smLoad{\texttt{global2}}$\\
    \end{tabular}
  }
  {\hfill}
  \subfigure[][IC atomic]{
    \begin{tabular}{l}
      \hspace{-5mm}\texttt{
        \begin{tabular}{l}
          v1 = 5;\\
          v1 = global1;\\
          v2 = global2;\\
          assert(v1 == v2);\\
        \end{tabular}
      }\\
      \\
      $\smLoad{\texttt{global2}} = 5$\\
    \end{tabular}
  }\\
  {\hfill}
  }
  \vspace{12pt}
  \centerline{Inferred assumption: $\smLoad{\texttt{global2}} = 5 \wedge \smLoad{\texttt{global1}} = 5$}
  \caption{Deriving the \gls{inferredassumption} for the other bug in
    the multi\_variable test.}
  \label{fig:eval:multi_variable:other_bug}
\end{figure}

The reported bug can also be converted into a fix.  This fix correctly
fixes the reported bug, but does not, of course, fix the one which is
not reported.  As such, the program might still crash.  In this case,
the fix actually causes the bug to reproduce more frequently,
increasing the reproduction rate from 27\% within three minutes to
81\%, from one hundred runs, because of the way in which it alters the
timing of the test program.  If the \texttt{STOP\_ANALYSIS()} marker
is removed then both bugs are discovered, and the resulting fix
correctly prevents the program from crashing.

\subsection{Write-to-read hazard (write\_to\_read)}


This test investigates a form of write-write-read race, whereas all of
the previous ones have considered only write-read ones.  It is shown
in Figure~\ref{fig:eval:write_to_read}.  The crashing thread loops
setting a global variable to point at some location and then proceeds
to use that global variable, while at the same time the interfering
thread loops setting that global variable to \texttt{NULL}.  In other
words, there is a write-to-read hazard in the crashing thread which
might be interrupted by the interfering thread, leading to a crash.

This test program is surprisingly reliable, given that it runs both
sides of the race in a tight loop with no delays or synchronisation,
and can often run for several minutes without encountering an error on
an otherwise idle system, during which time the buggy code might run
billions of times.  A more realistic test would run the two sections
far less frequently, and so might easily require hundreds of years of
CPU time to reproduce the bug.  I suspect that this is because the
store and load instructions in the crashing thread are close enough
together that the load is always satisfied from the processor's write
buffer, and so only returns \texttt{NULL} when the processor receives
an interrupt in precisely the wrong place.  The {\technique}-generated
bug enforcer, by contrast, can reliably reproduce the bug in a few
hundred milliseconds.

\begin{figure}
  \centerline{
    {\hfill}
  \subfigure[][Crashing thread]{
    \texttt{
      \begin{tabular}{ll}
        \multicolumn{2}{l}{while (1) \{}\\
        &STOP\_ANALYSIS();\\
        &global\_ptr = \&t;\\
        &*global\_ptr = 5;\\
        &STOP\_ANALYSIS();\\
        \multicolumn{2}{l}{\}}\\
      \end{tabular}
    }
  }
  \hfill
  \subfigure[][Interfering thread]{
    \texttt{
      \begin{tabular}{ll}
        \multicolumn{2}{l}{while (1) \{}\\
        &STOP\_ANALYSIS();\\
        &global\_ptr = NULL;\\
        &STOP\_ANALYSIS();\\
        \multicolumn{2}{l}{\}}\\
        \\
      \end{tabular}
    }
  }
    {\hfill}
    }
  \caption{The write\_to\_read test case.}
  \label{fig:eval:write_to_read}
\end{figure}

{\Technique} can also generate a fix for this bug.  The difficulty of
reproducing the bug makes it difficult to validate experimentally that
the generated fix is correct, but manual inspection suggested that it
is.

\subsection{Multiple bugs (multi\_bugs)}


This test combines simple\_toctou and write\_to\_read into a single
test, demonstrating {\technique}'s ability to exercise several bugs
using a single enforcer.  The test program is shown in
Figure~\ref{fig:eval:multi_bugs}.  The behaviour of
\texttt|select\_test| is configurable at run time, and will either
always select simple\_toctou, always select write\_to\_read, or select
randomly.  The dynamic analysis phases were run with it configured to
select randomly.

{\Implementation} is able to generate a candidate bug for each of the
component bugs in the test, each of which can be instantiated into an
enforcer, and these enforcers succeed in reproducing their respective
bugs.  As expected, neither enforcer is able to reproduce both bugs.
On the other hand, if both bugs' happens-before graphs are loaded into
the enforcer then it is able to reproduce either bug, as desired.
Perhaps more surprisingly, when the test is configured to exercise
both bugs the enforcer consistently reproduces the write\_to\_read
bug.  This is simply because the enforcer reproduces the
write\_to\_read bug before the simple\_toctou interfering critical
section ever runs.

In the same way, both candidate bugs can be instantiated into
individual fixes, both of which correctly fix their respective bug
while leaving the other bug in place.  Alternatively, a combined fix
can be generated, and this successfully fixes both bugs.

\begin{figure}
  \subfigure[][Crashing thread]{
    \texttt{
      \begin{tabular}{lll}
        \multicolumn{3}{l}{while (1) \{}\\
        & \multicolumn{2}{l}{r = select\_test();}\\
        & \multicolumn{2}{l}{STOP\_ANALYSIS();}\\
        & \multicolumn{2}{l}{if (r) \{}\\
        && simple\_toctou\_crashing();\\
        & \multicolumn{2}{l}{\} else \{}\\
        && write\_to\_read\_crashing();\\
        & \multicolumn{2}{l}{\}}\\
        & \multicolumn{2}{l}{STOP\_ANALYSIS();}\\
        \multicolumn{3}{l}{\}}\\
      \end{tabular}
    }
  }
  \subfigure[][Interfering thread]{
    \texttt{
      \begin{tabular}{lll}
        \multicolumn{3}{l}{while (1) \{}\\
        & \multicolumn{2}{l}{for (i = 0; i < 2000000; i++) \{}\\
        && STOP\_ANALYSIS(); \\
        && write\_to\_read\_interfering();\\
        && STOP\_ANALYSIS(); \\
        & \multicolumn{2}{l}{\}}\\
        & \multicolumn{2}{l}{STOP\_ANALYSIS();}\\
        & \multicolumn{2}{l}{simple\_toctou\_interfering();}\\
        & \multicolumn{2}{l}{STOP\_ANALYSIS();}\\
        \multicolumn{3}{l}{\}}\\
      \end{tabular}
    }
  }
  \caption{The multi\_bugs test.  The constant \texttt{2000000} was
    chosen so that the two bugs reproduce with roughly equal
    probability.}
  \label{fig:eval:multi_bugs}
\end{figure}

\begin{figure}
  \input{eval/artificial_bugs/special/multi_bugs.tex}
  \caption{Reproduction times CDFs for the multi\_bug test.}
\end{figure}

\subsection{Multiple crashing and interfering threads (multi\_threads)}

This test investigates {\implementation}'s behaviour in programs with
a very large number of threads.  The racing code is the same as in
indexed\_toctou (Figure~\ref{fig:eval:indexed_toctou}) with
$\texttt{NR\_PTRS} = 100$, except that rather than having a single
thread running the crashing and interfering critical sections, this
test has 32 threads running each.  {\Technique} behaves roughly as
expected here: it is able to generate an enforcer and a fix, with the
enforcer making the bug happen more quickly and the fix preventing it
from happening at all.

This test illustrates one subtlety in the implementation of the
enforcer: it must be (at least reasonably) fair in order for the time
to reproduce the bug to be predictable.  An initial version of the
enforcer internally used an unfair lock implementation, and this was
not able to reproduce the bug in reasonable time.  The
{\implementation} enforcers use a simple global lock to protect the
interpreter state, and in this test most threads spend almost all of
their time in the interpreter, and so the actual execution of the test
is effectively single-threaded.  Combining that with a naive unfair
lock implementation lead to such severe starvation that the bug could
not reproduce.  Switching to a fair lock implementation avoided the
issue.

\begin{figure}
  \input{eval/artificial_bugs/special/multi_threads.tex}
  \caption{CDF of reproduction times with different enforcer lock
    implementations.}
  \label{fig:eval:multi_threads}
\end{figure}

\todo{Currently running all of these tests on a four-processor machine,
  which might have an interesting effects on the results of a
  64-thread test.}

\todo{I kind of feel like I ought to say \emph{which} fair and unfair
  lock implementations I'm using here, since I just strongly implied
  that it matters, but it's a bit of a tedious thing to have to
  explain.}

\subsection{Complicated happens-before graphs ($\textrm{complex\_hb}_{\{5,11,17\}}$)}

\begin{figure}
  \subfigure[][Crashing thread]{
    \texttt{
      \begin{tabular}{ll}
        \multicolumn{2}{l}{while (1) \{}\\
        &STOP\_ANALYSIS();\\
        &x1 = global;\\
        &x2 = global;\\
        &x3 = global;\\
        &assert(!(x1 == 0 \&\& x2 == 1 \&\& x3 == 2));\\
        &STOP\_ANALYSIS();\\
        \multicolumn{2}{l}{\}}\\
      \end{tabular}
    }
  }
  \subfigure[][Interfering thread]{
    \texttt{
      \begin{tabular}{ll}
        \multicolumn{2}{l}{while (1) \{}\\
        &STOP\_ANALYSIS();\\
        &global = 0;\\
        &global = 1;\\
        &global = 2;\\
        &STOP\_ANALYSIS();\\
        \multicolumn{2}{l}{\}}\\
        \\
      \end{tabular}
    }
  }
  \caption{Crashing and interfering threads for the
    $\textrm{complex\_hb}_5$ test.  The $\textrm{complex\_hb}_{11}$
    and $\textrm{complex\_hb}_{17}$ tests are generated by extending
    this pattern to require additional happens-before edges.}
  \label{fig:eval:complex_hb}
\end{figure}

\begin{figure}
  \subfigure[][]{
    \begin{tabular}{ll}
      \begin{tikzpicture}
        \node (dummy) {};
        \node (ld1) [CfgInstr, below = of dummy] {\texttt{x1 = global;}};
        \node (ld2) [CfgInstr, below = 2 of ld1] {\texttt{x2 = global;}};
        \node (ld3) [CfgInstr, below = 2 of ld2] {\texttt{x3 = global;}};
        \node (st1) [CfgInstr, right = of dummy] {\texttt{global = 0;}};
        \node (st2) [CfgInstr, below = 2 of st1] {\texttt{global = 1;}};
        \node (st3) [CfgInstr, below = 2 of st2] {\texttt{global = 2;}};
        \draw[->] (ld1) -- (ld2);
        \draw[->] (ld2) -- (ld3);
        \draw[->] (st1) -- (st2);
        \draw[->] (st2) -- (st3);
        \draw[->,happensBeforeEdge] (st1) -- (ld1);
        \draw[->,happensBeforeEdge] (ld1) -- (st2);
        \draw[->,happensBeforeEdge] (st2) -- (ld2);
        \draw[->,happensBeforeEdge] (ld2) -- (st3);
        \draw[->,happensBeforeEdge] (st3) -- (ld3);
      \end{tikzpicture}\\
      Side condition: \true
    \end{tabular}
  }
  \hfill
  \subfigure[][]{
    \begin{tabular}{ll}
      \begin{tikzpicture}
        \node (ld1) [CfgInstr] {\texttt{x1 = global;}};
        \node (dummy) [right = of ld1] {};
        \node (ld2) [CfgInstr, below = 3 of ld1] {\texttt{x2 = global;}};
        \node (ld3) [CfgInstr, below = 2 of ld2] {\texttt{x3 = global;}};
        \node (st1) [CfgInstr, below = of dummy] {\texttt{global = 0;}};
        \node (st2) [CfgInstr, below = 0.5 of st1] {\texttt{global = 1;}};
        \node (st3) [CfgInstr, below = 1.7 of st2] {\texttt{global = 2;}};
        \draw[->] (ld1) -- (ld2);
        \draw[->] (ld2) -- (ld3);
        \draw[->] (st1) -- (st2);
        \draw[->] (st2) -- (st3);
        \draw[->,happensBeforeEdge] (ld1) -- (st1);
        \draw[->,happensBeforeEdge] (st2) -- (ld2);
        \draw[->,happensBeforeEdge] (ld2) -- (st3);
        \draw[->,happensBeforeEdge] (st3) -- (ld3);
      \end{tikzpicture}\\
      Side condition: $\smLoad{\mathrm{global}} = 0$
    \end{tabular}
  }
  \caption{Happens-before graphs generated for the $\textrm{complex\_hb}_5$ test.}
  \label{fig:eval:complex_hb:hb}
\end{figure}

This test, shown in Figure~\ref{fig:eval:complex_hb} is intended to
evaluate {\technique}'s ability to handle more complicated
happens-before graphs which require more than two context switches.
As expected, {\implementation} is able to generate both an enforcer
and a fix for this bug, which can either make the bug reproduce easily
or not at all.

The happens-before graphs and side conditions generated for this test
are shown in Figure~\ref{fig:eval:complex_hb:hb}.  The one on the left
is the expected graph, and will trigger the bug.  The one on the right
is not.  It describes the case in which the first store in the
interfering thread happens after the first load in the crashing
thread, but the initial contents of memory happens to contain the
right value.  It is impossible to enforce the desired happens-before
graph when \texttt{global} is initially 0, as the program structure
means that first store will only run when \texttt{global} is 2, but
the {\technique} analysis is not powerful enough to show that.
Without the side condition, the enforcer would have to try to enforce
both graphs at run-time, and so it would take longer to reproduce the
bug; with it, the enforcer can easily discard the spurious
happens-before graph at run time, partially compensating for the
incompleteness of the main analysis.

\subsection{A simple double-free bug (double\_free)}

\begin{figure}
  \subfigure[][Active threads]{
    \texttt{
      \begin{tabular}{lll}
        \multicolumn{3}{l}{while (1) \{}\\
        &\multicolumn{2}{l}{STOP\_ANALYSIS();}\\
        &\multicolumn{2}{l}{t = global\_ptr;}\\
        &\multicolumn{2}{l}{if (t != NULL) \{}\\
        &&free(t);\\
        &\multicolumn{2}{l}{\}}\\
        &\multicolumn{2}{l}{global\_ptr = NULL;}\\
        &\multicolumn{2}{l}{STOP\_ANALYSIS();}\\
        &\multicolumn{2}{l}{sleep(1 millisecond);}\\
        \multicolumn{3}{l}{\}}\\
      \end{tabular}
    }
  }
  \hfill
  \subfigure[][Environmental thread]{
    \texttt{
      \begin{tabular}{lll}
        \multicolumn{3}{l}{while (1) \{}\\
        &\multicolumn{2}{l}{STOP\_ANALYSIS();}\\
        &\multicolumn{2}{l}{if (global\_ptr == NULL) \{}\\
        &&global\_ptr = malloc(64);\\
        &\multicolumn{2}{l}{\}}\\
        &\multicolumn{2}{l}{STOP\_ANALYSIS();}\\
        \multicolumn{3}{l}{\}}\\
        \\
        \\
        \\
      \end{tabular}
    }
  }
  \caption{Threads involved in the double\_free bug.}
  \label{fig:eval:double_free}
\end{figure}

This test demonstrates {\technique}'s ability to handle some simple
double-free bugs.  The test program is shown in
Figure~\ref{fig:eval:double_free}.  Note that in this test, the
crashing and interfering threads (collectively, the active threads)
both run the same code, shown on the left of the figure, while a third
thread, known as the environmental thread, modifies the environment in
which they are operating.  The two active threads loop reading
\texttt{global\_ptr} and, if it is non-\texttt{NULL}, releasing it and
setting it to \texttt{NULL}.  The environment thread is meanwhile
undoing their work by examining \texttt{global\_ptr} and, when it is
\texttt{NULL}, setting it to a newly-allocated block.  Every time it
does so, the two active threads will race trying to release the block
and reset \texttt{global\_ptr}.  In some interleavings, both active
threads will try to release the same block, leading to a double-free
bug.

Note that the program threads here do not map directly on to the
threads in the candidate bug: the interfering thread is whichever of
the two active threads wins the race and releases the block first; the
crashing thread is the other active thread, which \texttt{free}s a
block which has already been released; and the environmental thread
does not appear in the candidate bug at all, despite being necessary
for the bug to reproduce.

{\Implementation} is able to build an enforcer and a fix for this bug,
and they behave as expected.  The critical sections for the fix are
moderately interesting.  One runs from immediately before the load of
\texttt{global\_ptr} to immediately after the \texttt{free}, while the
other runs from immediately before the \texttt{free} to immediately
after the store of \texttt{global\_ptr}.  The two sections overlap,
and so are the fix generating machinery merges them into a single
critical section covering the entire region between the
\texttt{STOP\_ANALYSIS()} markers.  This correctly fixes the bug.

Note that the critical region includes the call to \texttt{free}, and
so the generated patch will call into libc while holding the patch
lock.  In this case, that is safe (and in fact necessary to fix the
bug), but might carry a risk of deadlock for some other library
functions.  As discussed previously, {\technique}'s fixes detect such
deadlocks using a timeout and recover by simply allowing the critical
sections to proceed unprotected, avoiding the deadlock at the risk of
re-introducing the fixed bug.

\subsection{A program with existing synchronisation (existing\_sync\_visible, existing\_sync\_invisible)}

\begin{figure}
  \subfigure[][Crashing thread with {\technique}-visible synchronisation]{
    \texttt{
      \begin{tabular}{lll}
        \multicolumn{3}{l}{while (1) \{}\\
        &\multicolumn{2}{l}{STOP\_ANALYSIS();}\\
        &\multicolumn{2}{l}{lock(\&global\_lock);}\\
        &\multicolumn{2}{l}{if (global\_ptr != NULL) \{}\\
        &&*global\_ptr = 5;\\
        &\multicolumn{2}{l}{\}}\\
        &\multicolumn{2}{l}{unlock(\&global\_lock);}\\
        &\multicolumn{2}{l}{STOP\_ANALYSIS();}\\
        \multicolumn{3}{l}{\}}\\
      \end{tabular}
    }
    \label{fig:eval:existing_sync:visible}
  }
  \subfigure[][Crashing thread with {\technique}-invisible synchronisation]{
    \texttt{
      \begin{tabular}{lll}
        \multicolumn{3}{l}{while (1) \{}\\
        &\multicolumn{2}{l}{lock(\&global\_lock);}\\
        &\multicolumn{2}{l}{STOP\_ANALYSIS();}\\
        &\multicolumn{2}{l}{if (global\_ptr != NULL) \{}\\
        &&*global\_ptr = 5;\\
        &\multicolumn{2}{l}{\}}\\
        &\multicolumn{2}{l}{STOP\_ANALYSIS();}\\
        &\multicolumn{2}{l}{unlock(\&global\_lock);}\\
        \multicolumn{3}{l}{\}}\\
      \end{tabular}
    }
    \label{fig:eval:existing_sync:invisible}
  }
  \subfigure[][Interfering thread]{
    \texttt{
      \begin{tabular}{ll}
        \multicolumn{2}{l}{while (1) \{}\\
        &global\_ptr = \&t;\\
        &sleep(1 second);\\
        &STOP\_ANALYSIS();\\
        &lock(\&global\_lock);\\
        &global\_ptr = NULL;\\
        &unlock(\&global\_lock);\\
        &STOP\_ANALYSIS();\\
        \multicolumn{2}{l}{\}}\\
      \end{tabular}
    }
  }
  \caption{Threads for the existing\_sync\_visible and existing\_sync\_invisible tests.  \todo{Ugly diagram.}}
  \label{fig:eval:existing_sync}
\end{figure}

These tests explore {\technique}'s interactions with the program's
existing synchronisation.  It is the same as simple\_toctou, except
that the program contains calls to \texttt{pthread\_mutex\_lock} and
\texttt{pthread\_mutex\_unlock} which prevent the bug from ever
reproducing.  As discussed previously\editorial{ref?}, {\technique}
has no global model of the program's synchronisation, and so can only
analyse synchronisation within the \gls{analysiswindow}.  The analysis
will therefore be aware of the synchronisation in the
existing\_sync\_visible crashing thread,
Figure~\ref{fig:eval:existing_sync:visible}, but not that in the
existing\_sync\_invisible crashing thread,
Figure~\ref{fig:eval:existing_sync:invisible}.  As such, it generates
a candidate bug for the existing\_sync\_invisible test but not for the
existing\_sync\_visible one.

\begin{figure}
  \centerline{
    \begin{tikzpicture}
      \node (LD1) {First load};
      \path (node cs:name=LD1,anchor=west) -- ++(0,-2) node (LD2) [right] {Second load};
      \node (dummy) [right = 2 of LD1] {};
      \node (ST) [below = 0.6 of dummy] {Store};
      \draw (node cs:name=LD1,anchor=north west) -- ++(-0.2,0) |- (node cs:name=LD2,anchor=south west);
      \draw (node cs:name=ST,anchor=north east) -- ++(0.2,0) |- (node cs:name=ST,anchor=south east);
      \draw[->, happensBeforeEdge] (LD1) -- (ST);
      \draw[->, happensBeforeEdge] (ST) -- (LD2);
      \draw[->] (LD1) -- ++(0,-1.8);
      \path (-1.5,-2) to node [sloped] {locked region} (-1.5,0);
      \node at (5.3,-1.05) {locked region};
    \end{tikzpicture}
  }
  \caption{Happens-before graph which must be enforced for the
    existing\_sync\_invisible test, along with the program's existing
    locked regions.  This bug cannot be reproduced.}
  \label{fig:eval:existing_sync:hb}
\end{figure}

This candidate bug can be instantiated into a bug enforcer, but, as
might be expected, this enforcer cannot cause any bugs to reproduce.
The happens-before graph for the bug is shown in
Figure~\ref{fig:eval:existing_sync:hb}; this clearly contradicts the
program's existing synchronisation strategy, and so trying to enforce
it will lead to a deadlock.  The unbound message operations in the
crash enforcement plan will therefore all time out, preventing
successful plan completion.

It can similarly be instantiated into a fix.  This fix does not fix
any actual bugs, as there are none, but does not otherwise harm the
program's execution.

\subsection{The broken publish pattern (broken\_publish)}

\begin{figure}
  \subfigure[][Crashing thread]{
    \texttt{
      \begin{tabular}{lll}
        \multicolumn{3}{l}{while (1) \{}\\
        &\multicolumn{2}{l}{STOP\_ANALYSIS();}\\
        &\multicolumn{2}{l}{p = global\_ptr;}\\
        &\multicolumn{2}{l}{if (p != NULL) \{}\\
        &&assert(p->v == 7);\\
        &\multicolumn{2}{l}{\}}\\
        &\multicolumn{2}{l}{STOP\_ANALYSIS();}\\
        \multicolumn{3}{l}{\}}\\
      \end{tabular}
    }
  }
  \subfigure[][Interfering thread]{
    \texttt{
      \begin{tabular}{lll}
        \multicolumn{2}{l}{while (1) \{}\\
        &STOP\_ANALYSIS();\\
        &s = malloc();\\
        &global\_ptr = s;\\
        &s->v = 7;\\
        &STOP\_ANALYSIS();\\
        \multicolumn{2}{l}{\}}\\
        \\
      \end{tabular}
    }
  }
  \caption{The two sides of the broken\_publish test bug.  Garbage
    collection-related code is not shown.}
  \label{fig:eval:broken_publish}
\end{figure}

This test implements a common bug in the widely-used publish
pattern\needCite{}.  Were it correctly implemented, this pattern would
involves the interfering thread allocating and initialising a data
structure, and then storing a pointer to it in a thread-shared
location so that the crashing thread could access it.  In the test,
though, the interfering thread does not finish initialising the
structure (the store of 7 to \texttt{v}) until after it has published
the structure by storing to \texttt{global\_ptr}.  {\Implementation}
is able to find this bug and to generate an enforcer and a fix, and
they both work in the expected way.

\todo{I was trying to get malloc() to return something which just
  happened to have a 7 in the right place, so that I could say
  something about only being able to handle bugs which are down to
  concurrency and not other forms of undefined behaviour, but it looks
  like this version of libc zeroes out the buffer.  Bah.}

\subsection{A bug which lacks the W isolation property (w\_isolation)}

\begin{figure}
  \centerline{
    {\hfill}
  \subfigure[][Crashing thread]{
    \texttt{
      \begin{tabular}{ll}
        \multicolumn{2}{l}{while (1) \{}\\
        &STOP\_ANALYSIS();\\
        &s = malloc();\\
        &s->v = 7;\\
        &global\_ptr = s;\\
        &assert(s->v == 7);\\
        &STOP\_ANALYSIS();\\
        \multicolumn{2}{l}{\}}\\
      \end{tabular}
    }
  }
    {\hfill}
  \subfigure[][Interfering thread]{
    \texttt{
      \begin{tabular}{lll}
        \multicolumn{3}{l}{while (1) \{}\\
        &\multicolumn{2}{l}{STOP\_ANALYSIS();}\\
        &\multicolumn{2}{l}{p = global\_ptr;}\\
        &\multicolumn{2}{l}{if (p != NULL) \{}\\
        &&p->v = 5;\\
        &\multicolumn{2}{l}{\}}\\
        &\multicolumn{2}{l}{STOP\_ANALYSIS();}\\
        \multicolumn{3}{l}{\}}\\
      \end{tabular}
    }
  }
    {\hfill}
  }
  \caption{Racing threads for the w\_isolation test.  Garbage
    collection-related code is not shown.}
  \label{fig:w_isolation}
\end{figure}

This test illustrates a bug which lacks the W isolation property (see
Section~\ref{sect:derive:w_isolation}).  In this test, the crashing
thread allocates a fresh data structure, initialises the field
\texttt{v}, publishes it via a global pointer, and then asserts that
\texttt{v} is unchanged.  Meanwhile, the interfering thread loops
checking for any published structures and, if it finds one, changing
the value of \texttt{v}.  This bug can only be reproduced when the
interfering thread is able to access the structure which was stored by
the crashing one, and hence lacks the W isolation property.
{\Implementation} is able to correctly analyse this bug when
configured to not assume the W isolation property, but cannot
otherwise.

This bug is quite similar to the broken\_publish bug, which does have
the W isolation property.  The important difference is that in
broken\_publish, data only flows only in one direction between the two
threads, whereas in this bug data flows in both directions and there
are no sub-fragments of the critical section which have a bug and do
not have bi-directional data flow.

\subsection{Comparison to DataCollider}
\label{sect:eval:datacollider}

As a point of comparison, I implemented a tool which explores
alternative thread schedules at random, without first analysing the
program to obtain {\StateMachines} and verification conditions,
inspired by DataCollider\needCite{}.  When the program starts, this
tool places breakpoints at a randomly selected subset of the program's
memory accesses.  When one of the breakpoints is hit, it determines
what memory location the instruction is accessing and sets a processor
watch point\needCite{} on that location.  If the watch point is hit by
any other threads, the tool has discovered a race, and it selects one
of the two racing threads to go first at random.  In this way, the
tool encourages the program to explore its available schedules much
more quickly than it otherwise would.

The effectiveness of this tool is obviously highly dependent on the
fraction of instructions which are covered by breakpoints and the
length of the delays inserted.  I used the following parameters in all
experiments:

\begin{itemize}
\item
  At any given point, half of the program's store instructions will
  have an instruction breakpoint, excluding stack accesses.
\item
  When a breakpoint is hit, the tool will wait for up to 100$\mu$s for
  a matching read or write to arrive.  When a thread does arrive, it
  will select which to resume first at random, with equal probability.
  Note that this is a factor of a thousand smaller than the timeout
  used for {\technique} enforcers in most of this evaluation.
\item
  Every 100ms, the tool discards its current instruction breakpoint
  set and generates a new one.
\end{itemize}

These parameters were chosen because they minimise the median
reproduction time for the simple\_toctou test\editorial{I should maybe
  give some actual experiments I did to confirm that.}.

It is instructive to compare these parameters to those used in the
DataCollider paper.  DataCollider uses timeouts of between one and
fifteen milliseconds, depending on the type of instruction, and so the
100$\mu$s timeout is of broadly similar order of magnitude.  Setting
breakpoints on half of instructions, on the other hand, is not.  The
actual DataCollider implementation adjusts the breakpoint density
dynamically so as to achieve a particular breakpoint rate, and their
paper does not specify a numerical breakpoint density, which makes a
direct comparison difficult.  It is, however, possible to estimate the
breakpoint density from the information which they do give.  Their
evaluation shows breakpoint rates of up to 1500 breakpoints per second
in a virtual machine with two processors running at 2.4GHz, so $4.8
\times 10^9$ cycles per second.  If one assumes roughly one non-stack
memory access every hundred cycles that translates to 4.8 million
memory accessing instructions per second, and so they must trap
roughly 0.03\% of memory accessing instructions.  Even allowing for
the fact that their implementation preferentially places breakpoints
on instructions which execute infrequently this is still likely to
translate to a breakpoint ratio several orders of magnitude lower than
that used in this evaluation.

Compared to {\technique}, and to the original DataCollider, these
parameters have two interesting effects: the tool intervenes in the
program's execution much more frequently, because of the high
breakpoint ratio, but makes relatively small changes to the execution
each time, because of the short timeout.  In other words, these
parameters mean that the DataCollider-like tool gets a very large
number of opportunities to reproduce the bug, but has a low
probability of reproduction at each opportunity.  This is a sensible
strategy for these test bugs, as the test harness will ensure that the
buggy code is run very frequently, but would perhaps be slightly less
effective in more realistic programs where the buggy code runs less
often.  It is therefore quite unlikely that DataCollider would perform
as well on more realistic tests as it does on these artificial bugs.

\todo{More fundamentally, using DataCollider to explore alternative
  schedules is itself somewhat unfair, as DataCollider was originally
  intended to discover races rather than to permute schedules, and,
  while it is trivial to extend it to perform schedule exploration, it
  is not entirely surprising that the results are somewhat poor.
  Nevertheless, it represents the approach which is conceptually close
  to {\technique}'s in the existing literature, and so I consider it
  to be an interesting reference point.}

\todo{I'd really like to do a comparison with CHESS, as well, but
  that's only implemented for 32-bit Windows programs, and
  {\technique} is only implemented for 64-bit Linux ones, which makes
  a bit of a mess of direct comparisons.  It'd be kind of fun to
  re-implement it for Linux, but I don't really have time to do that
  right now.}

The results of these experiments are shown in
Figure~\ref{fig:eval:summary_cdfs}.  For the majority of experiments,
the DataCollider-like tool gives a modest reduction in the time taken
to reproduce the bug, but is less effective than {\technique}, as
expected.  There are a few exceptions, however:

\begin{itemize}
\item The cross\_function bug actually reproduces less frequently with
  the DataCollider-like tool than it does with no reproduction-aiding
  tools.  This is because the tool, in this test, often places a delay
  in the \gls{crashingthread} at a place which is irrelevant to the
  bug under investigation.  This slows down the \gls{crashingthread},
  and so there are fewer opportunities to reproduce the bug, but,
  because it is mispositioned, does not increase the probability of
  each opportunity actually succeeding.  The end result is that it
  takes slightly longer to reproduce the bug.
\item The double\_free test reproduces more quickly with the
  DataCollider-like tool than with {\technique}.  This is in large
  part an artifact of the test harness.  This test is implemented such
  that the program will crash precisely when the two racing threads
  start their critical section at the same time, and so the harness
  includes a small delay between starting the two threads so as to
  avoid crashing on the very first iteration.  This is effective for
  the {\technique} enforcers and for the baseline case, but the way
  the DataCollider-like tool is implemented tends to resynchronise the
  threads so that they crash very quickly.  In fact, more than 90\% of
  the time this test program crashes before the tool has detected any
  races or inserted any delays.
\item The multi\_variable test also reproduces more quickly under the
  DataCollider-like tool than under the {\technique} enforcer.  With
  the DataCollider-like tool, the mean time to reproduction is 75ms,
  the median time 69ms, and the $95^{th}$ percentile 172ms; with a
  {\technique} enforcer, the mean time is 322ms, the median 330ms, and
  the $95^{th}$ percentile 386ms.  This is primarily because
  {\technique}'s timeouts are not well-suited to this test; reducing
  the timeout from 100ms to 1ms reduced the mean time to 50ms, the
  median to 50ms, and the $95^{th}$ percentile to 58ms, which would
  represent a useful improvement on the DataCollider-like tool.  On
  the other hand, setting the timeouts that short would, as discussed
  above, be likely to harm reproduction performance on more realistic
  tests.
\item Finally, the multi\_threads test reproduces more quickly under
  the DataCollider-like tool than under the {\technique} enforcer.
  the problem here is simply that the instruction interpreter used in
  {\technique} enforcers is rather slow, taking more than 1.5 seconds
  to step every thread far enough through their critical section to
  actually try to perform message operations, and then several hundred
  milliseconds more to reach a point where they can reproduce the bug.
  The DataCollider-like tool, on the other hand, runs most of the
  program unmodified, and so does not suffer from this problem.  In a
  more realistic program the critical sections would probably cover a
  smaller proportion of the entire program, and so the program would
  spend less time in the interpreter, which would somewhat mitigate
  this effect.
\end{itemize}

Overall, the {\technique} enforcers are usually able to reproduce most
of the target bugs more quickly than the DataCollider-like tool, and
the cases where the DataCollider-like tool is more effective are
mostly caused by the tests being quite small relative to realistic
programs.

\subsection{Summary tables}

I now give some a quantitative characterisation of {\technique}'s
performance on these test programs.  This includes CDFs of the time
taken to reproduce the various bugs, in
\autoref{fig:eval:summary_cdfs}, the time taken to perform the various
analysis steps, in \autoref{table:eval:summary_analysis_times}, and
the performance effects of the fixes, in
\autoref{table:eval:fix_overheads}.  The important points here are:

\begin{itemize}
\item {\Technique}'s enforcers make these bugs reproduce more quickly,
  often much more quickly.
\item The various analysis passes are generally very fast, usually
  taking a few hundred milliseconds.  The most time-consuming program
  to analyse is complex\_hb\_17, which takes 1.711 seconds to analyse.
\item The fixes which {\technique} generates for these bugs have
  tolerable overhead, generally from a few tens of percent to a small
  factor.  More realistic tests would probably show lower overheads,
  as the cost of the fix could be more easily amortised over a larger
  program.
\end{itemize}

In other words, {\technique} is an effective and efficient technique
for both finding and fixing bugs in small programs.  The next few
sections will explore how well it scales up to more complex programs.

The performance table shows a measure of the performance impact of the
various fixes generated by {\technique}.  Performance here is measured
in the number of loop iterations completed per second in the two
threads, excluding any threads for which the test harness deliberately
inserts delays.  Each test program was run 20 times for ten seconds,
with any crashing runs repeated.  I report the number of iterations
per second with and without a fix, and the ratio of those two
quantities.  The table reports each quantity as $[a; b; c]$, where $b$
is an estimate of the quantity of interest (the mean number of
iterations per second for the raw performance numbers, or the ratio of
those two means for the ratio column), and $[a;b]$ forms a 95\%
confidence interval for that quantity.  The confidence intervals for
the raw performance measures were calculated using the central limit
theorem (and simply assuming that 20 samples are sufficient for that
to be valid).  The confidence interval for the ratio was calculated
using Luxburg's approximation\needCite{}.  Ratios greater than one
indicate that the program ran more slowly with the fix applied; those
less than one indicate that it ran more quickly.  In the case of the
multi\_threads test, the performance metric given is summed across all
threads.

\input{eval/artificial_bugs/crash_time_cdfs}

\begin{sidewaystable}
  \begin{tabular}{lllll}
    Test name                      & Static analysis & Generating verification conditions & Building the enforcers & Building the fixes \\
    simple\_toctou                 & $0.571 \pm 0.060$ &  $0.195 \pm 0.018$ &  $0.126 \pm 0.006$ &  $0.142 \pm 0.005$\\
    indexed\_toctou                & $0.604 \pm 0.089$ &  $0.275 \pm 0.004$ &  $0.146 \pm 0.005$ &  $0.142 \pm 0.004$\\
    crash\_indexed\_toctou         & $0.566 \pm 0.068$ &  $0.289 \pm 0.009$ &  $0.149 \pm 0.007$ &  $0.138 \pm 0.004$\\
    interfering\_indexed\_toctou   & $0.624 \pm 0.055$ &  $0.288 \pm 0.008$ &  $0.149 \pm 0.007$ &  $0.142 \pm 0.010$\\
    context                        & $0.682 \pm 0.058$ &  $0.233 \pm 0.005$ &  $0.132 \pm 0.007$ &  $0.139 \pm 0.005$\\
    cross\_function                & $0.669 \pm 0.036$ &  $0.186 \pm 0.004$ &  $0.132 \pm 0.005$ &  $0.138 \pm 0.006$\\
    double\_free                   & $0.447 \pm 0.029$ &  $0.186 \pm 0.005$ &  $0.130 \pm 0.002$ &  $0.135 \pm 0.003$\\
    multi\_variable                & $0.792 \pm 0.115$ &  $0.228 \pm 0.006$ &  $0.157 \pm 0.004$ &  $0.135 \pm 0.004$\\
    write\_to\_read                & $0.439 \pm 0.035$ &  $0.182 \pm 0.003$ &  $0.125 \pm 0.004$ &  $0.135 \pm 0.003$\\
    broken\_publish                & $0.521 \pm 0.037$ &  $0.181 \pm 0.004$ &  $0.123 \pm 0.005$ &  $0.139 \pm 0.004$\\
    complex\_hb\_5                 & $0.426 \pm 0.011$ &  $0.212 \pm 0.006$ &  $0.149 \pm 0.007$ &  $0.141 \pm 0.008$\\
    complex\_hb\_11                & $0.457 \pm 0.021$ &  $0.343 \pm 0.005$ &  $0.162 \pm 0.004$ &  $0.141 \pm 0.005$\\
    complex\_hb\_17                & $0.451 \pm 0.025$ &  $1.711 \pm 0.012$ &  $0.200 \pm 0.003$ &  $0.140 \pm 0.004$\\
    existing\_sync\_visible        & $0.566 \pm 0.020$ &  $0.156 \pm 0.005$ & \multicolumn{2}{c}{\emph{Nothing generated}} \\
    existing\_sync\_invisible      & $0.572 \pm 0.023$ &  $0.206 \pm 0.006$ &  $0.136 \pm 0.003$ &  $0.140 \pm 0.004$\\
    multi\_bugs                    & $0.625 \pm 0.024$ &  $0.288 \pm 0.005$ & \\
    \hspace{5mm}Bug 1 & & &  $0.161 \pm 0.006$ &  $0.136 \pm 0.003$\\
    \hspace{5mm}Bug 2 & & &  $0.143 \pm 0.006$ &  $0.140 \pm 0.005$\\
    multi\_threads                 & $0.562 \pm 0.025$ &  $0.232 \pm 0.002$ &  $0.160 \pm 0.022$ &  $0.142 \pm 0.009$\\
    w\_isolation                   & $0.509 \pm 0.031$ &  $0.143 \pm 0.004$ &  $0.117 \pm 0.003$ &  $0.137 \pm 0.005$\\
    glibc                          & $0.337 \pm 0.030$ &  $0.193 \pm 0.004$ &  $0.128 \pm 0.004$ &  $0.134 \pm 0.003$\\
  \end{tabular}
  \caption{Time taken in the various analysis phases for the
    artificial bugs.  Times are given as mean and standard deviation
    of ten runs.  The W isolation assumption was enabled for all tests
    except w\_isolation.}
  \label{table:eval:summary_analysis_times}
\end{sidewaystable}

\begin{sidewaystable}
  \begin{tabular}{lllll}
    \multicolumn{2}{l}{Test name} & Performance without fix & Performance with fix & Ratio\\
    \multicolumn{2}{l}{simple\_toctou                }  & $[352,\!540,\!000; 352,\!630,\!000; 352,\!730,\!000]$  &  $[95,\!430,\!000; 95,\!600,\!000; 95,\!770,\!000]$  &  $[3.681; 3.689; 3.696]$ \\
    \multicolumn{2}{l}{indexed\_toctou               } \\
    & \multicolumn{1}{l}{Crashing thread} & $[9,\!280,\!000; 9,\!540,\!000; 9,\!800,\!000]$  &  $[9,\!100,\!000; 9,\!600,\!000; 10,\!100,\!000]$  &  $[0.92; 0.99; 1.08]$ \\
    & \multicolumn{1}{l}{Interfering thread} & $[8,\!140,\!000; 8,\!340,\!000; 8,\!540,\!000]$  &  $[8,\!300,\!000; 8,\!760,\!000; 9,\!220,\!000]$  &  $[0.88; 0.95; 1.03]$ \\
    \multicolumn{2}{l}{crash\_indexed\_toctou        }  & $[156,\!726,\!000; 156,\!735,\!000; 156,\!744,\!000]$  &  $[70,\!650,\!000; 70,\!710,\!000; 70,\!770,\!000]$  &  $[2.2147; 2.2166; 2.2185]$ \\
    \multicolumn{2}{l}{interfering\_indexed\_toctou  }  & $[69,\!770,\!000; 70,\!000,\!000; 70,\!230,\!000]$  &  $[42,\!660,\!000; 43,\!040,\!000; 43,\!420,\!000]$  &  $[1.607; 1.626; 1.646]$ \\
    \multicolumn{2}{l}{double\_free                  }  & $[17,\!800; 18,\!110; 18,\!420]$  &  $[18,\!770; 18,\!860; 18,\!940]$  &  $[0.940; 0.960; 0.981]$ \\
    \multicolumn{2}{l}{write\_to\_read               }  & $[487,\!800,\!000; 489,\!700,\!000; 491,\!600,\!000]$  &  $[91,\!400,\!000; 94,\!600,\!000; 97,\!800,\!000]$  &  $[4.99; 5.18; 5.38]$ \\
    \multicolumn{2}{l}{broken\_publish               } \\
    & \multicolumn{1}{l}{Crashing thread} & $[1,\!840,\!000; 1,\!980,\!000; 2,\!120,\!000]$  &  $[1,\!910,\!000; 2,\!070,\!000; 2,\!230,\!000]$  &  $[0.82; 0.96; 1.11]$ \\
    & \multicolumn{1}{l}{Interfering thread} & $[1,\!840,\!000; 1,\!980,\!000; 2,\!120,\!000]$  &  $[1,\!910,\!000; 2,\!070,\!000; 2,\!230,\!000]$  &  $[0.82; 0.96; 1.11]$ \\
    \multicolumn{2}{l}{complex\_hb\_5                } \\
    & \multicolumn{1}{l}{Crashing thread} & $[50,\!000,\!000; 56,\!000,\!000; 62,\!000,\!000]$  &  $[7,\!950,\!000; 8,\!420,\!000; 8,\!890,\!000]$  &  $[5.6; 6.6; 7.8]$ \\
    & \multicolumn{1}{l}{Interfering thread} & $[50,\!900,\!000; 52,\!800,\!000; 54,\!700,\!000]$  &  $[13,\!470,\!000; 13,\!890,\!000; 14,\!300,\!000]$  &  $[3.56; 3.80; 4.06]$ \\
    \multicolumn{2}{l}{complex\_hb\_11               } \\
    & \multicolumn{1}{l}{Crashing thread} & $[41,\!000,\!000; 46,\!000,\!000; 51,\!000,\!000]$  &  $[7,\!700,\!000; 8,\!300,\!000; 9,\!000,\!000]$  &  $[4.5; 5.5; 6.6]$ \\
    & \multicolumn{1}{l}{Interfering thread} & $[41,\!300,\!000; 43,\!200,\!000; 45,\!000,\!000]$  &  $[13,\!420,\!000; 13,\!840,\!000; 14,\!250,\!000]$  &  $[2.89; 3.12; 3.36]$ \\
    \multicolumn{2}{l}{complex\_hb\_17               } \\
    & \multicolumn{1}{l}{Crashing thread} & $[40,\!800,\!000; 45,\!100,\!000; 49,\!300,\!000]$  &  $[8,\!100,\!000; 8,\!800,\!000; 9,\!400,\!000]$  &  $[4.3; 5.1; 6.1]$ \\
    & \multicolumn{1}{l}{Interfering thread} & $[41,\!600,\!000; 43,\!000,\!000; 44,\!400,\!000]$  &  $[13,\!500,\!000; 14,\!100,\!000; 14,\!800,\!000]$  &  $[2.82; 3.04; 3.28]$ \\
    \multicolumn{2}{l}{existing\_sync\_invisible     }  & $[147,\!041,\!000; 147,\!064,\!000; 147,\!087,\!000]$  &  $[68,\!305,\!000; 68,\!339,\!000; 68,\!374,\!000]$  &  $[2.1505; 2.1520; 2.1534]$ \\
    \multicolumn{2}{l}{multi\_threads                } \\
    & \multicolumn{1}{l}{Crashing thread} & $[7,\!320,\!000; 7,\!390,\!000; 7,\!470,\!000]$  &  $[7,\!300,\!000; 7,\!370,\!000; 7,\!430,\!000]$  &  $[0.985; 1.004; 1.023]$ \\
    & \multicolumn{1}{l}{Interfering thread} & $[6,\!970,\!000; 7,\!040,\!000; 7,\!110,\!000]$  &  $[7,\!130,\!000; 7,\!200,\!000; 7,\!260,\!000]$  &  $[0.959; 0.978; 0.997]$ \\
    \multicolumn{2}{l}{multi\_bugs} & $[145,\!000,\!000; 148,\!400,\!000; 151,\!900,\!000]$ \\
    & simple\_toctou fix only & & $[91,\!000,\!000; 93,\!200,\!000; 95,\!300,\!000]$  &  $[1.52; 1.59; 1.67]$ \\
    & write\_to\_read fix only & & $[52,\!600,\!000; 54,\!500,\!000; 56,\!500,\!000]$  &  $[2.57; 2.72; 2.89]$\\
    & Both fixes & &  $[59,\!900,\!000; 63,\!400,\!000; 66,\!800,\!000]$  &  $[2.17; 2.34; 2.54]$ \\
  \end{tabular}
  \caption{Performance overheads of automatically-generated fixes,
    measured in loop iterations per second.  Further details of the
    experiments are given in the test.}
  \label{table:eval:fix_overheads}
\end{sidewaystable}

\section{Semi-artificial bugs}
\label{sect:eval:semiartificial}

I now present the results of running the tool on some bugs which are
partly artificial and partly real.  This includes the kernel of a real
bug and two cases produced by deliberately introducing bugs into
programs from the STAMP benchmark suite\needCite{}.

\subsection{A kernel of a real bug (glibc)}
\label{sect:eval:glibc}

\todo{I'm tempted to treat this as an artificial bug instead of a
  semi-artificial one.}

glibc is a kernel of glibc bug 2644 \cite{Cambell2006}, which
affected versions of glibc up to 2.5 and could lead to a crash if
multiple threads were shut down at the same time.  A simplified
version of the code involved is shown in Figure~\ref{fig:eval:glibc},
where \texttt{forcedunwind} and \texttt{done\_init} are global
variables.  The program will crash if the load of
\texttt{forcedunwind} on line 1 loads a \texttt{NULL} pointer and the
load of \texttt{done\_init} on line 3 loads 1.  This is clearly
possible due to interleaving with the stores on lines 5 and 6.

Note that the bug here depends on the compiler's optimizer, and is not
apparent at the source-code level\footnote{Unfortunately, only the
  32-bit x86 version of gcc optimizes the function like this, and my
  implementation of {\technique} only supports 64-bit programs, which
  prevented me from testing with the real bug.}.  {\Technique}
operates entirely at the machine-code level, and so this does not
present any additional complexity.

\begin{figure}
  \subfigure[][Before optimisation]{
    \texttt{
      \begin{tabular}{lll}
        \multicolumn{3}{l}{\_Unwind\_ForcedUnwind() \{}\\
        & \multicolumn{2}{l}{if (forcedunwind == NULL) \{} \\
        &&pthread\_cancel\_init();\\
        & \multicolumn{2}{l}{\}} \\
        & \multicolumn{2}{l}{forcedunwind();}\\
        \multicolumn{3}{l}{\}}\\
        \multicolumn{3}{l}{pthread\_cancel\_init() \{}\\
        & \multicolumn{2}{l}{if (done\_init) return;}\\
        & \multicolumn{2}{l}{forcedunwind = \_forcedunwind\_impl;}\\
        & \multicolumn{2}{l}{done\_init = 1;}\\
        \multicolumn{3}{l}{\}}\\
      \end{tabular}
    }
  }
  \subfigure[][After optimisation.  Blue indicates the crashing fragment and red the interfering one.]{
    \texttt{
      \begin{tikzpicture}
        \path [use as bounding box] (0,0) rectangle (0,1);
        \draw [fill, color=blue!20] (1.3,0.4) rectangle (7,2.8);
        \path [pattern color=red, pattern=diagonal hatch] (1.3,1) rectangle (7,0.4);
        \draw [fill, color=red!20] (1.3,0.4) rectangle (7,-0.8);
        \draw [fill, color=blue!20] (1.3,-2.1) rectangle (2.5,-1.4);
      \end{tikzpicture}
      \begin{tabular}{lllll}
          & \multicolumn{4}{l}{\_Unwind\_ForcedUnwind() \{}\\
        1 & & \multicolumn{3}{l}{l = forcedunwind;}\\
        2 & & \multicolumn{2}{l}{if} & (l == NULL \&\&\\
        3 & & & &\hspace{3mm}\!done\_init) \{\\
        4 & & & \multicolumn{2}{l}{l = \_forcedunwind\_impl;} \\
        5 & & & \multicolumn{2}{l}{forcedunwind = l;} \\
        6 & & & \multicolumn{2}{l}{done\_init = 1;}\\
        7 & & \multicolumn{3}{l}{\}}\\
        8 & & \multicolumn{3}{l}{l();}\\
          & \multicolumn{4}{l}{\}}\\
        \\
      \end{tabular}
    }
  }
  \caption{Source code for the glibc test case.}
  \label{fig:eval:glibc}
\end{figure}


\subsection{labyrinth}

This test consists of the labyrinth component of the STAMP benchmark
suite\needCite{}, converted to use locks rather than transactional
memory and with one of its critical sections removed.  It is shown in
Figure~\ref{fig:eval:labyrinth}.  This program is structured as a
read-modify-writeback operation: the \texttt{memcpy} on line 8 takes a
local copy of a shared structure, which is then worked on by
\texttt{PdoExpansion} and \texttt{PdoTraceback}, generating results
which can be written back by \texttt{TMgrid\_addPath}.  Before
performing the writeback, \texttt{TMgrid\_addPath} first checks that
no other threads have generated the path which it is adding, crashing
if they have.  Removing the lock operations on lines 7 and 15
introduces a race which can trigger this crash.

\begin{figure}
  \texttt{
    \begin{tabular}{lllll}
        & \multicolumn{4}{l}{TMgrid\_addPath(grid\_t *gridPtr, vector\_t *pointVectorPtr) \{}\\
      1 & & \multicolumn{3}{l}{for (i = 1; i < pointVectorPtr->size() - 1; i++) \{}\\
      2 & & & \multicolumn{2}{l}{long *gridPointPtr = pointVectorPtr->get(i);}\\
      3 & & & \multicolumn{2}{l}{assert(*gridPointPtr == GRID\_POINT\_EMPTY);}\\
      4 & & & \multicolumn{2}{l}{*gridPointPtr = GRID\_POINT\_PTR;}\\
      5 & & \multicolumn{3}{l}{\}}\\
      6 & \multicolumn{4}{l}{\}}\\
      \\
      7  & \multicolumn{4}{l}{acquire\_lock();}\\
      8  & \multicolumn{4}{l}{memcpy(myGridPtr, gridPtr);}\\
      9  & \multicolumn{4}{l}{if (PdoExpansion(myGridPtr)) \{}\\
      10 & & \multicolumn{3}{l}{pointVectorPtr = PdoTraceback(myGridPtr);}\\
      11 & & \multicolumn{3}{l}{if (pointVectorPtr) \{}\\
      12 & & & \multicolumn{2}{l}{TMgrid\_addPath(gridPtr, pointVectorPtr);}\\
      13 & & \multicolumn{3}{l}{\}}\\
      14 & \multicolumn{4}{l}{\}}\\
      15 & \multicolumn{4}{l}{release\_lock();}\\
    \end{tabular}
  }
  \caption{The labyrinth test. \texttt{PdoExpansion} and
    \texttt{PdoTraceback} are thread-local functions.}
  \label{fig:eval:labyrinth}
\end{figure}

{\Implementation} is not able to generate a candidate bug for this
test, and is therefore unable to generate an enforcer or a fix for it.
The reason is simple: \texttt{PdoExpansion} and \texttt{PdoTraceback}
are both large functions, and {\implementation} takes an unreasonable
amount of time and memory to analyse them.  It is therefore unable to
link the load of \texttt{*gridPointPtr} on line 3 to the store in the
\texttt{memcpy} on line 8, preventing it from discovering the bug's
concurrency behaviour.

Manual inspection of the machine code suggested in order to correctly
analyse this bug, \gls{alpha} would need to be set to at least 570
instructions, of which 265 correspond to \texttt{PdoExpansion} and 195
to \texttt{PdoTraceback}.  As discussed in \autoref{sect:eval:alpha},
it is difficult to push \gls{alpha} above a few dozen with the current
implementation, and so analysing the entire path is unlikely to be
feasible.  On the other hand, if {\technique} could use procedure
summaries\needCite{} to represent \texttt{PdoExpansion} and
\texttt{PdoTraceback}, it would only have to analyse 76 instructions,
which is far more plausible.  I have not investigated this possibility
in any detail.

\subsection{bayes}

This test is, like labyrinth, formed by taking one of the STAMP
benchmark programs and removing one of the critical sections.  A
simplified version of the original program is shown in
Figure~\ref{fig:eval:bayes}.  It implements a task queue as a
singly-linked list with a C++-like iterator protocol\needCite{}.  When
a worker thread is ready to start a new task, it gets the first
element from the list (line 12), checks whether the list was empty
(line 13), and, if it was not, removes the element from the list (line
14) and proceeds to use it.  The \texttt{list\_remove} operation works
by scanning the list (\texttt{findPrevious}, line 1) to find the
element prior to the one to be removed (which in this case will be a
special dummy head element), using that to check whether the element
to be removed is present (line 3) and, if it is, removing the target
element from the list (lines 4 to 6).

\begin{figure}
  \texttt{
    \begin{tabular}{llll}
      & \multicolumn{3}{l}{bool list\_remove(list, what) \{}\\
      1 & & \multicolumn{2}{l}{prev = findPrevious(list, what);}\\
      2 & & \multicolumn{2}{l}{node = prev->next;}\\
      3 & & \multicolumn{2}{l}{if ((node != NULL) \&\& list->compare(node->data, what) == 0) \{} \\
      4 & & & prev->next = node->next;\\
      5 & & & node->next = NULL;\\
      6 & & & free(node);\\
      7 & & & return TRUE;\\
      8 & & \multicolumn{2}{l}{\}}\\
      9 & & \multicolumn{2}{l}{return FALSE;}\\
      10 & \multicolumn{3}{l}{\}}\\
      \\
      11 & \multicolumn{3}{l}{acquire\_lock();}\\
      12 & \multicolumn{3}{l}{it = taskList->begin();}\\
      13 & \multicolumn{3}{l}{if (it != taskList->end();) \{}\\
      14 & & \multicolumn{2}{l}{status = list\_remove(taskList, it->get());}\\
      15 & & \multicolumn{2}{l}{assert(status);}\\
      16 & \multicolumn{3}{l}{\}}\\
      17 & \multicolumn{3}{l}{release\_lock();}\\
    \end{tabular}
  }
  \caption{The bayes test program.}
  \label{fig:eval:bayes}
\end{figure}

For this test, I removed the lock operations on lines 11 and 17.  This
introduced several possible crashing bugs.  I ran the program 100
times without any modifications so as to establish a baseline:

\begin{itemize}
\item The assertion on line 15 fired 12 times.
\item The \texttt{free} on line 6 caused a double-free crash in the C
  library 12 times.
\item All other runs completed without apparent errors.
\end{itemize}

The Bayes program was invoked with the parameters \texttt{-v32 -r1024
  -n2 -p20 -i2 -e2 -s1 -t 2}.

Both of these bugs are in the current form to be investigated by
{\technique}, and so I produced \glspl{bugenforcer} for both of them.
I investigated the assertion failure first.  The race here is
essentially a time-of-check, time-of-use one, where the list element
might be removed by another thread in between the check on lines 12
and 13 in the caller and the call to \texttt{findPrevious} on line 1.
The path from line 12 to the crash on line 15 is 48 instructions long,
and so I analysed this bug with an \gls{analysiswindow} of 50
instructions.  This produced a single \gls{verificationcondition},
which I converted to a \gls{bugenforcer}.  The enforcer was moderately
effective: of 100 runs of the program, 26 suffered the bug under
investigation, while 74 suffered the double-free bug.  This is not
quite the desired result, which would have been for the program to
suffer the bug under investigation every time, but would still
probably be useful to a programmer investigating the bug.

I next investigated the double-free bug on line 6.  The race in this
case is that another thread might release the node in between the load
on line 2 and the free on line 6.  The path in this case is 28
instructions, and so I analysed this bug with \gls{alpha} set to 30
instructions\footnote{Unfortunately, attempting to analyse this bug
  with a \gls{analysiswindow} of 50 instructions ran out of memory on
  my 8GiB test machine without producing any
  \glspl{verificationcondition}.}.  This, again, produced a single
\gls{bugenforcer}.  I ran the program 100 times with this enforcer
and all of these runs suffered the double free bug, precisely as
desired.

{\Technique} can generate fixes for both of these bugs.  I ran the
program 100 times with each of the fixes applied.  With the
double-free only fix applied, the program completed successfully 90
times, suffered the assertion failure five times, and suffered five
other crashes elsewhere in the program.  With the assertion-only fix
applied, the program completed 95 runs successfully but failed an
assertion elsewhere in the program in the remaining 5 cases.  As
expected, both partial fixes eliminate the bug which they are designed
for, and marginally improve the program's reliability over the
baseline case, but do not fix any of the other bugs in the program.

{\Technique} can also generate a combined fix which fixes both bugs.
I ran the program 100 times with that combined fix applied.  In that
case, every run completed successfully.  This is because the union of
the critical sections for the two bugs is the same as the critical
section which I removed in order to produce the test program, and so
the combined fix restores the program to its correct functionality.

\todo{Could maybe give some more details here?  e.g. perf numbers,
  details of fixes and enforcers?}

\section{Experiments with real programs}
\label{sect:eval:real}

\subsection{pbzip2}

\todo{Writing in this section is horrible.}  I ran {\implementation}
in its bug-finding mode on pbzip2.  This included building a program
model, generating all {\StateMachines}, converting them to enforcers,
and then running pbzip2 under each enforcer in turn.  This did not
find any bugs in pbzip2.

Building the \glspl{bugenforcer} took a total of $2090 \pm 50$
seconds, mean and standard deviation of mean for ten runs.  The
largest single component of this was building the
\glspl{verificationcondition}, which took $1140 \pm 30$ seconds,
followed by converting the \glspl{verificationcondition} to
\glspl{bugenforcer}, which took $950 \pm 20$ seconds.  The static
analysis phase was much faster, taking $4.29 \pm 0.05$ seconds.

{\Implementation} produced a total of 24 \glspl{bugenforcer} for this
test:

\begin{itemize}
\item There were 154 potentially-crashing instructions in the
  program binary.
\item Of those, 7 failed to generate a crashing {\StateMachine} due to
  a timeout.
\item The remaining 147 \glspl{crashingthread} generated 244
  interfering \glspl{cfg}.  There were no timeouts or other failures
  generating interfering \glspl{cfg}.
\item Of those 244 interfering \glspl{cfg}, 8 timed out. Ignoring
  those timeouts, all of the remaining 236 interfering \glspl{cfg}
  were successfully symbolically executed to produce
  \glspl{verificationcondition}.
\item 210 of those \glspl{verificationcondition} were shown to be
  unsatisfiable by the satisfiability checker, leaving 26 in need of
  dynamic checking.
\item There were two further timeouts converting
  \glspl{verificationcondition} to \glspl{bugenforcer}.
\end{itemize}

All timeouts were set to one minute in this experiment.  The same
operations suffered timeouts in every repeat of the experiment.

\todo{How long do the enforcers take to run?} \todo{Version of
  pbzip2?} \todo{Compile settings?} \todo{Value of $\alpha$?}

\subsection{Thunderbird}

\begin{figure}
  \subfigure[][Crashing thread]{
    \texttt{
      \begin{tabular}{llll}
        & \multicolumn{3}{l}{ProcessCurrentURL() \{}\\
        & & \multicolumn{2}{l}{if (m\_transport) \{}\\
        & & & m\_transport->SetTimeout();\\
        & & \}\\
        & \}\\
      \end{tabular}
    }
  }
  \subfigure[][Interfering thread]{
    \texttt{
      \begin{tabular}{lll}
        & \multicolumn{2}{l}{CloseStreams() \{}\\
        & & m\_transport = NULL;\\
        & \multicolumn{2}{l}{\}}\\
        \\
        \\
      \end{tabular}
    }
  }
  \caption{The thunderbird bug.}
  \label{fig:eval:thunderbird}
\end{figure}

\verb|thunderbird| is Mozilla bug number 391259\cite{Mery2007}, a
simple time-of-check, time-of-use race in the IMAP client component of
Thunderbird, a popular open-source e-mail client.  The relevant parts
of the program are shown in Figure~\ref{fig:eval:thunderbird}.  If
\verb|m_transport| is set to \verb|NULL| by \verb|CloseStreams()| in
between the two accesses in \verb|ProcessCurrentURL| then the program
will crash.  The simplest way to trigger this behaviour is for the
user to click on an IMAP folder and to then immediately close
Thunderbird.  For these experiments, I assumed that the crashing
instruction had already been identified, and so I had
{\implementation} investigate that one instruction rather than
generating every possible \glspl{verificationcondition} and
\glspl{bugenforcer}.

This is essentially the same bug as simple\_toctou, but embedded in a
much large program.  As such, {\implementation} produces a very
similar \gls{verificationcondition}, \gls{bugenforcer}, and fix.  The
enforcer is, however, much more difficult to use in this case.
Triggering this bug requires user interaction, and it is difficult to
perform the necessary operations in {\implementation}'s default 200ms
timeout.  Increasing the timeout to five seconds made it trivial to
trigger the desired behaviour: of ten attempts with the enforcer
loaded and using a five second timeout, every one reproduced the bug.
Without the enforcer, or using the 200ms timeout, ten attempts to
reproduce the bug all failed.  The code involved in the bug runs quite
rarely, and in a background thread, and so the user interface was
perfectly usable even with this very long delay.

{\Implementation} produced two \glspl{verificationcondition} for this
bug: one representing the expected race, and one representing a race
with another part of Thunderbird which can also set
\texttt{m\_transport} to \texttt{NULL}.  The second
\gls{verificationcondition} is a false positive: that part of
Thunderbird is correctly synchronised, and so cannot reproduce the
bug.  For all other stores to \texttt{m\_transport}, the
\gls{programmodel} is able to show that the stored value is a valid
pointer, and so no \glspl{verificationcondition} are generated.
{\Implementation} can generated \glspl{bugenforcer} for both
\glspl{verificationcondition}, or it can generate a combined
\gls{bugenforcer} which tries to check both bugs.  All of these
\glspl{bugenforcer} have the expected effect: the false positive one
has no apparent effect, whereas the true positive and combined ones
both successfully reproduce the bug.

The analysis time was quite reasonable here:

\begin{itemize}
\item The static analysis phase of building the \gls{programmodel}
  took $1240 \pm 10$ seconds, or about twenty minutes.
\item Building the \glspl{verificationcondition} took $390 \pm 10$
  milliseconds.
\item It took $640 \pm 10$ milliseconds to build each of the two
  single-\gls{verificationcondition} \glspl{bugenforcer}.
\item Building the combined \gls{bugenforcer} took $710 \pm 20$
  milliseconds.
\end{itemize}

All times mean and standard deviation of ten runs.  The time taken
building the \gls{programmodel} clearly completely dominates the other
phases.  This is unsurprising; it must analyse the entire program,
whereas the other phases analyse only the small window which is
relevant to the bug.  Twenty minutes is still a perfectly reasonable
analysis time for a program as complicated as Thunderbird.

This bug does illustrate one weakness of the {\technique} approach to
finding bugs: it is difficult to apply to programs which do not have a
good automated test suite.  {\Technique} generates a very large number
of false positives in its main analysis phase and relies on the
enforcers to distinguish them from real bugs, but this is only
effective if the enforcers can be run automatically, which is in turn
only possible when it is possible to exercise some reasonable
proportion of the program automatically.  Using {\technique}'s
bug-finding mode on Thunderbird would be completely infeasible for
this reason: it produces several thousand candidate bugs, which is
tolerable if each can be dismissed with a few seconds of further
computation, but not if each requires several minutes of user
interaction.

\subsection{MySQL}
\label{sect:eval:mysql}

This is MySQL bug 56324\needCite{}.  A simplified version of the buggy
code is shown in Figure~\ref{fig:eval:mysqld}.  This is, again, a
variant of the simple\_toctou bug, embedded in a much larger program.
As expected, {\implementation} is able to find the bug, generate an
enforcer, and generate a fix, and they all behave as expected.  For
this experiment, I selected a test out of the MySQL test suite which
exercises the buggy code (rpl\_change\_master) and ran it under the
dynamic analysis.  I then used the results of that dynamic analysis to
build a \gls{programmodel}, and used the resulting \gls{programmodel}
to analyse the crashing instruction, with \gls{alpha} set to 20.  This
produced two \glspl{verificationcondition}: the expected one
representing the dangerous interleaving, and a spurious one
representing an interleaving with the code which initialises
\texttt{PSI\_server} when the program starts up.  I converted these to
a single combined \gls{bugenforcer} and ran the rpl\_change\_master
test ten times under that test, and every run reproduced the bug.  By
comparison, ten runs without an enforcer failed to reproduce the bug
at all.

\todo{Put in some performance numbers here.}

{\Technique} can also generate a fix for this bug.  In this case, the
fix consists of two critical sections, one which covers the store in
the \gls{interferingthread} and another which covers the two loads of
\texttt{PSI\_server} in the \gls{crashingthread}.  This is sufficient
to eliminate the identified bad interleaving, but is probably not the
fix which a human programmer would have produced, as it does not cover
the actual call to \texttt{delete\_current\_thread}.  As such, it is
possible for the fixed program to call
\texttt{delete\_current\_thread} after \texttt{PSI\_server} is set to
\texttt{NULL}, or to call \texttt{delete\_current\_thread} multiple
times.  Fortunately, \texttt{delete\_current\_thread} itself only
accesses thread-local variables and is idempotent, and so the fix is
still effective.

\begin{figure}
  \subfigure[][Crashing thread]{
    \texttt{
      \begin{tabular}{ll}
        \multicolumn{2}{l}{if (PSI\_server) \{}\\
        & PSI\_server->delete\_current\_thread();\\
        \}\\
      \end{tabular}
    }
  }
  \subfigure[][Interfering thread]{
    \texttt{
      \begin{tabular}{l}
        \\
        PSI\_server = NULL;\\
        \\
      \end{tabular}
    }
  }
  \caption{The MySQL bug.}
  \label{fig:eval:mysqld}
\end{figure}

{\Implementation} can also be used in bug-finding mode on MySQL, as it
has an extensive automated test suite.  In this mode it was able to
locate both the bug given here, and was also able to locate several
other, similar, races on \texttt{PSI\_server} which were unknown to
the author before writing the tool.  This suggests that
{\implementation} can sometimes be used to find previously-unknown
bugs, as desired.

\todo{Complication: this bug is only present in builds of MySQL which
  don't have compiler optimisations, because otherwise the compiler
  caches \texttt{PSI\_server} in a register and avoids the crash.}

\section{Performance exploration}
\label{sect:eval:time_details}

\subsection{Finding bugs}
This section looks at how the time taken by the analysis breaks down
across the various phases and components of {\implementation}.  I look
first at the component which must be run for each potentially-crashing
instruction.  For this experiment, I selected 1000 memory-accessing
instructions at random from the optimised build of MySQL, analysed
them with {\implementation}, and timed how long the various components
of the analysis took.  The results are shown in
Figure~\ref{fig:eval:time_breakdown:crashing}.  This chart shows
several important properties of the analysis:

\begin{figure}
  \input{eval/bubble_charts/bubble1.tex}
  \caption{Time taken by the different phases.  See text for detailed
    description.}
  \label{fig:eval:time_breakdown:crashing}
\end{figure}

\begin{itemize}
\item Each phase of the analysis is represented by a distinct blob.
  Each blob is formed from a number of horizontal lines, with each
  line representing a single potentially-crashing instruction which
  required non-trivial work at that phase of the analysis.  The width
  of these lines gives the amount of time which that instruction spent
  in that phase, using the scale immediately below the chart.  This means
  that several useful properties can be read off from the blobs:

  \begin{itemize}
  \item The area of a blob is proportional to the total amount of time
    spent in the corresponding phase, across all potentially-crashing
    instructions.
  \item The width of a blob is proportional to the maximum time spent
    in that phase by any potentially-crashing instruction.
  \item The shape of the blob shows the level of variability in the
    time spent in the phase.
  \item The height of the blob shows how many instructions required
    analysis in that phase.
  \end{itemize}

  The lines within a blob are ordered so that they increase
  monotonically and then decrease monotonically but is otherwise
  arbitrary.  In particular, the height of a blob's centre is not
  meaningful.

\item The horizontal position of a blob is given by the average time
  at which that phase finished, for instructions which required
  non-trivial work in that phase, using the scale given on the x axis.
  Note that this is a different scale to the one used for the length
  of horizontal lines within a blob; this is necessary because the
  mean and maximum for some of the distributions displayed differ by
  two orders of magnitude.

\item Not every instruction requires analysis, and the solid line
  shows how many instructions can be dismissed early.  For instance,
  roughly 19\% of the instructions analysed were accessing either the
  stack or a fixed address which is guaranteed to be mapped, and there
  is no need to perform any analysis on these instructions.
  Similarly, for 48\% of instructions the {\StateMachine} simplifiers
  were able to reduce the {\StateMachine} to the point where it did
  not depend on any memory loads for which the dynamic aliasing model
  could find \glspl{interferingstore}, and so there were no
  interfering \glspl{cfg} for these potentially-crashing instructions.

\item Similarly, some instructions do not start later phases because
  of timeouts earlier in the analysis phase.  The number of timeouts
  before a particular phase is given by the difference between the
  solid and dashed lines.  In this case, the only timeout occurred
  while simplifying the crashing {\StateMachine}.  The timeout was set
  to one minute.

\item The chart only shows time spent in the most important phases.
  Any remaining time is shown in an ``Other'' blob on the right of the
  chart.
\end{itemize}

The chart shows three features which are worth noting:

\begin{itemize}
\item Most of the time is spent in the ``Process interfering
  \glspl{cfg}'' phase.  This is discussed in more detail later.
\item The ``Derive C-atomic'' phase is tiny.  This indicates that the
  {\StateMachine} simplifiers are doing a good job for these
  {\StateMachines}, and so the symbolic execution engine has very
  little to do.
\item All of the distributions have very large positive outliers,
  indicating that most potential bugs can be dismissed very easily but
  the remaining ones require a significant amount of analysis.  This
  is unsurprising: many of the algorithms involved have very bad but
  very rare worst-case complexity; steps which avoid those worst cases
  are processed quickly, but those which hit them perform very badly.
\end{itemize}

I now investigate the time taken processing each interfering
\gls{cfg}.  Figure~\ref{fig:eval:time_breakdown:interfering} shows the
time taken by the various sub-phases of this, in the same style as the
previous figure.  The data for this chart was collected in the same
run of {\implementation} as that for the previous chart.  There were
2174 interfering \glspl{cfg} in total.

The third phase of this chart, ``rederive crashing {\StateMachine}'',
perhaps requires further explanation.  Fixing the interfering
\gls{cfg} fixes the set of instructions which the \gls{crashingthread}
might race with, which can often allow further simplifications in the
crashing {\StateMachine} and in C-atomic.  This phase is responsible
for discovering such simplifications.  In this case, those
simplifications were sufficient to completely dismiss 56\% of the
interfering \glspl{cfg}.

\begin{figure}
  \input{eval/bubble_charts/bubble2.tex}
  \caption{Time taken by the different phases.  See text for detailed
    description.}
  \label{fig:eval:time_breakdown:interfering}
\end{figure}

Perhaps the most surprising property of this chart is that
symbolically executing the cross-product {\StateMachine} is less
expensive than executing the \gls{ic-atomic} one, despite having to
handle far more complex instruction schedules.  There are two reasons
for this.  First, the cross-product construction converts the
program's concurrency behaviour into control flow operations, and the
{\StateMachine} simplifiers are particularly good at simplifying
control flow.  The symbolic execution phase therefore never sees most
of the interleaving-related complexity (although the cross-product
simplification phase does, and therefore takes modestly longer than
the other simplification phases).  Second, the cross-product symbolic
execution has more information than the \gls{ic-atomic} one.  The
entire reason for deriving \gls{ic-atomic} is to constraint the set of
environments in which the threads might execute, and the cross-product
symbolic execution can take advantage of that to eliminate some
potential paths from consideration.

\todo{I'd like to say more about what's causing those timeouts, but
  that's tricky because I don't really know; my guess is that they're
  mostly cause by bad BDD orderings.}

\subsection{Building bug enforcers}
The next thing to look at is how long it takes to convert
\glspl{verificationcondition} to \glspl{bugenforcer}.  For this
experiment, I chose 5000 \glspl{verificationcondition} at random from
those generated by analysing the whole of MySQL, converted each one to
an enforcer, and timed how long each phase took; the results are shown
in Figure~\ref{fig:eval:time_breakdown:convert_to_enforcer}\footnote{I
  also performed a similar experiment using the 297
  \glspl{verificationcondition} generated by the previous phase.  The
  results were broadly similar, but the smaller sample set meant that
  some low-probability outcomes were completely unrepresented in the
  results.}.

Several of the \glspl{verificationcondition} could not be converted to
\glspl{bugenforcer} due to running out of memory.  36 (0.7\%) failed
while slicing the \gls{verificationcondition} according to the
happens-before graph and 13 (0.3\%) failed while placing the side
conditions.  The system used to run the test had 8GiB of memory.  To
some extent, this reflects a problem with my implementation
{\implementation} rather than the technique {\technique}:
{\implementation} manages most of the data structures involved using a
garbage collector which only runs in between the phases, which
simplified the implementation but will have significantly increased
the system's memory consumption.  It would therefore be reasonable to
suppose that a more refined implementation could usefully reduce the
number of \glspl{verificationcondition} which fail.  It is unlikely,
however, that it would be possible to eliminate all failures
completely, as both phases make heavy use of BDD reordering, which has
an exponential worst-case memory consumption.  The algorithm is design
such that most of the time the BDDs are in nearly the correct order to
begin with, and so this worst case can usually be avoided, but the
minority of cases which do hit it will always require a very large
amount of memory.

The only other phase which requires a non-trivial amount of time is
the final one, compiling the enforcer.  This involves converting the
generated enforcer to C and passing it off to the system compiler (in
this case, gcc version 4.4.3) to convert it into an ELF library.  The
time taken by this phase is completely dominated by the time taken by
the external compiler.

\begin{figure}
  \input{eval/bubble_charts/bubble4.tex}
  \caption{Time taken by the different phases involved in converting a
    \gls{verificationcondition} to a \gls{bugenforcer}.}
  \label{fig:eval:time_breakdown:convert_to_enforcer}
\end{figure}

\subsubsection{Effects of the $\alpha$ parameter}
\label{sect:eval:alpha}

\todo{This is for the unoptimised build of MySQL; should also do it
  for the optimised one, because the results are different in an
  interesting way.}

\todo{Some of these have no error bars, and they also have a quite
  large uncertainty.}

\begin{figure}
  \input{eval/alpha/unopt/bpm.tex}
  \caption{Effect of the $\alpha$ parameter on the time taken to build
    the \gls{crashingthread} {\StateMachines}.  Note log scale.}
  \label{fig:perf:alpha:bpm:unopt}
\end{figure}

The value of the $\alpha$ parameter has a large effect on the
behaviour of the \gls{verificationcondition} generating process.  I
now investigate these effects in slightly more detail.  For these
experiments, I selected 1000 instructions at random in an unoptimised
build of MySQL and ran the analysis on each of them in turn.  I then
repeated the analysis for a number of different values of \gls{alpha}.

The first thing I investigated was the effect of \gls{alpha} on the
time taken to derive the \gls{crashingthread} {\StateMachines},
including deriving the \gls{cfg}, compiling it to a {\StateMachine},
performing the initial simplification, and deriving C-atomic.  The
results are shown in the box plot in
Figure~\ref{fig:perf:alpha:bpm:unopt}.  For this experiment, I used a
sixty second timeout and treated any instructions which took longer
than that to generate the crashing {\StateMachine} as having failed.
Note that the means were calculated discarding timeouts, whereas the
quantiles were calculated treating timeouts as having taken precisely
sixty seconds.  The overall shape of this chart is roughly what one
would expect: the time taken to analyse a particular
potentially-crashing instruction rises roughly exponentially with the
value of \gls{alpha}, and the proportion of timeouts rises rapidly
once \gls{alpha} exceeds a certain value, which in this case is about
40.

The next thing I investigated was the effect of \gls{alpha} on the
size of the \glspl{crashingthread}.  This chart shows the sizes of the
\glspl{crashingthread}, in terms of the number of (dynamic) program
instructions represented in the \gls{cfg}, the number of states in the
{\StateMachine} before simplification, and the number of states in the
{\StateMachine} after simplification.  These results are again easy to
explain:

\begin{itemize}
\item The number of instructions in the thread is always at least
  \gls{alpha}, and the average number grows roughly exponentially with
  \gls{alpha}.
\item The number of states in the initial {\StateMachine} is generally
  larger than the number of instructions.  There are two reasons for
  this.  The first is that the machine code being analysed is in a
  CISC instruction set, whereas the {\StateMachine} language is very
  RISC-like, and so each instruction usually requires multiple states.
  The second is that {\implementation} uses some initial special
  states, such as \state{ImportRegister} and some not described here,
  to set up an initial environment for the {\StateMachine} and to
  import properties from the \gls{programmodel}.  The number of these
  special states is usually around 30, and so they represent an
  important source of states when \gls{alpha} is small but not when it
  becomes larger.
\item The number of states in the simplified {\StateMachine} follows
  roughly the same trend as the other measures, but is usually about a
  factor of ten smaller.  This simply indicates that the simplifier is
  working as desired.
\end{itemize}

Note that the distributions shown in all three charts exclude any
\gls{crashingthread} which suffered a timeout before generating the
relevant object (so, for instance, the ``Simplified states'' chart
excludes any \glspl{crashingthread} which did not complete
simplification).  This will have caused some selection bias, as the
larger \glspl{crashingthread} are most likely to suffer a timeout.
This will be particularly important for the high quantiles of the high
\gls{alpha} distributions.

\begin{figure}
  \input{eval/alpha/unopt/crashing_size.tex}
  \caption{Effect of the $\alpha$ parameter on the size of the
    \gls{crashingthread}.}
\end{figure}

\begin{figure}
  \input{eval/alpha/unopt/gsc.tex}
  \caption{Effect of the $\alpha$ parameter on the derivation of the
    \gls{interferingthread} \glspl{cfg}.}
  \label{fig:perf:alpha:gsc:unopt}
\end{figure}

Once the \gls{crashingthread} {\StateMachine} has been generated, the
next step is to derive \glspl{cfg} for the \glspl{interferingthread}.
The response of this process to the $\alpha$ parameter is shown in
Figure~\ref{fig:perf:alpha:gsc:unopt}.  As before, increasing $\alpha$
leads to an increase in the both the time taken to perform the
analysis and the number of timeouts.  This behaviour needs little
explanation.

The lower part of the chart, showing the number of interfering
\glspl{cfg} per \gls{crashingthread}, is more surprising.  Many of
\glspl{crashingthread} do not produce any interfering \gls{cfg} at
all.  This indicates that the potential crash could be dismissed based
on contents of the crashing {\StateMachine} and the dynamic aliasing
model, with no need for any further analysis.  There are several ways
that this could happen:

\begin{itemize}
\item It might be that none of the \gls{alpha} instructions leading up
  to the crashing instruction loaded from thread-shared memory, in
  which case they trivially cannot suffer a concurrency bug of this
  form.
\item Similarly, the simplifiers can sometimes show that a the value
  of a load from shared memory cannot possibly influence whether the
  thread will crash, and if so the \gls{crashingthread} can be
  dismissed without needing to derive the interfering \glspl{cfg}.
\item If the \gls{crashingthread} makes only a single load from memory
  then it is safe to dismiss any interfering \glspl{cfg} which make
  only a single store, as in that case there are no interleavings
  which are not equivalent to running the two threads atomically.
\end{itemize}

All of these special cases are clearly more likely when \gls{alpha} is
small, and indeed we observe that many of the small \gls{alpha} tests
do not generate any interfering \glspl{cfg}.  This does not, however,
explain why the large \gls{alpha} tests also show a large number of
\glspl{crashingthread} which do not generate any interfering
\glspl{cfg}.  The explanation here is selection bias: only simple
\glspl{crashingthread} avoid timing out before generating the
interfering \glspl{cfg}, and these tend to generate fewer interfering
\glspl{cfg}.

The next step of the analysis is to take these pairs of crashing
{\StateMachines} and interfering \glspl{cfg} and convert them into
\glspl{verificationcondition}.  This is shown in
Figure~\ref{fig:perf:alpha:gvc:unopt}.  Note that the denominator for
the timeout rate has changed: previously, it was the total number of
potentially-crashing instructions, whereas now it is the number of
pairs generated by the previous analysis stages.

The effect of $\alpha$ on the number of \glspl{verificationcondition}
generated is easiest to understand if it is expressed as the fraction
of \glspl{cfg} which generate a
\gls{verificationcondition}\footnote{Recall that there is at most one
  \gls{verificationcondition} for each interfering \gls{cfg}.}  This
initially increases with $\alpha$ and then begins to fall off again
when $\alpha$ increases further.  Increasing $\alpha$ allows
{\technique} to consider interleaving the program's threads over a
larger window of instructions, and so to consider more complicated
concurrency behaviour, and this leads to a modest increase in the
fraction of interfering \glspl{cfg} which might cause undesirable
behaviour in the \gls{crashingthread}, accounting for the initial
increase.  The subsequent decrease is, again, a selection effect, as
only threads with simple aliasing and concurrency behaviour will reach
this stage of the analysis.  The time taken by the analysis follows
broadly the same pattern, for essentially similar reasons.

\begin{figure}
  \input{eval/alpha/unopt/gvc.tex}
  \caption{Effect of the $\alpha$ parameter on the number of
    \glspl{verificationcondition} generated and the time taken to do
    so.  All timeouts in earlier stages of the analysis are excluded.
    The mean excludes timeouts in this stage.  The time taken includes
    building the interfering {\StateMachine}, rederiving crashing
    {\StateMachine}, deriving \gls{ic-atomic} constraints, building
    the \gls{verificationcondition}, and the final satisfiability
    check.}
  \label{fig:perf:alpha:gvc:unopt}
\end{figure}

\begin{figure}
  \input{eval/alpha/opt/bpm.tex}
  \caption{Effect of the $\alpha$ parameter on the time taken to build
    the \gls{crashingthread} {\StateMachines} for an optimised build
    of MySQL.}
  \label{fig:perf:alpha:bpm:opt}
\end{figure}

\begin{figure}
  \input{eval/alpha/opt/gsc.tex}
  \caption{Effect of the $\alpha$ parameter on the number of
    \gls{interferingthread} \glspl{cfg} per \gls{crashingthread}
        {\StateMachine} for an optimised build of MySQL.  \todo{Urk}}
  \label{fig:perf:alpha:gsc:opt}
\end{figure}

\begin{figure}
  \input{eval/alpha/opt/gvc.tex}
  \caption{Effect of the $\alpha$ parameter on the number of
    \glspl{verificationcondition} generated and the time taken to do
    so for an optimised build of MySQL.}
  \label{fig:perf:alpha:gvc:opt}
\end{figure}

The results with an optimised version of MySQL are broadly similar,
and are shown in Figures~\ref{fig:perf:alpha:bpm:opt},
\ref{fig:perf:alpha:gsc:opt}, and \ref{fig:perf:alpha:gvc:opt}.  The
most obvious difference between the optimised and unoptimised graphs
is that the optimised program takes longer to analyse, and suffers a
higher timeout rate.  This is not particular surprising: the optimised
program fits more complicated behaviour into the \gls{analysiswindow}
and so takes longer to process.  \todo{Not sure there's much to say
  about that.}

\subsection{Generating fixes}

The final component of {\technique} which I investigate here is fix
generation.  The performance behaviour here is far simpler than for
the other phases.  It is summarised in
\autoref{tab:eval:gen_fix_perf}.  The time taken is clearly completely
dominated by the time taken by the system compiler, and is perfectly
reasonable for these patches.

\begin{table}
  \begin{tabular}{|l|l|l|l|l|}
    \hline
    Phase & Mean & $5^{th}$\% & $95^{th}$\% & Population standard deviation \\
    \hline
    Find critical sections & $0.50 \pm 0.02$ & 0.1 & 1.6 & 1.1 \\
    Identify patch points & $0.805 \pm 0.008$ & 0.3 & 1.4 & 0.6 \\
    Partial evaluation & $0.75 \pm 0.06$ & 0.2 & 1.5 & 4.4 \\
    System compiler & $105.1 \pm 0.2$ & 85 & 128 & 14 \\
    \hline
    Total & $107.2 \pm 0.2$ & 87 & 131 & 15 \\
    \hline
  \end{tabular}
  \caption{Time taken to convert 5000 \gls{verificationcondition}
    generated from MySQL to fixes.  All times in milliseconds.
    Standard deviation of mean calculated using central limit
    theorem.}
  \label{tab:eval:gen_fix_perf}
\end{table}

Table~\ref{tab:eval:gen_fix_perf:props} gives some statistics related
to the complexity of the patches generated.  The properties examined
are:

\begin{itemize}
\item The number of program instructions in the protected region.
\item The number of lock and unlock operations present in the
  generated machine code fragment.
\item The size the generated machine code fragment, in terms of both
  the number of instructions and the number of bytes.  This is further
  broken down into the number of instructions needed to acquire the
  lock, the number to release it, and the number in other forms of
  patching overhead.
\item The number of instructions in the original program which are
  patched to gain control of the program.
\item The number of instructions in the $\mathit{Cont}$ set (see
  \autoref{sect:enforce:gain_control}).
\item The number of relocations.  This is the number of places in the
  generated machine code which must be modified at run-time when the
  patch is loaded, to account for the patch being loaded at an address
  which is unknown at compile time.
\end{itemize}

The table shows the mean and central limit theorem standard deviation
of the mean, the $5^{th}$ and $95^{th}$ percentiles, and the
population standard deviation ($\sigma$) of the parameter.  As can be
seen, the generated patches usually consist of a few dozen
instructions or a few hundred bytes of machine code.  The size is
therefore perfectly manageable.  Note that most of the distributions
shown in the table are non-Gaussian and so, for instance, the $\sigma$
column should be treated with caution.

\begin{table}
  \begin{tabular}{|l|l|l|l|l|l|l|l|}
    \hline
                           & \multicolumn{4}{c|}{Distribution summary} & \multicolumn{3}{c|}{Effect on time taken} \\
    \hline
    Property of fix        & Mean & $5^{th}$\% & $95^{th}$\% & $\sigma$ & $\beta$, ms & $\gamma$ & R\\
    \hline
    Protected instructions & $17.9 \pm 0.4$  & 3   & 40  & 30  & 0.2  & 0.04 & 0.28 \\
    Lock operations        & $2.58 \pm 0.04$ & 1   & 4   & 2.7 & 1.6  & 0.009& 0.13 \\
    Unlock operations      & $2.88 \pm 0.02$ & 1   & 5   & 1.4 & 0.12 & 0.00001& 0.005 \\
    Patch instructions     & $70 \pm 1$      & 27  & 127 & 72  & 0.06 & 0.009& 0.13\\
    \hspace{5mm}Acquiring lock&$15.5 \pm 0.2$& 6   & 24  & 16  &$\ast$&$\ast$& $\ast$\\
    \hspace{5mm}Releasing lock&$17.3 \pm 0.1$& 6   & 30  & 8   &$\ast$&$\ast$& $\ast$\\
    \hspace{5mm}Overhead   & $19.1 \pm 0.5$  & 0   & 50  & 32  &$\ast$&$\ast$& $\ast$\\
    Patch bytes            & $326 \pm 5$     & 126 & 594 & 346 & 0.01 & 0.009& 0.13\\
    Patch points           & $2.26 \pm 0.01$ & 2   & 3   & 0.88& 0.8  & 0.0008& 0.04\\
    $\mathit{Cont}$        &$0.034 \pm 0.003$& 0   & 0   & 0.2 &$\ast$&$\ast$& $\ast$\\
    Relocations            & $11.0 \pm 0.1$  & 6   & 18  & 8.0 & 0.5  & 0.03 & 0.24\\
    \hline
  \end{tabular}
  \caption{Summaries of some gross properties of the generated fixes,
    and their effects on the time taken by all phases of fix
    generation.  See text for details. $\ast$: Not meaningful.}
  \label{tab:eval:gen_fix_perf:props}
\end{table}

The table also shows the dependence of the time taken to generate the
patch on some of these measurements.  For these calculations, I
performed a simple linear regression of the total time taken on the
parameter.  The table shows $\beta$, the gradient of the resulting
line, $\gamma$, the proportion of the standard deviation of time which
can be explained by that linear regression, and R, the Pearson
product-moment correlation coefficient.  As can be seen, the number of
protected instructions and the number of relocations are modestly
positively correlated with the total time taken, while the other
parameters show weaker correlations, or none at all.  This gives an
indication of how the patch process would scale to more complex
critical sections.  These results should be treated with some caution,
though: none of the correlations are very strong, and many of the
error distributions are non-Gaussian, the sample size is only modest,
and there are strong correlations between the different properties of
the fix in addition to between the properties and the time taken.

\section{W isolation}
\label{sect:eval:w_isolation}

\todo{No error bars on some of those measurements!  This section is
  very noddy.}

I now investigate the effect of the \gls{w-isolation} on the analysis,
in terms of the time taken and the set of
\glspl{verificationcondition} generated.  I took the same 1000
instructions from the optimised build of MySQL and analysed them with
and without the \gls{w-isolation} assumption, with \gls{alpha} set to
20 in both cases.  The results are shown in
Table~\ref{table:eval:w-isolation}.  As expected, the
\gls{w-isolation} assumption modestly decreases both the number of
\glspl{verificationcondition} which must be checked by run-time
\glspl{bugenforcer} and the time taken to generate them.

\begin{table}
  \begin{tabular}{lll}
    & \multicolumn{2}{c}{\gls{w-isolation}} \\
                                                           & Enabled & Disabled \\
    Timeouts building \gls{crashingthread} {\StateMachines}& 245                    & 245 \\
    Time to build \gls{crashingthread} {\StateMachine} & $0.5 \pm 0.1$* & $0.6 \pm 0.1$* \\
    \Glspl{crashingthread} generating no \glspl{interferingthread} & 456                    & 406 \\
    Time to build \gls{interferingthread} \glspl{cfg}, per \gls{crashingthread} & $1.5 \pm 0.3$* & $1.5 \pm 0.3$* \\
    Number of \gls{interferingthread} \glspl{cfg} per \gls{crashingthread} & $5.1 \pm 0.5\dagger$ & $6.3 \pm 0.6\dagger$ \\
    \hspace{5mm}Total number \gls{interferingthread} \gls{cfg} & 3819 & 4778 \\
    Timeouts building \gls{interferingthread} {\StateMachines} & 17 & 71 \\
    \hspace{5mm}As percentage of \gls{interferingthread} \glspl{cfg} & 0.4\% & 1.5\% \\
    Time to build \glspl{interferingthread} {\StateMachines} & $0.11 \pm 0.02$* & $0.12 \pm 0.01$* \\
    Timeouts in symbolic execution & 35 & 281 \\
    \hspace{5mm}As percentage of \glspl{interferingthread} & 0.9\% & 7.6\% \\
    Symbolic execution time per \glspl{interferingthread} & $0.96 \pm 0.07$* & $1.31 \pm 0.08$* \\
    Number of \glspl{verificationcondition} & 682 & 863 \\
    \hspace{5mm}As percentage of \glspl{interferingthread} & 18\% & 18\% \\
    \hspace{5mm}Per \gls{crashingthread} & 0.9 & 1.1 \\
  \end{tabular}
  \caption{Effects of the \gls{w-isolation} assumption.  *: mean and
    standard deviation of mean for all runs which did not time out,
    seconds.  $\dagger$: mean and standard deviation of mean}
  \label{table:eval:w-isolation}
\end{table}

\section{Dynamic analysis}
\label{sect:eval:dynamic_analysis}

\subsection{Effects on analysis performance and correctness}

\todo{This is another one which is lacking error bars.}

Similarly to the investigation of the effects of the \gls{w-isolation}
assumption in Section~\ref{sect:eval:w_isolation}, I now investigate
the effects of the dynamic aliasing analysis on the generation of
\glspl{verificationcondition}\editorial{gurk}.  It is difficult to
completely avoid using the information from the dynamic analysis while
generating \glspl{verificationcondition}, as it is needed to generate
the \gls{interferingthread} \glspl{cfg}, but it is possible to disable
the parts of the {\StateMachine} simplifications which depend on it
and the parts of the symbolic execution engine which use it, and that
is what I did in this test.  The results are shown in
Table~\ref{table:eval:effect_of_dyn}.  Note that the aliasing enabled
column of this table was produced from the same experiment as produced
the \gls{w-isolation} enabled column in
Table~\ref{table:eval:w-isolation}.

There are two main observations to draw from this table: discarding
the information from the dynamic analysis dramatically reduces the
number of timeouts building the \gls{crashingthread} {\StateMachines},
but even more dramatically increases the number of timeouts and time
taken in the other phases of the analysis.  These two observations
have the same explanation.  The various {\StateMachine} simplification
steps depend almost completely on the information from the dynamic
aliasing analysis in order to determine which memory accesses alias.
Without that information, they can achieve very little, ensuring that
they exit very quickly and the \gls{crashingthread} {\StateMachines}
are all built very quickly, but also ensuring that the later analysis
phases which make use of the {\StateMachines} perform very badly.

\begin{table}
  \begin{tabular}{lll}
    & \multicolumn{2}{c}{Dynamic aliasing analysis} \\
                                                           & Enabled & Disabled \\
    Timeouts building \gls{crashingthread} {\StateMachines}& 245                    & 9 \\
    Time to build \gls{crashingthread} {\StateMachine} & $0.5 \pm 0.1$* & $0.7 \pm 0.1$* \\
    \Glspl{crashingthread} generating no \glspl{interferingthread} & 456                    & 7 \\
    Timeouts building \gls{interferingthread} \glspl{cfg} & 0 & 2 \\
    Time to build \gls{interferingthread} \glspl{cfg}, per \gls{crashingthread} & $1.5 \pm 0.3$* & $1.5 \pm 0.4$* \\
    Number of \gls{interferingthread} \glspl{cfg} per \gls{crashingthread} & $5.1 \pm 0.5\dagger$ & $12 \pm 1\dagger$ \\
    \hspace{5mm}Total number \gls{interferingthread} \gls{cfg} & 3819 & 4778 \\
    Timeouts building \gls{interferingthread} {\StateMachines} & 17 & 130\\
    \hspace{5mm}As percentage of \gls{interferingthread} \glspl{cfg} & 0.4\% & 3\% \\
    Time to build each \glspl{interferingthread} {\StateMachine} & $0.11 \pm 0.02$* & $1.31 \pm 0.04$* \\
    Timeouts in symbolic execution & 35 & 1601 \\
    \hspace{5mm}As percentage of \glspl{interferingthread} & 0.9\% & 34\% \\
    Symbolic execution time per \glspl{interferingthread} & $0.96 \pm 0.07$* & $3.9 \pm 0.3$* \\
    Number of \glspl{verificationcondition} & 682 & 656 \\
    \hspace{5mm}As percentage of \glspl{interferingthread} & 18\% & 14\% \\
    \hspace{5mm}Per \gls{crashingthread} & 0.9 & 0.7 \\
  \end{tabular}
  \caption{Effects of the dynamic aliasing analysis.  *: mean and
    standard deviation of mean for all runs which did not time out,
    seconds.  $\dagger$: mean and standard deviation of mean.}
  \label{table:eval:effect_of_dyn}
\end{table}

\subsection{Convergence times}

One obvious question with a dynamic analysis is how long it takes to
achieve reasonable coverage.  This experiment aims to investigate
that.  For this experiment, I modified the dynamic analysis tool so
that it periodically recorded a snapshot of the aliasing table, and I
then examined these snapshots to quantify the rate at which aliasing
pairs were added to the table.  The results for mysql, Thunderbird,
and pbzip2 are shown in Figures~\ref{fig:eval:dyn_convergence:mysqld},
\ref{fig:eval:dyn_convergence:thunderbird}, and
\ref{fig:eval:dyn_convergence:pbzip2}, respectively.  These graphs
show the number of aliasing pairs in the table at time $t$, as a
percentage of the number of pairs in the final table.  These tests
involved starting and restarting the program several times; such
restarts are shown as dashed vertical lines.  The most important
result here is that the dynamic aliasing analysis achieves reasonable
coverage reasonably quickly, usually within ten minutes to quarter of
an hour.  Asking users to run this dynamic analysis before using
{\technique} is therefore not an unreasonable burden.

For all tests, the dynamic analysis tool was configured to take one
snapshot every 1,000,000 Valgrind basic blocks, which corresponds to a
couple of hundred milliseconds of execution time. This is arguably a
somewhat unfair test: taking a snapshot can itself take several
hundred milliseconds, depending on the size of aliasing table is at
the time, and so this experiment actually spent most of its time
taking snapshots rather than running the test program.  A more
realistic test would probably show faster convergence.

The precise behaviour of the dynamic analysis depends, in part, on the
workload which is being used to exercise the program.  For these
experiments, I used the following workloads:

\begin{itemize}
\item For MySQL, I used four tests out of the MySQL test suite. The
  first of these, rpl\_change\_master, was chosen to match the bug
  discussed in \autoref{sect:eval:mysql}.  The other three were chosen
  at random.
\item For Thunderbird, I tried to use the program in roughly the way
  an ordinary user would for a few minutes.  It is difficult to
  specify workloads precisely in a highly interactive GUI
  program\editorial{Excuses...}.
\item For pbzip2, I compressed three randomly generated 10MiB files in
  series.
\end{itemize}

\begin{figure}
  \subfigure[][rpl\_change\_master]{
    \input{eval/dyn_convergence/rpl_change_master.tex}
  }
  \subfigure[][innodb\_multi\_update]{
    \input{eval/dyn_convergence/innodb_multi_update.tex}
  }
  \subfigure[][binlog\_stm\_drop\_tbl]{
    \input{eval/dyn_convergence/binlog_stm_drop_tbl.tex}
  }
  \subfigure[][timestamp\_basic]{
    \input{eval/dyn_convergence/timestamp_basic.tex}
  }
  \caption{Dynamic aliasing coverage against time for MySQL, using
    some tests out of the test suite.}
  \label{fig:eval:dyn_convergence:mysqld}
\end{figure}

\begin{figure}
  \input{eval/dyn_convergence/thunderbird.tex}
  \caption{Dynamic aliasing coverage against time for Thunderbird.}
  \label{fig:eval:dyn_convergence:thunderbird}
\end{figure}

\begin{figure}
  \input{eval/dyn_convergence/pbzip2.tex}
  \caption{Dynamic aliasing coverage against time for pbzip2.}
  \label{fig:eval:dyn_convergence:pbzip2}
\end{figure}

\subsection{Performance concerns}

I also briefly investigated the performance cost of the analysis.  The
time taken by pbzip2 to compress a 100MiB file without the dynamic
analysis was $7.8 \pm 0.1$ seconds on this machine, mean and standard
deviation of mean for ten runs with independent randomly-generated
files; with the dynamic analysis, this increased to $274 \pm 2$
seconds, a factor of nearly thirty-five.  This is a rather large
overhead, and would be completely infeasible in a production
environment, but is probably tolerable for something which needs to
run for a few tens of minutes in a development one.

For comparison, a null Valgrind skin completed this test in $226.6 \pm
0.5$ seconds, an overhead of a factor of twenty-nine.  That suggests
that most of the overhead of the dynamic analysis tool comes simply
from the fact that it is implemented as a Valgrind skin, and that
re-implementing it in a faster analysis framework such as
PIN~\cite{Luk2005} might provide a useful speed-up.

\todo{Do I need perf numbers for any other test programs?  It'd make
  it much more convincing, but they'd be a nuisance to produce.}
