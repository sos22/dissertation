\chapter{Fixing bugs}
\label{sect:fix_global_lock}

\newcommand{\one}{~\hspace{-1.2pt}1\hspace{3.0pt}~}

In addition to finding bugs, {\technique} can also be used to fix them
by introducing a new global lock.  This ensures that the instructions
involved in the thread \glspl{cfg} cannot execute in parallel,
preventing the bug from happening.  This chapter describes how this is
done in detail.

\section{Identifying the instructions which must be protected}
\label{sect:fix:identify}

\begin{sanefig}
  \hspace{-5mm}\subfigure[][\gls{crashingthread}]{
    \texttt{
      \begin{tabular}{lll}
        1 & \multicolumn{2}{l}{ptr = complex\_local\_condition();}\\
        2 & \multicolumn{2}{l}{dptr = *ptr;}\\
        3 & \multicolumn{2}{l}{if (dptr != NULL) \{}\\
        4 & &dptr = *ptr1;\\
        5 & &*dptr = 5;\\
        6 & \multicolumn{2}{l}{\}}\\
      \end{tabular}
    }
  }
  \hspace{-10mm}\subfigure[][\Gls{interferingthread}]{
    \texttt{
      \begin{tabular}{ll}
        \\
        \\
        7 & ptr = complex\_local\_condition();\\
        8 & *ptr = NULL;\\
        \\
        \\
      \end{tabular}
    }
  }
  \caption{An example bug}
  \label{fig:fix_bug:complex_local}
\end{sanefig}

The first step in producing such a fix is determining which parts of
the dynamic \glspl{cfg} are to be included in the critical sections.
In principle, it is sufficient to use the entire dynamic \gls{cfg} for
both threads: the analysis of \autoref{sect:derive} will only report
atomicity violation bugs, and so making both threads execute
atomically will always eliminate reported bugs.  This is not always a
good choice, though.  Consider, for instance, the threads in
\autoref{fig:fix_bug:complex_local}.  In this example, the
\gls{crashingthread} will crash if statement \texttt{8} intercedes
between statements \texttt{2} and \texttt{4}, and so a reasonable
strategy for fixing this bug would be to make statements \texttt{2},
\texttt{3}, and \texttt{4} execute atomically with respect to
statement \texttt{8}.  The dynamic \glspl{cfg} will include those
instructions, and so making them atomic would correctly fix the bug,
but are also likely to include at least a suffix of
\texttt{complex\_local\_condition()}, if not the entire function,
and will therefore protect more than necessary. This will lead to a
higher than necessary loss of concurrency, and hence potentially
higher than necessary performance overhead.

\todo{Not the best bit of writing I've ever done.} 

It is possible to derive a smaller set of critical sections using the
\gls{verificationcondition}.  As discussed in \autoref{sect:derive},
the \gls{verificationcondition} includes all of the conditions on
instruction interleaving which influence whether the bug will
reproduce, and so any instruction not mentioned in a happens-before
test $\happensBeforeEdge$ cannot be involved in a race which might
cause a crash.  It can be omitted from the atomic regions.  It is
therefore sufficient to restrict the protected regions to contain only
paths which start and end with instructions which are mentioned in
$\happensBeforeEdge$ tests.  This eliminates
\texttt{complex\_local\_condition()} from the critical sections
while leaving statements \texttt{2}, \texttt{3}, \texttt{4}, and
\texttt{8} protected, as desired.

\begin{sanefig}
  \centerline{
    \texttt{
      \begin{tabular}{l}
        ptr = complex\_local\_calculation();\\
        *ptr = NULL;\\
        *ptr = NULL;\\
      \end{tabular}
    }
  }
  \caption{A slightly extended version of
    \autoref{fig:fix_bug:complex_local}'s
    \texttt{interfering\_thread}.}
  \label{fig:fix_bug:complex_local2}
\end{sanefig}

Note that this is not guaranteed to produce an optimally small
selection of critical sections.  Consider, for example, a program with
the same crashing thread as the previous example but an interfering
thread in which the pointer is assigned to twice, as shown in
\autoref{fig:fix_bug:complex_local2}.  There are two obvious ways of
protecting this program:
\begin{itemize}
\item
  Place both loads in the crashing thread in a single critical section and
  both stores in the interfering thread in another one.
\item
  Place both loads in the crashing side in a single critical section,
  but give each store in the interfering thread its own critical
  section.  In other words, drop and re-acquire the lock in between
  the two stores.
\end{itemize}
Both approaches correctly eliminate the bug, but they will have
different performance characteristics.  In particular, dropping and
re-acquiring the lock reduces the size of the critical section, which
might improve concurrency and reduce starvation, but imposes higher
overheads due to the greater number of lock operations.  {\Technique}
uses a simple rule to decide which strategy to use: if the operations
are within \gls{alpha} instructions then it uses a single critical
section and if they are further apart then it uses two.  This is the
simplest approach to implement, as it matches the clustering rules
used in \autoref{sect:derive:write_side}, and generally produces
results which are at least broadly reasonable\editorial{That kind of
  implies I had much more choice in the matter than I actually did.}.

\section{Structure of fixes}

Once {\technique} has identified which fragments of dynamic \gls{cfg}
should be mutually exclusive, it must somehow arrange that they are.
{\Technique}'s fixes modify the program such that it acquires and
releases a new global lock when entering or leaving one of these
critical sections, without otherwise altering the program's behaviour.
This is similar to the problem faced by the crash enforcement
interpreter, which must also insert additional operations into a
program's execution, and one simple way to impose the necessary
synchronisation would be to take the enforcement interpreter and
replace the \textbf{RX} and \textbf{TX} phases with lock acquire and
release operations, respectively.  That would be effective, but would
have two major disadvantages.  Most importantly, it would have
potentially high overhead, as machine code interpreters are inherently
very slow\needCite{}, and this would prevent {\technique} from being
used to fix bugs in frequently-used code paths; what is an acceptable
slow-down in a tool used only in development would be completely
unacceptable in one intended for a production environment.
Furthermore, the machine code interpreter is itself a large and
complicated piece of software, and hence almost certainly contains
bugs of its own, and so any tool based on it carries the risk of
introducing other possible forms of program misbehaviour; an
acceptable risk in development, but not in production.

{\Technique} avoids these issues, by, in effect, partially evaluating
the interpreter, holding the program text and desired synchronisation
fixed and allowing the contents of registers and memory to vary.  The
result is the ``\gls{shadowfragment}'': a new version of the relevant
part of the program which has all necessary synchronisation added.
The program is patched to branch to this \gls{shadowfragment} when
appropriate, much as crash enforcers patch in branches to the
enforcement interpreter.  Crucially, the partial evaluation of the
\textbf{Emul} phase, which emulates the program's original
instructions, is almost always just the original instruction itself,
which makes it easy to achieve good performance and reducing the scope
for errors emulating obscure processor instructions.

\begin{sanefig}
  {\hfill}
  \subfigure[][Static \protect\gls{cfg}]{
    \begin{tikzpicture}
      \node (root) {};
      \node (A) [below = of root, CfgInstr] {A};
      \node (B) [below = of A, CfgInstr] {B};
      \node (C) [below = of B, CfgInstr] {C};
      \node (D) [below = of C, CfgInstr] {D};
      \node [left = .5 of B] {};
      \node [right = .5 of B] {};
      \node [below = .35 of D] {\phantom{$\bot$}};
      \draw[->] (root) -- (A);
      \draw[->] (A) -- (B);
      \draw[->] (B) -- (C);
      \draw[->,ifTrue] (C) -- (D);
      \draw[->,ifFalse] (C.west) ..controls +(-.3,0) and +(0,-0.3) .. ++(-.5,.5)
           -- ++(0,2.228)
           ..controls +(0,0.3) and +(-.3,0) .. ++(.49,.5)
           -- (A.west);
    \end{tikzpicture}
    \label{fig:fix:eg:static}
  }
  {\hfill}
  \subfigure[][Dynamic \protect\gls{cfg}]{
    \begin{tikzpicture}
      \node (root) {};
      \node (B0) [below = of root, CfgInstr] {$B_0$};
      \node (C0) [below = of B0, CfgInstr] {$C_0$};
      \node (A0) [below = of C0, CfgInstr] {$A_0$};
      \node (B1) [below = of A0, CfgInstr] {$B_1$};
      \node (bot) [below = of B1] {$\bot$};
      \node [left = of B0] {};
      \draw[->] (root) -- (B0);
      \draw[->] (B0) -- (C0);
      \draw[->,ifTrue]
          (C0.east) ..controls +(.3,0) and +(0,.3).. ++ (.5,-.5)
          -- ++(0,-3.89)
          ..controls +(0,-.3) and +(.3,0).. ++(-.49,-.5)
          -- (bot.east);
      \draw[->,ifFalse] (C0) -- (A0);
      \draw[->] (A0) -- (B1);
      \draw[->] (B1) -- (bot);
    \end{tikzpicture}
    \label{fig:fix:eg:dynamic}
  }
  {\hfill}
  \subfigure[][Patch strategy]{
    \raisebox{38mm}{
      $\left\{
      \begin{tabular}{rl}
        $\mathit{Patch}$: & $\{\textrm{A}\}$\\
        $\mathit{Cont}$:  & $\{\textrm{A}, \textrm{B}\}$\\
      \end{tabular}
      \right\}$
    }
    \label{fig:fix:eg:strategy}
  }
  {\hfill}
  \subfigure[][\protect\Gls{shadowfragment}]{
    \begin{tikzpicture}[outer sep=-2pt, node distance = 1cm and .5cm]
      \node (A0) [CfgInstr] {A};
      \node (Acq) [below = of A0, CfgInstr] {Acquire lock};
      \node (B0) [below = of Acq, CfgInstr] {B};
      \node (C) [below = of B0, CfgInstr, outer sep = -3pt] {C};
      \node (A1) [below left = of C, CfgInstr] {A};
      \node (B1) [below = of A1, CfgInstr] {B};
      \node (Rel) [below right = 1 and -.5 of C, CfgInstr] {Release lock};
      \node (D) [below = of Rel] {\texttt{jmp} D};
      \draw[->] (A0) -- (Acq);
      \draw[->] (Acq) -- (B0);
      \draw[->] (B0) -- (C);
      \draw[->,ifFalse] (C.south west) -- (A1);
      \draw[->,ifTrue] (C.south east) -- (Rel.north);
      \draw[->] (A1) -- (B1);
      \draw[->] (B1.west) ..controls +(-.3,0) and +(0,-.3) .. ++(-.5,.5)
        -- ++(0,1.88)
        .. controls +(+0,.3) and +(-.3,0) .. ++(.5,.5)
        -- (C.west);
      \draw[->] (Rel) -- (D);
    \end{tikzpicture}
    \label{fig:fix:eg:shadow}
  }
  {\hfill}
  \caption{Example of the \protect\gls{shadowfragment} algorithm; see
    text for full explanation.}
  \label{fig:fix:eg}
\end{sanefig}

\autoref{fig:fix:eg} illustrates this process for a simple example
bug.  The first diagram, \autoref{fig:fix:eg:static}, shows the
program's static \gls{cfg}; the second, \autoref{fig:fix:eg:dynamic},
shows the dynamic one, assuming that the critical section to be
enforced runs from the start of one instance of instruction B to the
end of the instance in the following loop iteration.  Ideally, the fix
would gain control of the program at instruction B, but in this case
the patch strategy algorithm (\autoref{sect:enforce:gain_control})
elected to gain control at instruction A and force the program to
remain in the \gls{shadowfragment} until reaching B
(\autoref{fig:fix:eg:strategy})\footnote{This might, for instance, be
  because B and C are very small, so that patching B directly would
  corrupt D, and there are a large number of branches from elsewhere
  in the program to D, so that recovering from that corruption would
  require a large number of further patches}.  For these inputs,
{\technique} will generate the \gls{shadowfragment} shown in
\autoref{fig:fix:eg:shadow}.  This fragment emulates the behaviour of
the static \gls{cfg} starting from instruction A with the addition of
synchronisation operations which ensure that the lock is held when
necessary, and so replacing the program's original instruction A with
a branch to this fragment will impose the desired additional
synchronisation.

These \glspl{cfg} can usefully be read as a kind of regular automata.
The static and shadow \glspl{cfg} effectively generate streams of
instructions, while the dynamic one recognises fragments of those
streams.  This provides a convenient way of describing the correctness
requirements for the \gls{shadowfragment}:
\begin{enumerate}
\item It must correctly emulate the static \gls{cfg}, in the sense of
  generating the same sequences of instructions (ignoring the lock
  acquire and release operations), starting from any instruction in
  the patch strategy's $\mathit{Patch}$ set.  It must also correctly
  maintain the ``sense'' of conditional branches i.e. which successor
  corresponds to the true branch (shown as a solid line) and which the
  false one (shown as a dashed one)\editorial{ugg}.
\item It must correctly enforce the desired synchronisation, in the
  sense that any subsequences which match the dynamic \gls{cfg} must
  hold the lock for their entire duration.
\item It must follow correct lock discipline, in the sense that it
  must never generate any sequence which double-acquires the lock,
  releases the lock without acquiring it, or terminates without
  releasing the lock.
\end{enumerate}
The example in \autoref{fig:fix:eg} follows all three rules; the next
section explains how to generate \glspl{shadowfragment} which do so in
the general case.

\section{Generating the \protect\gls{shadowfragment}}

\begin{sanefig}
  \newcommand{\lookaside}[1]{\multicolumn{5}{c}{\leavevmode\leaders\hbox{\vrule height .55ex depth -.45ex width 2pt \hspace{2pt}}\hfill\hspace{-3pt}\textit{#1\!}
      {\leavevmode\xleaders\hbox{\vrule height .55ex depth -.45ex width 2pt \hspace{2pt}}\hfill \hspace{-2pt}}
    } \vspace{-4pt}\\}
  \newcommand{\sectsep}[2]{\multicolumn{5}{c}{\leavevmode\cleaders\vrule height .55ex depth -.45ex\hfill ~#1
      {\leavevmode\leaders\vrule height .5ex depth -.45ex\hfill}
    } \vspace{#2}\\}
  \newcommand{\twolineNT}[2]{%
    \raisebox{1.35ex}{%
      \graphNT{%
        \raisebox{-2.7ex}{%
          \shortstack[c]{%
            #1\\
            #2
          }}}}}
  \renewcommand{\arraystretch}{1.2}
  \begin{tabular}{ccc p{3.21cm} c}
    \sectsep{\textbf{CheckForEntry} to \textbf{Lock}}{-4pt}
    \twolineNT{$\mathrm{X}, \alpha, l$}{\textbf{CheckForEntry}} & $\Rightarrow$ & \twolineNT{$\mathrm{X}, \alpha \cup \mathit{startingAt}(x),$}{$l, \textbf{Lock}$} & & \circled{~1~~}\\

    \sectsep{\textbf{Lock} to \textbf{Return}}{-4pt}
    \graphNT{$\mathrm{X}, \alpha, \false, \textbf{Lock}$} & $\Rightarrow$ & \begin{tikzpicture}[baseline = (current bounding box.center)]
      \node (r) {Acquire lock};
      \node (s) [style=graphNT, below = .3 of r] {$\mathrm{X}, \alpha, \true, \textbf{Return}$};
      \draw[->] (r) -- (s);
    \end{tikzpicture} & $\alpha \not= \varnothing$ & \circled{$2_a$}\\
    \graphNT{$\mathrm{X}, \alpha, l, \textbf{Lock}$} & $\Rightarrow$ & \graphNT{$\mathrm{X}, \alpha, l, \textbf{Return}$} & If $\neg$\circled{$2_a$} & \circled{$2_b$}\\

    \sectsep{\textbf{Return} to \textbf{Emul}}{-4pt}
    \graphNT{$\mathrm{X}, \varnothing, \false, \textbf{Return}$} & $\Rightarrow$ & \texttt{jmp} x & $\mathrm{x} \not\in \mathit{Cont}$ & \circled{$3_a$} \\
    \graphNT{$\mathrm{X}, \alpha, l, \textbf{Return}$} & $\Rightarrow$ & \graphNT{$\mathrm{X}, \alpha, l, \textbf{Emul}$} & If $\neg$\circled{$3_a$} & \circled{$3_b$} \\

    \sectsep{\textbf{Emul} to \textbf{Unlock}}{-8pt}
    \lookaside{Indirect control flow instructions described in text}
    \raisebox{2.5mm}{\graphNT{\raisebox{-2.5mm}{$\begin{tikzpicture}[baseline = (current bounding box.center)]
        \node (r) {\texttt{jcc}};
        \node (s) at (-.3,-.7) {X};
        \node (t) at ( .3,-.7) {Y};
        \draw[->,ifTrue] (r) -- (s);
        \draw[->,ifFalse] (r) -- (t);
      \end{tikzpicture}, \alpha, l, \textbf{Emul}$}}} & $\Rightarrow$ & 
      \begin{tikzpicture}[baseline = (current bounding box.center)]
        \node (r) at (.8,0) {\texttt{jcc}};
        \node (s) at (-.7,-1.1) {\twolineNT{$\mathrm{X}, \mathit{adv}(\mathrm{x}, \alpha),$}{$l, \textbf{Unlock}$}};
        \node (t) at ( 2.3,-1.1) {\twolineNT{$\mathrm{Y}, \mathit{adv}(\mathrm{y}, \alpha),$}{$l, \textbf{Unlock}$}};
        \draw[->,ifTrue] (r) -- (s);
        \draw[->,ifFalse] (r) -- (t);
      \end{tikzpicture}
      \renewcommand{\arraystretch}{1.5}
    & & \circled{$4_a$}\\
    \graphNT{\texttt{jmp} X, $\alpha$, $l$, \textbf{Emul}} & $\Rightarrow$ & \twolineNT{$\mathrm{X}, \mathit{adv}(\mathrm{x}, \alpha),$}{$l, \textbf{Unlock}$} & & \circled{$4_b$} \\
    \raisebox{2.5mm}{\graphNT{\raisebox{-2.5mm}{$\begin{tikzpicture}[baseline = (current bounding box.center)]
          \node (r) {\texttt{call} X};
          \node (s) [below = .3 of r] {Y};
          \draw[->] (r) -- (s);
      \end{tikzpicture}, \alpha, l, \textbf{Emul}$}}} & $\Rightarrow$ &
    \begin{tikzpicture}[baseline = (current bounding box.center)]
      \node (r) {\texttt{push} y};
      \node (s) [below = .3 of r] {\twolineNT{$\mathrm{X}, \mathit{adv}(\mathrm{x},\alpha),$}{$l, \textbf{Unlock}$}};
      \draw[->] (r) -- (s);
    \end{tikzpicture} & X not a library function & \circled{$4_c$} \\
    \raisebox{2.5mm}{\graphNT{\raisebox{-2.5mm}{$\begin{tikzpicture}[baseline = (current bounding box.center)]
          \node [minimum height=3ex, outer sep = -2pt] (r) {x};
          \node (s) [below = .3 of r] {Y};
          \draw[->] (r) -- (s);
      \end{tikzpicture}, \alpha, l, \textbf{Emul}$}}} & $\Rightarrow$ &
    \begin{tikzpicture}[baseline = (current bounding box.center)]
      \node (r) {x};
      \node (s) [below = .3 of r] {\twolineNT{$\mathrm{Y}, \mathit{adv}(\mathrm{y},\alpha),$}{$l, \textbf{Unlock}$}};
      \draw[->] (r) -- (s);
    \end{tikzpicture} & x some other instruction type & \circled{$4_d$} \\

    \sectsep{\textbf{Unlock} to \textbf{CheckForEntry}}{-4pt}
    \graphNT{$\mathrm{X}, \varnothing, \true, \textbf{Unlock}$} & $\Rightarrow$ & \begin{tikzpicture}[baseline = (current bounding box.center)]
      \node (r) {Release lock};
      \node (s) [below = .3 of r] {\twolineNT{$\mathrm{X}, \varnothing, \false,$}{\textbf{CheckForEntry}}};
      \draw[->] (r) -- (s);
    \end{tikzpicture} & & \circled{$5_a$}\\
    \graphNT{$\mathrm{X}, \alpha, l, \textbf{Unlock}$} & $\Rightarrow$ & \twolineNT{$\mathrm{X}, \alpha, l,$}{\textbf{CheckForEntry}} & If $\neg$\circled{$5_a$} & \circled{$5_b$}\\
  \end{tabular}

  \caption{The patch generating algorithm as a graph grammar.  The
    terminals of this grammar are instructions in a newly-generated
    static \protect\gls{cfg} representing the shadow program.  The
    non-terminal type, \textsc{Configuration}, is described in
    \autoref{fig:fixing:nt_type}.  The ``If
    $\neg$\protect\circled{$2_a$}'' annotation indicates that the
    production is used only if \protect\circled{$2_a$} does not match.
    \textrm{X} and \textrm{Y} match fragments of the static
    \protect\gls{cfg}; \textrm{x} and \textrm{y} match individual
    static instructions.  Where \textrm{X} matches a fragment of
    static \protect\gls{cfg}, \textrm{x} is the first instruction in
    that fragment, and similarly for \textrm{Y} and \textrm{y}.
    $\alpha$ matches a (possibly empty) set of dynamic
    \protect\gls{cfg} nodes.  $\mathit{startingAt}$ and $\mathit{adv}$
    are defined in the text.  Instruction-pointer relative addressing
    updated as appropriate when copying instructions.}
  \label{fig:fix:graph_grammar}
\end{sanefig}

\begin{sanefig}
  \hfill
  \begin{tikzpicture}
    \node (typename) {\textsc{Configuration}\!};
    \node (eqsign) [right = 0 of typename] {\raisebox{-2.5mm}{=}\hspace{.8mm}~};
    \node [graphNT, right = 0 of eqsign] {
      \begin{tabular}{rl}
        $\mathit{static}$: & \textsc{StaticCfgNode},\\
        $\mathit{dynamic}$: & \textsc{Set}(\textsc{DynCfgNode}),\\
        $\mathit{holdLock}$: & \textsc{Bool},\\
        $\mathit{phase}$: & $\left\{\raisebox{-1.25ex}{\shortstack[l]{
            $\mathbf{CheckForEntry}, \mathbf{Lock},$\\
            $\mathbf{Emul}, \mathbf{Unlock}$}}\right\}$\\
      \end{tabular}
    };
  \end{tikzpicture}
  \hfill{}
  \caption{The \textsc{Configuration} type, used as the non-terminal
    type in the shadow generating algorithm in
    \autoref{fig:fix:graph_grammar}.}
  \label{fig:fix:configuration}
\end{sanefig}

\begin{sanefig}
  {\hfill}
  \begin{tikzpicture}
    \node (a) {\raisebox{0.5mm}{$\mathit{initial}$}\!};
    \node (eqsign) [right = 0 of a] {\raisebox{-2.5mm}{=}\!};
    \node [right = 0 of eqsign] {
      $\left\{
      \graphNT{
        \begin{tabular}{r@{=}l}
          $\mathit{static}$ & $p$ \\
          $\mathit{dynamic}$ & $\varnothing$ \\
          $\mathit{holdLock}$ & \false \\
          $\mathit{phase}$ & $\mathbf{CheckForEntry}$ \\
        \end{tabular}
      }\,
      \middle|\,p \in Patch\right\}$};
  \end{tikzpicture}
  {\hfill}
  \caption{The initial set of non-terminals for the graph grammar in
    \autoref{fig:fix:graph_grammar}.  There is one initial
    non-terminal for every entry in the $\mathit{Patch}$ set.}
  \label{fig:fix:initial_configuration}
\end{sanefig}

\begin{sanefig}
  \begin{tabbular}{| p{7.33cm} | p{7.33cm} |}
    \hline
    \Gls{shadowfragment} phase & LLI stage \\
    \hline
    \textbf{CheckForEntry} & \textit{None; the enforcer equivalent is only present in the HLI} \\
    \textbf{Lock} & \textbf{RX} \\
    \textbf{Emul} & \textbf{Emul} \\
    \textbf{Unlock} & \textbf{TX} \\
    \textit{None; equivalent computation performed in the $\mathit{adv}$ function} & \textbf{Succ} \\
    \hline
  \end{tabbular}
  \caption{Correspondence between phases in the
    \protect\gls{shadowfragment} algorithm and stages in the crash
    enforcer's low-level interpreter (see
    \autoref{sect:enforce:llis}).}
  \label{fig:fix:state_correspondence}
\end{sanefig}

\noindent
The algorithm used to generate the \gls{shadowfragment} is shown in
\autoref{fig:fix:graph_grammar} as a graph generating grammar (see
\autoref{sect:intro:graph_grammar}), using the non-terminal type
\textsc{Configuration} defined in \autoref{fig:fix:configuration}.
The set of initial \textsc{Configuration}s is given in
\autoref{fig:fix:initial_configuration}.  A \textsc{Configuration}
contains the parts of the interpreter state which are to be partially
evaluated away:
\begin{itemize}
\item $\mathit{static}$ gives the current position in the program's
  static \gls{cfg}, interpreted as a regular automaton which generates
  the stream of instructions executed by the current thread.
\item $\mathit{dynamic}$ gives the set of possible states of the dynamic
  \gls{cfg}, interpreted as a regular automaton which matches the
  stream of instructions executed by the current thread.  This is
  analogous to the set of low-level interpreters maintained by the
  crash enforcement interpreter.
\item $\mathit{holdLock}$ is a simple flag indicating whether the
  current thread currently holds the global fix lock.
\item $\mathit{phase}$ is used to track how much of the interpreter's
  processing has been completed.
\end{itemize}
The grammar makes use of two utility functions:
\begin{itemize}
\item $\mathit{startingAt}(r)$, of type $\textsc{StaticCfgNode}
  \rightarrow \textsc{Set}(\textsc{DynCfgNode})$, maps from a static
  \gls{cfg} node $r$ to the set of dynamic \gls{cfg} nodes which match
  $r$ and which start a new dynamic \gls{cfg} fragment.  In the
  example of \autoref{fig:fix:eq}, $\mathit{startingAt}$ maps B to
  $\{B_0\}$ and every other instruction to $\varnothing$.
\item $\mathit{adv}(\mathrm{x}, \alpha)$, of type
  $(\textsc{StaticCfgNode}, \textsc{Set}(\textsc{DynCfgNode}))
  \rightarrow \textsc{Set}(\textsc{DynCfgNode})$, is defined by
  \begin{displaymath}
    \mathit{adv}(\mathrm{x}, \alpha) = \left\{r \middle| s \in \alpha \wedge r \in s.\mathit{next} \wedge r ~ \mathrm{x}\right\}
  \end{displaymath}
  $n.\mathit{next}$ is the set of \gls{cfg} nodes which can follow the
  \gls{cfg} node $n$ and $n ~ \mathrm{x}$ is true precisely when $n$
  matches $\mathrm{x}$.
\end{itemize}
\autoref{fig:fix:eg:application} shows how to apply this grammar to
the example.  Note that this figure shows both the output
\gls{shadowfragment} (using unboxed nodes and single arrows) and the
way in which it is generated (using boxed and double arrows).  For
instance, the \graphNT{A, $\varnothing$, \false, \textbf{Emul}}
non-terminal at the top-right of the figure is transformed by
production \circled{$4_d$} into the terminal A followed by the
non-terminal \graphNT{B, $\varnothing$, \false, \textbf{Unlock}}, and
the new non-terminal is itself transformed by production
\circled{$5_b$} into \graphNT{B, $\varnothing$, \false,
  \textbf{CheckForEntry}}.  This process of transforming non-terminals
continues until only terminals remain, at which point the resulting
graph forms the \gls{shadowfragment} shown in
\autoref{fig:fix:eg:shadow}.

Note that the entire loop, except for the first A instruction, will be
run under the lock, and so the lock could be held for an arbitrarily
large number of instructions even though the original dynamic
\gls{cfg} only required it to be held for four instructions.  To
understand why this is necessary, consider the infinite execution
$A_0B_0C_0A_1B_1C_1{\ldots}A_nB_nC_n{\ldots}$.  This produces an
infinite set of dynamic \gls{cfg} matches $\{B_iC_iA_{i+1}B_{i+1} | i
\in \mathbb{Z}\}$, and, by the second correctness criterion given
above, the \gls{shadowfragment} cannot release the lock whilst in any
of these fragments.  The fragments overlap, as $B_j$ will be present
in both the fragment for $i=j$ and that for $i=j-1$, and so, by
induction, the \gls{shadowfragment} cannot release the lock once
it has been acquired.

This has unfortunate implications for the program's liveness, as it
means that {\technique} fixes can introduce unbounded additional
delays into the program's execution; in the worst case, they could
even introduce a complete deadlock.  {\Technique} mitigates this
problem using a timeout in the lock acquire operation.  If this
timeout expires then the \gls{shadowfragment} returns the program to
normal operation\footnote{Assuming, of course, that the instruction to
  which the \protect\gls{shadowfragment} is to return is not a member
  of $\mathit{Cont}$.  Otherwise, the \protect\gls{shadowfragment}
  continues until the program exits $\mathit{Cont}$.}, abandoning its
attempt to impose the desired synchronisation.  This risks
re-introducing the original bug, but avoids introducing any additional
deadlocks.

\begin{sanefig}
  \newcommand{\graphLine}[3]{\draw [->,double] (#1) -- node [above = -.05] {\circled{#2}} (#3);}
  \newcommand{\nextline}[3]{\draw [->,double] (#1.east) -- ++(.2,0) -- ++(0,-.5) -- node [above = -.05] {\circled{#2}} ++(-14.5,0) --  ++(0,-.8) -- (#3.west);}
  \begin{tikzpicture}
    \tikzstyle{graphNT}+=[minimum width=2.2cm];
    \node at (0,0) (CFE) {\textbf{CheckForEntry}};
    \node at (3,0) (Lock) {\textbf{Lock}};
    \node at (6,0) (Return) {\textbf{Return}};
    \node at (9,0) (Emul) {\textbf{Emul}};
    \node at (11.8,0) (Unlock) {\textbf{Unlock}};

    \node at (0,-1) (r0) {};
    \node at (0,-2.3) (r1) {};

    \node at (0,-3.6) (r2) {};
    \node at (0,-4.95) (r3) {};
    \node at (0,-6.25) (r4) {};

    \node at (0,-7.55) (r5) {};

    \path (r5) ++(0,-2.86) node (r9) {}
               ++(0,-1) node (r10) {}
               ++(0,-1.9) node (r11) {}
               ++(0,-1) node (r12) {}
               ++(0,-1.3) node (r13) {};

    \node at (CFE |- r0) (AnfCFE) [graphNT] {A,$\varnothing$,\false};
    \node at (Lock |- r0) (AnfL) [graphNT] {A,$\varnothing$,\false};
    \node at (Return |- r0) (AnfR) [graphNT] {A,$\varnothing$,\false};
    \node at (Emul |- r0) (AnfE) [graphNT] {A,$\varnothing$,\false};
    \node at (Unlock |- r0) (A0) [CfgInstr] {A};
    \node at (Unlock |- r1) (BnfU) [graphNT] {B,$\varnothing$,\false};
    \graphLine{AnfCFE}{\one}{AnfL}
    \graphLine{AnfL}{$2_b$}{AnfR}
    \graphLine{AnfR}{$3_b$}{AnfE}
    \graphLine{AnfE}{$4_d$}{A0}
    \draw[->] (A0) -- (BnfU);

    \node at (CFE |- r2) (BnfCFE) [graphNT] {B,$\varnothing$,\false};
    \node at (Lock |- r2) (B0fL) [graphNT] {B,$B_0$,\false};
    \node at (Return |- r2) (Acq) [CfgInstr, inner sep = 0, outer sep=-1pt] {Acquire lock};
    \node at (Return |- r3) (B0tR) [graphNT] {B,$B_0$,\true};
    \node at (Emul |- r3) (B0tE) [graphNT] {B,$B_0$,\true};
    \node at (Unlock |- r3) (B0) [CfgInstr] {B};
    \node at (Unlock |- r4) (C0tU) [graphNT] {C,$C_0$,\true};
    \graphLine{BnfCFE}{\one}{B0fL}
    \graphLine{B0fL}{$2_a$}{Acq}
    \draw[->] (Acq) -- (B0tR);
    \graphLine{B0tR}{$3_b$}{B0tE}
    \graphLine{B0tE}{$4_d$}{B0}
    \draw[->] (B0) -- (C0tU);

    \nextline{BnfU}{$5_b$}{BnfCFE};

    \node at (CFE |- r5) (C0tCFE) [graphNT] {C,$C_0$,\true};
    \node at (Lock |- r5) (C0tL) [graphNT] {C,$C_0$,\true};
    \node at (Return |- r5) (C0tR) [graphNT] {C,$C_0$,\true};
    \node at (Emul |- r5) (C0tE) [graphNT] {C,$C_0$,\true};
    \node at (Unlock |- r5) (C0) [CfgInstr] {C};
    \path (C0.south) ++ (-.6,-1.4) node (DntU) [graphNT, minimum width = 0] {\shortstack[c]{D,$\varnothing$,\\\true}};
    \path (C0.south) ++ (.6,-.8) node (A0tU) [graphNT, minimum width = 0] {\shortstack[c]{A,$A_0$,\\\true}};
    \graphLine{C0tCFE}{\one}{C0tL}
    \graphLine{C0tL}{$2_b$}{C0tR}
    \graphLine{C0tR}{$3_b$}{C0tE}
    \graphLine{C0tE}{$4_a$}{C0}
    \draw[->,ifTrue] (C0.south) -- (DntU);
    \draw[->,ifFalse] (C0.south) -- (A0tU);
    \nextline{C0tU}{$5_b$}{C0tCFE};

    \node at (CFE |- r9) (Rel) [CfgInstr, outer sep = -1pt] {\!Release lock\!};
    \node at (CFE |- r10) (D0fCFE) [graphNT] {D,$\varnothing$,\false};
    \node at (Lock |- r10) (D0fL) [graphNT] {D,$\varnothing$,\false};
    \node at (Return |- r10) (D0fR) [graphNT] {D,$\varnothing$,\false};
    \node at (Emul |- r10) (D) [CfgInstr] {\texttt{jmp} D};
    \draw[->] (Rel) -- (D0fCFE);
    \graphLine{D0fCFE}{\one}{D0fL}
    \graphLine{D0fL}{$2_b$}{D0fR}
    \graphLine{D0fR}{$3_a$}{D}
    \draw [->,double] (DntU.east) -- ++(.2,0) -- ++(0,-.65) -- node [pos=.455, above = -.05] {\circled{$5_a$}} ++(-13.31,0) --  ++(0,-.5) -- (Rel.west);

    \node at (CFE |- r11) (A0tCFE) [graphNT] {A,$A_0$,\true};
    \node at (Lock |- r11) (A0tL) [graphNT] {A,$A_0$,\true};
    \node at (Return |- r11) (A0tR) [graphNT] {A,$A_0$,\true};
    \node at (Emul |- r11) (A0tE) [graphNT] {A,$A_0$,\true};
    \node at (Unlock |- r11) (A1) [CfgInstr] {A};
    \node at (Unlock |- r12) (B1tU) [graphNT] {B,$B_1$,\true};
    \graphLine{A0tCFE}{\one}{A0tL}
    \graphLine{A0tL}{$2_b$}{A0tR}
    \graphLine{A0tR}{$3_b$}{A0tE}
    \graphLine{A0tE}{$4_d$}{A1}
    \draw[->] (A1) -- (B1tU);
    \draw [->,double] (A0tU.east) -- ++(.2,0) -- ++(0,-3.9) -- node [pos=.505, above = -.05] {\circled{$5_b$}} ++(-14.58,0) --  ++(0,-.75) -- (A0tCFE.west);

    \node at (CFE |- r13) (B1tCFE) [graphNT] {\shortstack[c]{B,$B_1$,\\\true}};
    \node at (Lock |- r13) (B01tL) [graphNT] {\shortstack[c]{B,$\{B_0;B_1\}$,\\\true}};
    \node at (Return |- r13) (B01tR) [graphNT] {\shortstack[c]{B,$\{B_0;B_1\}$,\\\true}};
    \node at (Emul |- r13) (B01tE) [graphNT] {\shortstack[c]{B,$\{B_0;B_1\}$,\\\true}};
    \node at (Unlock |- r13) (B1) [CfgInstr] {B};
    \graphLine{B1tCFE}{\one}{B01tL}
    \graphLine{B01tL}{$2_a$}{B01tR}
    \graphLine{B01tR}{$3_b$}{B01tE}
    \graphLine{B01tE}{$4_d$}{B1}
    \draw[->] (B1.east) -- ++(1.4,0) -- ++(0,10.0) -| (C0tU.north);
    \nextline{B1tU}{$5_b$}{B1tCFE}

  \end{tikzpicture}
  \caption{Application of the algorithm in
    \autoref{fig:fix:graph_grammar} to the example in
    \autoref{fig:fix:eg}.  In this figure, boxed nodes show
    \textsc{Configuration}s and double arrows show the productions
    used to transform between them, while unboxed nodes and single
    arrows show the terminals and output edges.  The final
    \protect\gls{shadowfragment}, \autoref{fig:fix:eg:shadow}, can be
    recovered from this figure by removing all of the boxed nodes and
    double arrows.  The $\mathit{phase}$ field of non-terminals is
    shown in the headings above the graph, rather than in the
    non-terminals themselves, for reasons of clarity and space.}
  \label{fig:fix:eg:application}
\end{sanefig}

\section{Control flow}
Control flow instructions require special care in the \textbf{Emul}
phase, as they are the only ones which cannot be copied verbatim from
the input static \gls{cfg} to the shadow fragment.

Direct branches are simple to handle, as the target of the branch is
available statically and can be easily represented in the graph
structure of the shadow fragment.  This means that direct branch
instructions become no-op instructions (production \circled{$4_b$}),
while conditional branches can be handled by simply copying the
conditional branch instruction into the output and updating its
targets (production \circled{$4_a$}).

\todo{Assuming that the lock is held at the start of the call...}
Direct function calls are only moderately more difficult.  In the
common case, the called function will be in the main program binary,
and these can be treated as pushing the return address on to the stack
and then branching to the start of the function, reducing the problem
to the previous case (production \circled{$4_c$}).  Calls to library
functions, such as \texttt{malloc} or \texttt{printf}, are more
problematic.  The fix generation process has two options for handling
these, both bad:
\begin{itemize}
\item It could call into the library function whilst holding the lock.
  This ensures that the desired atomic blocks behave atomically, but
  at the cost of a significant risk of deadlocks if the called
  function is something like \texttt{pthread\_mutex\_lock}.
\item It could release the lock while calling into the library
  function.  This is much less likely to deadlock, but also much
  less likely to correctly eliminate the bug.
\end{itemize}
{\Implementation} always chooses to hold the lock when calling library
functions from a patch, by simply copying the \texttt{call}
instruction into the shadow fragment (production \circled{$4_d$}), and
relies on the \textbf{Lock} timeout to mitigate the damage from
deadlocks.  These timeouts might themselves lead to incomplete
protection, but generally happen far less frequently than library
calls, and so present a far less severe problem.

Indirect control flow transfers, which set the instruction pointer to
a computed value, form the final class of instructions requiring
special handling; these include computed branches, computed calls, and
the \texttt{ret} instruction.  They are difficult to handle as the
target of the branch is not available at compile time, and so cannot
be represented in the shadow fragment's graph structure.  The grammar
must instead emit terminals which perform appropriate actions at
run-time; in particular, it must either select an appropriate
non-terminal to branch to or return the program to normal operation,
releasing the lock first if necessary.  There are three cases to
consider:
\begin{itemize}
\item The indirect branch might select a successor instruction X such
  that $\mathit{adv}(\textrm{X}, \alpha)$ is non-empty.  In that case,
  the grammar must select the successor non-terminal \graphNT{X,
    $\mathit{adv}(\textrm{X}, \alpha)$, $l$, \textbf{Unlock}}.  There
  set of such possible successors is finite, known, and, usually,
  small, and so the grammar can simply emit run-time tests to
  recognise each and branch to an appropriate non-terminal.
\item It might select an X such that $\mathit{adv}(\textrm{X},
  \alpha)$ is empty but $\mathrm{X} \in \mathit{Cont}$.  In that case,
  the grammar must drop the lock, but must continue expanding the
  shadow fragment, as it is not safe to return control to the program
  at any instruction in the $\mathit{Cont}$ set.  It can achieve this
  by selecting the non-terminal \graphNT{X, $\varnothing$, $l$,
    \textbf{Unlock}}.  Again, the set of possible X matching this case
  is known and finite, and so the grammar can again emit tests for
  each\editorial{These two first cases are kind of redundant, but I
    think it's easier to understand if they're split out.}.
\item The selected X might not fit into either of the other two cases,
  and in that case the grammar must return control to the original
  program, releasing the lock first if necessary.
\end{itemize}
\autoref{fig:fix:eg_indirect} shows how this works for indirect branch
instructions using an example \textsc{Configuration} and
$\mathit{Cont}$ set.  Indirect calls and \texttt{ret} instructions
behave in a broadly similar way, except that they also need to include
some updates to the stack.

\begin{sanefig}
  \shortstack[c]{
    \subfigure[][Inputs]{
      \graphNT{
        a: \texttt{jmp} $\mathit{exp}$,
        \begin{math}
          \left\{
            \begin{tikzpicture}[node distance = .2 and .3,outer sep = -2pt, inner sep = 0pt, baseline = (C.north)]
              \node (A) [CfgInstr] {$a_0$};
              \node (B) [CfgInstr, below left = of A] {$B_0$};
              \node (C) [CfgInstr, below = of A] {$C_0$};
              \node (D) [CfgInstr, below right = of A] {$D_0$};
              \draw[->] (A) -- (B);
              \draw[->] (A) -- (C);
              \draw[->] (A) -- (D);
            \end{tikzpicture},
            \begin{tikzpicture}[node distance = .2 and .3,outer sep = -2pt, inner sep = 0pt, baseline = (C.north)]
              \node (A) [CfgInstr] {$a_1$};
              \node (C) [CfgInstr, below = of A] {$C_1$};
              \draw[->] (A) -- (C);
            \end{tikzpicture},
            \begin{tikzpicture}[node distance = .2 and -.1,outer sep = -2pt, inner sep = 0pt, baseline = (C.north)]
              \node (A) [CfgInstr] {$a_2$};
              \node (C) [CfgInstr, below left = of A] {$C_1$};
              \node (D) [CfgInstr, below right = of A] {$D_1$};
              \draw[->] (A) -- (C);
              \draw[->] (A) -- (D);
            \end{tikzpicture}
          \right\}
        \end{math},
        $l$,
        \textbf{Emul}
      }
      $\mathit{Cont} = \{\mathrm{E}\}$
    }\\
    \hspace{\textwidth}
  }\\
  \shortstack[c]{
    \subfigure[][Output]{
      \begin{tikzpicture}[node distance = 0.3 and 1]
        \node (r) [CfgInstr] {\texttt{R0} $\leftarrow \mathit{exp}$};
        \node (b) [below = of r, CfgInstr] {\texttt{if R0 = b}};
        \node (b0) [right = of b, graphNT] {B, $B_0$, $l$, \textbf{Unlock}};
        \node (c) [below = of b, CfgInstr] {\texttt{if R0 = c}};
        \node (c0) [right = of c, graphNT] {B, $\{C_0; C_1\}$, $l$, \textbf{Unlock}};
        \node (d) [below = of c, CfgInstr] {\texttt{if R0 = d}};
        \node (d0) [right = of d, graphNT] {D, $\{D_0; D_1\}$, $l$, \textbf{Unlock}};
        \node (e) [below = of d, CfgInstr] {\texttt{if R0 = e}};
        \node (e0) [right = of e, graphNT] {E, $\varnothing$, $l$, \textbf{Unlock}};
        \node (rel) [rectangle, dashed, draw, below = of e, CfgInstr] {Release lock};
        \node (end) [below = of rel, CfgInstr] {\texttt{jmp R0}};
        \draw[->] (r) -- (b);
        \draw[->,ifTrue] (b) -- (b0);
        \draw[->,ifTrue] (c) -- (c0);
        \draw[->,ifTrue] (d) -- (d0);
        \draw[->,ifTrue] (e) -- (e0);
        \draw[->,ifFalse] (b) -- (c);
        \draw[->,ifFalse] (c) -- (d);
        \draw[->,ifFalse] (d) -- (e);
        \draw[->,ifFalse] (e) -- (rel);
        \draw[->] (rel) -- (end);
      \end{tikzpicture}
    }\\
    \hspace{\textwidth}
  }
  \caption{An example indirect control-flow operation.  The release
    lock operation is only present if $l=\true$.  Not shown: restoring
    the previous value of register {\tt R0}.}
  \label{fig:fix:eg_indirect}
\end{sanefig}

\section{Discussion}
\todo{Some kind of discussion goes here}
