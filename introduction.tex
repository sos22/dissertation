\chapter{Introduction}

\section{Motivation and overview}
\label{sect:intro:overview}

Commodity hardware is becoming increasingly concurrent, whether due to
more packages per machine, more cores per package, or more threads per
core, and software is adapting to make use of this greater
concurrency.  This promises potentially greatly increased performance.
Unfortunately, it also promises greatly reduced reliability.
Highly-concurrent software is particularly prone to complex,
unpredictable, and hard-to-reproduce bugs, and as concurrent software
development techniques become more widespread, especially amongst less
able developers, we should expect to see the frequency of serious bugs
in important
\begin{wrapfigure}{r}{5.4cm}
  \vspace{-14pt}
  \begin{figgure}
    \centerline{
      \begin{tikzpicture}
        [block/.style = {rectangle,draw,fill=white},
          node distance = 1.2]
        \node [block] (dynamic) {Running program};
        \node [block,below = of dynamic] (model) {\Gls{programmodel}};
        \node [block,below = of model] (statemachines) {\STateMachines};
        \node [block,below = of statemachines] (candidates) {\Glspl{verificationcondition}};
        \node [block,below = of candidates] (repro) {Reproduction};
        \node [block, below = of repro] (fix) {Fixes};
        \draw [->] (dynamic) to node [right] {\shortstack[l]{Dynamic\\analysis}} (model);
        \draw [->] (model) to node [right] {\shortstack[l]{Program\\abstraction}} (statemachines);
        \draw [->] (statemachines) to node [right] {\shortstack[l]{Symbolic\\execution}} (candidates);
        \draw [->] (candidates) to node [right] {\shortstack[l]{Crash\\enforcement}} (repro);
        \draw [->] (dynamic.west)
          -- ++(-.1,0)
          ..controls +(-0.3,0) and +(0, 0.3) .. ++(-0.5,-0.5)
          -- ++(0,-6.29)
          ..controls +(0,-0.3) and +(-0.3,0) .. ++( 0.5,-0.5)
          -- (repro.west);
        \draw [->] (repro) to node [right] {\shortstack[l]{Fix\\generation}} (fix);
      \end{tikzpicture}
    }
    \vspace{-4pt}
    \caption{System overview}
    \label{fig:basic_pipeline}
  \end{figgure}
  \vspace{-24pt}
\end{wrapfigure}
software increase.  This dissertation presents automated
techniques to help developers discover, characterise, reproduce, and
fix a certain class of concurrency bug.

The basic approach is shown in \autoref{fig:basic_pipeline}.  The
process starts by building an \gls{programmodel}, showing how the
program accesses memory when it is operating normally, using primarily
dynamic analysis (\autoref{sect:program_model}).  This
\gls{programmodel} is used to locate potentially relevant fragments of
the program and to build {\StateMachines} which model their behaviour
(\autoref{sect:derive}).  These {\StateMachines} can then be
symbolically executed to determine whether they might exhibit the bug
under investigation, and, if so, under what circumstances; the results
are summarised as a set of \glspl{verificationcondition}.  This set
usually contains a large number of false positives, and so the next
step is to prune it back down using \glspl{bugenforcer}: special
schedulers which, when applied to the running program, make it far
more likely that the bug will reproduce quickly
(\autoref{sect:reproducing_bugs}).  Any bugs which \emph{do} reproduce
can then be passed to the final fix generation phase which generates
binary patches which eliminate the bug
(\autoref{sect:fix_global_lock}).

An important decision which must be made when designing this kind of
tool is how much higher-level semantic information to use, and in
particular whether to operate at the level of machine code or source
code.  Working with machine code gives a tool the most precise
description of the program's behaviour, as concurrency bugs often
depend on the precise details of compiler optimisations; on the other
hand, source code provides far more useful information, and so is
usually far easier to analyse.  {\Technique} takes the extreme
position of operating on machine code as far as possible, only relying
on access to source code when absolutely necessary.  For some
non-trivial programs, including Thunderbird and pbzip2, the technique
can produce useful results without any access to source code at all,
and for most others very little information is needed (MySQL, for
instance, required just two functions to be manually annotated).

\section{Contributions}

This dissertation makes several contributions:
\begin{itemize}
\item
  Suggest a novel method of finding concurrency-related bugs given
  only a binary program and some way of running it.
\item
  Describe how the bug description produced when trying to find a bug
  can be used to automatically fix the bug or to make it more easily
  reproducible.
\item
  Evaluate these techniques, showing that they can find and fix bugs
  in simple programs quickly, and that the analysis techniques scale
  up to find real bugs in realistic one.
\end{itemize}
I give a detailed description of {\technique} and some results
obtained using \implementation, my prototype implementation.  These
include details of the fixes generated for a selection of bugs, both
artificial ones and some from real programs (including two which were
unknown to me before writing the tool), along with a demonstration
that the analysis scales to realistically large programs with
acceptable computational cost.  I also show that the fixes generated
typically have sufficiently low overhead to be useful in practice.

\section{Type of bug considered}
\label{sect:types_of_bugs}

{\Technique} considers only a subset of concurrency bugs: those where
one thread, referred to as the ``\gls{crashingthread}'', is reading
from a shared data structure while another thread, the
``\gls{interferingthread}'', simultaneously modifies it, and these
concurrent updates cause the crashing thread to crash quickly.  In a
little more detail:
\begin{itemize}
\item The threads must be operating on a data structure located
  somewhere in shared memory.  This structure does not need to be in
  contiguous memory, and does not need to correspond to any
  higher-level concept of a data structure such as a C++
  \texttt{class} or \texttt{struct}, but it does need to be in
  process-accessible memory.  Structures on the filesystem, for
  instance, are not considered.
\item The crashing thread must crash in a detectable way.  The
  simplest case is a hardware-detected fault such as referencing bad
  memory or dividing by zero, but more complex types of fault could
  also be supported, if a suitable detector can be implemented.
  {\Implementation} includes detectors for hardware-detected faults,
  assertion-failure type errors, and some types of double-free error.
\item The crash must be caused by the concurrent updates.  There must
  be some regions of the crashing and interfering threads such that
  running those regions in parallel can crash but running them
  atomically, in either order, will not.
\item The crashing thread must crash ``quickly''.  {\Technique} uses a
  finite \gls{analysiswindow} \gls{alpha} and will only consider
  reordering concurrent operations which occur at most \gls{alpha}
  instructions before the crash, and bugs which require knowledge of
  the program behaviour beyond that window cannot be analysed;
  equivalently, {\technique} only considers bugs which can be fixed by
  small critical sections containing fewer than $\alpha$ dynamic
  instructions.  \gls{alpha} can, in principle, be arbitrarily large,
  but computational constraints mean that in practice it will be
  limited to a few dozen to a few hundred instructions, depending on
  the program to be analysed and how much information about the bug is
  available before analysis starts.
\end{itemize}
This clearly does not include every possible type of concurrent bug
(it does not, for instance, include any but the most trivial deadlock
bugs, and complicated memory corruption bugs are difficult to handle),
but does include some interesting ones.

\subsection{Order-violation bugs}

\begin{sanefig}
{\hfill}
\begin{tabular}{p{8cm}l}
Crashing thread:\hfill         & Interfering thread: \\
\\
1: Load $t_0$ from loc1        & 6: Load $t_3$ from loc1 \\
2: Store $t_0$ to loc2         & 7: Store $t_3$ to loc2 \\
\textit{Complicated local computation} & 8: Store $t_3 + 1$ to loc2 \\
3: Load $t_1$ from loc1        & \\
4: Load $t_2$ from loc2        & \\
5: Crash if $t_1 = t_2$ & \\
\end{tabular}
{\hfill}
\caption{An order violation bug. The complicated local computation
  does not modify loc1 or loc2.}
\label{fig:mandatory_concurrency1}
\end{sanefig}

\begin{sanefig}
\begin{centering}
\hfill
\begin{tabular}{p{8cm}l}
Crashing thread:          & Interfering thread: \\
\\
1: Load $t_0$ from loc1        & 6: Load $t_3$ from loc1 \\
2: Store $t_0+1$ to loc2       & 7: Store $t_3$ to loc2 \\
\textit{Complicated local computation} & 8: Store $t_3 + 1$ to loc2 \\
3: Load $t_1$ from loc1        & \\
4: Load $t_2$ from loc2        & \\
5: Crash if $t_1 = t_2$ & \\
\end{tabular}
\hfill
\end{centering}
\caption{Partial fix for the bug in
  \autoref{fig:mandatory_concurrency1}.}
\label{fig:mandatory_concurrency2}
\end{sanefig}

\noindent
The class of bugs described above does not include order violation
bugs, and so {\technique} will never report any.  Order violation bugs
in the program can, however, still sometimes affect the results.
Consider, for instance, the threads shown in
\autoref{fig:mandatory_concurrency1}.  These threads have an order
violation bug, in that the thread on the left will crash if it gets
from statement 2 to statement 3 before the thread on the right
executes.  As expected, {\technique} will discover that running the
left-hand thread in isolation always leads to a crash, and so will not
report a bug here.

Suppose now that the ordering violation bug is fixed as shown in
\autoref{fig:mandatory_concurrency2}.  This program is ``more
correct'' than the previous one, in the sense that any instruction
which causes the new program to crash would also have crashed the old
one and some which crashed the old one will not crash the new one, but
{\technique} \emph{will} report a potential bug in the new program.
Running the two new threads atomically, in either order, will avoid
the crash, but interleaving them might not (consider, for instance,
the order 1, 2, 6, 7, 3, 4, 5).  The ordering violation bug
effectively hid the atomicity violation one and {\technique} will not
find either.

This is an undesirable property for {\technique} to have, but it is
unlikely to be a serious problem in practice.  Most concurrency bugs
in real programs tend to be at least moderately difficult to
reproduce, as otherwise the developers of the software will fix them
quickly, and for this bug that means that the complicated local
computation between must take long enough that the interfering thread
is almost certain to intercede and prevent the crash.  That means, at
a minimum, that the local computation must be large relative to the
system's scheduling jitter.  This will usually be at least tens of
microseconds, and is often several milliseconds, which is usually
sufficient to execute thousands to hundreds of thousands of
instructions.  At the same time, there is only any possibility of an
ordering violation bug hiding an atomicity violation one if both bugs
fit into {\technique}'s \gls{analysiswindow}, which, in practice,
cannot exceed a couple of hundred instructions.  As such, it is quite
unlikely that real programs would be able to trigger this
behaviour\editorial{Could do with some evidence for that.}.

\section{Execution model}

In addition to restricting the class of bugs, {\technique} also
restricts the execution environment by assuming a strongly-ordered
memory model, so that memory accesses are seen by all processors in
the order in which they appear in the program.  This is a reasonable
approximation for the widely-used x86 architecture\cite[Section
  8.2]{Intel2013}, as that platform rarely reorders memory accesses
issued by a single processor.  Architectures with a weaker memory
ordering, such as Alpha\cite[Section 5.6]{Compaq2002} or
ARM\cite[Section 5.3.4]{ARM2007}, would require more involved
processing to correctly capture the more complicated concurrency
semantics.

\section{Model of program modification}
\label{sect:intro:theory_of_fixing}

{\Technique} relies on being able to modify a program's behaviour in
order to reproduce and then to fix bugs, and aims to do so soundly, in
the sense that it should never introduce additional bugs.  This is not
entirely well-defined without access to a formal specification of the
program's desired behaviour.  It might, for instance, be that the
program is designed to investigate the possible ways in which a
particular processor can interleave memory accesses, and to report its
results by either exiting normally or crashing with an unhandled page
fault.  There is no general way for an automated tool to distinguish
such a program from one which is intended to always exit normally but
occasionally crashes due to an unintended race condition.  Any fix for
the latter would break the former.

This issue is most easily understood by recasting the problem from
changing the behaviour of the program to changing the behaviour of the
system on which it is running.  A program will usually be designed to
work with a broad class of computers, rather than with one particular
physical system, and so if it exhibits behaviour A on one system and
behaviour B on another then forcing it to exhibit behaviour B on both
systems is unlikely to introduce a bug into a correct program.  This
gives a reasonable definition of what it means to soundly modify a
program without a specification: a modification is safe if it is
equivalent to moving the program from one computer system to another
similar one.  For {\technique}, two systems are similar if they differ
only in the time taken to run instructions, and so modifications which
amount to simply delaying a particular operation are safe.  This makes
{\technique} safe for use on any systems which do not have real-time
constraints.  Prior systems have used different definitions; this is
discussed in more detail in \autoref{sect:rw:theory_of_fixing}.

\section{Graph generating grammars}
\label{sect:intro:graph_grammar}

\begin{sanefig}
  {\hfill}
  \tikzstyle{graphNT}+=[text width=1cm,fill=white]
  \begin{tabular}{lcclccrcc}
    \graphNT{$n$} & $\Rightarrow$ & \raisebox{-6mm}{\begin{tikzpicture}
        \node (n) {A};
        \node (nn) [style=graphNT, below=.5 of n] {$3n+1$};
        \draw[->] (n) -- (nn);
      \end{tikzpicture}} & \circled{1} & \hspace{1cm} &
    \graphNT{2} & $\Rightarrow$ & \raisebox{-6mm}{
      \begin{tikzpicture}
        \node (2) {C};
        \node (1) [style = graphNT, below = .5 of 2] {1};
        \draw[->] (2) -- (1);
      \end{tikzpicture}
    } & \circled{3} \\
    \graphNT{$m$} & $\Rightarrow$ & \raisebox{-6mm}{\begin{tikzpicture}
        \node (m) {B};
        \node (mm) [style=graphNT, below left =.5 of m] {$\frac{m}{2}$};
        \node (mmm) [style=graphNT, below right = .5 of m] {$\frac{m}{2} - 2$};
        \draw[->] (m) -- (mm);
        \draw[->] (m) -- (mmm);
    \end{tikzpicture}} & \circled{2} & &
    \graphNT{4} & $\Rightarrow$ & \raisebox{-6mm}{
      \begin{tikzpicture}
        \node (4) {D};
        \node (2) [style=graphNT, below = .5 of 4] {2};
        \draw[->] (4) -- (2);
      \end{tikzpicture}
    } & \circled{4} \\
  \end{tabular}
  {\hfill}
  \caption{Productions for the example graph generating grammar.  The
    terminals of this grammar are capital letters and the
    non-terminals are positive integers in boxes. $n$ matches odd
    integers and $m$ matches even integers other than two and four.
    Circled numbers are labels used to refer to the productions in the
    text.}
  \label{fig:intro:graph_grammar}
\end{sanefig}
\begin{sanefig}
  \newcommand{\arrowwidth}{0.03}
  \newcommand{\arrowhead}{0.05}
  \newcommand{\arrowlength}{0.8}
  \newcommand{\arrowdecoration}{}
  \newcommand{\labelledarrow}[1]{
    \hspace{-3.5mm}
    \begin{tikzpicture}
      \draw [\arrowdecoration] (0,-\arrowwidth) -- ++(\arrowlength,0);
      \draw [\arrowdecoration] (0,\arrowwidth) -- ++(\arrowlength,0);
      \draw [\arrowdecoration] (\arrowlength - \arrowhead + 0.01, 0 - \arrowwidth - \arrowhead) -- (\arrowlength + \arrowhead / 3 + \arrowwidth / 3, 0) -- (\arrowlength - \arrowhead + 0.01, \arrowwidth + \arrowhead);
      \node at (\arrowlength / 2,0) [above] {#1};
    \end{tikzpicture}\hspace{-3.5mm}
  }
  \tikzstyle{graphNT}+=[text width=1em, fill=white]
  \centerline{
  \begin{tikzpicture}[baseline=(r.base)]
    \node [style=graphNT] (r) {3};
  \end{tikzpicture}
  \labelledarrow{\circled{1}}
  \begin{tikzpicture}[baseline=(r.base)]
    \node (r) {A\!\!};
    \node [left=-8pt of r] (r3) {\graphNT{3}:};
    \node [below = of r, style=graphNT] (s) {10};
    \draw[->] (r) -- (s);
  \end{tikzpicture}
  \labelledarrow{\circled{2}}
  \begin{tikzpicture}[baseline=(r.base)]
    \node (r) {A\!\!};
    \node [left=-8pt of r] (r3) {\graphNT{3}:};
    \node [below=.57 of r] (s) {B\!\!};
    \node [left=-8pt of s] (r10) {\graphNT{10}:};
    \node [below=of s, style=graphNT] (t) {5};
    \draw[->] (r) -- (s);
    \draw[->] (s) -- (t);
    \draw[->] (s.east) .. controls +(.2,0) and +(0,-.2) .. ++(.33,.3) -- ++(0,0.55) .. controls +(0,.2) and +(.2,0) .. ++(-.3,.3) -- (r.east);
  \end{tikzpicture}
  \labelledarrow{\circled{1}}
  \begin{tikzpicture}[baseline=(r.base)]
    \node (r) {A\!\!};
    \node [left=-8pt of r] (r3) {\graphNT{3}:};
    \node [below=.57 of r] (s) {B\!\!};
    \node [left=-8pt of s] (r10) {\graphNT{10}:};
    \node [below=.57 of s] (t) {A\!\!};
    \node [left=-8pt of t] (r10) {\graphNT{5}:};
    \node [below=of t,style=graphNT] (u) {16};
    \draw[->] (r) -- (s);
    \draw[->] (s) -- (t);
    \draw[->] (t) -- (u);
    \draw[->] (s.east) .. controls +(.2,0) and +(0,-.2) .. ++(.33,.3) -- ++(0,0.55) .. controls +(0,.2) and +(.2,0) .. ++(-.3,.3) -- (r.east);
  \end{tikzpicture}
  \labelledarrow{\circled{2}}
  \begin{tikzpicture}[baseline=(r.base)]
    \node (r) {A\!\!};
    \node [left=-8pt of r] (r3) {\graphNT{3}:};
    \node [below=.57 of r] (s) {B\!\!};
    \node [left=-8pt of s] (r10) {\graphNT{10}:};
    \node [below=.57 of s] (t) {A\!\!};
    \node [left=-8pt of t] (r10) {\graphNT{5}:};
    \node [below=.57 of t] (u) {B\!\!};
    \node [left=-8pt of u] (r16) {\graphNT{16}:};
    \path (u.south) ++(-.7,-1) node [style=graphNT] (v) {8};
    \path (u.south) ++(.6,-1) node [style=graphNT] (w) {6};
    \draw[->] (r) -- (s);
    \draw[->] (s) -- (t);
    \draw[->] (t) -- (u);
    \draw[->] (u) -- (v);
    \draw[->] (u) -- (w);
    \draw[->] (s.east) .. controls +(.2,0) and +(0,-.2) .. ++(.33,.3) -- ++(0,0.55) .. controls +(0,.2) and +(.2,0) .. ++(-.3,.3) -- (r.east);
  \end{tikzpicture}
  \labelledarrow{\circled{2}}
  \begin{tikzpicture}[baseline=(r.base)]
    \node (r) {A\!\!};
    \node [left=-8pt of r] (r3) {\graphNT{3}:};
    \node [below=.57 of r] (s) {B\!\!};
    \node [left=-8pt of s] (r10) {\graphNT{10}:};
    \node [below=.57 of s] (t) {A\!\!};
    \node [left=-8pt of t] (r10) {\graphNT{5}:};
    \node [below=.57 of t] (u) {B\!\!};
    \node [left=-8pt of u] (r16) {\graphNT{16}:};
    \path (u.south) ++(-.7,-.85) node [inner sep = 1.5 pt] (v) {B};
    \node [left=-7pt of v] (r8) {\graphNT{8}:};
    \path (v.south) ++(-.5,-1) node [style=graphNT] (x) {4};
    \path (v.south) ++(.5,-1) node [style=graphNT] (y) {2};
    \path (u.south) ++(.6,-1) node [style=graphNT] (w) {6};
    \draw[->] (r) -- (s);
    \draw[->] (s) -- (t);
    \draw[->] (t) -- (u);
    \draw[->] (u) -- (v);
    \draw[->] (u) -- (w);
    \draw[->] (v) -- (x);
    \draw[->] (v) -- (y);
    \draw[->] (s.east) .. controls +(.2,0) and +(0,-.2) .. ++(.33,.3) -- ++(0,0.55) .. controls +(0,.2) and +(.2,0) .. ++(-.3,.3) -- (r.east);
  \end{tikzpicture}
  }
  \centerline{
  \labelledarrow{$\cdots$}
  \begin{tikzpicture}[baseline=(r10.base)]
    \node (r) {A\!\!};
    \node [left=-8pt of r] (r3) {\graphNT{3}:};
    \node [below=.57 of r] (s) {B\!\!};
    \node [left=-8pt of s] (r10) {\graphNT{10}:};
    \node [below=.57 of s] (t) {A\!\!};
    \node [left=-8pt of t] (r10) {\graphNT{5}:};
    \node [below=.57 of t] (u) {B\!\!};
    \node [left=-8pt of u] (r16) {\graphNT{16}:};
    \path (u.south) ++(-.7,-.85) node [inner sep = 1.5 pt] (v) {B};
    \node [left=-7pt of v] (r8) {\graphNT{8}:};
    \path (v.south) ++(-.6,-.85) node [inner sep = 1.5pt] (x) {D};
    \node [left=-7pt of x] (r4) {\graphNT{4}:};
    \path (v.south) ++(.7,-.85) node [inner sep = 1.5pt] (y) {C};
    \node [left=-7pt of y] (r2) {\!\!\graphNT{2}:};
    \path (u.south) ++(.9,-.85) node (w) {\!\!\graphNT{6}:\!\!\!\!};
    \path (r2 -| w) node (z) {\!\!\graphNT{1}:\!\!\!\!};
    \node [right=1pt of z] [inner sep = 1.5pt] (z2) {A};
    \node [right=1pt of w] [inner sep = 1.5pt] (z3) {B};
    \draw[->] (r) -- (s);
    \draw[->] (s) -- (t);
    \draw[->] (t) -- (u);
    \draw[->] (u) -- (v);
    \draw[->] (u) -- (z3);
    \draw[->] (v) -- (x);
    \draw[->] (v) -- (y);
    \draw[->] (x) -- (r2);
    \draw[->] (y) -- (z);
    \draw[->] (z3) -- (z2);
    \draw[->] (s.east) .. controls +(.2,0) and +(0,-.2) .. ++(.33,.3) -- ++(0,0.55) .. controls +(0,.2) and +(.2,0) .. ++(-.3,.3) -- (r.east);
    \draw[->] (z2.south) .. controls +(0,-.2) and +(.2,0) .. ++(-.3,-.3) -- ++(-2.15,0) .. controls +(-.2,0) and +(0,-.2) .. ++(-.3,.25) -- (x.south);
    \draw[->] (z3.north) -- ++(0,4.08) .. controls +(0,.2) and +(.2,0) .. ++(-.3,.3) -- (r.east);
  \end{tikzpicture}
  \labelledarrow{}
  \begin{tikzpicture}[baseline=(r10.base)]
    \node (r) {\!\!A\!\!};
    \node [below=.57 of r] (s) {\!\!B\!\!};
    \node [below=.57 of s] (t) {\!\!A\!\!};
    \node [below=.57 of t] (u) {\!\!B\!\!};
    \path (u.south) ++(-.8,-.85) node [inner sep = 1.5 pt] (v) {B};
    \path (v.south) ++(-.55,-.85) node [inner sep = 1.5pt] (x) {D};
    \path (v.south) ++(.55,-.85) node [inner sep = 1.5pt] (y) {C};
    \path (u.south) ++(.8,-.85) node [inner sep = 1.5pt] (z3) {B};
    \path (y -| z3) node [inner sep = 1.5pt] (z2) {A};
    \draw[->] (r) -- (s);
    \draw[->] (s) -- (t);
    \draw[->] (t) -- (u);
    \draw[->] (u) -- (v);
    \draw[->] (u) -- (z3);
    \draw[->] (v) -- (x);
    \draw[->] (v) -- (y);
    \draw[->] (x) -- (y);
    \draw[->] (y) -- (z2);
    \draw[->] (z3) -- (z2);
    \draw[->] (s.east) .. controls +(.2,0) and +(0,-.2) .. ++(.33,.3) -- ++(0,0.55) .. controls +(0,.2) and +(.2,0) .. ++(-.3,.3) -- (r.east);
    \draw[->] (z2.south) .. controls +(0,-.2) and +(.2,0) .. ++(-.3,-.3) -- ++(-1.55,0) .. controls +(-.2,0) and +(0,-.2) .. ++(-.3,.25) -- (x.south);
    \draw[->] (z3.north) -- ++(0,4.08) .. controls +(0,.2) and +(.2,0) .. ++(-.3,.3) -- (r.east);
  \end{tikzpicture}
  }
  \caption[Expansion of a non-terminal using the productions in
    \autoref{fig:intro:graph_grammar}]{Expansion of the non-terminal
    \graphNT{$\mathrm{3}$} using the productions in
    \autoref{fig:intro:graph_grammar}.  Circled numbers above the
    arrows show which production was used at each step.}
  \label{fig:intro:graph_grammar:expansion}
\end{sanefig}
\noindent Several of the algorithms in this dissertation are described
in terms of graph generating node-replacement grammars, and so I now
give a brief overview of this formalism, with reference to the example
in \autoref{fig:intro:graph_grammar}.  This shows a simple grammar
which produces directed graphs of (terminal) capital letters starting
from a single (non-terminal) integer.  The grammar works by matching a
pattern (on the left of the $\Rightarrow$) against some non-terminal
in the graph, possibly defining some match variables (in the example,
$n$ and $m$), generating a new fragment of graph (on the right of the
$\Rightarrow$), and replacing the original non-terminal with the
fragment.  This repeats until there are no more non-terminals in the
graph.  A non-terminal can only be generated at most once; if a
non-terminal is generated multiple times then it is only added the
first time and subsequent instances re-use the first one.

\autoref{fig:intro:graph_grammar:expansion} shows how to apply this
grammar to the initial non-terminal \graphNT{3}.  The only production
that matches this initial graph is \circled{1} with $n = 3$, which
generates a new terminal A with a single successor non-terminal
\graphNT{$3n+1$} = \graphNT{10}.  This non-terminal matches production
\circled{2} with $m=10$, producing a terminal B and non-terminals
\graphNT{$\frac{m}{2}$}=\graphNT{5} and \graphNT{$\frac{m}{2}-2$} =
\graphNT{3}.  The \graphNT{3} non-terminal has already been generated
once, and so rather than adding a new node to the graph the grammar
adds an edge back to the previous one.  The grammar continues
expanding non-terminals in this way until none remain, producing the
graph at the bottom left of the figure.  It then forgets which
non-terminal generated each terminal, producing the final graph at the
bottom right of the figure.

\section{Structure of this dissertation}

This dissertation will, over the following chapters, present a
detailed description of how {\technique} works, starting with the
mechanism used to find and characterise bugs (\autoref{sect:derive})
and then moving on to describe how it first reproduces
(\autoref{sect:reproducing_bugs}) and then fixes
(\autoref{sect:fix_global_lock}) those bugs.  With the technique
itself described, I then give some experimental results obtained with
my prototype implementation {\implementation} (\autoref{chapter:eval})
and compare the technique to existing work in this area
(\autoref{chapter:related_work}).  Finally, I conclude and present
some possible avenues for future work (\autoref{sect:fw_concl}).
