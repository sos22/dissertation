\cleardoublepage
\newpage
\mbox{}
\newpage

\topskip0pt
\vspace*{\fill}
\centerline{{\bfseries \abstractname}}

\noindent
Various trends in computing mean that future software will make
increasingly use of concurrency, which implies that it will suffer
from increasingly frequent concurrency bugs.  At present, such bugs
are difficult for programmers to understand and fix, largely because
they tend to reproduce in complex and unpredictable ways.  This
dissertation presents {\technique}, a suite of complementary
techniques aimed at discovering, characterising, reproducing, and
ultimately fixing a particular class of concurrency bugs.  These
techniques require access only to the program binary, and not to its
source, and require minimal manual intervention.  I evaluate
{\implementation}, my prototype implementation of {\technique},
experimentally, showing that it can usefully be applied to real bugs
in large existing software projects, and characterising when it is
likely to fail.  I then place {\technique} in context by comparing it
to existing approaches to these problems, describing how it builds
upon or complements these other systems, before closing with a
discussion of possible future work and a summary of the conclusions
drawn.

The core idea in {\technique} is the \gls{bugenforcer}: a co-program
which runs alongside a running program and gently shepherds its
execution towards a schedule which is likely to reproduce a particular
bug.  {\Technique} uses these both to weed out false positives
produced by its initial (highly conservative) static analysis and to
confirm properties of the bug needed to generate its fix.
\Glspl{bugenforcer} are also useful in their own right: as I show in
the evaluation, they can often reduce the time taken to reproduce a
bug by many orders of magnitude when compared to stress testing alone,
which would potentially be of great help to a programmer tasked with
understanding and eliminating some undesirable behaviour.  I describe
the \gls{bugenforcer} mechanism in detail, showing both how it works
and some of its more important weaknesses.

\vspace*{\fill}

