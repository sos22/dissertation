Various trends in computing mean that future software will be
increasingly parallel, which means that concurrency bugs will be an
increasingly serious problem.  At present, such bugs are difficult for
programmers to understand and fix, largely because they tend to
reproduce in complex and unpredictable ways.  This dissertation
presents {\technique}, a suite of complementary techniques aimed at
discovering, characterising, reproducing, and ultimately fixing a
particular class of concurrency bugs.  These techniques require access
only to the program binary, and not to its source, and require minimal
manual intervention.  I evaluate {\implementation}, my prototype
implementation of {\technique}, experimentally, showing that it can
usefully be applied to real bugs in large existing software projects,
and giving an characterising when it is likely to fail.  I then place
{\technique} in context by comparing it to existing approaches to
these problems, describing how it builds upon or complements these
other systems, before closing with a discussion of possible future
work and a summary of the conclusions drawn.
