Various trends in computing mean that future software will be
increasingly parallel, which means that concurrency bugs will be an
increasingly serious problem.  At present, such bugs are difficult for
programmers to understand and fix, largely because they tend to
reproduce in complex and unpredictable ways.  I present a set of
techniques which produce a modified version of a program which
reproduces a particular concurrency bug far more easily, hence
assisting developers in fixing that bug.  I also show how the
technique can, in some cases, be generalised to fix the bug, or to
find completely unknown bugs.  These techniques are all completely
automated and require only the program binary as input.  I show that
they can be used to both reproduce and fix real bugs in large software
systems.

