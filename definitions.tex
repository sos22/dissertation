\tikzset{BddNode/.style={rectangle,minimum size=6mm}}
\tikzset{BddLeaf/.style={rectangle,minimum size=6mm}}
\tikzset{BddTrue/.style={}}
\tikzset{BddFalse/.style={dotted,thick}}

\tikzset{CfgInstr/.style={rectangle,minimum size=6mm}}
\tikzset{CommCfgInstr/.style={rectangle,minimum size=6mm,fill=blue!50}}
\tikzset{DupeCommCfgInstr/.style={rectangle,minimum size=6mm,fill=blue!10}}
\tikzset{InterferingCfgInstr/.style={rectangle,minimum size=6mm,fill=green!50}}
\tikzset{DupeInterferingCfgInstr/.style={rectangle,minimum size=6mm,fill=green!10}}
\tikzset{NewCfgInstr/.style={rectangle,minimum size=6mm,fill=red!50}}
\tikzset{stateSideEffect/.style={rectangle,draw,fill=white, inner sep = 2pt}}
\tikzset{stateIf/.style={ellipse,draw,fill=white,inner sep = 2pt}}
\tikzset{stateTerminal/.style={ellipse,draw,fill=white, inner sep = 2pt}}
\tikzset{killEdge/.style={decorate,decoration={snake,segment length=1mm,post length=0.4mm}}}
\tikzset{newCfgEdge/.style={color=red}}
\tikzset{happensBeforeEdge/.style={dashed}}
\tikzset{ifTrue/.style={}}
\tikzset{ifFalse/.style={dotted,thick}}

\newcommand{\draftonly}[1]{#1}
%\newcommand{\draftonly}[1]{}
\newcommand{\smh}[1]{\textcolor{green}{Steve says: #1}}
\newcommand{\editorial}[1]{\draftonly{\textcolor{red}{\footnote{\textcolor{red}{#1}}}}}

\newcommand{\hgreyline}{\arrayrulecolor{black!20}\hline\arrayrulecolor{black}}

\newlength{\footnoteadjustlength}
\setlength{\footnoteadjustlength}{2pt}
\newcommand{\fnote}[1]{\hspace{-\footnoteadjustlength}\footnote{#1}}
\newcommand{\randsched}{\textbf{randsched}}
\newcommand{\StateMachine}{approximant}
\newcommand{\StateMachines}{approximants}
\newcommand{\AStateMachine}{an approximant}
\newcommand{\STateMachine}{Approximant}
\newcommand{\STateMachines}{Approximants}
\newcommand{\implementation}{Shepgestaler}
\newcommand{\Implementation}{Shepgestaler}
\newcommand{\atechnique}{a Raft}
\newcommand{\technique}{Raft}
\newcommand{\Technique}{Raft}

\newcommand{\false}{\ensuremath{\mathit{false}}}
\newcommand{\true}{\ensuremath{\mathit{true}}}

\newcommand{\mai}[2]{\mathrm{#1}\!:\!\mathrm{#2}}
\newcommand{\happensBeforeEdge}{\leftarrowtail}
\newcommand{\happensBefore}[2]{#1\,{\happensBeforeEdge}\,#2}
\newcommand{\controlEdgeName}{\mathsf{Control}}
\newcommand{\controlEdge}[3]{{\controlEdgeName}(#1\!:\!#2\!\rightarrow\!#3)}
\newcommand{\entryExpr}[1]{\mathsf{Entry}(#1)}
\newcommand{\smVar}[1]{v_{#1}}
\newcommand{\smLoad}[1]{\mathsf{InitMemory}(#1)}
\newcommand{\smBadPtr}[1]{\mathsf{BadPtr}(#1)}
\newcommand{\smReg}[2]{\mathsf{InitReg}(\textsc{#1}_{#2})}
\newcommand{\stPhi}{\state{$\Phi$}}
\newcommand{\stLoad}[2]{\state{Load} $\ast(#2) \rightarrow \smVar{#1}$}
\newcommand{\stCopy}[2]{\state{Copy} $#2 \rightarrow \smVar{#1}$}
\newcommand{\stStore}[2]{\state{Store} $#1 \rightarrow \ast(#2)$}
\newcommand{\stIf}[1]{\state{If} ($#1$)}
\newcommand{\stAssert}[1]{\state{Assume} ($#1$)}
\newcommand{\stSurvive}{\state{Survive}}
\newcommand{\stCrash}{\state{Crash}}
\newcommand{\stUnreached}{\state{Unreached}}
\newcommand{\stStartAtomic}{\state{StartAtomic}}
\newcommand{\stEndAtomic}{\state{EndAtomic}}

\newcommand{\bugname}[1]{\textsf{#1}}

\newcommand{\queue}[1]{\{#1\}}
\newcommand{\map}[1]{\{#1\}}
\newcommand{\mapIndex}[2]{#1[#2]}
\newcommand{\state}[1]{\textsc{#1}}
\newcommand{\subfigureautorefname}{\figureautorefname}

\newcommand*\circled[1]{\tikz[baseline=(char.base)]{
    \node[shape=circle,draw,inner sep=1pt,fill=white] (char) {#1};}}
\tikzstyle{graphNT}=[shape=rectangle,draw,double,text centered, inner sep=2pt, fill=white]
\newcommand{\graphNT}[1]{\tikz[baseline=(char.base)]{\node [style=graphNT] (char) {#1};}}
\newcommand{\production}[1]{\tikz[baseline=(char.base)]{
    \node[shape=circle,draw,inner sep=1pt,fill=white,text width=1em, text centered] (char) {
      \ifthenelse{\equal{#1}{2}%
      }{%
        \hspace{-.3pt}2%
      }{%
        \ifthenelse{\equal{#1}{4}%
        }{%
          \hspace{-0.9pt}{4}%
        }{$#1$}}%
    };
  }%
}

\newenvironment{literalC}{\tt \begin{tabbing}}{\end{tabbing}\vspace{-12pt}{\hfill}}
\newcommand{\clbrace}{\{ \\ \hspace{3pt} \=\+ }
\newcommand{\crbrace}{\- \\ \}}
